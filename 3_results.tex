\subsection{Main result: Minority hosts have lower prices than white hosts} 

Before analysis, the data set used was restricted to hosts who have profile pictures and manage less than 20 listings, and listings priced at less than \$800 per night. 64,611 listings were left after restricting the data set. There were only 20 hosts who did not have profile pictures.

Table 3 presents OLS estimates of the effect of host race and gender on the listing price. The specification is of the form: 

\[ Price_{i,j} = \beta_1 Race_{i}\,X \,Sex_i + \beta_2 Age_i + \beta_3 x_{i,j}\]

The $Price_{i,j}$ is host $i$'s price from their Airbnb listing $j$. For hosts with multiple listings, each listing is treated separately. The $Race_{j}\,X \,Sex_j$ is the interaction of the race and sex of the host. White males are the omitted category. $Age_i$ is the age of the host (young, middle-aged, or senior). Young hosts are the omitted category for age. $x_{i,j}$ is vector of other covariates that grows from left to right in the columns of Table 3. The columns are additive in their covariates, so each column controls for everything in the previous columns, plus a new set of covariates. Standard errors are clustered by neighborhood throughout.

The first column, Model 1, in Table 3 presents the raw effect of host race and sex on the price of a listing. These are consistent with the mean listing prices by race presented in Table 2, except now also broken down by male and female hosts within each racial category.

Model 2 adds city and neighborhood fixed effects.%
	\footnote{Neighborhoods are assigned in accordance with each city's designations. In Chicago, for example, the fixed effect granularity is at the level of locating a listing within Hyde Park vs. Woodlawn (The University of Chicago is located in Hyde Park, and Woodlawn is directly adject to it).}
Location is an important proxy for income levels, crime rates, and distance from downtown, which are all predictors of property prices and therefore listing prices on Airbnb. As expected, controlling for location substantially decreases the estimated racial gaps in prices. The coefficients for minority hosts decrease from a range of \$20-40 to a range of \$10-20 (these are all negative, and relative to white male hosts). I observe the largest decrease in the coefficients on black hosts, which go down from \$40 to roughly \$15. Coefficients of Hispanic hosts decrease by around \$10; Asian hosts by about \$20. 

It is well-documented that blacks in urban populations are nearly four times more likely than whites to live in neighborhoods where the poverty rate is 40\% or higher.\cite{firebaugh} In fact, minorities at every income level live in poorer neighborhoods than do whites with comparable incomes. For example, a black household earning \$75,000 a year resides in a higher-poverty neighborhood than a white household with earnings of less than \$40,000 a year.\cite{logan} It is therefore expected that a large part of the variation in Airbnb prices between those groups can be explained by their listing's location. The coefficients of white females, on the other hand, persist at around \$4 with the addition of location controls. This is most likely because white females tend to live in the same areas as white males and therefore have little to no variation in price that can be explained by differences in neighborhood.  

Model 3 adds controls for listing-specific characteristics. Listing characteristics include fixed effects for the property type and room type, the listing's duration on the market, the number of guests the listing accommodates, the number of bathrooms, bedrooms, and beds, the bed type, the number of amenities, the number of minimum nights, any extra fees, whether the listing is instantly bookable, and the cancellation policy.% 
	\footnote{The listing's duration on the market is proxied by fixed effects for the month and year of the listing's first review.}
Controlling for these listing characteristics decreases all effects to \$5-10, depending on the race of the host. Asian female hosts have the largest decrease in coefficient after controlling for listing characteristics, which indicates that a substantial part of their effect is driven by owning properties with worse observable characteristics. The effects on middle-aged and senior hosts are almost eliminated by controlling for property characteristics, indicating that their higher listing prices are primarily driven by better observable characteristics. The effects for Hispanic males and white women largely disappear with the addition of property controls. 

In general, from Model 1 to Model 3, coefficients steadily decrease in magnitude and the $R^2$ increases from .166 with neighborhood controls to .621 with listing controls. Most of the variation in price between minority hosts and white male hosts can be explained by either the property's location or observable property characteristics. The $R^2$ jumps substantially to .621 in Model 3, so adding property characteristics explains much more of the total price variation than the location. This might be because Airbnb listings tend to be more concentrated in certain areas of each city (North Side in Chicago, lower and middle Manhattan in New York City, etc). If listings in a city cluster together instead of being uniformly dispersed, then controlling for location won't explain as much of the variation as controlling for property characteristics. LISTING OR PROPERTY PICK ONE - FIX

Model 4 in the last column presents my full, preferred specification. It adds host-specific characteristics to Model 3, including the host response time and the host response rate, whether the host is a Superhost, whether the host identity was verified by Airbnb, and if the host requires a guest's profile picture or phone to book. 

Importantly, Model 4 also controls for variation in host effort. I attempt to account for the idea that some hosts may have higher prices not because of better observable characteristics, but just because they are better hosts. There are several host-written fields on each listing page, the ``Summary", ``Description", ``Space", ``Neighborhood Overview", ``Transit", and ``Notes". By filling out these fields, hosts not only describe their listing, but have the opportunity to provide guests with helpful tips and information about the surrounding area. How well a host writes these descriptions is an indication of how much effort they are willing to put into hosting. To this end, I construct three variables to measure host effort. My first variable simply measures the length of each of these fields. Presumably, the longer the description, the more effort the host put into writing it. My second variable measures whether these fields had mostly long words or short words, so that a description that uses shorter words, such as ``My house is nice", would be counted as lower quality than ``My house is gorgeous". 

My third measure of host effort is a rudimentary sentiment analysis of the ``Description" field. Hu and Liu (2004) create a list of 2,006 positive words that commonly appear in customer reviews to aid in sentiment categorization.\cite{hu} I only include words that have substantial variation in the description, meaning that more than 5\% of descriptions had these words. This narrowed the list of viable words significantly. I take 7 positive words from that list that would be most relevant for Airbnb listings: ``spacious", ``beautiful", ``clean", ``comfort", ``great", ``love", and ``quiet". I then added a covariate for the number of these ``good words" in the host's ``Description" field. Together, these three ``host effort" variables control for hosts who write longer descriptions, use longer words in those descriptions, and put more words that are associated with positive reviews in their descriptions. 

After controlling for my final specification, I estimate that, across the board, minority hosts earn lower prices from their Airbnb listing than white hosts. The biggest effect is for Asian female hosts, whose prices are roughly \$9 less per day than white male hosts who own the same type of listing. The second biggest effect is for black males, with a coefficient of \$7. The coefficients on black women and Asian men are \$6 per day each, Hispanic females is \$5. This effect is statistically significant at the p $<$ .001 level for black hosts and Asian women, the p $<$ .01 level for Asian male hosts, and p $<$ .05 level for Hispanic women. For Hispanic men the effect is around \$2 and is not statistically significant. There is little effect for white females, and a small effect that's not statistically significant for middle-aged and senior hosts. An F-test shows that host race is jointly significant for price at the p $<$ .001 level after controlling for both property and host characteristics. CHECK THIS FTEST My results are stable to the addition of host characteristic controls while still clustering standard errors at the neighborhood level. The inclusion of these host characteristics does not improve the fit of the model substantially. Property characteristics and location still explain more of the variation in price than host characteristics. 

My results are consistent with Edelman and Luca's findings, but I find smaller effects (they found a 12\% price disparity, I found about a 7\% price disparity). This is most likely because I control for a larger set of covariates. To confirm this, I run a regression on listings in New York City, controlling for the same covariates that Edelman and Luca used in their main result. The results, presented in Table 4, show that I get the same coefficient as the one they found - an \$18 (X\%) price difference between black hosts and white hosts. This indicates that my main results in Table 3 were smaller because I controlled for more variation, not because of a structural change in the extent of discrimination in Airbnb.%
	\footnote{Airbnb has changed their user interface in the past four years, so I approximated several of their regressors with the closest variable available in my data. For example, instead of whether the host had social media accounts, I controlled for whether the host's contact information was verified by Airbnb.}

If one believed the price difference was driven by unobserved characteristics, one might have expected that the price gap between white and minority hosts would disappear with the addition of more controls. This is true up to a point, since when I add more covariates my coefficients shrink relative to Edelman and Luca's. However, after that, my coefficient of interest is stable to the addition of controls - adding host-specific controls does not substantially change any of the effects. As one might expect, the $R^2$ goes up to .621 with the addition of location and property controls, but adding my host controls increases the $R^2$ by only .006. 

There are a few possible sources of unexplained variation in the price of the listing - variation in the real, physical qualities of the listing that wasn't captured by the property controls, and variation in the quality of the listing's profile that was not captured by the host controls. Since I was able to control for all of the property-quality variables that Airbnb offers on a listing page, it is unlikely that there are unobserved property characteristics driving the price differences. Since adding host controls explained very little variation in the price, increasing the $R^2$ by only .006, it is unlikely that adding more sophisticated measures of host quality or effort would significantly help explain price disparities. While this does not eliminate the possibility that there is a set of controls not related to property type or host type that would have increased the $R^2$ drastically, this is still good evidence to believe that the price difference I estimate is a real difference, rather than purely caused by endogeneity. 

\subsection{Secondary result: Minority hosts have lower quantity demanded than white hosts}

\textbf{Are prices lower due to supply-side effects?}

Market discrimination laid out by Becker rests on the idea that groups who are discriminated against see lower demand in the market, which drives down their prices. I have estimated that prices are lower, but there could be multiple explanations in addition to discrimination for this result. One hypothesis is that, for the same type of listing, minority hosts \textit{choose} to price their listings lower than white hosts. This might happen if minority hosts earn lower wages because they experience discrimination in the labor market. A lower opportunity cost of time would mean they have a lower marginal cost of putting up and managing their listing than a white host who owns a similar property. If minority hosts choose to price lower for every quantity, this is, in effect, equivalent to the supply curve being lower for minority hosts relative to white hosts. We therefore know that we can test this hypothesis by looking at the quantity demanded. If prices are low because the supply curve is lower, then minority hosts would have a higher quantity demanded. Conversely, if the prices are low because the demand curve is lower - which would be in line with the presence of discrimination - then the quantity demanded should be lower than it is for white hosts. 

In order to test this hypothesis, I use number of reviews as a proxy for quantity demanded. In Table 6, I regress the number of reviews on host race, controlling for the same set of models as Table 3. I find that minority hosts have either the same or lower review numbers than white hosts for a similar listing that spent the same amount of time on the market. Most coefficients are either roughly zero, or negative and in the range of 1-2 reviews less than white hosts. The results are significant for white females and black hosts. While the coefficients were significant for Asian hosts under the less robust specifications, under the full specification the coefficient is not significant, but still slightly negative. In general, my results suggest that non-white male hosts do not see a higher quantity demanded than white male hosts. This is evidence against the supply-side explanation. 

It is important to keep in mind that this conclusion is only salient if the total number of reviews is a reasonable proxy for the demand of a listing. Yet, one can imagine that if reviewers systematically under-review minority hosts relative to white hosts, these groups would have lower numbers of reviews that do not necessarily represent a lower quantity demanded. There is no way to tell apart these mechanisms in my data, and no previous research has been done on race-based differences in review rates. 

My working assumption is that even if not every guest leaves a review, the review proportion is similar across host race, and a lower number of reviews therefore indicates a real difference between quantity demanded of minority hosts and white hosts. I substantiate this assumption with another supply-side metric, a measure of listing vacancy that I explore in the next section, that provides more evidence that lower prices are not driven by supply-side effects. Taken together, these two measures will provide strong evidence to reject the hypothesis that lower prices are due to supply-side effects. 


\textbf{Are number of reviews lower because minority hosts offer their listings a fewer number of days?}

In the previous section, I argued that minority hosts had a lower quantity demanded than comparable white hosts. However, they may have lower number of guests because they offer up their listing for fewer days of the month, not because people don't want to stay with minority hosts. In order to test this, I regress the availability of the listing out of 30 days on host race, controlling for my preferred specification. The availability of a listing is controlled by the host, who can update their availability calendar on their listing page. Potential guests can then see on which days the listing is available and book accordingly. When a guest books an available day, that day is removed from the availability calendar. Therefore, the availability out of 30 days measure is a true measure of the \textit{vacancy} rate of a listing.

The results, presented in Table 7, are striking. I find that the listings of black hosts spend about 20\% more time on the market vacant than the listings of white males. The effect is statistically significant, and amounts to about 2-3 days per month in real units. Interestingly, white females make their listing less available than white males, with approximately a .9 of a day statistically significant difference. This might explain why white females don't have lower prices than white males, but did have lower numbers of reviews. Perhaps white females simply offer their listing for fewer days than their white male counterparts. There were no statistically significant effects in availability for Hispanic hosts, like for most of my measures. Asian female hosts actually had a lower vacancy rate than white male hosts, which could help explain why they have a lower number of reviews. 

Overall, there is strong evidence that even though black hosts offer their listing for more days, they have less people staying with them than white hosts. This is significant evidence to reject the supply-side hypothesis for black hosts. This is not the case for Asian hosts - the coefficients for Asian hosts were negative, indicating that they actually are putting up their listing for rent less often than white hosts. This means that lower availability is another possible explanation, in addition to discrimination, for why Asian hosts have a lower number of reviews. However, I am not able to further distinguish between these two hypotheses in my data.  
