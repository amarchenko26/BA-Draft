% EMPIRICAL APPROACH

I use OLS to estimate the impact of host race on listing price on Airbnb, controlling for location, property, and host characteristics. My main specification is of the form:

\[ log(\text{Price}_{i,j}) = \alpha_i + \beta_1 \text{Race}_{i} \;  \text{X} \; \text{Sex}_i + \beta_2 \text{Age}_i + \beta_3 \text{Location} +  x_{i,j} \]

Where $log(\text{Price}_{i,j})$ is the log of host $i$'s price from their Airbnb listing $j$. For hosts with multiple listings, each listing is treated separately. $\text{Race}_{i} \;  \text{X} \; \text{Sex}_i$ is the interaction of the race and sex of the host, where White males are the omitted category throughout. $\text{Age}_i$ is a dummy variable for whether the host is young, middle-aged, or senior. $x_{i,j}$ is a vector of property and host controls that grows additively in each model. Together, I control for all features of the listing that are available to a potential guest, as well as additional metrics that aim to capture unobservable differences between hosts. Each column of Tables \ref{table:price} and \ref{table:quantity_demanded} controls for everything in the previous columns, plus a new set of covariates, as detailed below. All other tables control only for the full specification, Model 4. Standard errors are clustered by neighborhood throughout.

\begin{enumerate}
	\item \textit{Model 1} presents the raw effect of host race and sex on the price of a listing. These coefficients are consistent with the mean listing prices by race presented in Table \ref{table:listing_summary}, but further broken down by male and female hosts within each racial category.
	
	\item \textit{Model 2} adds city and neighborhood fixed effects, as well as zipcode-level Census information on demographics and various indicators of economic health. 
	
	Adding neighborhood fixed effects removes any variation in prices that are due to a property's location in a neighborhood with better amenities or proximity downtown. However, if Airbnb listings are clustered in certain areas of each city, neighborhood controls might not be very informative. To this end, I add zipcode-level data on property values, proxies for the desirability of the neighborhood (the unemployment rate, the occupancy rate, and other measures of poverty, see full details in \ref{data}). I also include population density and commuting time to work as proxies for distance to downtown. 
		
	\item \textit{Model 3} adds controls for listing-specific characteristics, as detailed in \ref{data}. See Table \ref{table:listing_summary} for a full list of property controls. I also control for the listing's duration on the market by proxying with fixed effects for the month and year of the listing's first review.
	
	\item \textit{Model 4} represents my full specification. I add all remaining host-dependent fields on the listing page, such as Superhost status, the host's response time, and their cancellation policy. I also include my constructed host quality controls in the form of sentiment analysis of the text on the listing page to control for hosts who write more objective descriptions of more positive valence. See Table \ref{table:host_summary} for a full list of these controls and Section \ref{data} for details of construction of host effort variables. 
\end{enumerate}




