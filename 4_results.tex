\subsection*{Minority hosts have lower prices than White hosts} 
	\label{result1}
	

Table \ref{table:price} presents OLS estimates of the effect of host race and gender on the listing price. 

[INSERT DISCUSSION HERE AS TO HOW COEFS DECREASE WITH ADDITIONAL MODELs, for example: Model 3 discussion: Controlling for listing characteristics decreases all effects to x - y, depending on the race of the host. The effects for Hispanic males and White women largely disappear with the addition of property controls.]

After controlling for my final specification, I estimate that, across the board, minority hosts, both male and female, earn lower prices from their Airbnb listing than White hosts. The biggest effect is for Asian hosts, whose prices are roughly 4 and 4.5\% for female and male hosts, respectively, less per day than White male hosts who own the same type of listing. The second biggest effect is for Black males, whose price is lower by 3.2\%.%
	\footnote{This effect is statistically significant at the p $<$ .001 level for Asian female and male hosts, and the p $<$ .01 level for Black male hosts.} 
For Hispanic, and Black female hosts, the effect is small, around 2\%, and is not statistically significant. My results are stable to the addition of host characteristic controls while still clustering standard errors at the neighborhood level. The inclusion of host characteristics does not improve the fit of the model substantially.  

[Should i insert second part of intro here (per my email)]

I also break up the effects of host race on listing price by city and controlled for my preferred specification. The results are in Table \ref{table:robustcity}. In general, no single city is driving all of the variation in my data. The effects on price are mostly negative for minority hosts, with a few positive coefficients in cities with fewer observations, none of which are significant. 

I use two different measures of quantity demanded - the number of reviews and the vacancy rate. Both confirm that Black hosts and White females have lower quantity demanded than White males. The results for Asian and Hispanic hosts are mixed. 

My first measure of quantity demanded is the number of reviews. In Table \ref{table:num_reviews}, I regress the number of reviews on host race, controlling for the same set of models as Table \ref{table:price} (including the listing's time on the market, an important driver of the number of reviews). I find that minority hosts have either the same or lower review numbers than White hosts for a listing that spends the same amount of time on the market. Specifically, Black hosts, White females, and Hispanic Males have 7 to 8\% fewer reviews than White males. Coefficients are roughly zero for Hispanic females and Asian hosts. 

A second proxy for quantity demanded is the number of days per month a listing remains vacant. Minority hosts may have a lower quantity demanded because they offer up their listing for fewer days of the month, not because they face lower demand. In order to test this, I regress the availability of the listing on host race, controlling for my preferred specification. The availability of a listing is a measure of vacancy for the following reason: availability is controlled by the host, who can update their availability calendar on their listing page. Potential guests can then see on which days the listing is available and book accordingly. When a guest books an available day, that day is removed from the availability calendar. Therefore, the availability out of 30 days measure is a true measure of the vacancy of a listing. It is  important to note that days could be unavailable on a listing's calendar for two reasons: either a host marks them as unavailable, or a guest books on that day. Therefore, if a host has few vacancies, it is not possible to tell if they have high demand, or simply no time to manage their listing. This is not inherently problematic unless the minority and White hosts mark days as unavailable at different rates. 

Vacancy rates can also be affected by the number of properties a host owns - if a host does not live in the property they rent out, they might make more days available on their calendar. If White hosts are more likely to own second or third properties, they would have more availability on their calendars, which would underestimate the gap in availability between White hosts and minority hosts.  

The results, presented in Table \ref{table:availability}, are striking. I find that the listings of Black hosts spend about 20\% more time vacant on the market than the listings of White males. The effect is statistically significant, and amounts to about 2 - 3 days per month in real units. Interestingly, White females make their listing less available than White males, with a statistically significant difference of one day. Asian females are similar to White females, making their listing available one day less than White males. 

Overall, evidence shows that even though Black hosts offer their listings for more days and charge lower prices, fewer guests stay with them. This is significant evidence for the presence of discrimination against Black hosts. Female Asian and female White hosts, on the other hand, choose to make their listing available less often than White hosts. Lower availability is therefore a possible explanation for why these groups have a lower number of reviews. I am not able to further distinguish between availability effects and discrimination in my data. Finally, the quantity demanded results for Hispanic hosts are not statistically significant.














\begin{comment}
	coThis might be because Airbnb listings tend to be more concentrated in certain areas of each city (North Side in Chicago, lower and middle Manhattan in New York City, etc). If listings in a city cluster together instead of being uniformly dispersed, then controlling for location won't explain as much of the variation as controlling for property characteristics. LISTING OR PROPERTY PICK ONE - FIX
	
	It is well-documented that Blacks in urban populations are nearly four times more likely than Whites to live in neighborhoods where the poverty rate is 40\% or higher \cite{firebaugh}. 
	
	The coefficients for minority hosts decrease from a range of \$20-40 to a range of \$10-20 (these are all negative, and relative to White male hosts). I observe the largest decrease in the coefficients on Black hosts, which go down from \$40 to roughly \$15. Coefficients of Hispanic hosts decrease by around \$10; Asian hosts by about \$20. In fact, minorities at every income level live in poorer neighborhoods than do Whites with comparable incomes. For example, a Black household earning \$75,000 a year resides in a higher-poverty neighborhood than a White household with earnings of less than \$40,000 a year \cite{logan}. The coefficients of White females, on the other hand, persist at around \$4 with the addition of location controls. This is most likely because White females tend to live in the same areas as White males and therefore have little to no variation in price that can be explained by differences in neighborhood.  
	
	Asian female hosts have the largest decrease in coefficient after controlling for listing characteristics, which indicates that a substantial part of their effect is driven by owning properties with worse observable characteristics. The effects on middle-aged and senior hosts are almost eliminated by controlling for property characteristics, indicating that their higher listing prices are primarily driven by better observable characteristics. 
	
	If one believed the price difference was driven by unobserved characteristics, one might have expected that the price gap between White and minority hosts would disappear with the addition of more controls. However, my coefficient of interest is stable to the addition of controls - adding host-specific controls does not substantially change any of the effects. 
	
	There are a few possible sources of unexplained variation in the price of the listing - variation in the real, physical qualities of the listing that wasn't captured by the property controls and variation in the quality of the listing's profile that was not captured by the host controls. Since I was able to control for all of the property-quality variables that Airbnb offers on a listing page, it is unlikely that there are unobserved property characteristics driving the price differences. Since adding host controls explained very little variation in the price, increasing the $R^2$ by only .006, it is unlikely that adding more sophisticated measures of host quality or effort would significantly help explain price disparities. While this does not eliminate the possibility that there is a set of controls not related to property type or host type that would have increased the $R^2$ drastically, this is still good evidence to believe that the price difference I estimate is a real difference, rather than purely caused by endogeneity. 
	
	The results are significant for White females and Black hosts. While the coefficients were significant for Asian hosts under the less robust specifications, under the full specification the coefficient is not significant, but still slightly negative.
\end{comment}