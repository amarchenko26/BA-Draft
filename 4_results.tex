\subsection*{Minority hosts have lower prices than White hosts} 
	\label{result1}
	

Table \ref{table:price} presents OLS estimates of the effect of host race and gender on the listing price according to the models described in Section \ref{empirical}. 

[INSERT DISCUSSION HERE AS TO HOW COEFS DECREASE WITH ADDITIONAL MODELs, for example: Model 3 discussion: Controlling for listing characteristics decreases all effects to x - y, depending on the race of the host. The effects for Hispanic males and White women largely disappear with the addition of property controls.]

The point estimates in Model 4 show that Asian hosts, both male and female, earn lower prices from their Airbnb listing than White hosts. These effects are 4 and 4.5\% for female and male hosts, respectively, less per day than White male hosts who own the same type of property. The second biggest effect is for Black males, whose price is lower by roughly 3.5\%.%
	\footnote{This effect is statistically significant at the p $<$ .001 level for Asian female and male hosts, and the p $<$ .01 level for Black male hosts.} 
For Hispanic, and Black female hosts, the point estimate is negative but not statistically significant. My results are stable to the addition of host characteristic controls while still clustering standard errors at the neighborhood level. The inclusion of host characteristics does not improve the fit of the model substantially.  

Model 1 is a naive regression and presented solely to show the baseline. I add the listing's location in Model 2 . Since a listing's price is strongly correlated with its location, it is unsurprising that a large amount of variation in Airbnb prices between racial groups can be explained by these location controls. [address R2 here]
This is consistent with the idea that where individuals of different races live is not randomly distributed across neighborhoods for a variety of reasons, including persistent urban segregation and in-group preferences [MELODY CITE HERE]. For example, it is well-documented that Blacks in urban populations are nearly four times more likely than Whites to live in neighborhoods where the poverty rate is 40\% or higher \citep{firebaugh}. This is consistent with my results, as Black hosts had the largest drop in point estimate with the addition of location controls. Asian hosts were similarly affected, this is also unsurprising, as urban Asian populations [MELODY - is there a paper or stat about urban asian poverty? Like something that shows that asian americans live in worse neighborhoods that white people?can you cite  it, and insert a sentence here  like "despite the stereotype of asian americans as the model  minority, there are persistent gaps in the income/wealth/quality of neighborhoods ofasians and whites.]


Property characteristics have the most explanatory power in accounting for price differences, as the $R^2$ increases from around .25 to .72 with the addition of listing controls in Model 3. 

The effects of Asian host race on listing price is less than what was measured by \cite{wang} and \cite{kakar}. \cite{wang} measured a price disparity of  Both studies 





There are several possible sources of unexplained variation that could remain. One is the quality of the text, such as the reviews and descriptions. Another potential source is the quality of pictures that the host takes, both as a profile picture and of hte listing. 

I control for all possible property-specific variables that Airbnb shows on the listing page, as well as extension demographic, property values, and occupancy controls, so it is unlikely that there remain important property characteristics that drive the price differences. 

The host controls explain very little variation in the price, increasing the $R^2$ by only .004, it is unlikely that adding more sophisticated measures of host quality or effort would significantly help explain price disparities. While this does not eliminate the possibility that there is a set of controls not related to property type or host type that would have increased the $R^2$ drastically, this is still good evidence to believe that the price difference I estimate is a real difference, rather than purely caused by endogeneity. 









I also break up the effects of host race on listing price by city and controlled for my preferred specification. The results are in Table \ref{table:robustcity}. In general, no single city is driving all of the variation in my data. The effects on price are mostly negative for minority hosts, with a few positive coefficients in cities with fewer observations, none of which are significant. 

I use two different measures of quantity demanded - the number of reviews and the vacancy rate. Both confirm that Black hosts and White females have lower quantity demanded than White males. The results for Asian and Hispanic hosts are mixed. 

My first measure of quantity demanded is the number of reviews. In Table \ref{table:num_reviews}, I regress the number of reviews on host race, controlling for the same set of models as Table \ref{table:price} (including the listing's time on the market, an important driver of the number of reviews). I find that minority hosts have either the same or lower review numbers than White hosts for a listing that spends the same amount of time on the market. Specifically, Black hosts, White females, and Hispanic Males have 7 to 8\% fewer reviews than White males. Coefficients are roughly zero for Hispanic females and Asian hosts. 

A second proxy for quantity demanded is the number of days per month a listing remains vacant. Minority hosts may have a lower quantity demanded because they offer up their listing for fewer days of the month, not because they face lower demand. In order to test this, I regress the availability of the listing on host race, controlling for my preferred specification. The availability of a listing is a measure of vacancy for the following reason: availability is controlled by the host, who can update their availability calendar on their listing page. Potential guests can then see on which days the listing is available and book accordingly. When a guest books an available day, that day is removed from the availability calendar. Therefore, the availability out of 30 days measure is a true measure of the vacancy of a listing. It is  important to note that days could be unavailable on a listing's calendar for two reasons: either a host marks them as unavailable, or a guest books on that day. Therefore, if a host has few vacancies, it is not possible to tell if they have high demand, or simply no time to manage their listing. This is not inherently problematic unless the minority and White hosts mark days as unavailable at different rates. 

Vacancy rates can also be affected by the number of properties a host owns - if a host does not live in the property they rent out, they might make more days available on their calendar. If White hosts are more likely to own second or third properties, they would have more availability on their calendars, which would underestimate the gap in availability between White hosts and minority hosts.  

The results, presented in Table \ref{table:availability}, are striking. I find that the listings of Black hosts spend about 20\% more time vacant on the market than the listings of White males. The effect is statistically significant, and amounts to about 2 - 3 days per month in real units. Interestingly, White females make their listing less available than White males, with a statistically significant difference of one day. Asian females are similar to White females, making their listing available one day less than White males. 

Overall, evidence shows that even though Black hosts offer their listings for more days and charge lower prices, fewer guests stay with them. This is significant evidence for the presence of discrimination against Black hosts. Female Asian and female White hosts, on the other hand, choose to make their listing available less often than White hosts. Lower availability is therefore a possible explanation for why these groups have a lower number of reviews. I am not able to further distinguish between availability effects and discrimination in my data. Finally, the quantity demanded results for Hispanic hosts are not statistically significant.














