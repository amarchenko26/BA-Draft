\label{result1}
	

Table \ref{table:price} presents OLS estimates of the effect of host race and gender on the listing price according to the models described in Section \ref{empirical}. 

Model 1 is a naive regression and presented solely to show baseline effects of host race and gender on price. Model 2 adds the property's location. Since a listing's price is strongly correlated with its location, it is unsurprising that a large amount of variation in Airbnb prices between racial groups can be explained by these location controls. This is consistent with the idea that individuals of a particular race are not randomly distributed across neighborhoods of a given city for a variety of reasons.%
	\footnote{These include persistent urban segregation and in-group preferences [MELODY CITE HERE].}
For example, Black hosts had the largest drop in point estimate with the addition of location controls. Consistent with this result, it is well-documented that Blacks in urban populations are nearly four times more likely than Whites to live in neighborhoods where the poverty rate is 40\% or higher \citep{firebaugh}. The effects for Asian hosts behave similarly. While this may be surprising, as urban Asian populations [MELODY - is there a paper or stat about urban asian poverty? Like something that shows that asian americans live in worse neighborhoods that white people?can you cite  it, and insert a sentence here  like "despite the stereotype of asian americans as the model  minority, there are persistent gaps in the income/wealth/quality of neighborhoods ofasians and whites.]

Property characteristics have the most explanatory power in accounting for price differences, as the $R^2$ increases from around .25 to .72 with the addition of listing controls in Model 3. Relative to apartments, houses are around 10\% more expensive, bed and breakfasts are 15\% more expensive, and condos are 7\% more expensive (as opposed to tents, which are 100\% cheaper than apartments!). Relative to renting out the entire home, private rooms and shared room are 50\% and nearly 100\% less expensive, respectively. SOME SHIT. 

My results are stable to the addition of host chara`cteristic controls while still clustering standard errors at the neighborhood level. The inclusion of host characteristics does not improve the fit of the model substantially, increasing the $R^2$ by only .004. It is therefore unlikely that adding more sophisticated measures of host quality or effort would significantly help explain price disparities. 

The results in Model 4 indicate that Asian hosts, both male and female, earn lower prices from their Airbnb listing than White hosts. These effects are 4\% for Asian females and 4.5\% for Asian male hosts. The second biggest effect is for Black males, whose prices are roughly 3.5\% lower.%
	\footnote{This effect is statistically significant at the p $<$ .001 level for Asian female and male hosts, and the p $<$ .01 level for Black male hosts.} 
The estimates are negative but not statistically significant for Hispanic and Black female hosts. Overall, my point estimates suggest an effect, but as I control for more observables, these effects are weakening. This requires a cautious interpretation of these point estimates. 

These effects of Asian host race on listing price are lower than measured by \cite{wang} (a 20\% disparity) and \cite{kakar} (an 8\% disparity). This is unsurprising, as it is unlikely that we would measure the same disparity when there are important structural differences between this paper and previous literature. My sample is national, while both previous papers considered Airbnb listings in the Bay Area only. There are also vast difference in our sample sizes: \cite{wang} has a sample size of 101 observations, while \cite{kakar} has 715 observations. Similarly, the effects of Black host race that I measure are smaller than those of \cite{edelman}, who measured a price disparity of 12\%. But for similar reasons, this comparison is uninformative. I can confirm their results using my data, with the results presented in Table \ref{table:edelman}. 

My large sample size and [presence of reviewer data] gives me enough power to include more comprehensive controls to account for potential omitted variable bias, such as demographics of neighborhood, occupancy rates, the sentiment of the descriptions and reviews. 

In short, moving from left to right in the table, the coefficients become less negative. This raises the question of whether or not if I controlled for all unobservables, it would further reduce my estimates. I follow the logic of \cite{oster} and \cite{altonji} to explore this question.

I use the results from Model 3 and Model 4 in my calculations. The change in estimates from Model 1 to Model 2 is largely not relevant to the question of whether there is a price disparity on Airbnb and if it could be caused by discrimination. This is because the location controls account for a different kind of variation - the fact rental prices are highly correlated with location, and cities are highly segregated by race. 

%The move from Model 2 to Model 3 is similar, in the sense that Model 3 adds objective property qualities. The room for omitted variable bias is therefore in Model 4, 

I therefore use Model 2 and Model 3 as my comparison. Under the most extreme assumption that all of the variation in the outcome variable can be attributed to unobservables, indeed, I can no longer reject the Null hypothesis of no price disparity. However, as \cite{oster} notes, this is a very high standard. In her review of papers published in the American Economic Review, Journal of Political Economy, and the Quarterly Journal of Economics, only 40\% of the positive results for non-randomized studies survive this standard, and 30\% of randomized results. 

Using the change in coefficients going from Model 2 to 3, I lose statistical significance at Oster’s maximum $R^2$ of XX. It is worth noting that the pattern observed in my paper is remarkably similar to \cite{kakar}. For example, comparing Column 3 to 6 in Table 3 of their paper, their point estimate implied by Oster's analysis is 0. 






There are several possible sources of unexplained variation that could remain. One is the quality of the text, such as the reviews and descriptions. Another potential source is the quality of pictures that the host takes, both as a profile picture and of the listing. 

I control for all possible property-specific variables that Airbnb shows on the listing page, as well as extension demographic, property values, and occupancy controls, so it is unlikely that there remain important property characteristics that drive the price differences. 











I also break up the effects of host race on listing price by city and controlled for my preferred specification. The results are in Table \ref{table:robustcity}. In general, no single city is driving all of the variation in my data. The effects on price are mostly negative for minority hosts, with a few positive coefficients in cities with fewer observations, none of which are significant. 

I use two different measures of quantity demanded - the number of reviews and the vacancy rate. Both confirm that Black hosts and White females have lower quantity demanded than White males. The results for Asian and Hispanic hosts are mixed. 

My first measure of quantity demanded is the number of reviews. In Table \ref{table:quantity_demanded}, I regress the number of reviews on host race, controlling for the same set of models as Table \ref{table:price} (including the listing's time on the market, an important driver of the number of reviews). I find that minority hosts have either the same or lower review numbers than White hosts for a listing that spends the same amount of time on the market. Specifically, Black hosts, White females, and Hispanic Males have 7 to 8\% fewer reviews than White males. Coefficients are roughly zero for Hispanic females and Asian hosts. 

A second proxy for quantity demanded is the number of days per month a listing remains vacant. Minority hosts may have a lower quantity demanded because they offer up their listing for fewer days of the month, not because they face lower demand. In order to test this, I regress the availability of the listing on host race, controlling for my preferred specification. The availability of a listing is a measure of vacancy for the following reason: availability is controlled by the host, who can update their availability calendar on their listing page. Potential guests can then see on which days the listing is available and book accordingly. When a guest books an available day, that day is removed from the availability calendar. Therefore, the availability out of 30 days measure is a true measure of the vacancy of a listing. It is  important to note that days could be unavailable on a listing's calendar for two reasons: either a host marks them as unavailable, or a guest books on that day. Therefore, if a host has few vacancies, it is not possible to tell if they have high demand, or simply no time to manage their listing. This is problematic only if minority and White hosts mark days as unavailable at different rates, a plausible conclusion if 

Vacancy rates can also be affected by the number of properties a host owns - if a host does not live in the property they rent out, they might make more days available on their calendar. If White hosts are more likely to own second or third properties, they would have more availability on their calendars, which would underestimate the gap in availability between White hosts and minority hosts.  

The results, also presented in Table \ref{table:quantity_demanded}, are striking. I find that the listings of Black hosts spend about 20\% more time vacant on the market than the listings of White males. The effect is statistically significant, and amounts to about 2 - 3 days per month in real units. Interestingly, White females make their listing less available than White males, with a statistically significant difference of one day. Asian females are similar to White females, making their listing available one day less than White males. 

Overall, there is suggestive evidence to show that even though Black hosts offer their listings for more days and charge lower prices, fewer guests stay with them. Controlling for occupancy rates mitigates some, but not all, concerns that minority hosts are less likely to have a second property that they can rent out full time, increasing vacancy rates. 

Female Asian and female White hosts, on the other hand, have lower vacancies than White hosts. Lower availability is therefore a possible explanation for why these groups have a lower number of reviews. Unfortunately, I am not able to further distinguish between the effect of lower availability and potential discrimination in my data. 














