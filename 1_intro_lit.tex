

Economists and other social scientists have long been concerned with the fact that minorities, especially African-Americans, have experienced pervasively lower living standards in the United States. One potential cause of this is discrimination in the housing market. 

African-Americans have experienced pervasively lower outcomes in the housing market. 

Many efforts have been made to curb discrimination in the housing sector against African-Americans. While landmark federal legislation such as the Fair Housing Act of 1968 prohibits housing discrimination based on race, the enforcement of anti-discrimination legislation is difficult on the local level. 

Minorities in the U.S. still experience wide 



Racial discrimination in housing has remained a widespread issue.

A classic study measured discrimination using fair housing audits and found that black housing seekers are told that 30\% fewer available housing units. \cite{Yinger, John. "Measuring racial discrimination with fair housing audits: Caught in the act." The American Economic Review (1986): 881-893.}

Research has found that discrimination is prevalent amongst landlords renting apartments, who discriminate both because of their own prejudice and in response to the prejudice of their white renters. \cite{Ondrich, Jan, Alex Stricker, and John Yinger. "Do landlords discriminate? The incidence and causes of racial discrimination in rental housing markets." Journal of Housing Economics 8.3 (1999): 185-204.}

Most research has focused on discrimination against potential customers, or renters, by landlords. However, discrimination 

Some studies have found evidence that discrimination in rental markets is statistical in nature. That is, when an applicant inquires about a potential apartment, the level of discrimination is correlated with the level of income that the applicant signals. For black applicants who imply that they are of a higher social class, discrimination is virtually not present.\cite{Hanson, Andrew, and Zackary Hawley. "Do landlords discriminate in the rental housing market? Evidence from an internet field experiment in US cities." Journal of Urban Economics 70.2 (2011): 99-114.}



However, there are many other 

The average black household still has less mean wealth than a white household.\cite{oliver} Even middle-class black and Hispanic households still live in neighborhoods with median incomes similar to those of very poor white neighborhoods.\cite{reardon} For families who want to move out, it can be more difficult to raise capital, since black applicants get fewer loans for the same application as white applicants.\cite{pope} Residential preferences, differences in family structure and availability of affordable housing contribute to these disparities.\cite{krysan} Discrimination in housing has also been cited as one of the primary causes of these inequities. 

However, for economists measuring discrimination, it is difficult to disentangle the effect of present-day racial discrimination from historical сorrelation between race and income poor labor and housing outcomes that are correlated with race. Without an experimental set-up such as an audit study - where it is possible to isolate the effect of discrimination - a large data set and a robust way to test hypotheses to confirm the presence of discrimination is needed.

As such, the recent rise of sharing economies in the housing and short-term rental markets can provide an insight into present-day discrimination. Sharing economies are convenient for research because they have a standardized platform, so all agents obey a uniform set of rules and observe the same set of information about one another. Moreover, many small players have entered these markets who would have otherwise been unable to participate in traditional markets. Being a small landlord with Airbnb has much lower barriers to entry than owning a large apartment building; in just the 10 years since its founding, Airbnb has surpassed Marriott nearly three-fold in the number of rooms offered worldwide.\cite{sharing} Since Airbnb simply provides a new online platform for people to conduct their interactions online, it is reasonable to assume that agents will discriminate based on pictures and names in a similar way online as they do in the real world. This means that studying discrimination in sharing economies could be an important way to learn about the extent of discrimination in traditional markets as well. 

%this has been studied before
Specifically, measurements of discrimination on Airbnb are confounded by many other factors that affect the listing's price, which may also correlate with race and gender. Edelman and Luca (2014) estimated the effect of host race on the price of their listing in New York City, and added a small set of controls to address this endogeneity problem. Their results suggest that non-black hosts on Airbnb have prices roughly 12\% higher than black hosts. However, these controls included only a few property characteristics (the number of bedrooms, the number of people accommodated, and whether the room is shared), the quality of the host's reviews, and a measure of the reliability of the host. This is problematic because many other unobservables in the error term could be confounding the effect, such as the location of the listing, the type of listing (e.g. apartment, house, etc.), the number of bathrooms or other amenities, host-specific policies like cancellation or extra prices, or the quality of the host themselves. 

In this paper, I use a new, previously unexploited dataset from a webscrape of Airbnb to empirically measure discrimination. Using price and listing information for 70,000 Airbnb hosts throughout the country, I address the limitations of Edelman and Luca's research by controlling for location, previously missing property characteristics, comprehensive measures of listing size, the time on the market of the listing, text analyses of host-written descriptions of the listing, and more host-specific characteristics like their response time and availability, various fees, etc. 

% what i find
I find that non-white hosts, both male and female, across the board have lower prices than white hosts. The results are presented in Table 3. The biggest effect is for Asian female hosts, whose prices are roughly \$9 less per day than white male hosts who own the same type of listing. The second biggest effect is for black males, with a coefficient of \$7, followed by black women and Asian men with coefficients of \$6 per day, and Hispanic females with a coefficient of \$5.\footnote{This effect is statistically significant at the p $<$ .001 level for black hosts and Asian women, the p $<$ .01 level for Asian male hosts, and p $<$ .05 level for Hispanic women.} For Hispanic men the effect is small, around \$2, and is not statistically significant.

It is important to note that these coefficients are smaller than Edelman and Luca's. This raises the concern that the price disparity I measure could be erased altogether by adding more controls. I attempt to address this problem by adding measures of host quality and listing attractiveness that could differentiate one listing from another for a potential guest, holding property size and type constant.\footnote{See Table 3, Model 4.} I find that the price disparity is stable to the addition of these controls, providing confidence that adding more variables of a similar nature would not eliminate the disparity. 

After addressing the empirical robustness of this main finding, I test several different hypotheses that could explain the price disparity. Unlike previous research, in addition to price data, I also have information on the quantity demanded of a hosts' listing. This allows for a richer analysis by creating the opportunity to distinguish between several different mechanisms, in addition to discrimination, which could be causing the price disparity. In Section 5, I consider three such hypotheses. 

%I address this issue by creating a set of variables that evaluate the listing description, the host response time, and other measures of host quality. 

To test the first hypothesis, I consider the number of reviews that a host has as a proxy for quantity demanded of that listing. Since black and Hispanic workers earn less on average than their white counterparts, even for the same amount of education, the opportunity cost of their time would be lower.\cite{Economic Policy Institute, State of Working America Data Library, ?Wages by education,? 2016.} They would therefore have a lower marginal cost of putting up and managing their listing, which would mean that those listings would have a lower price than white hosts who own similar listings. This would effectively be like minority hosts' supply curve being lower at each quantity. Supply and demand forces mean that the quantity demanded of minority hosts' listings should be higher. To see if this is the case, I regress number of reviews as a proxy for quantity demanded on host race, controlling for my preferred specification. I find that minority hosts actually have a lower number of reviews than white hosts. This is the first piece of evidence that this hypothesis fails to explain the price disparity. The next paragraph discusses this further. Throughout this analysis, I assume that guests review hosts of different race at equal rates (for more discussion on this assumption, see Section 5, Part 1). 

The second hypothesis is that  minority hosts offer up their listing for fewer days out of the month than white hosts. To test this, I regress a listing's availability out of 30 days on host race, controlling for my preferred specification. A host controls how many days of the month they offer their listing up for rent via an availability calendar on the listing page. When a guest books that listing, the booked days disappear from the availability calendar. Therefore, my measure of availability is actually a measure of true vacancy of the listing. If minority hosts have lower numbers of reviews, perhaps this is because they offer their listing for fewer days of the month than white hosts. My results indicate that for black hosts, this is not the case. The listings of black hosts actually stay vacant on the market 1-2 days longer than the listings of white hosts. However, there is evidence that Asian hosts choose to make their listings available less frequently, which would contribute to their lower number of reviews. For more discussion, see Section 5, Part 2.      

The final hypothesis I test is that minority hosts have lower prices because they have listings of worse quality than white hosts. I test this by looking at the quality of the reviews of the hosts. To do this, I code the demographic information of the reviewers for a randomly-chosen subset of my Chicago hosts. By having both the reviewer-side and the host-side demographic information, I can follow reviewers as they go from host to host and compare the sentiment (how favorable or unfavorable the review is) of the reviews they leave for white hosts versus minority hosts. In order to analyze the text of these reviews, I use a sentiment-analysis package in R called sentimentr.\cite{sentimentr} Rather than observing that all minority hosts uniformly had lower quality reviews, the significance of the result was either negligible or depended on the particular reviewer-host pairing. While there is some evidence that male reviewers tend to rate male hosts higher, there is little within-race preference between reviewers and hosts. Taken as a whole, minority hosts do not have lower quality reviews, as measured by my methods. For a full discussion, see Section 5, Part 3. 






\subsection{About Airbnb} %%%%%%%%%%%
Airbnb is an online marketplace founded in 2008 that allows hosts to rent their private dwellings to guests as temporary accommodation. As of 2017, it has more than 3 million listings, nearly three times more than Marriott's 1.2 million rooms worldwide.\cite{aboutus} Just like traditional hotel chains, guests on Airbnb can browse listings by city and property type, and book a stay based on prices, location, past reviews, pictures of the listing, size, and amenities. Unlike traditional hotel chains, however, hosts create a profile for themselves and a page for each listing they are renting. Each listing page includes the name and picture of the host, the reviews left by previous guests, and those guests' profile pictures. Guests can therefore infer demographic information about the host through a host's picture and name, creating the opportunity for discrimination. Figures 1-5 present screenshots of a listing in Hyde Park with all of the information that would be available to a potential guest. 


\subsection{Previous Literature} %%%%%

Most relevant to this paper is the study by Edelman and Luca (2014), the first to identify and measure the extent of anti-host discrimination on Airbnb.\cite{edelman} They look at the effect of host race on the price of their listing using a snapshot of roughly 3,800 New York City hosts in 2012. Controlling for several confounders that influence price, their findings indicate that non-black hosts on Airbnb have prices roughly 12\% higher than black hosts. I build on Edelman and Luca's research in several important ways. First, their experiment was conducted on a relatively small sample of 3,800 hosts in a single city. My sample includes seven cities throughout America, which are all large urban centers, picked so that they cover most geographic regions in the US. It is important to have this variety because discrimination in a large, cosmopolitan city with a highly diverse population such as New York might look different from discrimination in Nashville, which is more racially homogenous.\footnote{According to the U.S. Census Bureau's QuickFacts, Nashvile is 60.4\% white, 28.4\% black, 10.0\% Hispanic, and 2.5\% Asian. New York City is 44\% white, 25.5\% black, 12.7\% Asian, and 28.6\% Hispanic.} Second, their set of controls was limited by the relatively sparse listing information available on the Airbnb website, which in 2013 was not as comprehensive as it is today. Their covariates only included a listing's location, the number of people the listing accommodates, the rating, the number of bedrooms, and whether or not the whole apartment is available to the guest. After confirming that I get the same result when I run a regression using their controls, I then control for a more complete set of covariates, which are fully outlined in Section 2.1 and Section 3.1.\footnote{See Table 4 for the results of my regression using their covariates. See Section 3.1 for a discussion of my controls.} Most importantly, I propose and test alternative hypotheses for these price disparities. I conclude that racial discrimination is the most likely cause for my results. 


%Moreover, Airbnb has since expanded geographically and made more information available about each listing, potentially influencing the extent of discrimination on the platform.


The idea that discrimination can be reflected in prices is called market discrimination, a definition proposed by Becker (1957).\cite{becker} In the Airbnb market, Becker's market discrimination would be reflected in the price that the guest (buyer) pays to the host (seller) to stay with them. If the guest is discriminating, then given two comparable listings, they would choose not to stay in the one owned by a person of color. Hosts in minority groups, responding to a lower demand, rationally post a lower price in a competitive market. If this is the case, people of color would have systematically lower prices than white males (the canonical ``default" group) for the same types of listings. 

In 1957, Becker was concerned with discrimination arising from face-to-face interactions between minority and majority groups. Since then, there has been a large amount of research indicating that Becker's theory holds for people participating in online labor, lending, rental, and other markets as well. In these cases, participants simply bring their prejudices online and use names and photos to discriminate. The canonical example is the Bertrand and Mullainathan (2004) study, which found that resumes with white sounding names received 50\% more callbacks from potential employers than identical resumes with African-American sounding names.\cite{bertrand} Doleac and Stein (2010) examined market outcomes when selling an iPod on various online marketplaces. In some pictures, a dark-skinned hand was holding the iPod, signaling a black seller, while in others, a light-skinned hand was holding the iPod, signaling a white seller.\cite{doleac} Hands which indicates a black seller received fewer and lower offers than white sellers. On Prosper.com, a popular peer-to-peer lending website, Pope and Sydnor (2011) find that loan listings with pictures of black petitioners get 25-35\% fewer loans than whites with similar credit records.\cite{pope} In sharing economies, a similar pattern occurs. Rides on Uber who use African-American sounding names end up experiencing longer wait times and more frequent cancellations than riders who use white names.\cite{knittel} A later study by Edelman and Luca (2016) found a similar result: guests with distinctively African-American names receive 16\% fewer responses from Airbnb hosts than those with white names.\cite{edelman2} These examples indicate that people often use online firms as a platform to transfer their prejudices from the real world into the online world.   










%%%%%%%%%%%%%%%%%%%%%%%%%%%%%%%%%%%%%%%%%%%%%%%%%%%%%%%%%%%%%%
%As more people have entered these new markets, they have become increasingly dependent on the supplementary income they provide. Hosting with Airbnb, a platform that allows people to rent out their apartment, house, or single room to short-term lodgers, is one such opportunity. A 2017 report released by Airbnb states that in rural areas, hosts get as much as 5 - 20\% of their income from their listing.\cite{rural} Airbnb's fastest growing demographic of hosts, women over 60 years of age, earn \$6,000 a year on average from hosting, often relying on that income to supplement retirement savings.\cite{elderly}\cite{nyt2} 

%Some residents of areas of New York City have started relying on hosting with Airbnb to pay for rent or fund retirement. \cite{nyt1} 

%The extent to which hosts have grown to rely on Airbnb as a source of income makes discrimination on the platform a relevant topic of research. The economic consequences of discrimination are substantial - hosts who are discriminated against would face lower demand, have higher vacancies, and earn less revenue from their listing. While one previous paper found evidence of discrimination against New York City hosts using data from 2013, no other more recent or comprehensive research has been done on this type of discrimination on Airbnb. 

%In this paper, I empirically investigate the existence and extent of anti-host discrimination in Airbnb. I start by measuring the effect of host race and sex on the price of the listing and on a constructed measure of host revenue. I use data from a webscrape of around 70,000 Airbnb listings across 7 U.S. cities.\footnote{The scrape includes all of the property, host, and review information on a listing profile. To see what information would be available, see Figures 1-5 for screenshots of a sample listing. All of the information seen on the sample listing is included as variables in the data set.} For each of the 70,000 listings, the race, sex, and age of the host from their profile picture was coded. 

%Next, I construct a measure of host revenue by multiplying the price a host charges by the total number of reviews for that listing (a proxy for the quantity demanded). Using this measure of revenue, I estimate that White female hosts, Black male hosts, Black female hosts, and Asian female hosts lose about \$100-\$300 in revenue over the course of a year as compared to White male hosts who own similar listings. The exact revenue loss depends on the coefficients on price and number of reviews of a particular host.\footnote{See Table 5 and Section 3.2 for the exact effects on revenue.} These effects are statistically significant at the p $<$ .05 level or higher, and significant at the p $<$ .001 level for White females and Black females. There are also negative effects on revenue for Hispanic hosts and Asian males, but they are not significant. In Section 4, I also conduct several robustness checks and show that these results hold across various cities, price ranges, time on the market, and property types.\footnote{See Tables 6, 7 and discussion in Section 4.}

% discrimination is hard to measure, 
% Understanding discrimination in this new housing market is important because it ties into racial discrepancies in housing widely observed by economists and other social scientists. 

% However, it is difficult to separate the effect of current racial discrimination from the confounding effect of these other economic realities. It is therefore unclear to what extent current discrimination, especially in the housing market, contributes to these long-standing economic disparities. 

%Even though the accurate identification and measurement of discrimination by social scientists is vital to creating policies and statutes to combat it, measuring discrimination is difficult. Unobservable variables in the error term make it hard to isolate the effect of discrimination on the outcome variable of interest. Audit studies are one way that researchers can isolate the effect of race, sex, or other demographic on the outcome of interest. However, these types of experiments are not always possible due to the large organizational, manpower, or time costs associated with them. In the absence of an experimental set-up, regression models with a carefully chosen set of controls can aid in the accurate measurement of discrimination.  

% Economists and other social scientists have long documented the poor housing outcomes for minorities, particularly African-Americans, in the housing market.