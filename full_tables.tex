\singlespacing

\begin{figure}[h]
	\includegraphics[width=1\textwidth]{tables/sample1-cover}
	\caption{Sample listing profile from Chicago}
	\label{fig:listing}
\end{figure}
\begin{figure}
	\includegraphics[width=1\textwidth]{tables/sample2-property}
	\caption{Sample property characteristics}
	\label{fig:property}
\end{figure}

%Histograms
\begin{figure}\centering
	\includegraphics[width=.8\textwidth]{figures/price_dist-CONT-300}
	\caption{Distribution of prices}
	\caption*{Notes: All the listings priced at \$300 or more are grouped together at price = 300}
	\label{fig:prices}
\end{figure}

\begin{figure}\centering
	\includegraphics[width=.8\textwidth]{figures/num_reviews_dist-DISC-100}
	\caption{Distribution of number of reviews}
	\caption*{Notes: All the listings with 100 reviews or more are grouped together at number of reviews = 100}
	\label{fig:reviews}
\end{figure}

\begin{figure}\centering
	\includegraphics[width=.8\textwidth]{figures/review_sentiment_dist}
	\caption{Distribution of review sentiment}
	\label{fig:sentiment}
\end{figure}


%1
%listing-level summary table
\small
{
\begin{longtable}{l*{6}{c|c|cccc}}
	\caption{Summary Statistics by Host Race: Listing Characteristics}\\
	\hline
     &\multicolumn{1}{c}{Full data}&\multicolumn{1}{c}{}&\multicolumn{1}{c}{}&\multicolumn{1}{c}{Regression sample}&\multicolumn{1}{c}{}&\multicolumn{1}{c}{}\\
      \cline{3-7}\\
     &\multicolumn{1}{c}{}&\multicolumn{1}{c}{All}&\multicolumn{1}{c}{White}&\multicolumn{1}{c}{Black}&\multicolumn{1}{c}{Hispanic}&\multicolumn{1}{c}{Asian}\\
     \hline\hline
             
\textit{Outcome variables} \\
Price (\$/day)        & 175.72  &           167.37         &           178.62         &           125.95         &           160.39       &   131.06\\
                  & (294.14) &         (277.7)         &         (289.4)         &         (208.1)         &         (275.0)     & (242.1)    \\
Number of reviews     & 17.51  &      16.57  &      17.14         &      15.06&      16.46 & 	14.08\\
                 & (31.86)  &     (30.8)         &     (31.9)         &     (27.2)         &     (29.7)        & (27.6) \\
                 
\textit{Covariates} \\
\hline
Property Type \\
\hspace{3mm} Apartments/Lofts     		&	.577 &      .598         &       .590         &      .654        &      .625 			& 	.601         \\
\hspace{3mm} Townhouses/Condos   &  .042 &      .042         &      .039         &      .041        &      .041 	& 		.055         \\
\hspace{3mm} Houses    				&.321	&      .321         &       .336        &      .279        &      .289				& 		.311         \\
\hspace{3mm} Other    				&.06	&      .039      &       .035        &      .026        &      .045	& 		.033        \\

Room Type \\
\hspace{3mm} Entire home/apt   &  .577 & .577   	&      .607	&      .449  &      .510		&    .418\\
\hspace{3mm} Private room       & .384 & 	.384		&      .363	&      .483  &      .434		&    .530\\
\hspace{3mm} Shared room      & .039 &	.04	 	&      .029	&      .067  &      .056		&    .052\\

Max Num. Guests   & 3.44   &      3.26	&      3.36  &      2.90		&    3.17 		&	 2.89\\
               & (2.41)    &     (2.3)         &     (2.3)         &     (2.1)         &     (2.4)         & (2.1)\\
Bedrooms    &  1.35 &      1.30 &      1.33         &      1.20         &      1.25   & 1.20      \\
              &   (.92)   &     (.88)         &     (.92)         &     (.72)         &     (.90)       & (.76)  \\
Bathrooms  & 1.30  &      1.27         &       1.29         &      1.20         &      1.26 & 1.21         \\
                &  (.69)  &     (.66)         &     (.68)         &     (.52)         &     (.69)         & (.58)\\
Beds       & 1.82  &      1.73 &      1.76         &      1.63         &      1.74         & 1.59\\
               &   (1.41)  &     (1.32)         &     (1.31)         &     (1.22)         &     (1.60)   & (1.21)      \\
Availability    & 11.53   &      11.30&      10.9&      14.4 &      11.46  	& 	10.88\\
(out of 30 days)         &  (10.93)    & (10.99)     &     (10.85)         &     (11.54)         &     (11.03)         &     (11.05)         \\
Number of Amenities   &   .81  &      .80		&      .81&      .77 &      .80  	& 	.75\\
           &  (1.10)  & (1.10)     &     (1.11)         &     (1.05)         &     (1.10)         &     (1.13)         \\
Cleaning Fee   &  FIX &      .80		&      .81&      .77 &      .80  	& 	.75\\
&  (FIX)  & (1.10)     &     (1.11)         &     (1.05)         &     (1.10)         &     (1.13)         \\
Extra Guests Charge   &   FIX  &      .80		&      .81&      .77 &      .80  	& 	.75\\
&  (FIX)  & (1.10)     &     (1.11)         &     (1.05)         &     (1.10)         &     (1.13)         \\
Instantly Bookable?   &   FIX  &      .80		&      .81&      .77 &      .80  	& 	.75\\
&  (FIX)  & (1.10)     &     (1.11)         &     (1.05)         &     (1.10)         &     (1.13)         \\
Minimum Nights   &   FIX  &      .80		&      .81&      .77 &      .80  	& 	.75\\
&  (FIX)  & (1.10)     &     (1.11)         &     (1.05)         &     (1.10)         &     (1.13)         \\
Cancellation Policy   &   FIX  &      .80		&      .81&      .77 &      .80  	& 	.75\\
&  (FIX)  & (1.10)     &     (1.11)         &     (1.05)         &     (1.10)         &     (1.13)         \\

Year first review    & FIX   &      3.26	&      3.36  &      2.90		&    3.17 		&	 2.89\\
& (FIX)    &     (2.3)         &     (2.3)         &     (2.1)         &     (2.4)         & (2.1)\\
Month first review    & FIX   &      3.26	&      3.36  &      2.90		&    3.17 		&	 2.89\\
& (FIX)    &     (2.3)         &     (2.3)         &     (2.1)         &     (2.4)         & (2.1)\\

\hline
Observations  & 69,007  & 68,983   &       43,988         &       5,023         &       3,524   & 5,893      \\
\hline\hline
\caption*{Notes: The values in the table are means and standard deviations of listing-level data in my full sample. Summary statistics for selected covariates are listed in the table. Categorical variables such as room type do not have standard deviations. Property types are explicitly listed if more than 1.5\% of listings are that type.}

\end{longtable}
}
\normalsize




% \include{code/output/ }

%2
%Host-level summary table
{
	\def\sym#1{\ifmmode^{#1}\else\(^{#1}\)\fi}
	\begin{longtable}{l*{6}{c}}
		\caption{Summary Statistics by Race: Host Demographics}\\
		
		\hline
		&\multicolumn{1}{c}{}&\multicolumn{1}{c}{}&\multicolumn{1}{c}{}&\multicolumn{1}{c}{Regression}&\multicolumn{1}{c}{}&\multicolumn{1}{c}{}\\
		\cline{3-7}\\
			&\multicolumn{1}{c}{Full data}&\multicolumn{1}{c}{Full Sample}&\multicolumn{1}{c}{White}&\multicolumn{1}{c}{Black}&\multicolumn{1}{c}{Hispanic}&\multicolumn{1}{c}{Asian}\\
		
		\hline\hline\endfirsthead\hline\endhead\hline\endfoot\endlastfoot
		&\multicolumn{1}{c}{(1)}&\multicolumn{1}{c}{(2)}&\multicolumn{1}{c}{(3)}&\multicolumn{1}{c}{(4)}&\multicolumn{1}{c}{(5)}&\multicolumn{1}{c}{(6)}\\
		&\multicolumn{1}{c}{Full data}&\multicolumn{1}{c}{Regression sample}&\multicolumn{1}{c}{White}&\multicolumn{1}{c}{Black}&\multicolumn{1}{c}{Hispanic}&\multicolumn{1}{c}{Asian}\\
		\hline
		\endhead             
		     
		\textit{Host demographics} \\
		\hline
		\textit{Race} \\
		White     & &      .637         &       1.00         &      0         &      0 	& 		0         \\
		Black     &  &    .073       &       0         &      1.00         &      0 	& 		0         \\
		Hispanic     & &      .051         &       0         &      0         &      1.00 	& 		0         \\
		Asian     &   &   .085      &       0         &      0         &      0 	& 		1.00         \\
		Unknown/Multiracial     & &      .152         &       0         &      0         &      0 	& 		0         \\
		[1em]
		\textit{Sex} \\
		Male     & &      .309         &       .356         &      .354         &      .417 	& 		.367        \\
		Female     & &      .378         &       .427        &      .541         &      .426 	& 		.476         \\
		Unknown/Two people   &  &      .312         &       .216         &      .104         &      .156 	& 		.157         \\
		[1em]
		\textit{Age} \\
		Young ($<$ 30)     & &      .427         &       .469         &      .514        &      .481 	& 		.587         \\
		Middle-aged     & &      .421         &       .491        &      .470         &      .490 		& 		.379         \\
		Old ($>$ 65)     & &      .018         &       .026         &      .004         &      .009	& 		.009         \\
		Unknown    &  &      .133         &       .013         &      .011         &      .018 	& 		.024         \\
		[1em]

		
		\hline
		Observations (full sample)    &  & 68,983   &       43,988         &       5,023         &       3,524         &       5,893         \\
		\hline\hline
		\caption*{Notes: The values in the table are means and standard deviations of host-level data in my full sample. Summary statistics for selected covariates are listed in the table. Categorical variables such as race, sex, and age do not have standard deviations. White refers only to Non-Hispanic Whites.}
		
	\end{longtable}
}



%3
%Host-level summary table
{
	\def\sym#1{\ifmmode^{#1}\else\(^{#1}\)\fi}
	\begin{longtable}{l*{6}{c}}
		\caption{Summary Statistics by Race: Host Characteristics}\\
		
		\hline
		&\multicolumn{1}{c}{}&\multicolumn{1}{c}{}&\multicolumn{1}{c}{}&\multicolumn{1}{c}{Regression}&\multicolumn{1}{c}{}&\multicolumn{1}{c}{}\\
		\cline{3-7}\\
			&\multicolumn{1}{c}{Full data}&\multicolumn{1}{c}{Full Sample}&\multicolumn{1}{c}{White}&\multicolumn{1}{c}{Black}&\multicolumn{1}{c}{Hispanic}&\multicolumn{1}{c}{Asian}\\
		
		\hline\hline\endfirsthead\hline\endhead\hline\endfoot\endlastfoot
		&\multicolumn{1}{c}{(1)}&\multicolumn{1}{c}{(2)}&\multicolumn{1}{c}{(3)}&\multicolumn{1}{c}{(4)}&\multicolumn{1}{c}{(5)}&\multicolumn{1}{c}{(6)}\\
		&\multicolumn{1}{c}{Full data}&\multicolumn{1}{c}{Regression sample}&\multicolumn{1}{c}{White}&\multicolumn{1}{c}{Black}&\multicolumn{1}{c}{Hispanic}&\multicolumn{1}{c}{Asian}\\
		\hline
		\endhead             
		     
		\textit{Outcome variables} \\
		Host listings count         & &      5.53&      5.50 &      10.5&    3.16 & 2.68\\
		&	&     (33.0)         &     (31.2)         &     (60.3)         &     (17.8) & 	(3.62)         \\
		[1em]
		\textit{Selected covariates} \\
		\hline\hline
		\textit{Host characteristics} \\
		\hline
		Review value      &   &      93.61	&      94.08	 	&      91.89		&    92.81	 & 		92.24\\
		(out of 100)              &      &     (8.00)         &     (7.49)         &     (9.42)         &     (8.72) 	&	 (9.27)         \\
		[1em]
		Host is a superhost    &    &      .124		&      .134&      .084 &      .108  	& 	.097\\
		& & (.329)     &     (.341)         &     (.277)         &     (.310)         &     (.296)         \\
		[1em]
		Response rate      &   &       .756		&       .756		&      .771         &      .756  	& 	.744\\
		& &     (.391)         &     (.393)         &     (.368)         &     (.386)         &		(.399)\\
		[1em]
		Acceptance rate      &     &      .453&      .463&       .357         &      .494    &	.446     \\
		& &     (.463)         &     (.463)         &     (.451)         &     (.466)         &		(.467)\\
		[1em]
		Total ``good" words       &    &      .655&      .656&       .686         &      .677    &	.604     \\
		& &     (.857)         &     (.843)         &     (.882)         &     (.867)         &		(.826)\\
		[1em]
		Length of ``Summary"      &     &      208.67 	&      210.20	&       203.25         &      206.74    &	205.81     \\
		& &       (64.99)  &    (64.08)         &     (70.59)         &     (65.62)         &     (65.87) \\
		[1em]
		Short words in ``Summary"          &  &      .182		&      .185		&       .187         &      .175    &	.175     \\
		& &     (1.19)         &     (1.15)         &     (1.24)         &     (1.26)         &		(1.32)\\                    
		
		
		\hline
		Observations (full sample)    &  & 68,983   &       43,988         &       5,023         &       3,524         &       5,893         \\
		\hline\hline
		\caption*{Notes: The values in the table are means and standard deviations of host-level data in my full sample. Summary statistics for selected covariates are listed in the table. Categorical variables such as race, sex, and age do not have standard deviations. White refers only to Non-Hispanic Whites. Length of ``Summary" and proportion of short words in the ``Summary'' refer to my constructed measures of host quality. These two measures were also calculated for the description, space, neighborhood overview, notes, and transit fields, but were not included in the table for the sake of clarity and because they follow a similar pattern as the ``Summary" field.}
		
	\end{longtable}
}



%4
% reviewer summary table

{
	\def\sym#1{\ifmmode^{#1}\else\(^{#1}\)\fi}
	\begin{longtable}{l*{5}{c}}
		\caption{Summary Statistics by Race: Reviewer Characteristics}\\
		\hline
		&\multicolumn{1}{c}{(1)}&\multicolumn{1}{c}{(2)}&\multicolumn{1}{c}{(3)}&\multicolumn{1}{c}{(4)}&\multicolumn{1}{c}{(5)}\\
		&\multicolumn{1}{c}{Full sample}&\multicolumn{1}{c}{White}&\multicolumn{1}{c}{Black}&\multicolumn{1}{c}{Hispanic}&\multicolumn{1}{c}{Asian}\\
		\hline\hline\endfirsthead\hline\endhead\hline\endfoot\endlastfoot
		&\multicolumn{1}{c}{(1)}&\multicolumn{1}{c}{(2)}&\multicolumn{1}{c}{(3)}&\multicolumn{1}{c}{(4)}&\multicolumn{1}{c}{(5)}\\
		&\multicolumn{1}{c}{Full sample}&\multicolumn{1}{c}{White}&\multicolumn{1}{c}{Black}&\multicolumn{1}{c}{Hispanic}&\multicolumn{1}{c}{Asian}\\
		\hline
		\endhead                  

		\textit{Reviewer characteristics (Chicago data only) } \\
		\hline
		Host race            &      1.00 &      .738&       .099         &      .079    &	.083     \\
		[1em]
		Reviewer race            &      1.00		&      .759	&       .041         &      .047    &	.153     \\
		[1em]
		Review sentiment            &      .510	&      .512&       .503         &      .509    &	.506     \\
		&     (.261)         &     (.254)         &     (.258)         &     (.276)         &		(.287)\\
		[1em]
		Listing sentiment            &      .507&      .509&       .502         &      .499    &	.506     \\
		&     (.072)         &     (.067)         &     (.089)         &     (.096)         &		(.094)\\
		
		\hline
		Observations (full sample)     & 68,983   &       43,988         &       5,023         &       3,524         &       5,893         \\
		\hline\hline
		\caption*{Notes: The values in the table are means and standard deviations of reviewer-level data of a randomly chosen set of hosts in Chicago. Categorical variables do not have standard deviations. White refers only to Non-Hispanic Whites. The host race and the reviewer race in that panel is the proportion of each race that are included in the Reviewer data. The review sentiment is the sentiment of each review, the listing sentiment is the average sentiment per listing.}
		
	\end{longtable}
}



%5 - Main
{
\def\sym#1{\ifmmode^{#1}\else\(^{#1}\)\fi}
\begin{longtable}{l*{4}{c}}
\caption{Main result: Estimates of effect of host's race and gender on price}\\
\hline\hline\endfirsthead\hline\endhead\hline\endfoot\endlastfoot
                    &\multicolumn{1}{c}{(1)}&\multicolumn{1}{c}{(2)}&\multicolumn{1}{c}{(3)}&\multicolumn{1}{c}{(4)}\\
\hline
\textit{White Male (omitted)}  \\

White Female        &      -3.727\sym{*}  &      -3.714\sym{*}  &      -0.827         &      -0.243         \\
                    &     (1.770)         &     (1.562)         &     (1.049)         &     (1.053)         \\
[1em]
Black Male          &      -39.43\sym{***}&      -14.24\sym{***}&      -6.891\sym{***}&      -6.917\sym{***}\\
                    &     (4.058)         &     (3.483)         &     (1.994)         &     (1.963)         \\
[1em]
Black Female        &      -41.69\sym{***}&      -11.20\sym{***}&      -6.271\sym{***}&      -5.793\sym{***}\\
                    &     (4.315)         &     (2.793)         &     (1.577)         &     (1.586)         \\
[1em]
Hispanic Male       &      -20.59\sym{***}&      -7.247\sym{**} &      -2.537         &      -2.165         \\
                    &     (3.727)         &     (2.759)         &     (2.051)         &     (2.073)         \\
[1em]
Hispanic Female     &      -23.05\sym{***}&      -11.38\sym{***}&      -5.284\sym{*}  &      -5.022\sym{*}  \\
                    &     (4.426)         &     (3.219)         &     (2.110)         &     (2.078)         \\
[1em]
Asian Male          &      -27.42\sym{***}&      -12.08\sym{***}&      -5.834\sym{**} &      -6.338\sym{**} \\
                    &     (4.954)         &     (3.553)         &     (2.206)         &     (2.185)         \\
[1em]
Asian Female        &      -39.18\sym{***}&      -21.20\sym{***}&      -8.611\sym{***}&      -8.604\sym{***}\\
                    &     (4.463)         &     (2.592)         &     (1.586)         &     (1.599)         \\
[1em]
Constant            &       147.4\sym{***}&       54.75\sym{***}&      -33.93\sym{***}&      -0.742         \\
                    &     (5.015)         &     (1.506)         &     (5.507)         &     (6.764)         \\
\hline
Controls:        \\
\hspace{3mm} Location  &                &       X         &       X         &       X         \\
\hspace{3mm} Property Characteristics  &                &                &       X         &       X         \\
\hspace{3mm} Host Characteristics  &                &                &                &       X         \\
\hline
Observations        &       45072         &       45072         &       45072         &       45072         \\
Adjusted \(R^{2}\)  &       0.019         &       0.166         &       0.621         &       0.627         \\
\hline\hline
\multicolumn{5}{l}{\footnotesize Standard errors in parentheses}\\
\multicolumn{5}{l}{\footnotesize \sym{*} \(p<0.05\), \sym{**} \(p<0.01\), \sym{***} \(p<0.001\)}\\
\caption*{Notes: The dependent variable is the price of the listing. All race coefficients are relative to white males. The unit of observation is a listing. The sample is the sample of listings across 7 US cities. Model 1 is the baseline effect of host demographics on price. Model 2 controls for listing location to the neighborhood level. Model 3 adds listing characteristics such as property type, time on market, number of bedrooms, availability, etc. Model 4 adds host characteristics such as response and acceptance rates, measures of host effort, Superhost status, etc. See Data Appendix for full description of covariates.}
\label{Table 4}


\end{longtable}
}


\begin{comment}
[1em]
Middle-aged         &       12.21\sym{***}&       10.62\sym{***}&       1.724         &       1.702         \\
&     (2.126)         &     (1.281)         &     (0.913)         &     (0.907)         \\
[1em]
Old ($>$ 65)           &       8.145         &       3.664         &      -1.752         &      -2.239         \\
&     (5.936)         &     (5.339)         &     (3.271)         &     (3.237)         \\
	
\end{comment}


\begin{table}[htbp]\centering
	\def\sym#1{\ifmmode^{#1}\else\(^{#1}\)\fi}
	\caption{New title here}
	\label{table:price_new}
	\begin{tabular}{c|ccccc}
		\toprule
		
		                    &\multicolumn{1}{c}{(1)}&\multicolumn{1}{c}{(2)}&\multicolumn{1}{c}{(3)}&\multicolumn{1}{c}{(4)}&\multicolumn{1}{c}{(5)}\\
                    &\multicolumn{1}{c}{Model 1}&\multicolumn{1}{c}{Model 2}&\multicolumn{1}{c}{Model 3}&\multicolumn{1}{c}{Model 4}&\multicolumn{1}{c}{Model 5}\\
\hline
White Female        &      -3.727\sym{*}  &      -3.714\sym{*}  &      -0.827         &      -0.207         &      -0.233         \\
                    &     (1.770)         &     (1.562)         &     (1.049)         &     (1.056)         &     (1.040)         \\
[1em]
Black Male          &      -39.43\sym{***}&      -14.24\sym{***}&      -6.891\sym{***}&      -7.071\sym{***}&      -7.137\sym{***}\\
                    &     (4.058)         &     (3.483)         &     (1.994)         &     (1.968)         &     (1.973)         \\
[1em]
Black Female        &      -41.69\sym{***}&      -11.20\sym{***}&      -6.271\sym{***}&      -5.845\sym{***}&      -5.737\sym{***}\\
                    &     (4.315)         &     (2.793)         &     (1.577)         &     (1.584)         &     (1.581)         \\
[1em]
Hispanic Male       &      -20.59\sym{***}&      -7.247\sym{**} &      -2.537         &      -2.189         &      -2.152         \\
                    &     (3.727)         &     (2.759)         &     (2.051)         &     (2.065)         &     (2.063)         \\
[1em]
Hispanic Female     &      -23.05\sym{***}&      -11.38\sym{***}&      -5.284\sym{*}  &      -5.038\sym{*}  &      -5.349\sym{**} \\
                    &     (4.426)         &     (3.219)         &     (2.110)         &     (2.082)         &     (2.069)         \\
[1em]
Asian Male          &      -27.42\sym{***}&      -12.08\sym{***}&      -5.834\sym{**} &      -6.310\sym{**} &      -6.498\sym{**} \\
                    &     (4.954)         &     (3.553)         &     (2.206)         &     (2.195)         &     (2.217)         \\
[1em]
Asian Female        &      -39.18\sym{***}&      -21.20\sym{***}&      -8.611\sym{***}&      -8.608\sym{***}&      -8.772\sym{***}\\
                    &     (4.463)         &     (2.592)         &     (1.586)         &     (1.596)         &     (1.616)         \\
\hline
Location Fixed Effects&                     &         Yes         &         Yes         &         Yes         &         Yes         \\
Property Fixed Effects&                     &                     &         Yes         &         Yes         &         Yes         \\
Host Fixed Effects  &                     &                     &                     &         Yes         &         Yes         \\
Observations        &       45072         &       45072         &       45072         &       45072         &       45072         \\
Adjusted ^2$        &      0.0197         &       0.178         &       0.627         &       0.633         &       0.633         \\

		
		\bottomrule
	\end{tabular}
	
	\begin{tablenotes}
		\item \footnotesize Standard errors in parentheses
		\item \footnotesize \sym{*} \(p<0.05\), \sym{**} \(p<0.01\), \sym{***} \(p<0.001\)
		
		\item Notes: The dependent variable is the price of the listing. All race coefficients are relative to white males. The unit of observation is a listing. The sample is the sample of listings across 7 US cities. Model 1 is the baseline effect of host demographics on price. Model 2 controls for listing location to the neighborhood level. Model 3 adds listing characteristics such as property type, time on market, number of bedrooms, availability, etc. Model 4 adds host characteristics such as response and acceptance rates, measures of host effort, Superhost status, etc. See Data Appendix for full description of covariates.  
	\end{tablenotes}
\end{table}



%6
{
\def\sym#1{\ifmmode^{#1}\else\(^{#1}\)\fi}
\begin{longtable}{l*{1}{c}}
\caption{Robustness check with controls from Edelman \& Luca (2014), NYC data}\\
\hline\hline\endfirsthead\hline\endhead\hline\endfoot\endlastfoot
                    &\multicolumn{1}{c}{(1)}\\
                    &\multicolumn{1}{c}{Price per night}\\
\hline
Black               &      -18.11\sym{***}\\
                    &     (1.813)         \\
[1em]
Accommodates        &       12.84\sym{***}\\
                    &     (0.488)         \\
[1em]
Bedrooms            &       33.60\sym{***}\\
                    &     (1.227)         \\
[1em]
Review Scores Location&      -74.66\sym{***}\\
                    &     (7.363)         \\
[1em]
Review Scores Location Squared           &       5.407\sym{***}\\
                    &     (0.421)         \\
[1em]
Review Scores Checkin&      -1.268         \\
                    &     (1.157)         \\
[1em]
Review Scores Communication&      -1.226         \\
                    &     (1.218)         \\
[1em]
Review Scores Cleanliness&       3.454\sym{***}\\
                    &     (0.706)         \\
[1em]
Review Scores Accuracy&      -1.479         \\
                    &     (0.973)         \\
[1em]
Host verified &       1.945         \\
                    &     (1.357)         \\
[1em]
Private room        &      -71.14\sym{***}\\
                    &     (1.400)         \\
[1em]
Shared room         &      -102.8\sym{***}\\
                    &     (3.109)         \\
\hline
Observations        &       11999         \\
Adjusted \(R^{2}\)  &       0.526         \\
\hline\hline
\multicolumn{2}{l}{\footnotesize Standard errors in parentheses}\\
\multicolumn{2}{l}{\footnotesize \sym{*} \(p<0.05\), \sym{**} \(p<0.01\), \sym{***} \(p<0.001\)}\\
\caption*{\footnotesize Notes: This table presents the effect on price of controlling for Edelman \& Luca's (2014) full specification using my NYC data. The results are nearly identical to theirs (their coefficient on Black hosts was -17.8) when controlling for similar covariates in the same city. The omitted category for race is white hosts. The omitted category for room type is Entire Apartment. I could not control for host social media accounts as a proxy for host reliability like Edelman \& Luca, because Airbnb no longer provides this information. Instead, I controlled for ``host verified", a boolean for whether Airbnb has the host's phone number and email. I was not able to control for ``picture quality" either, but picture quality did not significantly influence price in Edelman \& Luca's regression.}\\
\end{longtable}
}




\begin{table}[htbp]\centering
	\def\sym#1{\ifmmode^{#1}\else\(^{#1}\)\fi}
	\caption{New title here}
	\label{table:edelman_new}
	\begin{tabular}{c|c}
		\toprule
		
		                    &\multicolumn{1}{c}{(1)}&\multicolumn{1}{c}{(2)}&\multicolumn{1}{c}{(3)}\\
                    &\multicolumn{1}{c}{Edelman}&\multicolumn{1}{c}{Medium Avail.}&\multicolumn{1}{c}{High Availibility}\\
\hline
White               &           0         &           0         &           0         \\
                    &         (.)         &         (.)         &         (.)         \\
[1em]
Black               &      -18.11\sym{***}&      -7.771\sym{**} &      -9.969\sym{***}\\
                    &     (1.813)         &     (2.988)         &     (2.516)         \\
[1em]
Hispanic            &      -7.830\sym{**} &      -3.410         &      -5.232         \\
                    &     (2.502)         &     (3.026)         &     (3.669)         \\
[1em]
Asian               &      -5.242\sym{**} &      -5.371         &      -6.837\sym{*}  \\
                    &     (1.823)         &     (2.909)         &     (2.889)         \\
[1em]
accommodates        &       12.84\sym{***}&       6.293\sym{***}&       12.07\sym{***}\\
                    &     (0.488)         &     (0.928)         &     (1.149)         \\
[1em]
bedrooms            &       33.60\sym{***}&       34.79\sym{***}&       40.74\sym{***}\\
                    &     (1.227)         &     (2.310)         &     (3.771)         \\
[1em]
review\_scores\_location&      -74.66\sym{***}&                     &                     \\
                    &     (7.363)         &                     &                     \\
[1em]
rev\_loc\_2           &       5.407\sym{***}&                     &                     \\
                    &     (0.421)         &                     &                     \\
[1em]
review\_scores\_checkin&      -1.268         &                     &                     \\
                    &     (1.157)         &                     &                     \\
[1em]
review\_scores\_communication&      -1.226         &                     &                     \\
                    &     (1.218)         &                     &                     \\
[1em]
review\_scores\_cleanliness&       3.454\sym{***}&                     &                     \\
                    &     (0.706)         &                     &                     \\
[1em]
review\_scores\_accuracy&      -1.479         &                     &                     \\
                    &     (0.973)         &                     &                     \\
[1em]
1 if TRUE, 0 if FALSE&       1.945         &       1.951         &      -5.937\sym{**} \\
                    &     (1.357)         &     (2.031)         &     (1.974)         \\
[1em]
Entire home/apt     &           0         &           0         &           0         \\
                    &         (.)         &         (.)         &         (.)         \\
[1em]
Private room        &      -71.14\sym{***}&      -48.40\sym{***}&      -67.86\sym{***}\\
                    &     (1.400)         &     (2.894)         &     (2.674)         \\
[1em]
Shared room         &      -102.8\sym{***}&      -78.10\sym{***}&      -90.57\sym{***}\\
                    &     (3.109)         &     (5.282)         &     (5.403)         \\
[1em]
78701               &                     &           0         &           0         \\
                    &                     &         (.)         &         (.)         \\
[1em]
78702               &                     &      -60.29\sym{***}&      -55.08\sym{***}\\
                    &                     &     (3.925)         &     (4.798)         \\
[1em]
78703               &                     &      -38.44\sym{***}&      -22.09\sym{***}\\
                    &                     &     (3.747)         &     (4.298)         \\
[1em]
78704               &                     &      -44.45\sym{***}&      -43.31\sym{***}\\
                    &                     &     (3.699)         &     (4.161)         \\
[1em]
78705               &                     &      -76.72\sym{***}&      -64.81\sym{***}\\
                    &                     &     (3.881)         &     (3.532)         \\
[1em]
78721               &                     &      -113.3\sym{***}&      -123.3\sym{***}\\
                    &                     &     (4.381)         &     (5.471)         \\
[1em]
78722               &                     &      -82.15\sym{***}&      -61.95\sym{***}\\
                    &                     &     (5.169)         &     (5.309)         \\
[1em]
78723               &                     &      -96.42\sym{***}&      -76.86\sym{***}\\
                    &                     &     (4.928)         &     (7.631)         \\
[1em]
78724               &                     &      -92.41\sym{***}&      -140.5\sym{***}\\
                    &                     &     (5.411)         &     (7.312)         \\
[1em]
78725               &                     &      -13.86\sym{*}  &      -150.4\sym{***}\\
                    &                     &     (5.433)         &     (7.699)         \\
[1em]
78726               &                     &      -74.58\sym{***}&      -333.1\sym{***}\\
                    &                     &     (5.885)         &     (13.91)         \\
[1em]
78727               &                     &      -131.0\sym{***}&      -136.5\sym{***}\\
                    &                     &     (5.735)         &     (4.543)         \\
[1em]
78729               &                     &      -198.3\sym{***}&      -137.9\sym{***}\\
                    &                     &     (5.917)         &     (4.283)         \\
[1em]
78730               &                     &       21.31\sym{*}  &      -25.91\sym{*}  \\
                    &                     &     (9.726)         &     (11.23)         \\
[1em]
78731               &                     &      -72.54\sym{***}&      -96.53\sym{***}\\
                    &                     &     (3.790)         &     (4.861)         \\
[1em]
78732               &                     &      -51.61\sym{***}&      -94.89\sym{***}\\
                    &                     &     (6.934)         &     (6.984)         \\
[1em]
78733               &                     &      -115.6\sym{***}&      -137.0\sym{***}\\
                    &                     &     (6.484)         &     (5.994)         \\
[1em]
78736               &                     &      -181.1\sym{***}&      -145.5\sym{***}\\
                    &                     &     (9.643)         &     (7.739)         \\
[1em]
78737               &                     &      -98.97\sym{***}&       33.70\sym{***}\\
                    &                     &     (8.397)         &     (7.222)         \\
[1em]
78739               &                     &      -82.02\sym{***}&      -151.9\sym{***}\\
                    &                     &     (6.120)         &     (6.508)         \\
[1em]
78741               &                     &      -101.8\sym{***}&      -89.38\sym{***}\\
                    &                     &     (3.611)         &     (3.935)         \\
[1em]
78744               &                     &      -140.0\sym{***}&      -149.9\sym{***}\\
                    &                     &     (5.096)         &     (6.412)         \\
[1em]
78745               &                     &      -85.55\sym{***}&      -121.0\sym{***}\\
                    &                     &     (4.329)         &     (4.883)         \\
[1em]
78746               &                     &      -58.94\sym{***}&       1.219         \\
                    &                     &     (5.199)         &     (7.178)         \\
[1em]
78747               &                     &      -121.0\sym{***}&      -140.0\sym{***}\\
                    &                     &     (6.097)         &     (5.479)         \\
[1em]
78748               &                     &      -76.10\sym{***}&      -138.4\sym{***}\\
                    &                     &     (5.555)         &     (5.281)         \\
[1em]
78749               &                     &      -125.2\sym{***}&      -61.63\sym{***}\\
                    &                     &     (6.382)         &     (8.122)         \\
[1em]
78751               &                     &      -70.56\sym{***}&      -88.84\sym{***}\\
                    &                     &     (3.803)         &     (4.881)         \\
[1em]
78752               &                     &      -96.68\sym{***}&      -114.0\sym{***}\\
                    &                     &     (5.247)         &     (4.446)         \\
[1em]
78754               &                     &      -121.8\sym{***}&      -148.4\sym{***}\\
                    &                     &     (5.826)         &     (4.541)         \\
[1em]
78756               &                     &      -76.53\sym{***}&      -58.94\sym{***}\\
                    &                     &     (4.782)         &     (5.880)         \\
[1em]
78757               &                     &      -130.7\sym{***}&      -104.7\sym{***}\\
                    &                     &     (4.672)         &     (3.590)         \\
[1em]
78758               &                     &      -93.46\sym{***}&      -130.5\sym{***}\\
                    &                     &     (4.935)         &     (4.558)         \\
[1em]
78759               &                     &      -75.51\sym{***}&      -99.96\sym{***}\\
                    &                     &     (4.643)         &     (5.266)         \\
[1em]
Acton               &                     &      -79.71\sym{***}&                     \\
                    &                     &     (8.877)         &                     \\
[1em]
Adams-Normandie     &                     &      -47.70\sym{***}&                     \\
                    &                     &     (9.432)         &                     \\
[1em]
Agoura Hills        &                     &      -76.24\sym{***}&      -54.26\sym{***}\\
                    &                     &     (4.964)         &     (8.856)         \\
[1em]
Agua Dulce          &                     &      -66.79\sym{***}&      -78.54\sym{***}\\
                    &                     &     (7.495)         &     (11.31)         \\
[1em]
Albany Park         &                     &      -34.01\sym{***}&      -49.71         \\
                    &                     &     (8.193)         &     (25.71)         \\
[1em]
Algiers Point       &                     &      -83.58\sym{***}&      -45.93\sym{***}\\
                    &                     &     (8.367)         &     (13.39)         \\
[1em]
Alhambra            &                     &      -90.95\sym{***}&      -48.08\sym{***}\\
                    &                     &     (5.106)         &     (9.583)         \\
[1em]
Altadena            &                     &      -74.04\sym{***}&      -23.59\sym{*}  \\
                    &                     &     (4.515)         &     (10.11)         \\
[1em]
Arcadia             &                     &      -94.93\sym{***}&      -51.92\sym{***}\\
                    &                     &     (7.574)         &     (11.82)         \\
[1em]
Arlington Heights   &                     &      -85.36\sym{***}&      -34.88\sym{***}\\
                    &                     &     (4.531)         &     (9.766)         \\
[1em]
Armour Square       &                     &      -24.83\sym{**} &      -94.92\sym{***}\\
                    &                     &     (8.411)         &     (27.17)         \\
[1em]
Arrochar            &                     &      -108.5\sym{***}&                     \\
                    &                     &     (26.99)         &                     \\
[1em]
Arverne             &                     &      -79.94\sym{**} &      -47.57\sym{**} \\
                    &                     &     (30.29)         &     (18.09)         \\
[1em]
Astoria             &                     &      -75.10\sym{**} &      -37.51\sym{*}  \\
                    &                     &     (27.29)         &     (17.60)         \\
[1em]
Atwater Village     &                     &      -31.32\sym{***}&      -37.84\sym{***}\\
                    &                     &     (5.786)         &     (9.397)         \\
[1em]
Auburn Gresham      &                     &      -38.24\sym{***}&      -71.23\sym{**} \\
                    &                     &     (9.384)         &     (25.60)         \\
[1em]
Audubon             &                     &      -27.05\sym{***}&      -35.51\sym{**} \\
                    &                     &     (8.142)         &     (13.47)         \\
[1em]
Avalon              &                     &      -66.93\sym{***}&       212.5\sym{***}\\
                    &                     &     (6.544)         &     (22.81)         \\
[1em]
Avondale            &                     &      -52.24\sym{***}&      -43.34         \\
                    &                     &     (8.722)         &     (25.86)         \\
[1em]
Azusa               &                     &      -81.83\sym{***}&      -55.07\sym{***}\\
                    &                     &     (8.523)         &     (10.41)         \\
[1em]
Baldwin Hills/Crenshaw&                     &      -43.16\sym{***}&      -3.811         \\
                    &                     &     (6.308)         &     (9.058)         \\
[1em]
Battery Park City   &                     &       7.755         &       160.4\sym{***}\\
                    &                     &     (27.42)         &     (26.96)         \\
[1em]
Bay Ridge           &                     &      -85.72\sym{**} &      -65.93\sym{***}\\
                    &                     &     (27.94)         &     (18.23)         \\
[1em]
Bayou St. John      &                     &      -73.37\sym{***}&      -14.74         \\
                    &                     &     (7.682)         &     (13.27)         \\
[1em]
Bayside             &                     &      -125.4\sym{***}&      -124.1\sym{***}\\
                    &                     &     (27.37)         &     (18.89)         \\
[1em]
Bedford-Stuyvesant  &                     &      -98.45\sym{***}&      -46.15\sym{**} \\
                    &                     &     (27.02)         &     (17.43)         \\
[1em]
Bel-Air             &                     &       14.05         &       67.77\sym{***}\\
                    &                     &     (13.91)         &     (15.75)         \\
[1em]
Bell                &                     &      -84.29\sym{***}&                     \\
                    &                     &     (4.929)         &                     \\
[1em]
Bellflower          &                     &      -95.08\sym{***}&      -51.25\sym{***}\\
                    &                     &     (6.112)         &     (8.276)         \\
[1em]
Bensonhurst         &                     &      -91.88\sym{***}&      -79.22\sym{***}\\
                    &                     &     (26.98)         &     (17.86)         \\
[1em]
Beverly Crest       &                     &       90.68\sym{***}&       27.31\sym{*}  \\
                    &                     &     (8.075)         &     (11.11)         \\
[1em]
Beverly Grove       &                     &      -58.88\sym{***}&      -0.143         \\
                    &                     &     (4.129)         &     (8.652)         \\
[1em]
Beverly Hills       &                     &      -54.13\sym{***}&       17.36\sym{*}  \\
                    &                     &     (4.870)         &     (8.451)         \\
[1em]
Beverlywood         &                     &      -37.76\sym{***}&      -31.31\sym{**} \\
                    &                     &     (7.301)         &     (9.914)         \\
[1em]
Black Pearl         &                     &       31.02\sym{***}&       30.34\sym{*}  \\
                    &                     &     (8.172)         &     (12.57)         \\
[1em]
Boerum Hill         &                     &      -40.33         &      -1.738         \\
                    &                     &     (27.23)         &     (17.28)         \\
[1em]
Borough Park        &                     &      -98.27\sym{***}&      -50.06\sym{**} \\
                    &                     &     (27.50)         &     (17.51)         \\
[1em]
Boyle Heights       &                     &      -95.08\sym{***}&      -47.37\sym{***}\\
                    &                     &     (4.557)         &     (9.789)         \\
[1em]
Brentwood           &                     &      -14.15\sym{**} &       9.237         \\
                    &                     &     (4.852)         &     (8.727)         \\
[1em]
Briarwood           &                     &      -70.32\sym{*}  &                     \\
                    &                     &     (27.83)         &                     \\
[1em]
Bridgeport          &                     &      -41.35\sym{***}&      -39.09         \\
                    &                     &     (8.274)         &     (26.01)         \\
[1em]
Brighton Beach      &                     &      -105.5\sym{***}&      -30.96         \\
                    &                     &     (27.07)         &     (16.52)         \\
[1em]
Brightwood Park, Crestwood, Petw&                     &      -95.07\sym{***}&      -33.33\sym{***}\\
                    &                     &     (4.203)         &     (9.619)         \\
[1em]
Broadmoor           &                     &      -95.66\sym{***}&      -83.38\sym{***}\\
                    &                     &     (7.332)         &     (13.30)         \\
[1em]
Brookland, Brentwood, Langdon&                     &      -77.65\sym{***}&      -38.61\sym{***}\\
                    &                     &     (4.007)         &     (8.346)         \\
[1em]
Brooklyn Heights    &                     &      -38.57         &       11.22         \\
                    &                     &     (27.27)         &     (18.53)         \\
[1em]
Brownsville         &                     &      -67.49\sym{*}  &      -49.24\sym{**} \\
                    &                     &     (27.02)         &     (17.22)         \\
[1em]
Burbank             &                     &      -95.63\sym{***}&      -29.54\sym{**} \\
                    &                     &     (4.837)         &     (9.629)         \\
[1em]
Bushwick            &                     &      -95.45\sym{***}&      -51.65\sym{**} \\
                    &                     &     (27.10)         &     (17.13)         \\
[1em]
Bywater             &                     &      -58.96\sym{***}&      -29.78\sym{*}  \\
                    &                     &     (7.553)         &     (13.26)         \\
[1em]
Calabasas           &                     &      -90.94\sym{***}&      -38.67\sym{***}\\
                    &                     &     (4.720)         &     (9.885)         \\
[1em]
Canarsie            &                     &      -138.2\sym{***}&      -65.32\sym{***}\\
                    &                     &     (27.19)         &     (18.17)         \\
[1em]
Canoga Park         &                     &      -89.41\sym{***}&      -40.03\sym{***}\\
                    &                     &     (5.150)         &     (8.154)         \\
[1em]
Capitol Hill, Lincoln Park&                     &      -56.10\sym{***}&       11.09         \\
                    &                     &     (3.641)         &     (8.565)         \\
[1em]
Carroll Gardens     &                     &      -11.28         &      -13.31         \\
                    &                     &     (27.01)         &     (18.03)         \\
[1em]
Carthay             &                     &      -48.62\sym{***}&      -26.09\sym{**} \\
                    &                     &     (4.539)         &     (9.089)         \\
[1em]
Castaic Canyons     &                     &      -83.11\sym{***}&       50.74\sym{***}\\
                    &                     &     (21.80)         &     (10.98)         \\
[1em]
Cathedral Heights, McLean Garden&                     &      -53.00\sym{***}&      -29.44\sym{***}\\
                    &                     &     (6.890)         &     (7.884)         \\
[1em]
Central Business District&                     &      -36.72\sym{***}&       23.26         \\
                    &                     &     (7.490)         &     (13.18)         \\
[1em]
Central City        &                     &      -56.55\sym{***}&       6.738         \\
                    &                     &     (7.081)         &     (12.83)         \\
[1em]
Central-Alameda     &                     &      -102.1\sym{***}&                     \\
                    &                     &     (8.558)         &                     \\
[1em]
Century City        &                     &      -37.07\sym{***}&       54.64\sym{***}\\
                    &                     &     (5.142)         &     (9.040)         \\
[1em]
Cerritos            &                     &      -87.24\sym{***}&       204.9\sym{***}\\
                    &                     &     (7.556)         &     (15.61)         \\
[1em]
Chatham             &                     &      -41.07\sym{***}&      -47.47         \\
                    &                     &     (11.62)         &     (26.93)         \\
[1em]
Chatsworth          &                     &      -155.4\sym{***}&      -67.74\sym{***}\\
                    &                     &     (6.345)         &     (9.684)         \\
[1em]
Chelsea             &                     &       5.329         &       82.62\sym{***}\\
                    &                     &     (27.16)         &     (17.36)         \\
[1em]
Cheviot Hills       &                     &      -33.43\sym{***}&       29.19\sym{**} \\
                    &                     &     (4.884)         &     (9.145)         \\
[1em]
Chinatown           &                     &      -33.82         &       16.04         \\
                    &                     &     (23.53)         &     (14.52)         \\
[1em]
Citrus              &                     &      -88.89\sym{***}&      -59.36\sym{***}\\
                    &                     &     (5.550)         &     (10.51)         \\
[1em]
City Island         &                     &      -140.0\sym{***}&      -75.27\sym{***}\\
                    &                     &     (27.67)         &     (18.28)         \\
[1em]
City Park           &                     &      -85.27\sym{***}&      -33.67\sym{*}  \\
                    &                     &     (7.365)         &     (13.16)         \\
[1em]
Civic Center        &                     &      -50.03         &       6.846         \\
                    &                     &     (27.05)         &     (17.46)         \\
[1em]
Claremont           &                     &      -78.08\sym{***}&      -14.93         \\
                    &                     &     (5.540)         &     (11.24)         \\
[1em]
Clason Point        &                     &      -111.7\sym{***}&      -57.36\sym{**} \\
                    &                     &     (27.19)         &     (18.05)         \\
[1em]
Cleveland Park, Woodley Park, Ma&                     &      -53.24\sym{***}&      -32.51\sym{***}\\
                    &                     &     (4.635)         &     (7.828)         \\
[1em]
Clifton             &                     &      -183.4\sym{***}&      -83.78\sym{***}\\
                    &                     &     (27.15)         &     (18.33)         \\
[1em]
Clinton Hill        &                     &      -77.61\sym{**} &      -30.24         \\
                    &                     &     (27.09)         &     (18.18)         \\
[1em]
Cobble Hill         &                     &      -61.50\sym{*}  &      -4.556         \\
                    &                     &     (27.11)         &     (17.87)         \\
[1em]
Colonial Village, Shepherd Park,&                     &      -98.95\sym{***}&      -18.81\sym{*}  \\
                    &                     &     (8.334)         &     (9.103)         \\
[1em]
Columbia Heights, Mt. Pleasant,&                     &      -59.80\sym{***}&      -13.24         \\
                    &                     &     (3.676)         &     (8.138)         \\
[1em]
Columbia St         &                     &      -107.9\sym{***}&      -47.44\sym{*}  \\
                    &                     &     (28.00)         &     (18.51)         \\
[1em]
Compton             &                     &      -150.3\sym{***}&      -18.61         \\
                    &                     &     (6.414)         &     (10.77)         \\
[1em]
Concourse           &                     &       34.42         &      -45.09\sym{*}  \\
                    &                     &     (27.48)         &     (17.77)         \\
[1em]
Concourse Village   &                     &      -93.75\sym{***}&      -30.08         \\
                    &                     &     (27.56)         &     (17.16)         \\
[1em]
Coney Island        &                     &      -86.60\sym{**} &                     \\
                    &                     &     (27.80)         &                     \\
[1em]
Corona              &                     &      -113.9\sym{***}&      -78.99\sym{***}\\
                    &                     &     (27.13)         &     (18.52)         \\
[1em]
Crown Heights       &                     &      -79.90\sym{**} &      -39.46\sym{*}  \\
                    &                     &     (27.06)         &     (17.38)         \\
[1em]
Culver City         &                     &      -73.92\sym{***}&       1.041         \\
                    &                     &     (4.035)         &     (9.316)         \\
[1em]
Cypress Hills       &                     &      -146.3\sym{***}&      -73.92\sym{***}\\
                    &                     &     (27.29)         &     (18.83)         \\
[1em]
Cypress Park        &                     &      -135.4\sym{***}&      -135.6\sym{***}\\
                    &                     &     (5.000)         &     (14.52)         \\
[1em]
DUMBO               &                     &      -29.82         &       16.26         \\
                    &                     &     (26.97)         &     (18.66)         \\
[1em]
Del Aire            &                     &      -72.54\sym{***}&      -58.94\sym{***}\\
                    &                     &     (5.469)         &     (10.62)         \\
[1em]
Del Rey             &                     &      -63.95\sym{***}&      -23.55\sym{**} \\
                    &                     &     (4.032)         &     (8.905)         \\
[1em]
Diamond Bar         &                     &      -120.2\sym{***}&      -62.53\sym{***}\\
                    &                     &     (7.239)         &     (11.10)         \\
[1em]
Dillard             &                     &      -58.78\sym{***}&      -49.87\sym{***}\\
                    &                     &     (7.965)         &     (13.94)         \\
[1em]
District 1          &                     &      -74.15\sym{***}&      -77.94\sym{***}\\
                    &                     &     (6.245)         &     (11.40)         \\
[1em]
District 11         &                     &      -160.9\sym{***}&      -49.63\sym{***}\\
                    &                     &     (5.507)         &     (11.83)         \\
[1em]
District 12         &                     &      -184.9\sym{***}&      -63.82\sym{***}\\
                    &                     &     (6.563)         &     (9.851)         \\
[1em]
District 13         &                     &      -130.4\sym{***}&      -83.54\sym{***}\\
                    &                     &     (4.789)         &     (9.513)         \\
[1em]
District 14         &                     &      -91.26\sym{***}&      -41.07\sym{***}\\
                    &                     &     (6.170)         &     (10.29)         \\
[1em]
District 15         &                     &      -134.6\sym{***}&      -74.09\sym{***}\\
                    &                     &     (4.701)         &     (9.883)         \\
[1em]
District 16         &                     &      -128.6\sym{***}&      -53.57\sym{***}\\
                    &                     &     (4.686)         &     (10.60)         \\
[1em]
District 17         &                     &      -70.27\sym{***}&       1.713         \\
                    &                     &     (5.045)         &     (10.32)         \\
[1em]
District 18         &                     &      -40.39\sym{***}&       45.75\sym{***}\\
                    &                     &     (4.579)         &     (10.24)         \\
[1em]
District 19         &                     &      -6.167         &       36.15\sym{**} \\
                    &                     &     (4.244)         &     (11.27)         \\
[1em]
District 2          &                     &      -63.20\sym{***}&      -58.33\sym{***}\\
                    &                     &     (4.806)         &     (9.443)         \\
[1em]
District 20         &                     &      -93.24\sym{***}&      -109.5\sym{***}\\
                    &                     &     (5.381)         &     (10.88)         \\
[1em]
District 21         &                     &      -91.86\sym{***}&      -11.29         \\
                    &                     &     (4.428)         &     (9.898)         \\
[1em]
District 22         &                     &      -196.3\sym{***}&      -32.87\sym{***}\\
                    &                     &     (11.56)         &     (9.291)         \\
[1em]
District 23         &                     &      -101.2\sym{***}&      -91.27\sym{***}\\
                    &                     &     (6.192)         &     (10.27)         \\
[1em]
District 24         &                     &      -75.05\sym{***}&      -16.68         \\
                    &                     &     (5.071)         &     (9.932)         \\
[1em]
District 25         &                     &      -52.17\sym{***}&       24.31\sym{*}  \\
                    &                     &     (6.395)         &     (9.510)         \\
[1em]
District 26         &                     &      -94.37\sym{***}&      -46.31\sym{***}\\
                    &                     &     (5.236)         &     (10.12)         \\
[1em]
District 27         &                     &      -123.6\sym{***}&      -88.29\sym{***}\\
                    &                     &     (5.441)         &     (10.04)         \\
[1em]
District 28         &                     &      -85.11\sym{***}&      -85.45\sym{***}\\
                    &                     &     (5.038)         &     (9.584)         \\
[1em]
District 29         &                     &      -34.53\sym{***}&      -67.32\sym{***}\\
                    &                     &     (7.520)         &     (9.106)         \\
[1em]
District 3          &                     &      -109.1\sym{***}&      -96.37\sym{***}\\
                    &                     &     (5.974)         &     (11.34)         \\
[1em]
District 30         &                     &      -126.4\sym{***}&      -49.64\sym{***}\\
                    &                     &     (7.542)         &     (12.55)         \\
[1em]
District 31         &                     &      -77.96\sym{***}&      -43.98\sym{**} \\
                    &                     &     (5.539)         &     (14.63)         \\
[1em]
District 32         &                     &      -171.7\sym{***}&      -103.2\sym{***}\\
                    &                     &     (8.990)         &     (9.499)         \\
[1em]
District 33         &                     &      -103.2\sym{***}&      -76.38\sym{***}\\
                    &                     &     (5.609)         &     (10.31)         \\
[1em]
District 34         &                     &      -90.69\sym{***}&      -43.94\sym{***}\\
                    &                     &     (14.07)         &     (10.01)         \\
[1em]
District 35         &                     &      -148.5\sym{***}&      -58.02\sym{***}\\
                    &                     &     (4.870)         &     (10.10)         \\
[1em]
District 4          &                     &      -72.77\sym{***}&      -34.19\sym{**} \\
                    &                     &     (5.833)         &     (10.39)         \\
[1em]
District 5          &                     &      -99.85\sym{***}&      -24.74\sym{*}  \\
                    &                     &     (4.746)         &     (9.942)         \\
[1em]
District 6          &                     &      -87.34\sym{***}&      -0.926         \\
                    &                     &     (4.724)         &     (10.91)         \\
[1em]
District 7          &                     &      -97.34\sym{***}&      -36.21\sym{***}\\
                    &                     &     (4.692)         &     (10.18)         \\
[1em]
District 8          &                     &      -148.7\sym{***}&      -65.36\sym{***}\\
                    &                     &     (5.232)         &     (9.748)         \\
[1em]
District 9          &                     &      -104.3\sym{***}&      -11.73         \\
                    &                     &     (5.352)         &     (11.16)         \\
[1em]
Ditmars Steinway    &                     &      -96.07\sym{***}&      -32.19         \\
                    &                     &     (27.09)         &     (17.28)         \\
[1em]
Douglas             &                     &      -70.32\sym{***}&      -61.00\sym{*}  \\
                    &                     &     (9.772)         &     (27.13)         \\
[1em]
Douglas, Shipley Terrace&                     &      -40.45\sym{***}&      -41.96\sym{**} \\
                    &                     &     (9.687)         &     (12.77)         \\
[1em]
Downey              &                     &      -68.18\sym{***}&      -23.90\sym{*}  \\
                    &                     &     (7.510)         &     (10.66)         \\
[1em]
Downtown            &                     &      -49.26\sym{***}&       8.996         \\
                    &                     &     (4.252)         &     (8.655)         \\
[1em]
Downtown Brooklyn   &                     &      -74.83\sym{**} &      -15.41         \\
                    &                     &     (27.24)         &     (19.65)         \\
[1em]
Downtown, Chinatown, Penn Quarte&                     &      -17.28\sym{***}&       53.28\sym{***}\\
                    &                     &     (3.503)         &     (9.011)         \\
[1em]
Duarte              &                     &      -89.86\sym{***}&      -101.1\sym{***}\\
                    &                     &     (6.356)         &     (11.18)         \\
[1em]
Dunning             &                     &      -91.81\sym{***}&      -81.09\sym{**} \\
                    &                     &     (12.96)         &     (26.46)         \\
[1em]
Dupont Circle, Connecticut Avenu&                     &      -40.49\sym{***}&      -20.16\sym{*}  \\
                    &                     &     (3.709)         &     (9.409)         \\
[1em]
Eagle Rock          &                     &      -95.98\sym{***}&      -28.75\sym{**} \\
                    &                     &     (4.897)         &     (9.807)         \\
[1em]
East Carrollton     &                     &      -66.32\sym{***}&      -27.81\sym{*}  \\
                    &                     &     (7.509)         &     (13.71)         \\
[1em]
East Elmhurst       &                     &      -75.10\sym{**} &      -52.99\sym{**} \\
                    &                     &     (27.31)         &     (17.77)         \\
[1em]
East Flatbush       &                     &      -89.31\sym{**} &      -60.72\sym{***}\\
                    &                     &     (27.17)         &     (17.51)         \\
[1em]
East Garfield Park  &                     &      -55.31\sym{***}&      -12.53         \\
                    &                     &     (8.330)         &     (29.18)         \\
[1em]
East Harlem         &                     &      -66.62\sym{*}  &      -21.43         \\
                    &                     &     (27.12)         &     (17.13)         \\
[1em]
East Hollywood      &                     &      -81.88\sym{***}&      -35.16\sym{***}\\
                    &                     &     (3.812)         &     (8.704)         \\
[1em]
East Los Angeles    &                     &      -83.29\sym{***}&      -98.60\sym{***}\\
                    &                     &     (6.857)         &     (9.244)         \\
[1em]
East New York       &                     &      -138.0\sym{***}&      -74.99\sym{***}\\
                    &                     &     (27.02)         &     (17.25)         \\
[1em]
East Riverside      &                     &      -64.61\sym{***}&      -13.97         \\
                    &                     &     (7.688)         &     (14.22)         \\
[1em]
East San Gabriel    &                     &      -69.39\sym{***}&      -43.95\sym{***}\\
                    &                     &     (5.475)         &     (11.44)         \\
[1em]
East Village        &                     &      -16.88         &       34.72\sym{*}  \\
                    &                     &     (27.09)         &     (17.40)         \\
[1em]
Eastchester         &                     &      -67.62\sym{*}  &      -31.00         \\
                    &                     &     (27.56)         &     (17.44)         \\
[1em]
Eastland Gardens, Kenilworth&                     &      -91.66\sym{***}&      -22.28         \\
                    &                     &     (7.102)         &     (14.86)         \\
[1em]
Echo Park           &                     &      -68.44\sym{***}&      -30.72\sym{***}\\
                    &                     &     (3.731)         &     (9.008)         \\
[1em]
Edgewater           &                     &      -31.12\sym{***}&      -33.41         \\
                    &                     &     (8.099)         &     (25.60)         \\
[1em]
Edgewood, Bloomingdale, Truxton&                     &      -69.33\sym{***}&      -20.68\sym{*}  \\
                    &                     &     (4.064)         &     (8.913)         \\
[1em]
El Segundo          &                     &      -42.48\sym{***}&       12.56         \\
                    &                     &     (4.734)         &     (9.139)         \\
[1em]
El Sereno           &                     &      -76.65\sym{***}&      -15.71         \\
                    &                     &     (5.070)         &     (8.479)         \\
[1em]
Elmhurst            &                     &      -95.65\sym{***}&      -68.39\sym{***}\\
                    &                     &     (27.18)         &     (17.63)         \\
[1em]
Elysian Park        &                     &      -99.07\sym{***}&       6.796         \\
                    &                     &     (6.205)         &     (10.55)         \\
[1em]
Elysian Valley      &                     &      -154.0\sym{***}&      -80.23\sym{***}\\
                    &                     &     (7.167)         &     (9.580)         \\
[1em]
Encino              &                     &      -69.60\sym{***}&      -35.09\sym{***}\\
                    &                     &     (4.329)         &     (8.866)         \\
[1em]
Englewood           &                     &      -86.31\sym{***}&                     \\
                    &                     &     (11.95)         &                     \\
[1em]
Exposition Park     &                     &      -110.8\sym{***}&      -90.20\sym{***}\\
                    &                     &     (5.423)         &     (14.57)         \\
[1em]
Fairfax             &                     &      -46.75\sym{***}&      -16.93         \\
                    &                     &     (4.208)         &     (8.898)         \\
[1em]
Fairgrounds         &                     &      -56.81\sym{***}&      -30.18\sym{*}  \\
                    &                     &     (7.266)         &     (13.43)         \\
[1em]
Fillmore            &                     &      -171.1\sym{***}&      -62.90\sym{***}\\
                    &                     &     (8.579)         &     (13.51)         \\
[1em]
Financial District  &                     &      -16.51         &       126.7\sym{***}\\
                    &                     &     (27.21)         &     (18.13)         \\
[1em]
Flatbush            &                     &      -83.67\sym{**} &      -29.91         \\
                    &                     &     (27.12)         &     (17.37)         \\
[1em]
Flatiron District   &                     &       129.8\sym{***}&       220.2\sym{***}\\
                    &                     &     (27.78)         &     (18.30)         \\
[1em]
Flatlands           &                     &      -58.87\sym{*}  &      -64.35\sym{***}\\
                    &                     &     (27.26)         &     (18.02)         \\
[1em]
Florence            &                     &      -90.61\sym{***}&      -74.41\sym{***}\\
                    &                     &     (7.036)         &     (11.23)         \\
[1em]
Flushing            &                     &      -87.24\sym{**} &      -34.44         \\
                    &                     &     (27.00)         &     (18.22)         \\
[1em]
Fordham             &                     &      -88.88\sym{**} &                     \\
                    &                     &     (27.90)         &                     \\
[1em]
Forest Hills        &                     &      -71.70\sym{**} &      -45.00\sym{*}  \\
                    &                     &     (27.24)         &     (18.68)         \\
[1em]
Fort Greene         &                     &      -59.65\sym{*}  &      -11.62         \\
                    &                     &     (27.05)         &     (17.16)         \\
[1em]
Fort Hamilton       &                     &      -64.91\sym{*}  &                     \\
                    &                     &     (26.72)         &                     \\
[1em]
French Quarter      &                     &       39.01\sym{***}&       37.36\sym{**} \\
                    &                     &     (7.474)         &     (12.90)         \\
[1em]
Freret              &                     &      -106.1\sym{***}&      -62.87\sym{***}\\
                    &                     &     (7.826)         &     (13.92)         \\
[1em]
Friendship Heights, American Uni&                     &      -44.66\sym{***}&      -17.71         \\
                    &                     &     (4.383)         &     (13.57)         \\
[1em]
Garden District     &                     &      -22.16\sym{**} &      -52.24\sym{***}\\
                    &                     &     (7.402)         &     (12.93)         \\
[1em]
Gardena             &                     &      -87.56\sym{***}&      -42.36\sym{***}\\
                    &                     &     (5.997)         &     (12.17)         \\
[1em]
Gentilly Terrace    &                     &      -76.40\sym{***}&      -55.03\sym{***}\\
                    &                     &     (7.479)         &     (13.49)         \\
[1em]
Georgetown, Burleith/Hillandale&                     &      -32.07\sym{***}&       16.51         \\
                    &                     &     (4.080)         &     (9.969)         \\
[1em]
Gert Town           &                     &      -88.19\sym{***}&      -95.22\sym{***}\\
                    &                     &     (9.097)         &     (12.65)         \\
[1em]
Glassell Park       &                     &      -104.8\sym{***}&      -59.63\sym{***}\\
                    &                     &     (4.997)         &     (9.482)         \\
[1em]
Glendale            &                     &      -64.50\sym{***}&      -39.25\sym{***}\\
                    &                     &     (3.856)         &     (9.175)         \\
[1em]
Gowanus             &                     &      -80.06\sym{**} &       47.29\sym{**} \\
                    &                     &     (27.17)         &     (18.19)         \\
[1em]
Gramercy            &                     &       12.15         &       18.59         \\
                    &                     &     (27.25)         &     (17.26)         \\
[1em]
Granada Hills       &                     &      -105.4\sym{***}&      -46.79\sym{***}\\
                    &                     &     (6.153)         &     (10.52)         \\
[1em]
Grand Boulevard     &                     &      -47.40\sym{***}&      -36.55         \\
                    &                     &     (8.798)         &     (26.50)         \\
[1em]
Gravesend           &                     &      -139.8\sym{***}&       86.43\sym{***}\\
                    &                     &     (27.46)         &     (19.91)         \\
[1em]
Green Meadows       &                     &      -71.68\sym{***}&                     \\
                    &                     &     (6.128)         &                     \\
[1em]
Greenpoint          &                     &      -71.42\sym{**} &      -8.437         \\
                    &                     &     (26.98)         &     (17.27)         \\
[1em]
Greenwich Village   &                     &       34.83         &       48.40\sym{**} \\
                    &                     &     (27.15)         &     (17.68)         \\
[1em]
Grymes Hill         &                     &      -60.72\sym{*}  &                     \\
                    &                     &     (27.75)         &                     \\
[1em]
Hacienda Heights    &                     &      -105.6\sym{***}&      -94.13\sym{***}\\
                    &                     &     (5.621)         &     (12.41)         \\
[1em]
Hancock Park        &                     &      -36.44\sym{***}&      -13.96         \\
                    &                     &     (4.733)         &     (9.600)         \\
[1em]
Harbor City         &                     &      -97.27\sym{***}&      -95.63\sym{***}\\
                    &                     &     (6.574)         &     (10.85)         \\
[1em]
Harbor Gateway      &                     &      -116.5\sym{***}&      -33.77\sym{**} \\
                    &                     &     (5.110)         &     (11.07)         \\
[1em]
Harlem              &                     &      -66.95\sym{*}  &      -14.35         \\
                    &                     &     (27.14)         &     (17.33)         \\
[1em]
Harvard Heights     &                     &      -80.38\sym{***}&      -68.81\sym{***}\\
                    &                     &     (4.889)         &     (13.39)         \\
[1em]
Hawthorne           &                     &      -71.75\sym{***}&      -45.84\sym{***}\\
                    &                     &     (4.526)         &     (8.598)         \\
[1em]
Hawthorne, Barnaby Woods, Chevy&                     &      -56.23\sym{***}&      -36.19\sym{**} \\
                    &                     &     (6.288)         &     (11.00)         \\
[1em]
Hell's Kitchen      &                     &      -11.04         &       57.94\sym{**} \\
                    &                     &     (27.12)         &     (17.56)         \\
[1em]
Hermosa             &                     &      -40.65\sym{***}&                     \\
                    &                     &     (8.527)         &                     \\
[1em]
Hermosa Beach       &                     &       23.59\sym{***}&       64.21\sym{***}\\
                    &                     &     (4.973)         &     (9.527)         \\
[1em]
Highland Park       &                     &      -80.12\sym{***}&      -64.00\sym{***}\\
                    &                     &     (4.877)         &     (8.868)         \\
[1em]
Historic Anacostia  &                     &      -86.82\sym{***}&      -17.16         \\
                    &                     &     (6.208)         &     (10.71)         \\
[1em]
Historic South-Central&                     &      -74.25\sym{***}&      -76.70\sym{***}\\
                    &                     &     (5.205)         &     (11.88)         \\
[1em]
Hollis Hills        &                     &       67.84\sym{*}  &      -38.60\sym{*}  \\
                    &                     &     (27.51)         &     (19.45)         \\
[1em]
Holliswood          &                     &      -71.13\sym{*}  &                     \\
                    &                     &     (28.23)         &                     \\
[1em]
Hollygrove          &                     &      -142.4\sym{***}&      -129.7\sym{***}\\
                    &                     &     (10.06)         &     (14.91)         \\
[1em]
Hollywood           &                     &      -65.70\sym{***}&      -30.51\sym{***}\\
                    &                     &     (3.663)         &     (8.278)         \\
[1em]
Hollywood Hills     &                     &      -59.21\sym{***}&      -5.598         \\
                    &                     &     (3.923)         &     (9.079)         \\
[1em]
Hollywood Hills West&                     &      -10.88\sym{*}  &       40.10\sym{***}\\
                    &                     &     (4.943)         &     (9.992)         \\
[1em]
Holy Cross          &                     &      -72.63\sym{***}&      -140.0\sym{***}\\
                    &                     &     (8.125)         &     (13.79)         \\
[1em]
Howard Beach        &                     &      -137.2\sym{***}&                     \\
                    &                     &     (27.47)         &                     \\
[1em]
Howard University, Le Droit Park&                     &      -66.04\sym{***}&       13.18         \\
                    &                     &     (4.042)         &     (10.23)         \\
[1em]
Humboldt Park       &                     &      -101.6\sym{***}&      -65.82\sym{**} \\
                    &                     &     (9.206)         &     (25.01)         \\
[1em]
Hyde Park           &                     &      -15.69         &      -38.86\sym{*}  \\
                    &                     &     (8.170)         &     (18.14)         \\
[1em]
Iberville           &                     &      -14.57         &      -57.55\sym{***}\\
                    &                     &     (8.283)         &     (13.95)         \\
[1em]
Industry            &                     &      -83.49\sym{***}&      -106.5\sym{***}\\
                    &                     &     (7.037)         &     (10.98)         \\
[1em]
Inglewood           &                     &      -87.72\sym{***}&      -54.42\sym{***}\\
                    &                     &     (4.192)         &     (8.133)         \\
[1em]
Inwood              &                     &      -80.62\sym{**} &      -14.17         \\
                    &                     &     (27.14)         &     (17.46)         \\
[1em]
Irish Channel       &                     &      -80.57\sym{***}&      -37.44\sym{**} \\
                    &                     &     (7.457)         &     (13.86)         \\
[1em]
Irving Park         &                     &      -33.00\sym{***}&      -38.93         \\
                    &                     &     (8.111)         &     (25.96)         \\
[1em]
Ivy City, Arboretum, Trinidad, C&                     &      -91.69\sym{***}&      -42.03\sym{***}\\
                    &                     &     (4.369)         &     (10.51)         \\
[1em]
Jackson Heights     &                     &      -77.00\sym{**} &      -55.91\sym{**} \\
                    &                     &     (27.19)         &     (18.17)         \\
[1em]
Jamaica             &                     &      -114.6\sym{***}&      -45.77\sym{*}  \\
                    &                     &     (27.06)         &     (18.25)         \\
[1em]
Jamaica Estates     &                     &      -316.6\sym{***}&      -47.98\sym{**} \\
                    &                     &     (28.20)         &     (17.60)         \\
[1em]
Jefferson Park      &                     &      -68.92\sym{***}&      -80.31\sym{***}\\
                    &                     &     (4.443)         &     (13.38)         \\
[1em]
Kalorama Heights, Adams Morgan,&                     &      -57.95\sym{***}&       33.48\sym{***}\\
                    &                     &     (3.788)         &     (9.077)         \\
[1em]
Kensington          &                     &      -106.2\sym{***}&      -68.20\sym{***}\\
                    &                     &     (27.33)         &     (18.30)         \\
[1em]
Kenwood             &                     &      -13.77         &      -7.684         \\
                    &                     &     (8.795)         &     (27.82)         \\
[1em]
Kingsbridge         &                     &      -103.4\sym{***}&      -45.14\sym{**} \\
                    &                     &     (27.45)         &     (16.96)         \\
[1em]
Kips Bay            &                     &       6.324         &       19.34         \\
                    &                     &     (27.25)         &     (17.92)         \\
[1em]
Koreatown           &                     &      -59.91\sym{***}&      -20.15\sym{*}  \\
                    &                     &     (3.740)         &     (8.494)         \\
[1em]
La Ca��ada Flintridge&                     &      -39.73\sym{***}&      -65.27\sym{***}\\
                    &                     &     (6.903)         &     (12.63)         \\
[1em]
La Crescenta-Montrose&                     &      -61.95\sym{***}&      -32.43\sym{**} \\
                    &                     &     (7.137)         &     (10.01)         \\
[1em]
La Habra Heights    &                     &      -81.26\sym{***}&      -22.73         \\
                    &                     &     (5.314)         &     (17.47)         \\
[1em]
La Mirada           &                     &      -77.48\sym{***}&      -38.75\sym{***}\\
                    &                     &     (7.083)         &     (8.664)         \\
[1em]
La Puente           &                     &      -155.0\sym{***}&                     \\
                    &                     &     (7.100)         &                     \\
[1em]
Ladera Heights      &                     &      -36.97\sym{***}&       63.93\sym{***}\\
                    &                     &     (6.071)         &     (9.769)         \\
[1em]
Lake Balboa         &                     &      -111.8\sym{***}&      -48.34\sym{***}\\
                    &                     &     (5.153)         &     (9.289)         \\
[1em]
Lake Terrace & Oaks &                     &      -111.4\sym{***}&                     \\
                    &                     &     (9.954)         &                     \\
[1em]
Lake View           &                     &       0.241         &      -20.11         \\
                    &                     &     (7.867)         &     (25.54)         \\
[1em]
Lakeview            &                     &      -54.83\sym{***}&       11.22         \\
                    &                     &     (8.251)         &     (13.58)         \\
[1em]
Lakewood            &                     &      -143.1\sym{***}&      -80.50\sym{***}\\
                    &                     &     (5.545)         &     (10.13)         \\
[1em]
Lamont Riggs, Queens Chapel, For&                     &      -115.7\sym{***}&      -35.95\sym{***}\\
                    &                     &     (7.460)         &     (9.398)         \\
[1em]
Lancaster           &                     &      -97.39\sym{***}&      -38.91\sym{***}\\
                    &                     &     (5.396)         &     (10.15)         \\
[1em]
Larchmont           &                     &      -61.24\sym{***}&      -21.49\sym{*}  \\
                    &                     &     (3.965)         &     (8.629)         \\
[1em]
Lawndale            &                     &      -139.2\sym{***}&      -31.91\sym{***}\\
                    &                     &     (5.809)         &     (9.569)         \\
[1em]
Leimert Park        &                     &      -64.35\sym{***}&      -50.31\sym{***}\\
                    &                     &     (5.281)         &     (9.322)         \\
[1em]
Leonidas            &                     &      -74.77\sym{***}&      -67.52\sym{***}\\
                    &                     &     (7.063)         &     (13.04)         \\
[1em]
Lincoln Heights     &                     &      -116.0\sym{***}&      -31.06\sym{**} \\
                    &                     &     (5.886)         &     (9.984)         \\
[1em]
Lincoln Park        &                     &       16.88\sym{*}  &       31.35         \\
                    &                     &     (7.882)         &     (25.86)         \\
[1em]
Lincoln Square      &                     &      -31.48\sym{***}&      -39.56         \\
                    &                     &     (8.144)         &     (25.62)         \\
[1em]
Little Italy        &                     &      -50.42         &       45.70\sym{**} \\
                    &                     &     (27.62)         &     (17.28)         \\
[1em]
Little Woods        &                     &      -106.8\sym{***}&      -159.6\sym{***}\\
                    &                     &     (8.928)         &     (13.55)         \\
[1em]
Logan Square        &                     &      -33.63\sym{***}&      -48.43         \\
                    &                     &     (8.057)         &     (25.62)         \\
[1em]
Lomita              &                     &      -54.69\sym{***}&      -45.04\sym{***}\\
                    &                     &     (7.318)         &     (8.922)         \\
[1em]
Long Beach          &                     &      -65.41\sym{***}&      -34.68\sym{***}\\
                    &                     &     (3.732)         &     (8.691)         \\
[1em]
Long Island City    &                     &      -67.40\sym{*}  &      -27.26         \\
                    &                     &     (27.13)         &     (17.08)         \\
[1em]
Longwood            &                     &      -106.7\sym{***}&      -74.03\sym{***}\\
                    &                     &     (27.13)         &     (18.53)         \\
[1em]
Loop                &                     &       41.89\sym{***}&       64.09\sym{*}  \\
                    &                     &     (8.580)         &     (25.73)         \\
[1em]
Los Feliz           &                     &      -62.66\sym{***}&      -12.47         \\
                    &                     &     (3.965)         &     (9.139)         \\
[1em]
Lower East Side     &                     &      -31.63         &       40.98\sym{*}  \\
                    &                     &     (27.13)         &     (17.32)         \\
[1em]
Lower Garden District&                     &      -36.67\sym{***}&      -14.20         \\
                    &                     &     (7.373)         &     (12.72)         \\
[1em]
Lower West Side     &                     &      -42.09\sym{***}&      -45.03         \\
                    &                     &     (7.923)         &     (26.07)         \\
[1em]
Lynwood             &                     &      -115.5\sym{***}&      -76.42\sym{***}\\
                    &                     &     (7.421)         &     (11.88)         \\
[1em]
Malibu              &                     &       81.01\sym{***}&       127.4\sym{***}\\
                    &                     &     (4.708)         &     (10.27)         \\
[1em]
Manhattan Beach     &                     &       16.86\sym{**} &       69.62\sym{***}\\
                    &                     &     (5.314)         &     (9.940)         \\
[1em]
Mar Vista           &                     &      -80.73\sym{***}&      -26.83\sym{**} \\
                    &                     &     (4.263)         &     (8.936)         \\
[1em]
Marigny             &                     &      -23.68\sym{**} &      -13.05         \\
                    &                     &     (7.310)         &     (13.96)         \\
[1em]
Marina del Rey      &                     &      -35.43\sym{***}&       42.88\sym{**} \\
                    &                     &     (4.728)         &     (13.40)         \\
[1em]
Marlyville - Fontainbleau&                     &      -99.43\sym{***}&      -64.11\sym{***}\\
                    &                     &     (7.379)         &     (13.40)         \\
[1em]
Maspeth             &                     &      -122.3\sym{***}&      -67.91\sym{***}\\
                    &                     &     (27.05)         &     (17.62)         \\
[1em]
Mcdonogh            &                     &      -133.2\sym{***}&      -32.77\sym{*}  \\
                    &                     &     (8.376)         &     (13.04)         \\
[1em]
Mckinley Park       &                     &      -58.87\sym{***}&      -65.72\sym{*}  \\
                    &                     &     (8.795)         &     (25.53)         \\
[1em]
Mid-City            &                     &      -79.37\sym{***}&      -39.26\sym{***}\\
                    &                     &     (5.280)         &     (10.29)         \\
[1em]
Mid-Wilshire        &                     &      -56.24\sym{***}&      -25.86\sym{**} \\
                    &                     &     (3.808)         &     (8.454)         \\
[1em]
Middle Village      &                     &      -153.4\sym{***}&      -93.68\sym{***}\\
                    &                     &     (27.65)         &     (19.70)         \\
[1em]
Midtown             &                     &       3.981         &       77.45\sym{***}\\
                    &                     &     (27.16)         &     (17.84)         \\
[1em]
Midwood             &                     &      -88.17\sym{**} &      -48.97\sym{**} \\
                    &                     &     (27.41)         &     (17.36)         \\
[1em]
Milan               &                     &      -5.567         &      -31.76\sym{*}  \\
                    &                     &     (8.243)         &     (12.85)         \\
[1em]
Mill Basin          &                     &      -42.08         &                     \\
                    &                     &     (28.01)         &                     \\
[1em]
Milneburg           &                     &      -174.1\sym{***}&      -113.7\sym{***}\\
                    &                     &     (10.89)         &     (13.61)         \\
[1em]
Monrovia            &                     &      -85.97\sym{***}&      -20.94\sym{*}  \\
                    &                     &     (4.951)         &     (10.33)         \\
[1em]
Montclare           &                     &      -95.92\sym{***}&      -116.6\sym{***}\\
                    &                     &     (9.396)         &     (26.28)         \\
[1em]
Montebello          &                     &      -101.0\sym{***}&      -31.02\sym{***}\\
                    &                     &     (4.894)         &     (9.255)         \\
[1em]
Montecito Heights   &                     &      -114.9\sym{***}&      -41.41\sym{***}\\
                    &                     &     (4.973)         &     (8.695)         \\
[1em]
Monterey Park       &                     &      -94.20\sym{***}&      -70.38\sym{***}\\
                    &                     &     (4.985)         &     (10.72)         \\
[1em]
Morgan Park         &                     &      -80.39\sym{***}&      -144.8\sym{***}\\
                    &                     &     (8.631)         &     (27.24)         \\
[1em]
Morningside Heights &                     &      -48.03         &      -2.603         \\
                    &                     &     (27.03)         &     (17.17)         \\
[1em]
Mount Eden          &                     &      -120.5\sym{***}&                     \\
                    &                     &     (26.77)         &                     \\
[1em]
Mount Hope          &                     &      -71.88\sym{**} &      -44.93\sym{*}  \\
                    &                     &     (27.70)         &     (18.01)         \\
[1em]
Mount Washington    &                     &      -80.58\sym{***}&      -46.98\sym{***}\\
                    &                     &     (4.138)         &     (10.63)         \\
[1em]
Murray Hill         &                     &      -11.81         &       56.48\sym{**} \\
                    &                     &     (27.19)         &     (17.33)         \\
[1em]
Navarre             &                     &      -72.47\sym{***}&      -67.69\sym{***}\\
                    &                     &     (7.956)         &     (14.43)         \\
[1em]
Navy Yard           &                     &      -42.43         &                     \\
                    &                     &     (28.34)         &                     \\
[1em]
Near North Side     &                     &       16.18\sym{*}  &       38.19         \\
                    &                     &     (7.975)         &     (25.98)         \\
[1em]
Near South Side     &                     &       53.19\sym{***}&                     \\
                    &                     &     (13.38)         &                     \\
[1em]
Near Southeast, Navy Yard&                     &      -48.26\sym{***}&       24.17\sym{*}  \\
                    &                     &     (4.750)         &     (11.38)         \\
[1em]
Near West Side      &                     &      -6.838         &       6.236         \\
                    &                     &     (8.022)         &     (25.84)         \\
[1em]
New Brighton        &                     &      -128.6\sym{***}&                     \\
                    &                     &     (27.23)         &                     \\
[1em]
New City            &                     &      -9.021         &                     \\
                    &                     &     (11.48)         &                     \\
[1em]
New Dorp Beach      &                     &      -152.2\sym{***}&                     \\
                    &                     &     (28.51)         &                     \\
[1em]
NoHo                &                     &      -5.274         &       142.2\sym{***}\\
                    &                     &     (27.13)         &     (17.58)         \\
[1em]
Nolita              &                     &      -9.911         &       37.28\sym{*}  \\
                    &                     &     (27.09)         &     (17.57)         \\
[1em]
North Center        &                     &      -16.88\sym{*}  &       3.559         \\
                    &                     &     (8.129)         &     (25.91)         \\
[1em]
North Cleveland Park, Forest Hil&                     &      -76.18\sym{***}&      -28.96\sym{**} \\
                    &                     &     (4.234)         &     (9.906)         \\
[1em]
North Hollywood     &                     &      -86.12\sym{***}&      -34.02\sym{***}\\
                    &                     &     (3.917)         &     (8.379)         \\
[1em]
North Michigan Park, Michigan Pa&                     &      -89.52\sym{***}&      -54.63\sym{***}\\
                    &                     &     (4.496)         &     (10.79)         \\
[1em]
North Park          &                     &      -105.4\sym{***}&      -109.2\sym{***}\\
                    &                     &     (11.57)         &     (26.12)         \\
[1em]
North Riverdale     &                     &      -73.31\sym{**} &                     \\
                    &                     &     (27.64)         &                     \\
[1em]
Northridge          &                     &      -116.9\sym{***}&      -57.57\sym{***}\\
                    &                     &     (5.878)         &     (9.740)         \\
[1em]
Norwood Park        &                     &      -63.32\sym{***}&      -77.50\sym{**} \\
                    &                     &     (8.796)         &     (26.18)         \\
[1em]
Oakland             &                     &      -4.350         &      -50.23         \\
                    &                     &     (8.726)         &     (27.67)         \\
[1em]
Ozone Park          &                     &      -127.2\sym{***}&      -48.64\sym{**} \\
                    &                     &     (27.12)         &     (17.89)         \\
[1em]
Pacific Palisades   &                     &      -29.97\sym{***}&       64.98\sym{***}\\
                    &                     &     (4.625)         &     (10.08)         \\
[1em]
Pacoima             &                     &      -104.6\sym{***}&                     \\
                    &                     &     (7.475)         &                     \\
[1em]
Palmdale            &                     &      -107.4\sym{***}&      -88.40\sym{***}\\
                    &                     &     (6.007)         &     (10.65)         \\
[1em]
Palms               &                     &      -90.07\sym{***}&      -14.26         \\
                    &                     &     (3.748)         &     (7.749)         \\
[1em]
Palos Verdes Estates&                     &      -77.38\sym{***}&      -25.40\sym{**} \\
                    &                     &     (6.946)         &     (9.050)         \\
[1em]
Panorama City       &                     &      -69.30\sym{***}&      -40.74\sym{***}\\
                    &                     &     (6.668)         &     (9.855)         \\
[1em]
Paramount           &                     &      -55.63\sym{***}&      -88.84\sym{***}\\
                    &                     &     (6.373)         &     (9.404)         \\
[1em]
Park Slope          &                     &      -48.93         &       3.811         \\
                    &                     &     (26.98)         &     (17.29)         \\
[1em]
Pasadena            &                     &      -53.37\sym{***}&      -9.950         \\
                    &                     &     (3.803)         &     (9.655)         \\
[1em]
Pico Rivera         &                     &      -87.99\sym{***}&      -45.89\sym{***}\\
                    &                     &     (5.583)         &     (10.02)         \\
[1em]
Pico-Robertson      &                     &      -59.85\sym{***}&      -43.89\sym{***}\\
                    &                     &     (4.067)         &     (8.204)         \\
[1em]
Pico-Union          &                     &      -119.3\sym{***}&      -34.20\sym{***}\\
                    &                     &     (5.463)         &     (9.159)         \\
[1em]
Playa Vista         &                     &      -48.46\sym{***}&      -2.445         \\
                    &                     &     (4.467)         &     (9.519)         \\
[1em]
Playa del Rey       &                     &      -2.219         &       6.289         \\
                    &                     &     (4.325)         &     (9.390)         \\
[1em]
Pomona              &                     &      -76.33\sym{***}&      -52.56\sym{***}\\
                    &                     &     (4.671)         &     (10.81)         \\
[1em]
Portage Park        &                     &      -93.90\sym{***}&      -58.67\sym{*}  \\
                    &                     &     (8.513)         &     (26.18)         \\
[1em]
Porter Ranch        &                     &      -39.45\sym{***}&                     \\
                    &                     &     (6.524)         &                     \\
[1em]
Prospect Heights    &                     &      -61.14\sym{*}  &      -20.28         \\
                    &                     &     (27.06)         &     (17.01)         \\
[1em]
Prospect-Lefferts Gardens&                     &      -104.1\sym{***}&      -54.65\sym{**} \\
                    &                     &     (27.04)         &     (17.49)         \\
[1em]
Pullman             &                     &      -57.59\sym{***}&                     \\
                    &                     &     (9.459)         &                     \\
[1em]
Queens Village      &                     &      -89.47\sym{**} &      -82.17\sym{***}\\
                    &                     &     (27.62)         &     (18.43)         \\
[1em]
Rancho Palos Verdes &                     &      -19.74\sym{***}&       7.938         \\
                    &                     &     (5.363)         &     (10.53)         \\
[1em]
Rancho Park         &                     &      -63.50\sym{***}&       3.723         \\
                    &                     &     (4.853)         &     (9.535)         \\
[1em]
Red Hook            &                     &      -57.78\sym{*}  &                     \\
                    &                     &     (27.34)         &                     \\
[1em]
Redondo Beach       &                     &      -64.25\sym{***}&      -1.793         \\
                    &                     &     (4.128)         &     (9.462)         \\
[1em]
Rego Park           &                     &      -73.89\sym{**} &      -6.369         \\
                    &                     &     (27.51)         &     (18.24)         \\
[1em]
Reseda              &                     &      -121.7\sym{***}&      -60.70\sym{***}\\
                    &                     &     (5.480)         &     (9.989)         \\
[1em]
Richmond Hill       &                     &      -102.9\sym{***}&      -65.42\sym{***}\\
                    &                     &     (27.22)         &     (18.52)         \\
[1em]
Ridgewood           &                     &      -108.0\sym{***}&      -94.25\sym{***}\\
                    &                     &     (27.08)         &     (18.53)         \\
[1em]
River Terrace, Benning, Greenway&                     &      -195.6\sym{***}&      -1.785         \\
                    &                     &     (9.286)         &     (11.63)         \\
[1em]
Riverdale           &                     &      -117.3\sym{***}&       45.25\sym{*}  \\
                    &                     &     (27.24)         &     (18.25)         \\
[1em]
Rogers Park         &                     &      -46.75\sym{***}&      -62.79\sym{*}  \\
                    &                     &     (8.232)         &     (25.86)         \\
[1em]
Roosevelt Island    &                     &      -111.8\sym{***}&      -20.48         \\
                    &                     &     (27.24)         &     (19.64)         \\
[1em]
Rosemead            &                     &      -95.24\sym{***}&      -58.03\sym{***}\\
                    &                     &     (9.805)         &     (11.74)         \\
[1em]
Rowland Heights     &                     &      -129.5\sym{***}&      -62.25\sym{***}\\
                    &                     &     (6.357)         &     (11.41)         \\
[1em]
San Gabriel         &                     &      -87.02\sym{***}&      -29.12\sym{**} \\
                    &                     &     (8.005)         &     (11.05)         \\
[1em]
San Pasqual         &                     &      -120.1\sym{***}&      -21.30         \\
                    &                     &     (7.495)         &     (12.34)         \\
[1em]
San Pedro           &                     &      -49.10\sym{***}&      -33.31\sym{***}\\
                    &                     &     (4.352)         &     (9.411)         \\
[1em]
Santa Clarita       &                     &      -82.46\sym{***}&      -37.72\sym{***}\\
                    &                     &     (7.996)         &     (9.729)         \\
[1em]
Santa Fe Springs    &                     &      -48.41\sym{***}&                     \\
                    &                     &     (7.894)         &                     \\
[1em]
Santa Monica        &                     &      -31.10\sym{***}&       31.12\sym{***}\\
                    &                     &     (4.100)         &     (8.707)         \\
[1em]
Sawtelle            &                     &      -57.27\sym{***}&      -9.309         \\
                    &                     &     (3.997)         &     (8.084)         \\
[1em]
Schuylerville       &                     &      -157.3\sym{***}&                     \\
                    &                     &     (27.82)         &                     \\
[1em]
Seventh Ward        &                     &      -58.68\sym{***}&      -73.95\sym{***}\\
                    &                     &     (7.207)         &     (13.75)         \\
[1em]
Shaw, Logan Circle  &                     &      -33.43\sym{***}&       11.24         \\
                    &                     &     (3.782)         &     (8.951)         \\
[1em]
Sheepshead Bay      &                     &      -224.8\sym{***}&      -77.10\sym{***}\\
                    &                     &     (28.00)         &     (17.82)         \\
[1em]
Sherman Oaks        &                     &      -61.17\sym{***}&      -29.78\sym{***}\\
                    &                     &     (3.981)         &     (8.696)         \\
[1em]
Sierra Madre        &                     &      -66.30\sym{***}&      -45.45\sym{***}\\
                    &                     &     (8.752)         &     (7.180)         \\
[1em]
Signal Hill         &                     &      -118.8\sym{***}&       72.93\sym{***}\\
                    &                     &     (7.026)         &     (8.919)         \\
[1em]
Silver Lake         &                     &      -61.06\sym{***}&      -24.87\sym{**} \\
                    &                     &     (4.214)         &     (8.937)         \\
[1em]
SoHo                &                     &       12.01         &       73.26\sym{***}\\
                    &                     &     (27.09)         &     (17.43)         \\
[1em]
South Gate          &                     &      -113.0\sym{***}&      -64.65\sym{***}\\
                    &                     &     (5.572)         &     (9.751)         \\
[1em]
South Lawndale      &                     &      -30.65\sym{***}&      -15.09         \\
                    &                     &     (9.178)         &     (26.44)         \\
[1em]
South Ozone Park    &                     &      -191.3\sym{***}&       24.42         \\
                    &                     &     (27.24)         &     (17.86)         \\
[1em]
South Park          &                     &      -116.4\sym{***}&                     \\
                    &                     &     (5.581)         &                     \\
[1em]
South Pasadena      &                     &      -60.87\sym{***}&      -70.58\sym{***}\\
                    &                     &     (5.489)         &     (11.16)         \\
[1em]
South San Gabriel   &                     &      -87.29\sym{***}&      -102.9\sym{***}\\
                    &                     &     (6.141)         &     (9.136)         \\
[1em]
South San Jose Hills&                     &      -96.76\sym{***}&                     \\
                    &                     &     (8.945)         &                     \\
[1em]
South Shore         &                     &      -31.95\sym{***}&      -67.89\sym{**} \\
                    &                     &     (8.708)         &     (25.83)         \\
[1em]
South Slope         &                     &      -68.18\sym{*}  &      -46.18\sym{**} \\
                    &                     &     (27.07)         &     (17.60)         \\
[1em]
Southeast Antelope Valley&                     &      -149.4\sym{***}&      -162.6\sym{***}\\
                    &                     &     (13.65)         &     (12.30)         \\
[1em]
Southwest Employment Area, South&                     &      -27.79\sym{***}&      -17.10         \\
                    &                     &     (4.439)         &     (9.380)         \\
[1em]
Springfield Gardens &                     &      -91.08\sym{**} &      -78.87\sym{***}\\
                    &                     &     (29.03)         &     (18.35)         \\
[1em]
Spuyten Duyvil      &                     &      -181.4\sym{***}&                     \\
                    &                     &     (30.06)         &                     \\
[1em]
St.  Anthony        &                     &      -91.04\sym{***}&      -126.4\sym{***}\\
                    &                     &     (8.886)         &     (14.60)         \\
[1em]
St. Claude          &                     &      -94.61\sym{***}&      -73.84\sym{***}\\
                    &                     &     (7.346)         &     (13.52)         \\
[1em]
St. George          &                     &      -141.1\sym{***}&      -63.02\sym{**} \\
                    &                     &     (27.42)         &     (19.77)         \\
[1em]
St. Roch            &                     &      -112.5\sym{***}&      -53.24\sym{***}\\
                    &                     &     (7.202)         &     (13.61)         \\
[1em]
St. Thomas Dev      &                     &      -120.8\sym{***}&       7.527         \\
                    &                     &     (10.66)         &     (12.77)         \\
[1em]
Stevenson Ranch     &                     &      -137.3\sym{***}&      -42.65\sym{***}\\
                    &                     &     (6.375)         &     (10.79)         \\
[1em]
Studio City         &                     &      -37.45\sym{***}&       4.452         \\
                    &                     &     (4.627)         &     (9.047)         \\
[1em]
Sun Valley          &                     &      -84.91\sym{***}&      -35.72\sym{***}\\
                    &                     &     (4.240)         &     (8.966)         \\
[1em]
Sunland             &                     &      -117.4\sym{***}&      -59.81\sym{***}\\
                    &                     &     (6.530)         &     (9.701)         \\
[1em]
Sunnyside           &                     &      -84.84\sym{**} &      -49.62\sym{**} \\
                    &                     &     (27.09)         &     (17.30)         \\
[1em]
Sunset Park         &                     &      -82.04\sym{**} &      -39.10\sym{*}  \\
                    &                     &     (26.94)         &     (17.23)         \\
[1em]
Sylmar              &                     &      -112.6\sym{***}&      -140.8\sym{***}\\
                    &                     &     (5.549)         &     (11.35)         \\
[1em]
Takoma, Brightwood, Manor Park&                     &      -95.14\sym{***}&      -61.97\sym{***}\\
                    &                     &     (4.570)         &     (9.295)         \\
[1em]
Tall Timbers - Brechtel&                     &      -153.4\sym{***}&      -75.65\sym{***}\\
                    &                     &     (9.435)         &     (14.66)         \\
[1em]
Tarzana             &                     &      -74.68\sym{***}&      -40.27\sym{***}\\
                    &                     &     (5.638)         &     (10.82)         \\
[1em]
Temple City         &                     &      -114.3\sym{***}&      -86.34\sym{***}\\
                    &                     &     (7.458)         &     (11.38)         \\
[1em]
Theater District    &                     &       22.32         &      -6.939         \\
                    &                     &     (27.39)         &     (17.92)         \\
[1em]
Toluca Lake         &                     &      -20.78\sym{**} &      -51.38\sym{***}\\
                    &                     &     (6.997)         &     (8.911)         \\
[1em]
Tompkinsville       &                     &      -81.06\sym{**} &      -22.51         \\
                    &                     &     (27.08)         &     (18.30)         \\
[1em]
Topanga             &                     &      -51.36\sym{***}&       2.979         \\
                    &                     &     (4.056)         &     (10.23)         \\
[1em]
Torrance            &                     &      -88.98\sym{***}&      -26.38\sym{**} \\
                    &                     &     (4.652)         &     (8.924)         \\
[1em]
Touro               &                     &      -40.85\sym{***}&      -29.57\sym{*}  \\
                    &                     &     (7.524)         &     (13.08)         \\
[1em]
Treme - Lafitte     &                     &      -58.08\sym{***}&      -27.69\sym{*}  \\
                    &                     &     (7.133)         &     (13.53)         \\
[1em]
Tribeca             &                     &       39.96         &       124.3\sym{***}\\
                    &                     &     (27.42)         &     (17.91)         \\
[1em]
Tujunga             &                     &      -103.7\sym{***}&      -44.47\sym{***}\\
                    &                     &     (9.316)         &     (10.54)         \\
[1em]
Tulane - Gravier    &                     &      -81.75\sym{***}&      -115.9\sym{***}\\
                    &                     &     (7.624)         &     (13.06)         \\
[1em]
Twining, Fairlawn, Randle Highla&                     &      -63.15\sym{***}&      -88.00\sym{***}\\
                    &                     &     (7.969)         &     (11.30)         \\
[1em]
Two Bridges         &                     &      -82.87\sym{**} &      -11.58         \\
                    &                     &     (27.85)         &     (16.58)         \\
[1em]
Unincorporated Santa Monica Moun&                     &       39.51\sym{***}&       77.38\sym{***}\\
                    &                     &     (5.115)         &     (9.592)         \\
[1em]
Union Station, Stanton Park, Kin&                     &      -69.99\sym{***}&      -22.80\sym{*}  \\
                    &                     &     (3.654)         &     (9.535)         \\
[1em]
University Heights  &                     &      -103.4\sym{***}&      -71.77\sym{***}\\
                    &                     &     (29.64)         &     (17.74)         \\
[1em]
University Park     &                     &      -90.52\sym{***}&      -109.8\sym{***}\\
                    &                     &     (4.764)         &     (9.481)         \\
[1em]
Upper East Side     &                     &      -21.99         &       30.97         \\
                    &                     &     (27.14)         &     (17.41)         \\
[1em]
Upper West Side     &                     &      -23.32         &       18.84         \\
                    &                     &     (27.18)         &     (17.72)         \\
[1em]
Uptown              &                     &      -34.51\sym{***}&      -30.98         \\
                    &                     &     (7.744)         &     (20.29)         \\
[1em]
Val Verde           &                     &      -141.9\sym{***}&      -45.98\sym{**} \\
                    &                     &     (7.222)         &     (14.10)         \\
[1em]
Valley Glen         &                     &      -80.81\sym{***}&      -59.55\sym{***}\\
                    &                     &     (4.213)         &     (8.651)         \\
[1em]
Valley Village      &                     &      -70.34\sym{***}&      -36.16\sym{***}\\
                    &                     &     (4.229)         &     (8.146)         \\
[1em]
Van Nuys            &                     &      -78.02\sym{***}&      -46.46\sym{***}\\
                    &                     &     (4.605)         &     (8.517)         \\
[1em]
Venice              &                     &      -14.06\sym{***}&       40.93\sym{***}\\
                    &                     &     (4.070)         &     (9.324)         \\
[1em]
Vermont Square      &                     &      -112.7\sym{***}&      -36.84\sym{**} \\
                    &                     &     (4.135)         &     (11.34)         \\
[1em]
View Park-Windsor Hills&                     &      -42.47\sym{***}&      -47.43\sym{***}\\
                    &                     &     (6.041)         &     (10.43)         \\
[1em]
Village De Lest     &                     &      -83.13\sym{***}&      -83.86\sym{***}\\
                    &                     &     (10.97)         &     (15.33)         \\
[1em]
Vinegar Hill        &                     &      -78.77\sym{**} &       235.9\sym{***}\\
                    &                     &     (27.79)         &     (20.00)         \\
[1em]
Wakefield           &                     &      -42.09         &      -93.46\sym{***}\\
                    &                     &     (27.25)         &     (19.33)         \\
[1em]
Walnut              &                     &      -90.94\sym{***}&      -65.45\sym{***}\\
                    &                     &     (11.16)         &     (11.22)         \\
[1em]
Washington Heights  &                     &      -62.83\sym{*}  &      -28.59         \\
                    &                     &     (27.17)         &     (17.06)         \\
[1em]
Washington Park     &                     &      -33.58\sym{**} &      -110.1\sym{***}\\
                    &                     &     (11.41)         &     (27.97)         \\
[1em]
West Adams          &                     &      -78.24\sym{***}&      -76.18\sym{***}\\
                    &                     &     (4.180)         &     (8.139)         \\
[1em]
West Carson         &                     &      -66.85\sym{***}&       67.12\sym{***}\\
                    &                     &     (6.125)         &     (11.77)         \\
[1em]
West Covina         &                     &      -175.5\sym{***}&      -69.84\sym{***}\\
                    &                     &     (6.612)         &     (10.58)         \\
[1em]
West End            &                     &      -70.99\sym{***}&      -72.05\sym{***}\\
                    &                     &     (10.10)         &     (13.99)         \\
[1em]
West End, Foggy Bottom, GWU&                     &      -32.91\sym{***}&       31.34\sym{***}\\
                    &                     &     (4.275)         &     (9.132)         \\
[1em]
West Englewood      &                     &      -73.15\sym{***}&      -71.18\sym{*}  \\
                    &                     &     (12.55)         &     (29.86)         \\
[1em]
West Hills          &                     &      -105.8\sym{***}&      -51.57\sym{***}\\
                    &                     &     (4.184)         &     (10.60)         \\
[1em]
West Hollywood      &                     &      -44.60\sym{***}&      -8.665         \\
                    &                     &     (3.852)         &     (8.603)         \\
[1em]
West Los Angeles    &                     &      -43.19\sym{***}&      -25.69\sym{**} \\
                    &                     &     (4.558)         &     (8.864)         \\
[1em]
West Ridge          &                     &      -20.34\sym{*}  &      -113.2\sym{***}\\
                    &                     &     (9.480)         &     (25.53)         \\
[1em]
West Riverside      &                     &      -25.37\sym{**} &      -38.54\sym{**} \\
                    &                     &     (7.738)         &     (13.82)         \\
[1em]
West Town           &                     &      -8.990         &      -18.01         \\
                    &                     &     (8.127)         &     (25.62)         \\
[1em]
West Village        &                     &       19.53         &       86.67\sym{***}\\
                    &                     &     (27.09)         &     (17.85)         \\
[1em]
Westchester         &                     &      -70.95\sym{***}&      -18.56\sym{*}  \\
                    &                     &     (4.284)         &     (8.558)         \\
[1em]
Westlake            &                     &      -72.11\sym{***}&      -20.75\sym{*}  \\
                    &                     &     (3.789)         &     (8.519)         \\
[1em]
Westwood            &                     &      -35.66\sym{***}&       10.47         \\
                    &                     &     (4.138)         &     (8.299)         \\
[1em]
Whitney             &                     &      -68.67\sym{***}&      -57.81\sym{***}\\
                    &                     &     (9.077)         &     (16.77)         \\
[1em]
Whittier            &                     &      -122.4\sym{***}&      -56.12\sym{***}\\
                    &                     &     (5.626)         &     (10.88)         \\
[1em]
Williamsbridge      &                     &      -72.50\sym{**} &      -131.1\sym{***}\\
                    &                     &     (27.40)         &     (18.82)         \\
[1em]
Williamsburg        &                     &      -57.57\sym{*}  &      -8.148         \\
                    &                     &     (27.04)         &     (17.14)         \\
[1em]
Windsor Square      &                     &      -80.00\sym{***}&      -20.70         \\
                    &                     &     (8.300)         &     (12.41)         \\
[1em]
Windsor Terrace     &                     &      -85.08\sym{**} &      -19.31         \\
                    &                     &     (27.42)         &     (18.02)         \\
[1em]
Winnetka            &                     &      -101.9\sym{***}&      -37.69\sym{***}\\
                    &                     &     (5.275)         &     (9.911)         \\
[1em]
Woodhaven           &                     &      -96.58\sym{***}&      -20.74         \\
                    &                     &     (27.31)         &     (18.89)         \\
[1em]
Woodland Hills      &                     &      -102.8\sym{***}&      -48.46\sym{***}\\
                    &                     &     (4.933)         &     (9.085)         \\
[1em]
Woodlawn            &                     &      -52.54\sym{***}&      -34.07         \\
                    &                     &     (8.540)         &     (26.27)         \\
[1em]
Woodridge, Fort Lincoln, Gateway&                     &      -102.9\sym{***}&      -41.00\sym{***}\\
                    &                     &     (7.442)         &     (10.58)         \\
[1em]
Woodside            &                     &      -76.52\sym{**} &      -48.32\sym{**} \\
                    &                     &     (27.19)         &     (17.01)         \\
[1em]
Los Angeles         &                     &           0         &           0         \\
                    &                     &         (.)         &         (.)         \\
[1em]
Washington DC       &                     &           0         &           0         \\
                    &                     &         (.)         &         (.)         \\
[1em]
Chicago             &                     &      -34.56\sym{***}&       28.51         \\
                    &                     &     (7.199)         &     (23.86)         \\
[1em]
New Orleans         &                     &      -19.15\sym{**} &       14.90         \\
                    &                     &     (6.236)         &     (9.927)         \\
[1em]
New York City       &                     &       18.94         &       23.32         \\
                    &                     &     (26.80)         &     (14.56)         \\
[1em]
Nashville           &                     &           0         &           0         \\
                    &                     &         (.)         &         (.)         \\
[1em]
Austin              &                     &           0         &       73.96\sym{***}\\
                    &                     &         (.)         &     (8.158)         \\
[1em]
Apartment           &                     &           0         &           0         \\
                    &                     &         (.)         &         (.)         \\
[1em]
Bed & Breakfast     &                     &       28.70\sym{***}&       17.73\sym{**} \\
                    &                     &     (6.771)         &     (6.045)         \\
[1em]
Boat                &                     &      -25.12         &      -53.84         \\
                    &                     &     (16.75)         &     (31.94)         \\
[1em]
Bungalow            &                     &       23.00         &      -7.257         \\
                    &                     &     (12.48)         &     (6.723)         \\
[1em]
Cabin               &                     &       8.083         &       8.980         \\
                    &                     &     (9.262)         &     (13.54)         \\
[1em]
Camper/RV           &                     &      -26.02         &      -58.37\sym{***}\\
                    &                     &     (18.01)         &     (15.26)         \\
[1em]
Castle              &                     &       11.17         &       29.77\sym{**} \\
                    &                     &     (6.456)         &     (10.30)         \\
[1em]
Chalet              &                     &       482.5\sym{***}&           0         \\
                    &                     &     (7.176)         &         (.)         \\
[1em]
Condominium         &                     &      -5.780         &       9.974         \\
                    &                     &     (6.044)         &     (7.581)         \\
[1em]
Dorm                &                     &      -25.98         &       37.23         \\
                    &                     &     (27.69)         &     (26.26)         \\
[1em]
Entire Floor        &                     &      -60.57         &      -5.236         \\
                    &                     &     (45.63)         &     (23.64)         \\
[1em]
Guesthouse          &                     &      -9.860         &       1.245         \\
                    &                     &     (13.27)         &     (12.80)         \\
[1em]
House               &                     &       10.95\sym{***}&       16.79\sym{***}\\
                    &                     &     (2.580)         &     (3.369)         \\
[1em]
Hut                 &                     &      -30.13\sym{***}&       55.21\sym{***}\\
                    &                     &     (5.443)         &     (10.68)         \\
[1em]
Loft                &                     &       14.35\sym{*}  &       24.53\sym{**} \\
                    &                     &     (5.655)         &     (7.955)         \\
[1em]
Other               &                     &       9.263         &       17.68         \\
                    &                     &     (11.53)         &     (13.42)         \\
[1em]
Tent                &                     &      -34.06         &      -52.58\sym{*}  \\
                    &                     &     (21.30)         &     (25.65)         \\
[1em]
Tipi                &                     &       10.87         &      -53.02\sym{***}\\
                    &                     &     (6.446)         &     (9.624)         \\
[1em]
Townhouse           &                     &      -12.40         &      -11.48         \\
                    &                     &     (6.380)         &     (6.000)         \\
[1em]
Treehouse           &                     &       13.11         &       128.7         \\
                    &                     &     (76.93)         &     (72.88)         \\
[1em]
Villa               &                     &      -2.087         &       22.97         \\
                    &                     &     (20.97)         &     (18.87)         \\
[1em]
bathrooms           &                     &       44.89\sym{***}&       40.66\sym{***}\\
                    &                     &     (3.684)         &     (4.063)         \\
[1em]
beds                &                     &      -2.714\sym{*}  &      -6.795\sym{***}\\
                    &                     &     (1.216)         &     (1.859)         \\
[1em]
Airbed              &                     &           0         &           0         \\
                    &                     &         (.)         &         (.)         \\
[1em]
Couch               &                     &       2.974         &       0.881         \\
                    &                     &     (8.031)         &     (10.05)         \\
[1em]
Futon               &                     &      -2.407         &       1.808         \\
                    &                     &     (6.397)         &     (7.227)         \\
[1em]
Pull-out Sofa       &                     &      -5.818         &       1.301         \\
                    &                     &     (7.235)         &     (7.303)         \\
[1em]
Real Bed            &                     &       3.215         &       3.205         \\
                    &                     &     (5.490)         &     (6.425)         \\
[1em]
cleaning\_fee        &                     &       0.417\sym{***}&       0.305\sym{***}\\
                    &                     &    (0.0405)         &    (0.0404)         \\
[1em]
extra\_people        &                     &       0.160\sym{*}  &     -0.0404         \\
                    &                     &    (0.0690)         &    (0.0593)         \\
[1em]
num\_amenities       &                     &       0.216         &      -0.646         \\
                    &                     &     (0.845)         &     (0.824)         \\
[1em]
first\_review\_month=1&                     &           0         &           0         \\
                    &                     &         (.)         &         (.)         \\
[1em]
first\_review\_month=2&                     &       3.351         &      -1.070         \\
                    &                     &     (4.375)         &     (5.778)         \\
[1em]
first\_review\_month=3&                     &      -0.244         &       5.997         \\
                    &                     &     (4.094)         &     (5.899)         \\
[1em]
first\_review\_month=4&                     &       3.970         &      -3.354         \\
                    &                     &     (4.598)         &     (5.061)         \\
[1em]
first\_review\_month=5&                     &       3.704         &      -2.491         \\
                    &                     &     (3.913)         &     (4.812)         \\
[1em]
first\_review\_month=6&                     &       0.278         &       1.338         \\
                    &                     &     (4.404)         &     (5.036)         \\
[1em]
first\_review\_month=7&                     &       2.916         &       6.193         \\
                    &                     &     (3.995)         &     (4.921)         \\
[1em]
first\_review\_month=8&                     &       1.616         &       0.164         \\
                    &                     &     (4.454)         &     (4.810)         \\
[1em]
first\_review\_month=9&                     &       3.639         &       2.052         \\
                    &                     &     (4.104)         &     (4.900)         \\
[1em]
first\_review\_month=10&                     &       1.762         &      -3.183         \\
                    &                     &     (4.853)         &     (5.481)         \\
[1em]
first\_review\_month=11&                     &       2.089         &      -3.160         \\
                    &                     &     (4.862)         &     (5.918)         \\
[1em]
first\_review\_month=12&                     &      -9.383         &      -0.638         \\
                    &                     &     (4.840)         &     (5.373)         \\
[1em]
first\_review\_month=99&                     &      -1.166         &       79.93\sym{***}\\
                    &                     &     (17.24)         &     (21.62)         \\
[1em]
first\_review\_year=9 &                     &           0         &           0         \\
                    &                     &         (.)         &         (.)         \\
[1em]
first\_review\_year=10&                     &      -17.64         &       38.43         \\
                    &                     &     (16.10)         &     (24.02)         \\
[1em]
first\_review\_year=11&                     &      -21.50         &       48.93\sym{*}  \\
                    &                     &     (16.96)         &     (22.45)         \\
[1em]
first\_review\_year=12&                     &      -24.82         &       46.26\sym{*}  \\
                    &                     &     (14.07)         &     (21.36)         \\
[1em]
first\_review\_year=13&                     &      -19.59         &       50.84\sym{*}  \\
                    &                     &     (14.61)         &     (21.10)         \\
[1em]
first\_review\_year=14&                     &      -17.17         &       62.71\sym{**} \\
                    &                     &     (15.11)         &     (21.15)         \\
[1em]
first\_review\_year=15&                     &      -19.66         &       57.96\sym{**} \\
                    &                     &     (15.28)         &     (21.36)         \\
[1em]
first\_review\_year=16&                     &      -22.14         &       61.31\sym{**} \\
                    &                     &     (15.53)         &     (21.97)         \\
[1em]
first\_review\_year=99&                     &           0         &           0         \\
                    &                     &         (.)         &         (.)         \\
[1em]
group(cancellation\_policy)=1&                     &           0         &           0         \\
                    &                     &         (.)         &         (.)         \\
[1em]
group(cancellation\_policy)=3&                     &      -4.795\sym{*}  &      -8.068\sym{***}\\
                    &                     &     (2.192)         &     (2.371)         \\
[1em]
group(cancellation\_policy)=4&                     &      -33.32\sym{***}&       27.34\sym{**} \\
                    &                     &     (8.695)         &     (9.075)         \\
[1em]
group(cancellation\_policy)=5&                     &      -8.248\sym{***}&      -11.58\sym{***}\\
                    &                     &     (2.202)         &     (2.092)         \\
[1em]
group(cancellation\_policy)=6&                     &      -31.64\sym{***}&      -58.22\sym{***}\\
                    &                     &     (8.615)         &     (11.32)         \\
[1em]
1 if TRUE, 0 if FALSE&                     &      -4.375\sym{*}  &      -0.790         \\
                    &                     &     (2.164)         &     (2.511)         \\
[1em]
1 if TRUE, 0 if FALSE&                     &      -4.092         &      -1.682         \\
                    &                     &     (5.938)         &     (8.450)         \\
[1em]
1 if TRUE, 0 if FALSE&                     &      -2.871         &       4.630         \\
                    &                     &     (4.926)         &     (7.073)         \\
[1em]
minimum\_nights      &                     &      -0.523\sym{**} &      -0.208         \\
                    &                     &     (0.184)         &     (0.127)         \\
[1em]
availability\_30     &                     &       0.115         &       3.833\sym{***}\\
                    &                     &     (0.296)         &     (0.458)         \\
[1em]
availability\_60     &                     &       0.293\sym{***}&       0.507\sym{***}\\
                    &                     &    (0.0796)         &     (0.120)         \\
[1em]
Length of Description&                     &    -0.00224         &    -0.00817         \\
                    &                     &   (0.00364)         &   (0.00432)         \\
[1em]
Quality of review as measured by proportion of long words&                     &      -48.68         &       20.45         \\
                    &                     &     (60.30)         &     (37.22)         \\
[1em]
len\_desc2           &                     &     0.00805         &    -0.00137         \\
                    &                     &   (0.00953)         &   (0.00987)         \\
[1em]
short\_words2        &                     &      -48.51\sym{*}  &      -58.82         \\
                    &                     &     (23.83)         &     (35.57)         \\
[1em]
len\_desc3           &                     &     0.00174         &      0.0100\sym{**} \\
                    &                     &   (0.00321)         &   (0.00327)         \\
[1em]
short\_words3        &                     &      -14.37         &      -8.658         \\
                    &                     &     (16.38)         &     (16.24)         \\
[1em]
len\_desc4           &                     &   -0.000150         &    -0.00158         \\
                    &                     &   (0.00430)         &   (0.00454)         \\
[1em]
short\_words4        &                     &       0.130         &      -0.387         \\
                    &                     &     (0.171)         &     (0.249)         \\
[1em]
len\_desc5           &                     &     0.00142         &     0.00263         \\
                    &                     &   (0.00459)         &   (0.00621)         \\
[1em]
short\_words5        &                     &       0.255         &       0.236         \\
                    &                     &     (0.139)         &     (0.278)         \\
[1em]
len\_desc6           &                     &    -0.00108         &     -0.0147\sym{**} \\
                    &                     &   (0.00492)         &   (0.00529)         \\
[1em]
short\_words6        &                     &      -0.360\sym{*}  &     -0.0449         \\
                    &                     &     (0.147)         &     (0.484)         \\
[1em]
good\_word\_tot       &                     &      -0.197         &      -0.967         \\
                    &                     &     (0.525)         &     (0.551)         \\
[1em]
N/A                 &                     &           0         &           0         \\
                    &                     &         (.)         &         (.)         \\
[1em]
a few days or more  &                     &      -18.87\sym{*}  &      -56.30\sym{***}\\
                    &                     &     (7.977)         &     (9.844)         \\
[1em]
within a day        &                     &      -14.46         &      -65.83\sym{***}\\
                    &                     &     (10.72)         &     (13.63)         \\
[1em]
within a few hours  &                     &      -17.56         &      -68.41\sym{***}\\
                    &                     &     (12.27)         &     (15.89)         \\
[1em]
within an hour      &                     &      -15.25         &      -63.86\sym{***}\\
                    &                     &     (12.30)         &     (16.47)         \\
[1em]
(group\_host\_response\_time>=.)&                     &           0         &           0         \\
                    &                     &         (.)         &         (.)         \\
[1em]
host\_response\_rate  &                     &       5.138         &       11.44         \\
                    &                     &     (10.57)         &     (11.00)         \\
[1em]
1 if TRUE, 0 if FALSE&                     &       10.47\sym{***}&       8.757\sym{**} \\
                    &                     &     (2.351)         &     (2.972)         \\
[1em]
78717               &                     &                     &      -154.8\sym{***}\\
                    &                     &                     &     (4.430)         \\
[1em]
78728               &                     &                     &      -136.8\sym{***}\\
                    &                     &                     &     (8.113)         \\
[1em]
78734               &                     &                     &      -116.2\sym{***}\\
                    &                     &                     &     (5.602)         \\
[1em]
78735               &                     &                     &      -69.23\sym{***}\\
                    &                     &                     &     (4.816)         \\
[1em]
78738               &                     &                     &      -88.91\sym{***}\\
                    &                     &                     &     (5.088)         \\
[1em]
78750               &                     &                     &      -174.5\sym{***}\\
                    &                     &                     &     (6.949)         \\
[1em]
78753               &                     &                     &      -84.40\sym{***}\\
                    &                     &                     &     (5.940)         \\
[1em]
Allerton            &                     &                     &      -58.13\sym{**} \\
                    &                     &                     &     (18.38)         \\
[1em]
Alondra Park        &                     &                     &      -58.53\sym{***}\\
                    &                     &                     &     (9.610)         \\
[1em]
Angeles Crest       &                     &                     &      -119.2\sym{***}\\
                    &                     &                     &     (15.07)         \\
[1em]
Archer Heights      &                     &                     &      -29.06         \\
                    &                     &                     &     (27.61)         \\
[1em]
Arden Heights       &                     &                     &      -90.64\sym{***}\\
                    &                     &                     &     (18.74)         \\
[1em]
Ashburn             &                     &                     &      -147.6\sym{***}\\
                    &                     &                     &     (27.22)         \\
[1em]
Athens              &                     &                     &      -193.8\sym{***}\\
                    &                     &                     &     (15.76)         \\
[1em]
Austin              &                     &                     &      -135.0\sym{***}\\
                    &                     &                     &     (26.83)         \\
[1em]
Avocado Heights     &                     &                     &      -38.96\sym{**} \\
                    &                     &                     &     (12.70)         \\
[1em]
Baldwin Park        &                     &                     &      -36.05\sym{*}  \\
                    &                     &                     &     (14.91)         \\
[1em]
Bath Beach          &                     &                     &      -111.6\sym{***}\\
                    &                     &                     &     (18.28)         \\
[1em]
Bay Terrace         &                     &                     &      -4.910         \\
                    &                     &                     &     (17.02)         \\
[1em]
Bayswater           &                     &                     &      -60.07\sym{**} \\
                    &                     &                     &     (18.97)         \\
[1em]
Bellerose           &                     &                     &      -96.78\sym{***}\\
                    &                     &                     &     (18.54)         \\
[1em]
Belmont             &                     &                     &       31.91         \\
                    &                     &                     &     (16.83)         \\
[1em]
Belmont Cragin      &                     &                     &      -80.49\sym{**} \\
                    &                     &                     &     (27.01)         \\
[1em]
Bergen Beach        &                     &                     &      -98.12\sym{***}\\
                    &                     &                     &     (19.44)         \\
[1em]
Beverly             &                     &                     &      -140.2\sym{***}\\
                    &                     &                     &     (27.79)         \\
[1em]
Bradbury            &                     &                     &      -77.22\sym{***}\\
                    &                     &                     &     (11.32)         \\
[1em]
Broadway-Manchester &                     &                     &      -56.55\sym{***}\\
                    &                     &                     &     (10.56)         \\
[1em]
Bronxdale           &                     &                     &      -53.54\sym{**} \\
                    &                     &                     &     (17.72)         \\
[1em]
Burnside            &                     &                     &      -153.9\sym{***}\\
                    &                     &                     &     (26.68)         \\
[1em]
Calumet Heights     &                     &                     &      -94.51\sym{***}\\
                    &                     &                     &     (26.20)         \\
[1em]
Cambria Heights     &                     &                     &      -104.4\sym{***}\\
                    &                     &                     &     (18.68)         \\
[1em]
Capitol View, Marshall Heights,&                     &                     &      -30.51\sym{**} \\
                    &                     &                     &     (11.76)         \\
[1em]
Carson              &                     &                     &      -44.76\sym{***}\\
                    &                     &                     &     (12.43)         \\
[1em]
Chesterfield Square &                     &                     &       139.6\sym{***}\\
                    &                     &                     &     (15.54)         \\
[1em]
Claremont Village   &                     &                     &      -73.58\sym{***}\\
                    &                     &                     &     (16.84)         \\
[1em]
Co-op City          &                     &                     &      -5.191         \\
                    &                     &                     &     (17.36)         \\
[1em]
Concord             &                     &                     &      -39.37\sym{*}  \\
                    &                     &                     &     (18.68)         \\
[1em]
Congress Heights, Bellevue, Wash&                     &                     &      -9.024         \\
                    &                     &                     &     (10.03)         \\
[1em]
Covina              &                     &                     &       12.14         \\
                    &                     &                     &     (8.489)         \\
[1em]
Deanwood, Burrville, Grant Park,&                     &                     &      -75.39\sym{***}\\
                    &                     &                     &     (11.74)         \\
[1em]
Desire Area         &                     &                     &      -107.4\sym{***}\\
                    &                     &                     &     (12.63)         \\
[1em]
District 10         &                     &                     &      -106.7\sym{***}\\
                    &                     &                     &     (12.25)         \\
[1em]
Dyker Heights       &                     &                     &      -55.11\sym{**} \\
                    &                     &                     &     (17.19)         \\
[1em]
East La Mirada      &                     &                     &      -71.40\sym{***}\\
                    &                     &                     &     (13.21)         \\
[1em]
East Pasadena       &                     &                     &      -46.42\sym{***}\\
                    &                     &                     &     (10.69)         \\
[1em]
Edenwald            &                     &                     &      -117.4\sym{***}\\
                    &                     &                     &     (17.18)         \\
[1em]
El Monte            &                     &                     &      -52.39\sym{***}\\
                    &                     &                     &     (10.99)         \\
[1em]
Emerson Hill        &                     &                     &      -44.26\sym{*}  \\
                    &                     &                     &     (19.07)         \\
[1em]
Fairfax Village, Naylor Gardens,&                     &                     &      -84.69\sym{***}\\
                    &                     &                     &     (11.27)         \\
[1em]
Far Rockaway        &                     &                     &      -30.84         \\
                    &                     &                     &     (18.30)         \\
[1em]
Fieldston           &                     &                     &      -121.3\sym{***}\\
                    &                     &                     &     (20.34)         \\
[1em]
Forest Glen         &                     &                     &      -52.04         \\
                    &                     &                     &     (27.65)         \\
[1em]
Fresh Meadows       &                     &                     &      -44.36\sym{*}  \\
                    &                     &                     &     (18.84)         \\
[1em]
Gentilly Woods      &                     &                     &      -145.4\sym{***}\\
                    &                     &                     &     (14.19)         \\
[1em]
Glen Oaks           &                     &                     &      -33.47         \\
                    &                     &                     &     (19.17)         \\
[1em]
Glendora            &                     &                     &      -60.99\sym{***}\\
                    &                     &                     &     (11.38)         \\
[1em]
Gramercy Park       &                     &                     &      -94.02\sym{***}\\
                    &                     &                     &     (11.50)         \\
[1em]
Graniteville        &                     &                     &      -68.36\sym{***}\\
                    &                     &                     &     (19.58)         \\
[1em]
Greater Grand Crossing&                     &                     &      -74.64\sym{**} \\
                    &                     &                     &     (25.64)         \\
[1em]
Green Valley        &                     &                     &      -166.7\sym{***}\\
                    &                     &                     &     (20.01)         \\
[1em]
Hawaiian Gardens    &                     &                     &      -80.85\sym{***}\\
                    &                     &                     &     (13.08)         \\
[1em]
Highbridge          &                     &                     &      -57.81\sym{**} \\
                    &                     &                     &     (18.95)         \\
[1em]
Hollis              &                     &                     &      -51.95\sym{**} \\
                    &                     &                     &     (18.45)         \\
[1em]
Howland Hook        &                     &                     &      -42.05\sym{*}  \\
                    &                     &                     &     (17.66)         \\
[1em]
Kew Gardens         &                     &                     &      -59.94\sym{***}\\
                    &                     &                     &     (17.81)         \\
[1em]
La Verne            &                     &                     &      -40.72\sym{***}\\
                    &                     &                     &     (9.969)         \\
[1em]
Lake Hughes         &                     &                     &      -29.96\sym{**} \\
                    &                     &                     &     (10.81)         \\
[1em]
Laurelton           &                     &                     &      -39.06\sym{*}  \\
                    &                     &                     &     (18.69)         \\
[1em]
Lennox              &                     &                     &      -59.75\sym{***}\\
                    &                     &                     &     (11.70)         \\
[1em]
Lighthouse Hill     &                     &                     &      -47.81\sym{**} \\
                    &                     &                     &     (18.11)         \\
[1em]
Lower Ninth Ward    &                     &                     &      -25.07         \\
                    &                     &                     &     (14.51)         \\
[1em]
Manchester Square   &                     &                     &      -81.12\sym{***}\\
                    &                     &                     &     (11.24)         \\
[1em]
Marble Hill         &                     &                     &       13.31         \\
                    &                     &                     &     (18.54)         \\
[1em]
Mariners Harbor     &                     &                     &      -20.33         \\
                    &                     &                     &     (17.87)         \\
[1em]
Mayfair, Hillbrook, Mahaning Hei&                     &                     &      -65.90\sym{***}\\
                    &                     &                     &     (10.08)         \\
[1em]
Mayflower Village   &                     &                     &      -82.33\sym{***}\\
                    &                     &                     &     (11.46)         \\
[1em]
Morris Heights      &                     &                     &      -38.92\sym{*}  \\
                    &                     &                     &     (18.31)         \\
[1em]
Morrisania          &                     &                     &      -61.50\sym{***}\\
                    &                     &                     &     (17.78)         \\
[1em]
Mott Haven          &                     &                     &      -33.43         \\
                    &                     &                     &     (17.28)         \\
[1em]
New Springville     &                     &                     &      -66.05\sym{***}\\
                    &                     &                     &     (18.74)         \\
[1em]
North Hills         &                     &                     &      -60.05\sym{***}\\
                    &                     &                     &     (9.472)         \\
[1em]
North Lawndale      &                     &                     &      -99.08\sym{***}\\
                    &                     &                     &     (29.90)         \\
[1em]
North Whittier      &                     &                     &      -33.67\sym{**} \\
                    &                     &                     &     (11.59)         \\
[1em]
Northeast Antelope Valley&                     &                     &      -25.04         \\
                    &                     &                     &     (28.47)         \\
[1em]
Northwest Palmdale  &                     &                     &      -65.47\sym{***}\\
                    &                     &                     &     (12.96)         \\
[1em]
Norwalk             &                     &                     &      -63.99\sym{***}\\
                    &                     &                     &     (9.721)         \\
[1em]
Norwood             &                     &                     &      -15.46         \\
                    &                     &                     &     (18.10)         \\
[1em]
Ohare               &                     &                     &      -69.36\sym{**} \\
                    &                     &                     &     (26.16)         \\
[1em]
Old Aurora          &                     &                     &      -66.42\sym{***}\\
                    &                     &                     &     (14.92)         \\
[1em]
Parkchester         &                     &                     &      -36.41         \\
                    &                     &                     &     (18.58)         \\
[1em]
Pelham Gardens      &                     &                     &      -59.17\sym{***}\\
                    &                     &                     &     (17.60)         \\
[1em]
Pines Village       &                     &                     &      -144.4\sym{***}\\
                    &                     &                     &     (12.90)         \\
[1em]
Plum Orchard        &                     &                     &      -274.5\sym{***}\\
                    &                     &                     &     (17.17)         \\
[1em]
Pontchartrain Park  &                     &                     &      -108.9\sym{***}\\
                    &                     &                     &     (15.65)         \\
[1em]
Port Morris         &                     &                     &      -86.44\sym{***}\\
                    &                     &                     &     (18.11)         \\
[1em]
Quartz Hill         &                     &                     &      -175.1\sym{***}\\
                    &                     &                     &     (14.27)         \\
[1em]
Ramona              &                     &                     &      -15.66         \\
                    &                     &                     &     (10.79)         \\
[1em]
Randall Manor       &                     &                     &      -58.16\sym{**} \\
                    &                     &                     &     (18.89)         \\
[1em]
Rockaway Beach      &                     &                     &      -40.84\sym{*}  \\
                    &                     &                     &     (17.86)         \\
[1em]
Rosedale            &                     &                     &      -14.27         \\
                    &                     &                     &     (18.40)         \\
[1em]
San Dimas           &                     &                     &      -96.29\sym{***}\\
                    &                     &                     &     (11.21)         \\
[1em]
San Fernando        &                     &                     &      -109.8\sym{***}\\
                    &                     &                     &     (10.50)         \\
[1em]
San Marino          &                     &                     &       90.59\sym{***}\\
                    &                     &                     &     (13.70)         \\
[1em]
Sea Gate            &                     &                     &      -66.93\sym{**} \\
                    &                     &                     &     (20.55)         \\
[1em]
Shadow Hills        &                     &                     &      -58.29\sym{***}\\
                    &                     &                     &     (9.564)         \\
[1em]
Sheridan, Barry Farm, Buena Vist&                     &                     &      -134.9\sym{***}\\
                    &                     &                     &     (13.44)         \\
[1em]
Shore Acres         &                     &                     &      -69.94\sym{***}\\
                    &                     &                     &     (19.98)         \\
[1em]
South Chicago       &                     &                     &      -28.90         \\
                    &                     &                     &     (28.61)         \\
[1em]
South Deering       &                     &                     &      -132.2\sym{***}\\
                    &                     &                     &     (28.18)         \\
[1em]
South El Monte      &                     &                     &      -75.53\sym{***}\\
                    &                     &                     &     (22.84)         \\
[1em]
South Whittier      &                     &                     &      -91.48\sym{***}\\
                    &                     &                     &     (11.08)         \\
[1em]
Spring Valley, Palisades, Wesley&                     &                     &       39.19\sym{***}\\
                    &                     &                     &     (10.83)         \\
[1em]
St. Albans          &                     &                     &      -58.19\sym{**} \\
                    &                     &                     &     (18.88)         \\
[1em]
St. Bernard Area    &                     &                     &      -27.47         \\
                    &                     &                     &     (14.31)         \\
[1em]
Sun Village         &                     &                     &      -128.7\sym{***}\\
                    &                     &                     &     (13.88)         \\
[1em]
Throgs Neck         &                     &                     &      -15.65         \\
                    &                     &                     &     (17.97)         \\
[1em]
Tottenville         &                     &                     &      -123.2\sym{***}\\
                    &                     &                     &     (25.23)         \\
[1em]
Tremont             &                     &                     &      -57.09\sym{**} \\
                    &                     &                     &     (19.01)         \\
[1em]
Tujunga Canyons     &                     &                     &      -83.37\sym{***}\\
                    &                     &                     &     (10.52)         \\
[1em]
Unionport           &                     &                     &      -183.2\sym{***}\\
                    &                     &                     &     (24.56)         \\
[1em]
Universal City      &                     &                     &       60.65\sym{***}\\
                    &                     &                     &     (10.97)         \\
[1em]
Valinda             &                     &                     &      -33.11\sym{**} \\
                    &                     &                     &     (11.61)         \\
[1em]
Vermont Knolls      &                     &                     &      -98.81\sym{***}\\
                    &                     &                     &     (11.22)         \\
[1em]
Vermont-Slauson     &                     &                     &      -31.90\sym{***}\\
                    &                     &                     &     (9.037)         \\
[1em]
Veterans Administration&                     &                     &      -73.59\sym{***}\\
                    &                     &                     &     (8.112)         \\
[1em]
Vincent             &                     &                     &      -57.49\sym{***}\\
                    &                     &                     &     (12.39)         \\
[1em]
Watts               &                     &                     &      -42.22\sym{**} \\
                    &                     &                     &     (12.84)         \\
[1em]
West Elsdon         &                     &                     &      -13.37         \\
                    &                     &                     &     (26.41)         \\
[1em]
West Farms          &                     &                     &      -36.27\sym{*}  \\
                    &                     &                     &     (18.04)         \\
[1em]
West Lake Forest    &                     &                     &      -0.771         \\
                    &                     &                     &     (12.25)         \\
[1em]
West Puente Valley  &                     &                     &      -140.7\sym{***}\\
                    &                     &                     &     (15.59)         \\
[1em]
West Whittier-Los Nietos&                     &                     &      -172.9\sym{***}\\
                    &                     &                     &     (16.16)         \\
[1em]
Westerleigh         &                     &                     &      -82.78\sym{***}\\
                    &                     &                     &     (18.43)         \\
[1em]
Westlake Village    &                     &                     &      -34.36\sym{***}\\
                    &                     &                     &     (9.530)         \\
[1em]
Whitestone          &                     &                     &      -36.54\sym{*}  \\
                    &                     &                     &     (17.51)         \\
[1em]
Woodland/Fort Stanton, Garfield&                     &                     &      -52.12\sym{***}\\
                    &                     &                     &     (8.498)         \\
[1em]
Cave                &                     &                     &      -99.53\sym{***}\\
                    &                     &                     &     (7.317)         \\
[1em]
Earth House         &                     &                     &      -42.06\sym{***}\\
                    &                     &                     &     (10.58)         \\
[1em]
Yurt                &                     &                     &       162.1\sym{***}\\
                    &                     &                     &     (12.91)         \\
[1em]
group(cancellation\_policy)=2&                     &                     &      -50.18\sym{***}\\
                    &                     &                     &     (8.035)         \\
[1em]
Constant            &       324.9\sym{***}&       107.4\sym{***}&      -70.24\sym{*}  \\
                    &     (31.26)         &     (22.05)         &     (29.10)         \\
\hline
Location Fixed Effects&         Yes         &         Yes         &         Yes         \\
Property Fixed Effects&         Yes         &         Yes         &         Yes         \\
Host Fixed Effects  &         Yes         &         Yes         &         Yes         \\
\hline \vspace{-1.25em}&                     &                     &                     \\
Observations        &       11999         &        7970         &       11340         \\
Adjusted R2         &       0.527         &       0.720         &       0.663         \\

		
		\bottomrule
	\end{tabular}
	
	\begin{tablenotes}
		\item \footnotesize Standard errors in parentheses
		\item \footnotesize \sym{*} \(p<0.05\), \sym{**} \(p<0.01\), \sym{***} \(p<0.001\)
		
		\item Notes: This table presents the effect on price of controlling for Edelman \& Luca's (2014) full specification using my NYC data. My results are nearly identical to theirs (their coefficient on Black *hosts was -17.8) when controlling for similar covariates in the same city. The omitted category for race is White hosts. The omitted category for room type is Entire Apartment. I could not control for host social media accounts as a proxy for host reliability like Edelman \& Luca did, because Airbnb no longer provides this information. Instead, I controlled for ``host verified", a boolean for whether Airbnb has the host's phone number and email. I was not able to control for ``picture quality" either, but picture quality did not significantly influence price in Edelman \& Luca's regression.
	\end{tablenotes}
\end{table}


%7
\include{tables/table_8_numrev_reg_4_14_FINAL}

\begin{table}[htbp]\centering
	\def\sym#1{\ifmmode^{#1}\else\(^{#1}\)\fi}
	\caption{New title here}
	\label{table:numreviews_new}
	\begin{tabular}{c|ccccc}
		\toprule
		
		                    &\multicolumn{1}{c}{(1)}&\multicolumn{1}{c}{(2)}&\multicolumn{1}{c}{(3)}&\multicolumn{1}{c}{(4)}\\
                    &\multicolumn{1}{c}{Model 1}&\multicolumn{1}{c}{Model 2}&\multicolumn{1}{c}{Model 3}&\multicolumn{1}{c}{Model 4}\\
\hline
White Female        &      -0.804         &      -0.560         &      -1.420\sym{***}&      -1.310\sym{***}\\
                    &     (0.540)         &     (0.496)         &     (0.344)         &     (0.325)         \\
[1em]
Black Male          &      -2.049         &      -1.494         &      -2.034\sym{**} &      -1.146         \\
                    &     (1.137)         &     (0.970)         &     (0.617)         &     (0.592)         \\
[1em]
Black Female        &      -2.153         &      -1.439         &      -2.523\sym{***}&      -2.253\sym{***}\\
                    &     (1.125)         &     (1.004)         &     (0.559)         &     (0.536)         \\
[1em]
Hispanic Male       &      -1.404         &      -0.168         &      -0.183         &     0.00541         \\
                    &     (1.258)         &     (1.178)         &     (0.817)         &     (0.795)         \\
[1em]
Hispanic Female     &      -0.443         &       0.805         &      -0.867         &      -0.326         \\
                    &     (1.119)         &     (1.042)         &     (0.749)         &     (0.713)         \\
[1em]
Asian Male          &      -2.856\sym{**} &      -1.054         &      -1.024         &      -0.941         \\
                    &     (0.913)         &     (0.832)         &     (0.606)         &     (0.562)         \\
[1em]
Asian Female        &      -3.112\sym{**} &      -0.818         &      -1.201\sym{*}  &      -0.931         \\
                    &     (0.952)         &     (0.770)         &     (0.598)         &     (0.559)         \\
\hline
Location Fixed Effects&                     &         Yes         &         Yes         &         Yes         \\
Property Fixed Effects&                     &                     &         Yes         &         Yes         \\
Host Fixed Effects  &                     &                     &                     &         Yes         \\
\hline \vspace{-1.25em}&                     &                     &                     &                     \\
Observations        &       45072         &       45072         &       45072         &       45072         \\
Adjusted R2         &     0.00888         &      0.0630         &       0.408         &       0.455         \\

		
		\bottomrule
	\end{tabular}
	
	\begin{tablenotes}
		\item \footnotesize Standard errors in parentheses
		\item \footnotesize \sym{*} \(p<0.05\), \sym{**} \(p<0.01\), \sym{***} \(p<0.001\)
		
		\item Notes: The dependent variable is the number of reviews of the listing. The omitted category for race is white males, so all coefficients are relative to that group. The unit of observation is an Airbnb listing, so hosts who have multiple listings are treated separately each time. The sample is the sample of listings across 7 US cities. The specification is the same as Table 5. See Data Appendix for a discussion of my covariates.	\end{tablenotes}
\end{table}

%8
{
\def\sym#1{\ifmmode^{#1}\else\(^{#1}\)\fi}
\begin{longtable}{l*{1}{c}}
\caption{Effect of host's race on listing availability out of 30 days} \label{table:availability}\\
\hline\hline\endfirsthead\hline\endhead\hline\endfoot\endlastfoot
                    &\multicolumn{1}{c}{(1)}\\
                    &\multicolumn{1}{c}{Number of vacant days out of 30}\\
\hline
White Female        &      -0.861\sym{***}\\
                    &     (0.114)         \\
[1em]
Black Male          &       2.317\sym{***}\\
                    &     (0.308)         \\
[1em]
Black Female        &       1.785\sym{***}\\
                    &     (0.277)         \\
[1em]
Hispanic Male       &      -0.154         \\
                    &     (0.331)         \\
[1em]
Hispanic Female     &     -0.0906         \\
                    &     (0.341)         \\
[1em]
Asian Male          &      -0.195         \\
                    &     (0.299)         \\
[1em]
Asian Female        &      -1.191\sym{***}\\
                    &     (0.259)         \\
\hline
Controls:        \\
\hspace{3mm} Location  &                           X      \\
\hspace{3mm} Property Characteristics  &   X         \\
\hspace{3mm} Host Characteristics  &         X        \\
\hline
Observations        &       45779         \\
Adjusted \(R^{2}\)  &       0.215         \\
\hline\hline
\multicolumn{2}{l}{\footnotesize Standard errors in parentheses}\\
\multicolumn{2}{l}{\footnotesize \sym{*} \(p<0.05\), \sym{**} \(p<0.01\), \sym{***} \(p<0.001\)}\\
\caption*{Notes: This table presents the effect of host race on listing availability out of 30 days, controlling for my preferred specification throughout. When a listing is booked, this availability metric is updated on the Airbnb website to reflect that booking. Therefore, this measure actually represents the number of days out of the total available days that listings were vacant, relative to white male hosts.}\\
\end{longtable}
}




\begin{table}[htbp]\centering
	\def\sym#1{\ifmmode^{#1}\else\(^{#1}\)\fi}
	\caption{Table 8: Effect of host’s race on listing availability out of 30 days}
	\label{table:availability_30_new}
	\begin{tabular}{c|c}
		\toprule
		
		                    &\multicolumn{1}{c}{(1)}\\
                    &\multicolumn{1}{c}{Number of vacant days out of 30}\\
\hline
White Female        &      -0.889\sym{***}\\
                    &     (0.114)         \\
[1em]
Black Male          &       2.323\sym{***}\\
                    &     (0.306)         \\
[1em]
Black Female        &       1.770\sym{***}\\
                    &     (0.280)         \\
[1em]
Hispanic Male       &      -0.177         \\
                    &     (0.335)         \\
[1em]
Hispanic Female     &      -0.177         \\
                    &     (0.340)         \\
[1em]
Asian Male          &      -0.182         \\
                    &     (0.302)         \\
[1em]
Asian Female        &      -1.177\sym{***}\\
                    &     (0.261)         \\
\hline
Location Fixed Effects&         Yes         \\
Property Fixed Effects&         Yes         \\
Host Fixed Effects  &         Yes         \\
\hline \vspace{-1.25em}&                     \\
Observations        &       45076         \\
Adjusted R2         &       0.226         \\

		
		\bottomrule
	\end{tabular}
	
	\begin{tablenotes}
		\item \footnotesize Standard errors in parentheses
		\item \footnotesize \sym{*} \(p<0.05\), \sym{**} \(p<0.01\), \sym{***} \(p<0.001\)
		\tem This table presents the effect of host race on listing availability out of 30 days, controlling for my preferred specification in Table 5, Model 4. When a listing is booked, this availability metric is updated on the Airbnb website to reflect that booking. Therefore, this measure actually represents the number of days out of the total available days that listings were vacant, relative to a white male host.


repository/


%9
{
\def\sym#1{\ifmmode^{#1}\else\(^{#1}\)\fi}
\begin{longtable}{l*{7}{c}}
\caption{Robustness checks by city}\\
\hline\hline\endfirsthead\hline\endhead\hline\endfoot\endlastfoot
                    &\multicolumn{1}{c}{(1)}&\multicolumn{1}{c}{(2)}&\multicolumn{1}{c}{(3)}&\multicolumn{1}{c}{(4)}&\multicolumn{1}{c}{(5)}&\multicolumn{1}{c}{(6)}&\multicolumn{1}{c}{(7)}\\
                    &\multicolumn{1}{c}{LA}&\multicolumn{1}{c}{NYC}&\multicolumn{1}{c}{Austin}&\multicolumn{1}{c}{Chicago}&\multicolumn{1}{c}{New Orleans}&\multicolumn{1}{c}{DC}&\multicolumn{1}{c}{Nashville}\\
\hline
Black               &      -5.156\sym{*}  &      -3.977\sym{*}  &      -6.284         &      -2.942         &      -18.45\sym{*}  &      -7.426         &      -4.754         \\
                    &     (2.144)         &     (1.692)         &     (10.75)         &     (3.181)         &     (8.203)         &     (4.872)         &     (8.193)         \\
[1em]
Hispanic            &      -5.621\sym{*}  &      -1.246         &       0.877         &      -0.807         &       4.109         &       3.264         &      -38.58\sym{***}\\
                    &     (2.197)         &     (1.937)         &     (5.212)         &     (5.180)         &     (10.77)         &     (4.739)         &     (9.458)         \\
[1em]
Asian               &      -5.585\sym{**} &      -5.975\sym{**} &      -27.66\sym{***}&      -17.64\sym{***}&       3.805         &      -5.880         &       10.50         \\
                    &     (1.785)         &     (2.043)         &     (7.763)         &     (4.353)         &     (13.36)         &     (3.131)         &     (21.29)         \\
\hline
Observations        &       16825         &       14765         &        3636         &        3255         &        2563         &        2285         &        1747         \\
Adjusted \(R^{2}\)  &       0.684         &       0.616         &       0.611         &       0.613         &       0.568         &       0.586         &       0.670         \\
\hline\hline
\multicolumn{8}{l}{\footnotesize Standard errors in parentheses}\\
\multicolumn{8}{l}{\footnotesize \sym{*} \(p<0.05\), \sym{**} \(p<0.01\), \sym{***} \(p<0.001\)}\\
\caption*{Notes: The dependent variable is the price of a listing. This table breaks down the combined effects shown in the last column of Table 5 by city. The omitted category for race is white hosts, so all coefficients are relative to that group. For ease of reading, I did not include the gender of the host. I control for my preferred specification (referred to as Model 4 in Table 5) that includes host demographics, listing location, listing characteristics, and host characteristics. See Section 3.1 for a full discussion of the covariates included. Low number of observations for Black, Hispanic, and Asian hosts contribute to imprecise estimates in cities with less than 5,000 Airbnb hosts (New Orleans and Nashville have less than 100 Hispanic and Asian hosts, DC and Austin have less than 200 Hispanic and Asian hosts).}
\end{longtable}
}




\begin{table}[htbp]\centering
	\def\sym#1{\ifmmode^{#1}\else\(^{#1}\)\fi}
	\caption{New title here}
	\label{table:robustcity_new}
	\begin{tabular}{c|ccccccc}
		\toprule
		
		                    &\multicolumn{1}{c}{(1)}&\multicolumn{1}{c}{(2)}&\multicolumn{1}{c}{(3)}&\multicolumn{1}{c}{(4)}&\multicolumn{1}{c}{(5)}&\multicolumn{1}{c}{(6)}&\multicolumn{1}{c}{(7)}\\
                    &\multicolumn{1}{c}{LA}&\multicolumn{1}{c}{NYC}&\multicolumn{1}{c}{Austin}&\multicolumn{1}{c}{Chicago}&\multicolumn{1}{c}{New Orleans}&\multicolumn{1}{c}{DC}&\multicolumn{1}{c}{Nashville}\\
\hline
Los Angeles         &           0         &                     &                     &                     &                     &                     &                     \\
                    &         (.)         &                     &                     &                     &                     &                     &                     \\
[1em]
New York City       &                     &           0         &                     &                     &                     &                     &                     \\
                    &                     &         (.)         &                     &                     &                     &                     &                     \\
[1em]
Austin              &                     &                     &           0         &                     &                     &                     &                     \\
                    &                     &                     &         (.)         &                     &                     &                     &                     \\
[1em]
Chicago             &                     &                     &                     &           0         &                     &                     &                     \\
                    &                     &                     &                     &         (.)         &                     &                     &                     \\
[1em]
New Orleans         &                     &                     &                     &                     &           0         &                     &                     \\
                    &                     &                     &                     &                     &         (.)         &                     &                     \\
[1em]
Washington DC       &                     &                     &                     &                     &                     &           0         &                     \\
                    &                     &                     &                     &                     &                     &         (.)         &                     \\
[1em]
Nashville           &                     &                     &                     &                     &                     &                     &           0         \\
                    &                     &                     &                     &                     &                     &                     &         (.)         \\
\hline
Observations        &       16825         &       14765         &        3636         &        3255         &        2563         &        2285         &        1747         \\
Adjusted \(R^{2}\)  &       0.684         &       0.617         &       0.611         &       0.613         &       0.570         &       0.586         &       0.672         \\

		
		\bottomrule
	\end{tabular}
	
	\begin{tablenotes}
		\item \footnotesize Standard errors in parentheses
		\item \footnotesize \sym{*} \(p<0.05\), \sym{**} \(p<0.01\), \sym{***} \(p<0.001\)
		
		\item Notes: The effects for the combined data from Table 3 are now broken down across all 7 cities. The cities decrease in number of observations from left to right. Each set of coefficients represents the coefficient on host race, with price as the outcome variable. I control for my preferred specification that includes listing location, listing characteristics, and host characteristics. Low number of observations for Black, Hispanic, and Asian hosts contribute to imprecise estimates in smaller cities. New Orleans and Nashville have less than 100 Hispanic and Asian hosts. DC and Austin have less than 200 Hispanic and Asian hosts. \end{tablenotes}
\end{table}

%10
\begin{landscape}

{
\def\sym#1{\ifmmode^{#1}\else\(^{#1}\)\fi}
\begin{longtable}{l*{9}{c}}
\caption{Robustness checks by listing characteristics}\\
\hline\hline\endfirsthead\hline\endhead\hline\endfoot\endlastfoot
                    &\multicolumn{1}{c}{(1)}&\multicolumn{1}{c}{(2)}&\multicolumn{1}{c}{(3)}&\multicolumn{1}{c}{(4)}&\multicolumn{1}{c}{(5)}&\multicolumn{1}{c}{(6)}&\multicolumn{1}{c}{(7)}&\multicolumn{1}{c}{(8)}&\multicolumn{1}{c}{(9)}\\
                    &\multicolumn{1}{c}{Low \$ LA}&\multicolumn{1}{c}{High \$ LA}&\multicolumn{1}{c}{Low \$ NYC}&\multicolumn{1}{c}{High \$ NYC}&\multicolumn{1}{c}{Older listings}&\multicolumn{1}{c}{Newer listings}&\multicolumn{1}{c}{Apartment}&\multicolumn{1}{c}{Condo}&\multicolumn{1}{c}{House}\\
\hline
Black               &      -2.241         &      -12.05         &       0.499         &      -10.27\sym{**} &      -8.925\sym{***}&      -7.256\sym{***}&      -4.875\sym{***}&      -7.660         &      -11.74\sym{**} \\
                    &     (1.834)         &     (6.999)         &     (1.110)         &     (3.753)         &     (2.156)         &     (1.363)         &     (1.438)         &     (7.996)         &     (3.693)         \\
[1em]
Hispanic            &      -3.345\sym{**} &      -14.91\sym{*}  &      -1.307         &       0.286         &      -3.783         &      -3.089         &      -2.881         &      -8.052         &      -6.157         \\
                    &     (1.140)         &     (6.929)         &     (1.792)         &     (4.032)         &     (3.207)         &     (1.725)         &     (1.528)         &     (9.087)         &     (3.866)         \\
[1em]
Asian               &      -3.019\sym{**} &      -17.77\sym{**} &      -3.749\sym{*}  &      -8.495\sym{*}  &      -6.602\sym{**} &      -6.214\sym{***}&      -6.884\sym{***}&      -18.25\sym{*}  &      -6.895\sym{*}  \\
                    &     (1.077)         &     (6.005)         &     (1.578)         &     (3.487)         &     (2.124)         &     (1.743)         &     (1.501)         &     (7.687)         &     (2.803)         \\
\hline
Observations        &       12357         &        4468         &        8383         &        6382         &        9847         &       25883         &       28410         &        1854         &       13510         \\
Adjusted \(R^{2}\)  &       0.376         &       0.554         &       0.320         &       0.489         &       0.667         &       0.667         &       0.557         &       0.605         &       0.689         \\
\hline\hline
\multicolumn{10}{l}{\footnotesize Standard errors in parentheses}\\
\multicolumn{10}{l}{\footnotesize \sym{*} \(p<0.05\), \sym{**} \(p<0.01\), \sym{***} \(p<0.001\)}\\
\caption*{Notes: I break down my data by price, time on market, and property type. The categories, from left to right, are: listings in Los Angeles and New York whose price is below vs. above the mean predicted price in those cities; listings in the entire data set which have been on the market for more than 2 years vs. less than 2 years; and listings of different property types, including apartments (includes apartments and lofts), condos (includes condos and townhouse), and houses. I do not break up data by high/low prices for the other cities in my data because smaller sample sizes lead to skewed and less informative coefficients in those cities. I control for my preferred specification throughout. The outcome variable is price of the listing.}\\
\end{longtable}
}

\end{landscape}





\begin{table}[htbp]\centering
	\def\sym#1{\ifmmode^{#1}\else\(^{#1}\)\fi}
	\caption{Robustness checks by listing characteristics}
	\label{table:robustlisting_new}
	\begin{tabular}{c|ccccccccc}
		\toprule
		
		\input{code/tables/output/robustness_listing_chars.tex}
		
		\bottomrule
	\end{tabular}

\begin{tablenotes}
	\item \footnotesize Standard errors in parentheses
	\item \footnotesize \sym{*} \(p<0.05\), \sym{**} \(p<0.01\), \sym{***} \(p<0.001\)
\tem Notes: I break down my combined data by price, time on market, and property type. The categories, from left to right, are: listings whose price is below vs. above the mean price of \$147 and whose prices are above \$800, the price originally dropped, listings who have have been on the market for no more than 2 years vs. no more than 8 years, and listings of different property types, including apartments (includes apartments and lofts), condos (includes condos and townhouse), and houses. I control for my preferred specification throughout. The outcome variable is price of the listing.


%11
\small
\begin{landscape}
{
\def\sym#1{\ifmmode^{#1}\else\(^{#1}\)\fi}
\begin{longtable}{l*{8}{c}}
\caption{Estimates of effect of host demographics on review sentiment, by reviewer demographics}\\
\hline\hline\endfirsthead\hline\endhead\hline\endfoot\endlastfoot
                    &\multicolumn{1}{c}{(1)}&\multicolumn{1}{c}{(2)}&\multicolumn{1}{c}{(3)}&\multicolumn{1}{c}{(4)}&\multicolumn{1}{c}{(5)}&\multicolumn{1}{c}{(6)}&\multicolumn{1}{c}{(7)}&\multicolumn{1}{c}{(8)}\\
                    &\multicolumn{1}{c}{White M Rev.}&\multicolumn{1}{c}{White F Rev.}&\multicolumn{1}{c}{Black M Rev.}&\multicolumn{1}{c}{Black F Rev.}&\multicolumn{1}{c}{Hisp. M Rev.}&\multicolumn{1}{c}{Hisp. F Rev.}&\multicolumn{1}{c}{Asian M Rev.}&\multicolumn{1}{c}{Asian F Rev.}\\
\hline
White Female        &     -0.0280         &      0.0871         &       1.034\sym{*}  &      -0.277         &       0.189         &      0.0536         &       0.132         &      0.0961         \\
                    &    (0.0716)         &    (0.0484)         &     (0.454)         &     (0.283)         &     (0.326)         &     (0.202)         &     (0.126)         &     (0.111)         \\
Black Male          &      0.0490         &      0.0513         &       2.389         &     -0.0865         &       1.267         &      -3.605\sym{**} &       0.323         &      -1.272\sym{***}\\
                    &     (0.228)         &     (0.329)         &     (1.409)         &     (1.079)         &     (0.819)         &     (1.136)         &     (0.365)         &     (0.267)         \\
Black Female        &      -0.118         &      0.0165         &      -2.719\sym{**} &     -0.0154         &       0.232         &     -0.0402         &       0.968\sym{***}&      -0.370         \\
                    &     (0.160)         &     (0.103)         &     (0.916)         &     (0.666)         &     (0.548)         &     (0.743)         &     (0.249)         &     (0.205)         \\
Hispanic Male       &     0.00202         &       0.109         &      -0.345         &       0.852         &      -0.459         &      -0.521         &     -0.0522         &      0.0696         \\
                    &    (0.0956)         &     (0.126)         &     (0.905)         &     (0.616)         &     (0.573)         &     (0.906)         &     (0.198)         &     (0.287)         \\
Hispanic Female     &      0.0442         &     -0.0668         &       4.569\sym{*}  &      -0.867         &      -1.141         &       1.364\sym{**} &      0.0841         &      -0.146         \\
                    &     (0.325)         &    (0.0823)         &     (1.819)         &     (2.900)         &     (0.823)         &     (0.485)         &     (0.192)         &     (0.464)         \\
Asian Male          &      -0.270         &      -0.185         &       3.609\sym{***}&       0.373         &     -0.0566         &      -1.121         &       0.741\sym{**} &       0.229         \\
                    &     (0.235)         &     (0.167)         &     (0.546)         &     (0.819)         &     (1.064)         &     (1.815)         &     (0.251)         &     (0.273)         \\
Asian Female        &      -0.163         &      -0.135         &       7.498\sym{***}&       0.946         &      -0.633         &      -0.573         &      0.0775         &      -0.397         \\
                    &     (0.159)         &     (0.127)         &     (1.754)         &     (0.603)         &     (0.958)         &     (0.523)         &     (0.227)         &     (0.310)         \\
\hline
Controls:        \\
\hspace{3mm} Location  &                           X      & X & X & X & X & X &  X & X \\
\hspace{3mm} Property Characteristics  &   X  & X & X & X & X & X &  X & X \\
\hspace{3mm} Host Characteristics  &         X& X & X & X & X & X &  X & X \\
\hline
Observations        &        2690         &        2557         &         124         &         171         &         201         &         145         &         487         &         537         \\
Adjusted \(R^{2}\)  &       0.007         &       0.001         &       0.271         &       0.194         &       0.102         &       0.136         &       0.021         &      -0.006         \\
\hline\hline
\multicolumn{9}{l}{\footnotesize Standard errors in parentheses}\\
\multicolumn{9}{l}{\footnotesize \sym{*} \(p<0.05\), \sym{**} \(p<0.01\), \sym{***} \(p<0.001\)}\\
\caption*{Notes: This table measures the quality of a review that reviewers leave for hosts in Chicago. The demographics of the reviewers are the columns (male is ``M", female is ``F"), and the demographics of the host are the rows. The outcome variable is the sentiment of the review. Each coefficient is the standardized sentiment of a review. Review sentiment measures how positive or negative the review is. Reviews that are numerically positive are of positive sentiment and numerically negative are negative sentiment, relative to the mean sentiment score for each host type. The unit of observation is a single review. The data is a subsample of the Chicago hosts and their reviewers. I control for my preferred specification throughout (referred to as Model 4 in Table 5). See Data Appendix for a full discussion of the covariates included.}
\end{longtable}
}

\end{landscape}

\normalsize


%12
%  8 May 2017 23:05:33
{
\def\sym#1{\ifmmode^{#1}\else\(^{#1}\)\fi}
\begin{longtable}{l*{4}{c}}
\caption{Estimates of effect of host's race and gender on yearly revenue}\\
\hline\hline\endfirsthead\hline\endhead\hline\endfoot\endlastfoot
                    &\multicolumn{1}{c}{(1)}&\multicolumn{1}{c}{(2)}&\multicolumn{1}{c}{(3)}&\multicolumn{1}{c}{(4)}\\
                    &\multicolumn{1}{c}{Revenue}&\multicolumn{1}{c}{Revenue}&\multicolumn{1}{c}{Revenue}&\multicolumn{1}{c}{Revenue}\\
\hline
White Female        &      -199.0\sym{***}&      -156.4\sym{***}&      -151.9\sym{***}&      -144.1\sym{***}\\
                    &     (48.54)         &     (46.80)         &     (39.72)         &     (40.63)         \\
[1em]
Black Male          &      -655.0\sym{***}&      -329.5\sym{***}&      -261.8\sym{***}&      -182.9\sym{**} \\
                    &     (98.27)         &     (96.16)         &     (59.77)         &     (59.93)         \\
[1em]
Black Female        &      -814.7\sym{***}&      -365.0\sym{***}&      -319.5\sym{***}&      -298.3\sym{***}\\
                    &     (96.68)         &     (78.69)         &     (51.94)         &     (46.80)         \\
[1em]
Hispanic Male       &      -209.0         &      -44.58         &      -25.42         &      -6.391         \\
                    &     (112.5)         &     (97.48)         &     (88.24)         &     (83.84)         \\
[1em]
Hispanic Female     &      -280.8\sym{*}  &      -79.78         &      -119.0         &      -68.88         \\
                    &     (140.1)         &     (120.7)         &     (108.2)         &     (104.6)         \\
[1em]
Asian Male          &      -360.5\sym{**} &      -95.77         &      -15.85         &      -22.78         \\
                    &     (129.4)         &     (115.4)         &     (88.42)         &     (84.11)         \\
[1em]
Asian Female        &      -676.6\sym{***}&      -329.2\sym{***}&      -183.6\sym{**} &      -162.6\sym{**} \\
                    &     (98.12)         &     (74.93)         &     (62.31)         &     (60.08)         \\
[1em]
Constant            &      2301.2\sym{***}&      2384.3\sym{***}&      2009.5\sym{***}&       758.0\sym{**} \\
                    &     (109.2)         &     (42.90)         &     (187.8)         &     (235.3)         \\
\hline
Controls:        \\
\hspace{3mm} Location  &                &       X         &       X         &       X         \\
\hspace{3mm} Property Characteristics  &                &                &       X         &       X         \\
\hspace{3mm} Listing Characteristics  &                &                &                &       X         \\
\hline
Observations        &       45072         &       45072         &       45072         &       45072         \\
Adjusted \(R^{2}\)  &       0.006         &       0.082         &       0.350         &       0.401         \\
\hline\hline
\multicolumn{5}{l}{\footnotesize Standard errors in parentheses}\\
\multicolumn{5}{l}{\footnotesize \sym{*} \(p<0.05\), \sym{**} \(p<0.01\), \sym{***} \(p<0.001\)}\\
\caption*{Notes: The dependent variable is a measure of yearly host revenue, as measured by (price * number of reviews per month * 12) for each listing. The omitted category for race is white males, so all coefficients are relative to that group. The unit of observation is an Airbnb listing, so hosts who have multiple listings are treated separately each time. The sample is the sample of listings across 7 US cities. The specification is the same as Table 5. See Data Appendix for a full discussion of the covariates. }
\end{longtable}
}

\begin{comment}
[1em]
Middle-aged         &       18.31         &       98.64\sym{*}  &      -33.37         &      -121.1\sym{***}\\
&     (55.99)         &     (44.60)         &     (36.71)         &     (36.42)         \\
[1em]
Old ($>$ 65)                 &       222.9         &       249.5         &      -73.13         &      -266.0\sym{*}  \\
&     (158.6)         &     (135.3)         &     (113.0)         &     (112.5)         \\
\end{comment}


\begin{table}[htbp]\centering
	\def\sym#1{\ifmmode^{#1}\else\(^{#1}\)\fi}
	\caption{Estimates of effect of host's race and gender on yearly revenue}
	\label{table:revenue_new}
	\begin{tabular}{c|cccc}
		\toprule
		
		                    &\multicolumn{1}{c}{(1)}&\multicolumn{1}{c}{(2)}&\multicolumn{1}{c}{(3)}&\multicolumn{1}{c}{(4)}\\
                    &\multicolumn{1}{c}{Model 1}&\multicolumn{1}{c}{Model 2}&\multicolumn{1}{c}{Model 3}&\multicolumn{1}{c}{Model 4}\\
\hline
White Female        &      -206.2\sym{***}&      -157.8\sym{**} &      -150.4\sym{**} &      -137.8\sym{**} \\
                    &     (51.41)         &     (52.02)         &     (45.74)         &     (47.49)         \\
[1em]
White Two people or Unknown&       408.9\sym{***}&       335.2\sym{***}&       123.1\sym{**} &       10.57         \\
                    &     (76.17)         &     (51.60)         &     (47.52)         &     (46.72)         \\
[1em]
Black Male          &      -710.1\sym{***}&      -334.5\sym{***}&      -232.4\sym{***}&      -138.1\sym{*}  \\
                    &     (101.6)         &     (95.95)         &     (57.99)         &     (58.38)         \\
[1em]
Black Female        &      -815.5\sym{***}&      -345.1\sym{***}&      -238.6\sym{***}&      -195.7\sym{***}\\
                    &     (102.4)         &     (87.33)         &     (54.95)         &     (52.66)         \\
[1em]
Black Two people or Unknown&      -363.9\sym{*}  &       105.7         &      -195.7         &      -224.9\sym{*}  \\
                    &     (161.7)         &     (166.3)         &     (102.8)         &     (105.1)         \\
[1em]
Hispanic Male       &      -209.8         &      -35.64         &       4.983         &       26.26         \\
                    &     (114.4)         &     (101.2)         &     (92.49)         &     (88.10)         \\
[1em]
Hispanic Female     &      -250.5         &      -59.61         &      -126.9         &      -62.61         \\
                    &     (138.1)         &     (119.0)         &     (114.5)         &     (113.7)         \\
[1em]
Hispanic Two people or Unknown&      -307.3         &       124.0         &      -16.35         &      -44.26         \\
                    &     (198.0)         &     (192.5)         &     (158.5)         &     (162.3)         \\
[1em]
Asian Male          &      -377.6\sym{**} &      -71.74         &       7.878         &       12.88         \\
                    &     (128.3)         &     (110.4)         &     (87.06)         &     (83.89)         \\
[1em]
Asian Female        &      -705.9\sym{***}&      -295.9\sym{***}&      -165.8\sym{*}  &      -141.7\sym{*}  \\
                    &     (105.8)         &     (82.04)         &     (66.30)         &     (66.03)         \\
[1em]
Asian Two people or Unknown&      -207.9         &       241.2         &      -8.691         &      -122.2         \\
                    &     (144.1)         &     (128.7)         &     (94.61)         &     (96.72)         \\
[1em]
Multiracial or Unknown Male&      -60.11         &       10.06         &       13.69         &       13.33         \\
                    &     (253.7)         &     (210.9)         &     (180.2)         &     (176.9)         \\
[1em]
Multiracial or Unknown Female&      -246.8         &      -266.6         &      -247.5         &      -198.3         \\
                    &     (239.1)         &     (193.0)         &     (193.8)         &     (177.0)         \\
[1em]
Multiracial or Unknown Two people or Unknown&       194.3         &       254.1\sym{*}  &       53.01         &      -22.06         \\
                    &     (108.8)         &     (108.8)         &     (84.67)         &     (80.58)         \\
\hline
Location Fixed Effects&                     &         Yes         &         Yes         &         Yes         \\
Property Fixed Effects&                     &                     &         Yes         &         Yes         \\
Host Fixed Effects  &                     &                     &                     &         Yes         \\
\hline \vspace{-1.25em}&                     &                     &                     &                     \\
Observations        &       68984         &       68981         &       68981         &       68951         \\
Adjusted R2         &     0.00847         &      0.0863         &       0.330         &       0.374         \\

		
		\bottomrule
	\end{tabular}
	
	\begin{tablenotes}
		\item \footnotesize Standard errors in parentheses
		\item \footnotesize \sym{*} \(p<0.05\), \sym{**} \(p<0.01\), \sym{***} \(p<0.001\)
		\tem Notes: The dependent variable is a constructed measure of yearly host revenue, as measured by (price * number of reviews per month * 12) for each listing. The omitted category for race is white males, so all coefficients are relative to that group. The unit of observation is an Airbnb listing, so hosts who have multiple listings are treated separately each time. The sample is the sample of listings across 7 US cities. The specification is the same as Table 5. See Data Appendix for a full discussion of my covariates.



"$repository/code/tables/output/yearly_revenue.tex"

