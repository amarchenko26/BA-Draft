
% Summary Stats by Host Race: Listing Characteristics
\begin{table}[htbp]
\caption{Summary Statistics By Host Race: Listing Characteristics}
\begin{center}%
\small\begin{tabular}{l c | c | c c c c}
& \multicolumn{1}{c}{Full Data} & \multicolumn{5}{c}{Regression Sample}
\\
 \cmidrule(r){3-7}
\\
 & \multicolumn{1}{c}{Full data} & \multicolumn{1}{c}{All} & White & Black & Hispanic & Asian
\\
\hline\hline\noalign{\smallskip} 
 \textit{\textit{Outcome Variables}} & & & & & & \\ Price & 175.72 & 142.77 & 151.46 & 112.22 & 131.45 & 118.08 \\
 & (294.14) & (116.11) & (121.49) & (89.64) & (106.15) & (94.91) \\
 Number of Reviews & 17.51 & 16.87 & 17.46 & 15.39 & 16.75 & 14.23 \\
 & (31.86) & (31.23) & (32.37) & (27.52) & (29.92) & (26.77) \\
 \textit{Covariates} & & & & & & \\ \hline Property Type & & & & & & \\ \hspace{10bp}Apartments/Lofts    & 0.60 & 0.63 & 0.63 & 0.66 & 0.66 & 0.62 \\ \hspace{10bp}Townhouses/Condos   & 0.04 & 0.04 & 0.04 & 0.04 & 0.04 & 0.06 \\ \hspace{10bp}Houses                      & 0.32 & 0.30 & 0.30 & 0.27 & 0.26 & 0.30 \\ \hspace{10bp}Others                              & 0.04 & 0.03 & 0.03 & 0.03 & 0.03 & 0.03 \\Room Type &&&&&& \\ \hspace{10bp}Entire House/Apartment      & 0.58 & 0.55 & 0.58 & 0.43 & 0.51 & 0.42 \\ \hspace{10bp}Private Room                        & 0.38 & 0.42 & 0.39 & 0.50 & 0.44 & 0.53 \\ \hspace{10bp}Shared Room                         & 0.04 & 0.04 & 0.03 & 0.07 & 0.05 & 0.05 \\ Max Num. Guests & 3.44 & 3.15 & 3.24 & 2.94 & 3.06 & 2.84 \\
 & (2.41) & (2.13) & (2.15) & (2.06) & (2.15) & (2.00) \\
 Bedrooms & 1.34 & 1.26 & 1.28 & 1.19 & 1.21 & 1.18 \\
 & (0.92) & (0.80) & (0.83) & (0.69) & (0.78) & (0.72) \\
 Bathrooms & 1.30 & 1.23 & 1.24 & 1.19 & 1.21 & 1.19 \\
 & (0.69) & (0.55) & (0.56) & (0.49) & (0.52) & (0.53) \\
 Beds & 1.82 & 1.67 & 1.69 & 1.60 & 1.68 & 1.57 \\
 & (1.41) & (1.21) & (1.19) & (1.15) & (1.51) & (1.18) \\
 Cleaning Fee & 48.95 & 43.70 & 46.06 & 36.20 & 40.35 & 36.45 \\
 & (59.62) & (48.32) & (49.73) & (43.18) & (45.51) & (42.86) \\
 Extra Guests Charge & 13.74 & 13.43 & 13.26 & 15.13 & 13.94 & 12.72 \\
 & (23.65) & (22.67) & (23.00) & (22.71) & (22.48) & (20.36) \\
 Minimum Nights & 3.01 & 3.03 & 3.08 & 2.61 & 2.86 & 3.17 \\
 & (9.21) & (8.79) & (9.39) & (4.35) & (6.55) & (8.67) \\
 Availability (out of 30 days) & 11.54 & 11.04 & 10.64 & 14.19 & 11.24 & 10.79 \\
 & (10.93) & (10.91) & (10.75) & (11.49) & (10.94) & (11.01) \\
 Number of Amenities & 0.81 & 0.79 & 0.80 & 0.75 & 0.79 & 0.75 \\
 & (1.10) & (1.10) & (1.10) & (1.04) & (1.10) & (1.13) \\
 Instantly Bookable? & 0.15 & 0.15 & 0.14 & 0.21 & 0.17 & 0.16 \\
 & (0.36) & (0.36) & (0.34) & (0.41) & (0.38) & (0.37) \\
 Year of first review & 14.86 & 14.86 & 14.83 & 14.89 & 14.90 & 15.03 \\
 & (1.22) & (1.22) & (1.22) & (1.30) & (1.21) & (1.17) \\
 Strict Cancellation Policy & 0.43 & 0.40 & 0.40 & 0.41 & 0.41 & 0.40 \\\hline
Observations & \numprint{69007} & \numprint{45076} & \numprint{26391} & \numprint{3346} & \numprint{2274} & \numprint{3719}
\\
\hline\hline\noalign{\smallskip} \end{tabular} 
\begin{minipage}{6in}
{\it Note:} The values in the table are means and standard deviations of listing-level data in my full sample. Summary statistics for selected covariates are listed in the table. Categorical variables such as room type do not have standard deviations. Property types are explicitly listed if more than 1.5\% of listings are that type. Only the most popular cancellation policy type is listed - in the full sample, 99\% of listings have strict (43\%), flexible (31\%) or moderate (25\%) cancellation policies. Year of first review is a proxy for the time on the market - 14.86 indicates that the first review of the mean listing in the full sample occurred in October of 2014.
\end{minipage}
\end{center}
\end{table}

\newpage

% Summary Stats by Host Race: Host Demographics
\begin{table}[htbp]
\caption{Summary Statistics By Host Race: Host Demographics}
\begin{center}%
\small\begin{tabular}{l c | c | c c c c}
& \multicolumn{1}{c}{} & \multicolumn{5}{c}{Regression Sample}
\\
 \cmidrule(r){3-7}
\\
 & \multicolumn{1}{c}{Full data} & \multicolumn{1}{c}{All} & White & Black & Hispanic & Asian
\\
\hline\hline\noalign{\smallskip} 
 Race &&&&&& \\
 \hspace{10bp}White & 0.735 & 0.731 &     1 &    0 &     0 &    0 \\  \hspace{10bp}Black & 0.096 & 0.097 &     0 &    1 &     0 &    0 \\  \hspace{10bp}Hispanic & 0.063 & 0.065 &     0 &    0 &     1 &    0 \\  \hspace{10bp}Asian & 0.106 & 0.108 &     0 &    0 &     0 &    1 \\  \hspace{10bp}Unknown & 0.000 & {0.000} & {0} &  {0}  & {0}  & {0} \\  Sex &&&&&& \\
 \hspace{10bp}Male & 0.449 & 0.448 &  0.451 & 0.403 &  0.498 & 0.436 \\  \hspace{10bp}Female & 0.550 & 0.552 &  0.549 & 0.597 &  0.502 & 0.564 \\  \hspace{10bp}Unknown & 0.000 & {0} & {0} &  {0}  & {0}  & {0}\\  Age &&&&&& \\
 \hspace{10bp}Young($\<30$) & 0.507 & 0.513 &  0.494 & 0.542 &  0.518 & 0.612 \\  \hspace{10bp}Middle-aged & 0.469 & 0.464 &  0.478 & 0.449 &  0.471 & 0.378 \\  \hspace{10bp}Old ($/>65$) &              0.020 & 0.021 &  0.026 & 0.005 &  0.010 & 0.008 \\  \hspace{10bp}Unknown & 0.000 & {0} & {0} &  {0}  & {0}  & {0} \\ \hline
Observations & 46930 & 45076 & 32934 & 4354 & 2913 & 4875 
\\
\hline\hline\noalign{\smallskip} \end{tabular} 
\begin{minipage}{6in}
{\it Note:} The values in the table are summaries of host demographics inthe host-level data. Column 1 is the summary statistics for the full,unrestricted data set across 7 cities. Columns 2 $-$ 6 are the restricteddata used in the analysis. Column 2 is the full regression sample, andcolumns 3 $-$ 6 break down the regression sample by host race. The“Unknown” category was dropped from the regression and is therefore zerothroughout columns 2 $-$ 6. White refers only to Non-Hispanic Whites.\end{minipage}
\end{center}
\end{table}

\newpage

% Summary Stats by Host Race: Host Characteristics
\begin{table}[htbp]
\caption{Summary Statistics By Host Race: Host Characteristics}
\begin{center}%
\small\begin{tabular}{l c | c | c c c c}
& \multicolumn{1}{c}{} & \multicolumn{5}{c}{Regression Sample}
\\
 \cmidrule(r){3-7}
\\
 & \multicolumn{1}{c}{Full data} & \multicolumn{1}{c}{All} & White & Black & Hispanic & Asian
\\
\hline\hline\noalign{\smallskip} 
 \textit{\textit{Outcome Variables}} & & & & & & \\ Host Listings Count & 6.38 & 2.23 & 2.16 & 2.38 & 2.49 & 2.44 \\
 & (36.54) & (2.59) & (2.50) & (2.83) & (3.03) & (2.61) \\
 \textit{Covariates} & & & & & & \\ \hline Review scores rating & 93.56 & 93.68 & 94.18 & 91.91 & 92.80 & 92.26 \\
 & (8.13) & (7.90) & (7.33) & (9.44) & (8.71) & (9.27) \\
 Host is a Superhost & 0.13 & 0.13 & 0.14 & 0.09 & 0.11 & 0.10 \\
 & (0.34) & (0.33) & (0.34) & (0.28) & (0.31) & (0.30) \\
 Response rate & 0.77 & 0.76 & 0.76 & 0.78 & 0.76 & 0.74 \\
 & (0.38) & (0.39) & (0.39) & (0.37) & (0.39) & (0.40) \\
 Acceptance rate & 0.47 & 0.45 & 0.46 & 0.35 & 0.49 & 0.44 \\
 & (0.46) & (0.46) & (0.46) & (0.45) & (0.47) & (0.47) \\
 Polarity of Summary & 0.30 & 0.30 & 0.30 & 0.29 & 0.30 & 0.29 \\
 & (0.17) & (0.17) & (0.17) & (0.16) & (0.17) & (0.17) \\
 Subjectivity of Summary & 0.53 & 0.54 & 0.54 & 0.53 & 0.54 & 0.53 \\
 & (0.15) & (0.15) & (0.15) & (0.15) & (0.15) & (0.15) \\
 Host's Identity Verified? & 0.70 & 0.70 & 0.71 & 0.66 & 0.68 & 0.69 \\
 & (0.46) & (0.46) & (0.45) & (0.47) & (0.47) & (0.46) \\
 Guest Pic Required? & 0.04 & 0.04 & 0.04 & 0.06 & 0.04 & 0.04 \\
 & (0.19) & (0.19) & (0.19) & (0.23) & (0.19) & (0.19) \\
 Guest Phone Required? & 0.05 & 0.05 & 0.05 & 0.06 & 0.04 & 0.04 \\
 & (0.22) & (0.21) & (0.21) & (0.24) & (0.20) & (0.20) \\
 Response time $<$ 1 hour & 0.41 & 0.40 & 0.39 & 0.44 & 0.41 & 0.41 \\\hline
Observations & \numprint{69010} & \numprint{45076} & \numprint{32934} & \numprint{4354} & \numprint{2913} & \numprint{4875}
\\
\hline\hline\noalign{\smallskip} \end{tabular} 
\begin{minipage}{6in}
\label{table:host_summary}
{\it Note:} The values in the table are means and standard deviations of host-level data in the full sample. Summary statistics for selected covariates are listed in the table. Categorical variables such as response time do not have standard deviations. Statistics for only the most frequent response time (\say{within an hour}) are included. White refers only to non-Hispanic whites. Polarity of \say{Summary} and Subjectivity of \say{Summary} refer to the scores from a natural language processing algorithm that measures the sentiment and objectivity of that field. These two measures were also calculated for the description, space, neighborhood overview, notes, and transit fields, but were not included in the table for the sake of clarity and because they follow a similar pattern as the \say{Summary} field.
\end{minipage}
\end{center}
\end{table}

\newpage

% Summary Stats by Race: Reviewer Characteristics
\begin{table}[htbp]
\caption{Summary Statistics By Race: Reviewer Characteristics}
\begin{center}%
\small\begin{tabular}{l c | c | c c c c}
& \multicolumn{1}{c}{} & \multicolumn{5}{c}{Reviewer Race in \say{All} data} 
\\
 \cmidrule(r){3-7}
\\
 & \multicolumn{1}{c}{Full data} & \multicolumn{1}{c}{All} & White & Black & Hispanic & Asian
\\
\hline\hline\noalign{\smallskip} 
 Reviewer Race  & 1.00 & 1.00 & 0.66 & 0.03 & 0.04 & 0.11 \\\\
 Host race & & & & & & \\ \hspace{10bp}White &     0.73 & 0.83 & 0.84 & 0.70 & 0.75 & 0.75 \\ \hspace{10bp}Black &     0.06 & 0.06& 0.05 & 0.17 & 0.07 & 0.06 \\ \hspace{10bp}Hispanic &  0.04 & 0.05& 0.05 & 0.06 & 0.10 & 0.08 \\ \hspace{10bp}Asian &     0.05 & 0.05& 0.05 & 0.08 & 0.08 & 0.11 \\ \hspace{10bp}Unknown &   0.12 & 0.00& 0.00 & 0.00 & 0.00 & 0.00 \\\\
 Review Sentiment & 0.51 & 0.51 & 0.51 & 0.50 & 0.47 & 0.53 \\
 & (0.26) & (0.26) & (0.25) & (0.23) & (0.30) & (0.25) \\
\\
 Listing Sentiment & 0.51 & 0.51 & 0.51 & 0.50 & 0.50 & 0.51 \\
 & (0.07) & (0.07) & (0.07) & (0.07) & (0.07) & (0.09) \\
\\
\hline
Observations & \numprint{17050} &  \numprint{10573} & \numprint{6929} & \numprint{319} & \numprint{402} & \numprint{1153}
\\
\hline\hline\noalign{\smallskip} \end{tabular} 
\begin{minipage}{6in}
{Note:} This table demonstratates the summary statistics for data used in the \say{Estimates of effect of host demographics on review sentiment, by reviewer demographics} table. Column 1 contains statistics on the raw data. Column 2 contains statistics on the data used in the estimations. Columns 3 - 6 section Column 2 by reviewer race. Row 1: Reviewer race, indicates the proportion of the different races in the reviewer data. Row 2: Host race, indicates the marginal probability of a host race given reviewer race. The values in the table are means and standard deviations of reviewer-level data who left reviews for a randomly chosen set of hosts in Chicago. The review sentiment is the sentiment of each review, the listing sentiment is the average sentiment per listing. Observations in columns 2 - 5 do not add up to 17,050 because multiracial or unidentifiable reviewer pictures are excluded. White refers only to non-Hispanic whites.
\end{minipage}
\end{center}
\end{table}

\newpage

% Price
\begin{table}[htbp]\centering
	\def\sym#1{\ifmmode^{#1}\else\(^{#1}\)\fi}
	\caption{Main result: Estimates of effect of host race and gender on listing price}
	\begin{tabular}{l*{5}{c}}
		\hline\hline
		                    &\multicolumn{1}{c}{(1)}&\multicolumn{1}{c}{(2)}&\multicolumn{1}{c}{(3)}&\multicolumn{1}{c}{(4)}\\
                    &\multicolumn{1}{c}{Model 1}&\multicolumn{1}{c}{Model 2}&\multicolumn{1}{c}{Model 3}&\multicolumn{1}{c}{Model 4}\\
\hline
White Female        &     -0.0236\sym{*}  &     -0.0138         &     0.00201         &     0.00298         \\
                    &    (0.0106)         &   (0.00854)         &   (0.00496)         &   (0.00484)         \\
[1em]
Black Male          &      -0.276\sym{***}&     -0.0828\sym{**} &     -0.0360\sym{**} &     -0.0328\sym{**} \\
                    &    (0.0315)         &    (0.0259)         &    (0.0123)         &    (0.0123)         \\
[1em]
Black Female        &      -0.299\sym{***}&     -0.0586\sym{**} &     -0.0196         &     -0.0167         \\
                    &    (0.0296)         &    (0.0188)         &    (0.0102)         &   (0.00996)         \\
[1em]
Hispanic Male       &      -0.153\sym{***}&     -0.0521\sym{**} &     -0.0233\sym{*}  &     -0.0200         \\
                    &    (0.0259)         &    (0.0191)         &    (0.0113)         &    (0.0113)         \\
[1em]
Hispanic Female     &      -0.150\sym{***}&     -0.0653\sym{**} &     -0.0196         &     -0.0202         \\
                    &    (0.0280)         &    (0.0202)         &    (0.0115)         &    (0.0114)         \\
[1em]
Asian Male          &      -0.221\sym{***}&     -0.0987\sym{***}&     -0.0425\sym{**} &     -0.0446\sym{***}\\
                    &    (0.0336)         &    (0.0225)         &    (0.0134)         &    (0.0135)         \\
[1em]
Asian Female        &      -0.283\sym{***}&      -0.131\sym{***}&     -0.0409\sym{***}&     -0.0396\sym{***}\\
                    &    (0.0299)         &    (0.0161)         &   (0.00874)         &   (0.00893)         \\
[1em]
Constant            &       4.802\sym{***}&       4.979\sym{***}&       3.891\sym{***}&       4.003\sym{***}\\
                    &    (0.0300)         &     (0.398)         &     (0.343)         &     (0.344)         \\
\hline
Location Controls   &                     &         Yes         &         Yes         &         Yes         \\
Property Controls   &                     &                     &         Yes         &         Yes         \\
Host Controls       &                     &                     &                     &         Yes         \\
\hline \vspace{-1.25em}&                     &                     &                     &                     \\
Observations        &       45073         &       45073         &       45073         &       45073         \\
Adjusted R2         &      0.0263         &       0.246         &       0.716         &       0.720         \\

		\hline\hline
		\multicolumn{5}{l}{\footnotesize Standard errors in parentheses}\\
		\multicolumn{5}{l}{\footnotesize \sym{*} \(p<0.05\), \sym{**} \(p<0.01\), \sym{***} \(p<0.001\)}\\
	\end{tabular}	
\label{table:price}
	\begin{tablenotes}
		
		\item {\it Note:} This table presents the impact of host race on the price of a listing. The dependent variable is the log price. The omitted category is White males. The unit of observation is a listing. The sample is listings across seven US cities, whose prices are no more than \$800 per night, and whose hosts own no more than twenty properties. Model 1 is the baseline effect of host demographics on price. Model 2 includes fixed effects for the neighborhood of the listing and Census demographic, economic health characteristics, and occupancy rates on the zipcode-level. Model 3 adds listing characteristics such as the property type and size. Model 4 adds host characteristics such as response and acceptance rates, and measures of host effort.  
		
	\end{tablenotes}
\end{table}



% ALL MEASURES of QD
\begin{table}[htbp]\centering
	\def\sym#1{\ifmmode^{#1}\else\(^{#1}\)\fi}
	\caption{Effect of host race on two proxies of a listing's number of bookings}
	\begin{tabular}{l*{2}{c}}
		\hline\hline
		\input{code/tables/tex_output/individual_tables/quantity_demanded}
		\hline\hline
		\multicolumn{2}{l}{\footnotesize Standard errors in parentheses}\\
		\multicolumn{2}{l}{\footnotesize \sym{*} \(p<0.05\), \sym{**} \(p<0.01\), \sym{***} \(p<0.001\)}\\
	\end{tabular}
\label{table:quantity_demanded}
	\begin{tablenotes}
		
		\item {\it Note:} This table presents the effect of host race on two proxies for the quantity demanded of a listing: its number of reviews and its availability out of 30 days. The availability metric represents the number of days out of the total days available for booking that a listing is vacant. The omitted category is White males. I control for the specification in Table \ref{table:price}, Model 4.
		
	\end{tablenotes}
\end{table}



% Host demographics on review sentiment, by reviewer demographics
\begin{landscape}
	\begin{table}[htbp]\centering
		\def\sym#1{\ifmmode^{#1}\else\(^{#1}\)\fi}
		\caption{Estimates of effect of host demographics on review sentiment, by reviewer demographics}
		\begin{tabular}{l *{9}{c}}
			\hline\hline
			&\multicolumn{9}{c}{Reviewers} \\
			\cmidrule(r){3-10}\\
			                    &\multicolumn{1}{c}{(1)}&\multicolumn{1}{c}{(2)}&\multicolumn{1}{c}{(3)}&\multicolumn{1}{c}{(4)}&\multicolumn{1}{c}{(5)}&\multicolumn{1}{c}{(6)}&\multicolumn{1}{c}{(7)}&\multicolumn{1}{c}{(8)}&\multicolumn{1}{c}{(9)}\\
                    &\multicolumn{1}{c}{Full sample}&\multicolumn{1}{c}{White M}&\multicolumn{1}{c}{White F}&\multicolumn{1}{c}{Black M}&\multicolumn{1}{c}{Black F}&\multicolumn{1}{c}{Hispanic M}&\multicolumn{1}{c}{Hispanic F}&\multicolumn{1}{c}{Asian M}&\multicolumn{1}{c}{Asian F}\\
\hline
White Female        &    -0.00268         &     -0.0246         &      0.0640         &     -0.0718         &       2.215\sym{***}&      -1.032\sym{***}&      -0.898\sym{***}&       4.660\sym{***}&      -1.823\sym{***}\\
                    &    (0.0468)         &    (0.0758)         &    (0.0575)         &    (0.0902)         &  (2.03e-11)         &  (2.01e-10)         &  (2.73e-13)         &     (0.416)         &  (2.77e-08)         \\
Black Male          &      -0.172\sym{**} &      -0.176         &      -0.155         &      -0.168         &      -15.79\sym{***}&       22.87\sym{***}&       0.186\sym{***}&      -4.616         &      -18.00\sym{***}\\
                    &    (0.0621)         &     (0.192)         &     (0.312)         &     (0.459)         &  (1.20e-10)         &  (1.80e-09)         &  (5.36e-13)         &     (3.232)         &(0.000000824)         \\
Black Female        &       0.104         &     -0.0509         &      0.0863         &       0.144         &      -2.253\sym{***}&       6.886\sym{***}&       0.345\sym{***}&      -4.269\sym{***}&       3.430\sym{***}\\
                    &    (0.0685)         &     (0.185)         &     (0.126)         &     (0.195)         &  (3.76e-11)         &  (4.07e-10)         &  (1.60e-12)         &     (0.883)         &(0.000000275)         \\
Hispanic Male       &      -0.115\sym{**} &     -0.0477         &     -0.0130         &      -0.436\sym{***}&       6.431\sym{***}&       19.07\sym{***}&      -0.512\sym{***}&      -7.871\sym{***}&       6.453\sym{***}\\
                    &    (0.0411)         &    (0.0967)         &     (0.108)         &     (0.113)         &  (6.66e-11)         &  (1.11e-09)         &  (8.00e-13)         &     (1.597)         &(0.000000210)         \\
Hispanic Female     &      0.0711         &     0.00719         &      0.0117         &      0.0858         &      -37.98\sym{***}&       85.18\sym{***}&      -2.929\sym{***}&       6.073\sym{***}&      -4.928\sym{***}\\
                    &    (0.0951)         &     (0.401)         &     (0.139)         &     (0.187)         &  (2.99e-10)         &  (4.27e-09)         &  (8.01e-13)         &     (0.888)         &(0.000000109)         \\
Asian Male          &      0.0219         &      -0.281         &      -0.168         &       0.182         &       6.200\sym{***}&      -21.97\sym{***}&       0.792\sym{***}&       8.107\sym{***}&       11.59\sym{***}\\
                    &     (0.162)         &     (0.231)         &     (0.141)         &     (0.271)         &  (1.05e-10)         &  (1.28e-09)         &  (8.01e-13)         &     (0.736)         &(0.000000331)         \\
Asian Female        &      -0.147         &      -0.224         &      -0.321         &     -0.0774         &      -8.758\sym{***}&      -11.16\sym{***}&      -0.993\sym{***}&       7.884\sym{***}&      -2.325\sym{***}\\
                    &    (0.0882)         &     (0.201)         &     (0.162)         &     (0.307)         &  (6.98e-11)         &  (7.24e-10)         &  (1.60e-12)         &     (0.797)         &  (8.11e-08)         \\
\hline
Location Controls   &         Yes         &         Yes         &         Yes         &         Yes         &         Yes         &         Yes         &         Yes         &         Yes         &         Yes         \\
Property Controls   &         Yes         &         Yes         &         Yes         &         Yes         &         Yes         &         Yes         &         Yes         &         Yes         &         Yes         \\
Host Controls       &         Yes         &         Yes         &         Yes         &         Yes         &         Yes         &         Yes         &         Yes         &         Yes         &         Yes         \\
\hline \vspace{-1.25em}&                     &                     &                     &                     &                     &                     &                     &                     &                     \\
Observations        &       10573         &        2665         &        2527         &        1737         &         121         &         171         &          27         &         198         &         142         \\
Adjusted R2         &      0.0238         &      0.0548         &      0.0525         &      0.0922         &       0.838         &       0.719         &       0.970         &       0.642         &       0.786         \\
	
			\hline\hline
			\multicolumn{10}{l}{\footnotesize Standard errors in parentheses}\\
			\multicolumn{10}{l}{\footnotesize \sym{*} \(p<0.05\), \sym{**} \(p<0.01\), \sym{***} \(p<0.001\)}\\
		\end{tabular}
		\label{table:sentiment}
		
		\begin{tablenotes}
			
			\item {\it Note:} This table presents the quality of reviews that reviewers leave for hosts in Chicago. The columns are the demographics of the reviewers (male is \say{M}, female is \say{F}), and the rows are the demographics of the host, consistent with previous tables. The outcome variable is the standardized sentiment of the review, as assigned by a machine learning algorithm. Reviews that are numerically positive are of positive sentiment and numerically negative are negative sentiment, relative to the mean sentiment score for each host type. The unit of observation is a single review. The data is a subsample of the Chicago hosts and their reviewers. I control for the specification in Table \ref{table:price}, Model 4.
			
		\end{tablenotes}
		
	\end{table}
\end{landscape}


% Robustness city
\begin{table}[htbp]\centering
	\def\sym#1{\ifmmode^{#1}\else\(^{#1}\)\fi}
	\caption{Main results by city}
	\begin{tabular}{l*{7}{c}}
		\hline\hline
		\input{code/tables/tex_output/individual_tables/price_by_city}
		\hline\hline
		\multicolumn{8}{l}{\footnotesize Standard errors in parentheses}\\
		\multicolumn{8}{l}{\footnotesize \sym{*} \(p<0.05\), \sym{**} \(p<0.01\), \sym{***} \(p<0.001\)}\\
	\end{tabular}
	\label{table:robustcity}
	\begin{tablenotes}
		
		\item {\it Note:} This table estimates the main results in Table \ref{table:price} separately across the 7 cities in the sample. Each set of coefficients represents the coefficient on host race in a regression with log price as the outcome variable. Low number of observations for Black, Hispanic, and Asian hosts contribute to imprecise estimates in smaller cities (New Orleans, Nashville have less than 100 Hispanic and Asian hosts; DC and Austin have less than 200 such hosts). The omitted category is White males. I control for the specification in Table \ref{table:price}, Model 4.
		
	\end{tablenotes}
\end{table}


\begin{landscape}
	% Robustness Listing Characteristics 
	\begin{table}[htbp]\centering
		\def\sym#1{\ifmmode^{#1}\else\(^{#1}\)\fi}
		\caption{Main results by listing characteristics}
		\begin{tabular}{l*{8}{c}}
			\hline\hline
			\input{code/tables/tex_output/individual_tables/price_by_listing_type}
			\hline\hline
			\multicolumn{9}{l}{\footnotesize Standard errors in parentheses}\\
			\multicolumn{9}{l}{\footnotesize \sym{*} \(p<0.05\), \sym{**} \(p<0.01\), \sym{***} \(p<0.001\)}\\
		\end{tabular}
		\label{table:robustlisting}
		\begin{tablenotes}
			
			\item {\it Note:} This table estimates the main results in Table \ref{table:price} separately for listings of different price points, review numbers, age, and property types. The categories, from left to right, are: listings whose price is above the cutoff price in the original sample, listings of all prices, listings with more than 5 reviews, listings who have have been on the market for no more than 2 years versus no more than 8 years, and listings of different property types, including apartments (includes apartments and lofts), condos (includes condos and townhouse), and houses. The omitted category is White males. I control for the specification in Table \ref{table:price}, Model 4. The outcome variable is the log price of the listing.
			
		\end{tablenotes}
	\end{table}
\end{landscape}


%% APPENDIX REALLY 

% Edelman & Luca
\begin{table}[htbp]\centering
	\def\sym#1{\ifmmode^{#1}\else\(^{#1}\)\fi}
	
	\caption{Robustness check with controls from Edelman \& Luca (2014)}
	\begin{tabular}{l*{1}{c}}
		\hline\hline
		                    &\multicolumn{1}{c}{(1)}\\
                    &\multicolumn{1}{c}{Edelman}\\
\hline
0                   &           0         \\
                    &         (.)         \\
[1em]
Black               &      -14.68         \\
                    &     (73.64)         \\
[1em]
Hispanic            &      -8.702         \\
                    &     (73.73)         \\
[1em]
Asian               &       5.019         \\
                    &     (73.64)         \\
[1em]
Multiracial or Unknown&       6.590         \\
                    &     (73.62)         \\
\hline
Location Fixed Effects&         Yes         \\
Property Fixed Effects&         Yes         \\
Host Fixed Effects  &         Yes         \\
\hline \vspace{-1.25em}&                     \\
Observations        &       16592         \\
Adjusted R2         &       0.238         \\
 
		\hline\hline
		\multicolumn{2}{l}{\footnotesize Standard errors in parentheses}\\
		\multicolumn{2}{l}{\footnotesize \sym{*} \(p<0.05\), \sym{**} \(p<0.01\), \sym{***} \(p<0.001\)}\\
	\end{tabular}
	\label{table:edelman}
	\begin{tablenotes}
		
		\item {\it Note:} This table presents the effect on log price of controlling for \cite{edelman}'s full specification using my NYC data. The omitted category for race is White hosts. The omitted category for room type is Entire Apartment. I could not control for host social media accounts as a proxy for host reliability like \cite{edelman} did, because Airbnb no longer provides this information. Instead, I controll for ``host verified", a dummy for whether Airbnb has the host's phone number and email. I similarly can not control for ``picture quality", but picture quality did not significantly influence price in \cite{edelman}'s regression.
		
	\end{tablenotes}
\end{table}


% Robustness host listings count
\begin{table}[htbp]\centering
	\def\sym#1{\ifmmode^{#1}\else\(^{#1}\)\fi}
	\caption{Estimates of effect of host race on price, by host's listing count}
	\begin{tabular}{l*{6}{c}}
		\hline\hline
		&\multicolumn{6}{c}{Cutoff for the number of listings operated by a host} \\
		\cmidrule(r){2-7}\\
		\input{code/tables/tex_output/individual_tables/robustness_listings_count}
		\hline\hline
		\multicolumn{5}{l}{\footnotesize Standard errors in parentheses}\\
		\multicolumn{5}{l}{\footnotesize \sym{*} \(p<0.05\), \sym{**} \(p<0.01\), \sym{***} \(p<0.001\)}\\
	\end{tabular}
	\label{table:robustness_listings_count}
	\begin{tablenotes}

		\item {\it Note:} This table estimates the main results in Table \ref{table:price} separately for hosts who operate different numbers of listings. The categories, from left to right, are: listings operated by hosts who own 1 property on Airbnb, listings operated by hosts who have 2 or fewer properties on Airbnb, etc. The omitted category is White males. I control for the specification in Table \ref{table:price}, Model 4. The outcome variable is the log price of the listing.
		
	\end{tablenotes}
\end{table}




\begin{comment}

% Price, no interaction
\begin{table}[htbp]\centering
	\def\sym#1{\ifmmode^{#1}\else\(^{#1}\)\fi}
	\caption{Main result: Estimates of effect of host’s race and gender on price [Without interaction]}
	\begin{tabular}{l*{5}{c}}
		\hline\hline
		\input{code/tables/tex_output/individual_tables/price_no_int}
		\hline\hline
		\multicolumn{5}{l}{\footnotesize Standard errors in parentheses}\\
		\multicolumn{5}{l}{\footnotesize \sym{*} \(p<0.05\), \sym{**} \(p<0.01\), \sym{***} \(p<0.001\)}\\
	\end{tabular}
	\label{table:price_no_int}
	\begin{tablenotes}
		
		\item {\it Note:} The dependent variable is the log price of the listing. The omitted category for host race and gender is White hosts. The unit of observation is a listing. The sample is the sample of listings across 7 US cities. Model 1 is the baseline effect of host demographics on price. Model 2 controls for listing location to the neighborhood level and demographic and economic health characteristics on the zipcode-level. Model 3 adds listing characteristics such as the property type and size. Model 4 adds host characteristics such as response and acceptance rates and measures of host effort.  
	\end{tablenotes}
\end{table}


% Yearly revenue
\begin{table}[htbp]\centering
	\def\sym#1{\ifmmode^{#1}\else\(^{#1}\)\fi}
	\caption{Estimates of effect of host's race and gender on yearly revenue}
	\begin{tabular}{l*{4}{c}}
		\hline\hline
		                    &\multicolumn{1}{c}{(1)}&\multicolumn{1}{c}{(2)}&\multicolumn{1}{c}{(3)}&\multicolumn{1}{c}{(4)}\\
                    &\multicolumn{1}{c}{Model 1}&\multicolumn{1}{c}{Model 2}&\multicolumn{1}{c}{Model 3}&\multicolumn{1}{c}{Model 4}\\
\hline
White Female        &      -206.2\sym{***}&      -157.8\sym{**} &      -150.4\sym{**} &      -137.8\sym{**} \\
                    &     (51.41)         &     (52.02)         &     (45.74)         &     (47.49)         \\
[1em]
White Two people or Unknown&       408.9\sym{***}&       335.2\sym{***}&       123.1\sym{**} &       10.57         \\
                    &     (76.17)         &     (51.60)         &     (47.52)         &     (46.72)         \\
[1em]
Black Male          &      -710.1\sym{***}&      -334.5\sym{***}&      -232.4\sym{***}&      -138.1\sym{*}  \\
                    &     (101.6)         &     (95.95)         &     (57.99)         &     (58.38)         \\
[1em]
Black Female        &      -815.5\sym{***}&      -345.1\sym{***}&      -238.6\sym{***}&      -195.7\sym{***}\\
                    &     (102.4)         &     (87.33)         &     (54.95)         &     (52.66)         \\
[1em]
Black Two people or Unknown&      -363.9\sym{*}  &       105.7         &      -195.7         &      -224.9\sym{*}  \\
                    &     (161.7)         &     (166.3)         &     (102.8)         &     (105.1)         \\
[1em]
Hispanic Male       &      -209.8         &      -35.64         &       4.983         &       26.26         \\
                    &     (114.4)         &     (101.2)         &     (92.49)         &     (88.10)         \\
[1em]
Hispanic Female     &      -250.5         &      -59.61         &      -126.9         &      -62.61         \\
                    &     (138.1)         &     (119.0)         &     (114.5)         &     (113.7)         \\
[1em]
Hispanic Two people or Unknown&      -307.3         &       124.0         &      -16.35         &      -44.26         \\
                    &     (198.0)         &     (192.5)         &     (158.5)         &     (162.3)         \\
[1em]
Asian Male          &      -377.6\sym{**} &      -71.74         &       7.878         &       12.88         \\
                    &     (128.3)         &     (110.4)         &     (87.06)         &     (83.89)         \\
[1em]
Asian Female        &      -705.9\sym{***}&      -295.9\sym{***}&      -165.8\sym{*}  &      -141.7\sym{*}  \\
                    &     (105.8)         &     (82.04)         &     (66.30)         &     (66.03)         \\
[1em]
Asian Two people or Unknown&      -207.9         &       241.2         &      -8.691         &      -122.2         \\
                    &     (144.1)         &     (128.7)         &     (94.61)         &     (96.72)         \\
[1em]
Multiracial or Unknown Male&      -60.11         &       10.06         &       13.69         &       13.33         \\
                    &     (253.7)         &     (210.9)         &     (180.2)         &     (176.9)         \\
[1em]
Multiracial or Unknown Female&      -246.8         &      -266.6         &      -247.5         &      -198.3         \\
                    &     (239.1)         &     (193.0)         &     (193.8)         &     (177.0)         \\
[1em]
Multiracial or Unknown Two people or Unknown&       194.3         &       254.1\sym{*}  &       53.01         &      -22.06         \\
                    &     (108.8)         &     (108.8)         &     (84.67)         &     (80.58)         \\
\hline
Location Fixed Effects&                     &         Yes         &         Yes         &         Yes         \\
Property Fixed Effects&                     &                     &         Yes         &         Yes         \\
Host Fixed Effects  &                     &                     &                     &         Yes         \\
\hline \vspace{-1.25em}&                     &                     &                     &                     \\
Observations        &       68984         &       68981         &       68981         &       68951         \\
Adjusted R2         &     0.00847         &      0.0863         &       0.330         &       0.374         \\

		\hline\hline
		\multicolumn{5}{l}{\footnotesize Standard errors in parentheses}\\
		\multicolumn{5}{l}{\footnotesize \sym{*} \(p<0.05\), \sym{**} \(p<0.01\), \sym{***} \(p<0.001\)}\\
	\end{tabular}
	\label{revenue}
	\begin{tablenotes}
		\item {\it Note:} The dependent variable is a measure of yearly host revenue, as measured by for each listing. The omitted category for race is White males, so all coefficients are relative to that group. The unit of observation is an Airbnb listing, so hosts who have multiple listings are treated separately each time. The sample is the sample of listings across 7 US cities. The specification is the same as Table \ref{table:price}.
	\end{tablenotes}
\end{table}

%  Number of reviews
\begin{table}[htbp]\centering
\def\sym#1{\ifmmode^{#1}\else\(^{#1}\)\fi}
\caption{Estimates of effect of host’s race and gender on number of reviews}
\begin{tabular}{l*{4}{c}}
\hline\hline
                    &\multicolumn{1}{c}{(1)}&\multicolumn{1}{c}{(2)}&\multicolumn{1}{c}{(3)}&\multicolumn{1}{c}{(4)}\\
                    &\multicolumn{1}{c}{Model 1}&\multicolumn{1}{c}{Model 2}&\multicolumn{1}{c}{Model 3}&\multicolumn{1}{c}{Model 4}\\
\hline
White Female        &     -0.0646\sym{**} &     -0.0489\sym{*}  &     -0.0651\sym{***}&     -0.0516\sym{**} \\
                    &    (0.0246)         &    (0.0232)         &    (0.0164)         &    (0.0162)         \\
[1em]
Black Male          &     -0.0618         &     -0.0528         &     -0.0867\sym{**} &     -0.0619         \\
                    &    (0.0636)         &    (0.0542)         &    (0.0334)         &    (0.0321)         \\
[1em]
Black Female        &     -0.0626         &     -0.0633         &      -0.144\sym{***}&      -0.104\sym{***}\\
                    &    (0.0634)         &    (0.0575)         &    (0.0301)         &    (0.0271)         \\
[1em]
Hispanic Male       &     -0.0920         &     -0.0398         &     -0.0567         &     -0.0534         \\
                    &    (0.0530)         &    (0.0503)         &    (0.0353)         &    (0.0315)         \\
[1em]
Hispanic Female     &    0.000356         &      0.0468         &     -0.0277         &      0.0191         \\
                    &    (0.0589)         &    (0.0565)         &    (0.0377)         &    (0.0364)         \\
[1em]
Asian Male          &     -0.0872         &      0.0114         &     -0.0145         &     -0.0147         \\
                    &    (0.0542)         &    (0.0468)         &    (0.0374)         &    (0.0319)         \\
[1em]
Asian Female        &      -0.182\sym{***}&     -0.0662         &     -0.0998\sym{***}&     -0.0529\sym{*}  \\
                    &    (0.0523)         &    (0.0412)         &    (0.0278)         &    (0.0251)         \\
[1em]
Constant            &       1.251\sym{***}&       1.591\sym{**} &       4.876\sym{***}&       3.910\sym{***}\\
                    &     (0.231)         &     (0.558)         &     (0.531)         &     (0.416)         \\
\hline
Location Controls   &                     &         Yes         &         Yes         &         Yes         \\
Property Controls   &                     &                     &         Yes         &         Yes         \\
Host Controls       &                     &                     &                     &         Yes         \\
\hline \vspace{-1.25em}&                     &                     &                     &                     \\
Observations        &       35734         &       35734         &       35734         &       35734         \\
Adjusted R2         &      0.0102         &      0.0742         &       0.455         &       0.559         \\

\hline\hline
\multicolumn{5}{l}{\footnotesize Standard errors in parentheses}\\
\multicolumn{5}{l}{\footnotesize \sym{*} \(p<0.05\), \sym{**} \(p<0.01\), \sym{***} \(p<0.001\)}\\
\end{tabular}
\label{table:num_reviews}

\begin{tablenotes}
\item {\it Note:} The dependent variable is the log number of reviews of the listing. The omitted category for race is White males. The unit of observation is an Airbnb listing, so hosts who have multiple listings are treated separately each time. The sample is the sample of listings across 7 US cities. The specification is the same as Table \ref{table:price}.	
\end{tablenotes}
\end{table}


% GARBAGE Chicago price, ML sentiment controls
\begin{table}[htbp]\centering
\def\sym#1{\ifmmode^{#1}\else\(^{#1}\)\fi}
\caption{Main result: Estimates of effect of Chicago host’s race and gender on price, ML sentiment controls}
\begin{tabular}{l*{5}{c}}
\hline\hline
\input{code/tables/tex_output/individual_tables/chicago_price_sentiment}
\hline\hline
\multicolumn{5}{l}{\footnotesize Standard errors in parentheses}\\
\multicolumn{5}{l}{\footnotesize \sym{*} \(p<0.05\), \sym{**} \(p<0.01\), \sym{***} \(p<0.001\)}\\
\end{tabular}	
\label{table:chiprice}
\begin{tablenotes}

\item {\it Note:} This table presents the impact of host race on the price of a listing. The dependent variable is the log price. The omitted category is White males. The unit of observation is a listing. The sample is the sample of listings in Chicago. Model 1 is the baseline effect of host demographics on price. Model 2 controls for listing location to the neighborhood level and demographic and economic health characteristics on the zipcode-level. Model 3 adds listing characteristics such as the property type and size. Model 4 adds host characteristics such as response and acceptance rates and measures of host effort.  
\end{tablenotes}
\end{table}


% OLD Robustness City
\begin{table}[htbp]\centering
\def\sym#1{\ifmmode^{#1}\else\(^{#1}\)\fi}
\caption{Robustness City}
\begin{tabular}{l*{7}{c}}
\hline\hline
                    &\multicolumn{1}{c}{(1)}&\multicolumn{1}{c}{(2)}&\multicolumn{1}{c}{(3)}&\multicolumn{1}{c}{(4)}&\multicolumn{1}{c}{(5)}&\multicolumn{1}{c}{(6)}&\multicolumn{1}{c}{(7)}\\
                    &\multicolumn{1}{c}{LA}&\multicolumn{1}{c}{NYC}&\multicolumn{1}{c}{Austin}&\multicolumn{1}{c}{Chicago}&\multicolumn{1}{c}{New Orleans}&\multicolumn{1}{c}{DC}&\multicolumn{1}{c}{Nashville}\\
\hline
Black               &      -2.140         &       0.216         &      -22.44         &      -5.119         &      -24.97\sym{*}  &      -15.82\sym{**} &      -16.53         \\
                    &     (8.483)         &     (3.988)         &     (20.55)         &     (5.020)         &     (9.445)         &     (5.621)         &     (10.24)         \\
[1em]
Hispanic            &       8.654         &      -3.570         &       12.48         &     -0.0855         &      -4.189         &      -2.363         &      -40.36\sym{**} \\
                    &     (8.023)         &     (2.893)         &     (11.39)         &     (5.714)         &     (12.29)         &     (7.185)         &     (13.55)         \\
[1em]
Asian               &       0.939         &       1.329         &      -45.73\sym{***}&      -16.85\sym{***}&      -3.150         &      -14.33\sym{*}  &      -5.671         \\
                    &     (5.852)         &     (6.272)         &     (12.07)         &     (4.175)         &     (16.86)         &     (6.042)         &     (23.82)         \\
[1em]
Multiracial or Unknown&       7.168         &       1.392         &       2.283         &       13.52         &      -13.03         &       4.361         &      -43.06\sym{**} \\
                    &     (6.057)         &     (6.277)         &     (15.28)         &     (7.754)         &     (7.951)         &     (6.297)         &     (12.67)         \\
[1em]
Los Angeles         &           0         &                     &                     &                     &                     &                     &                     \\
                    &         (.)         &                     &                     &                     &                     &                     &                     \\
[1em]
New York City       &                     &           0         &                     &                     &                     &                     &                     \\
                    &                     &         (.)         &                     &                     &                     &                     &                     \\
[1em]
Austin              &                     &                     &           0         &                     &                     &                     &                     \\
                    &                     &                     &         (.)         &                     &                     &                     &                     \\
[1em]
Chicago             &                     &                     &                     &           0         &                     &                     &                     \\
                    &                     &                     &                     &         (.)         &                     &                     &                     \\
[1em]
New Orleans         &                     &                     &                     &                     &           0         &                     &                     \\
                    &                     &                     &                     &                     &         (.)         &                     &                     \\
[1em]
Washington DC       &                     &                     &                     &                     &                     &           0         &                     \\
                    &                     &                     &                     &                     &                     &         (.)         &                     \\
[1em]
Nashville           &                     &                     &                     &                     &                     &                     &           0         \\
                    &                     &                     &                     &                     &                     &                     &         (.)         \\
\hline
Location Fixed Effects&         Yes         &         Yes         &         Yes         &         Yes         &         Yes         &         Yes         &         Yes         \\
Property Fixed Effects&         Yes         &         Yes         &         Yes         &         Yes         &         Yes         &         Yes         &         Yes         \\
Host Fixed Effects  &         Yes         &         Yes         &         Yes         &         Yes         &         Yes         &         Yes         &         Yes         \\
\hline \vspace{-1.25em}&                     &                     &                     &                     &                     &                     &                     \\
Observations        &       26076         &       20417         &        5818         &        5144         &        4511         &        3722         &        3277         \\
Adjusted R2         &       0.438         &       0.280         &       0.468         &       0.366         &       0.475         &       0.450         &       0.616         \\

\hline\hline
\multicolumn{8}{l}{\footnotesize Standard errors in parentheses}\\
\multicolumn{8}{l}{\footnotesize \sym{*} \(p<0.05\), \sym{**} \(p<0.01\), \sym{***} \(p<0.001\)}\\
\end{tabular}
\label{table:robustcity_old}

\begin{tablenotes}
\item {\it Note:} This table breaks down the effects for the combined data in Table \ref{table:price} across the 7 cities in the sample. Each set of coefficients represents the coefficient on host race, with log price as the outcome variable. I control for my preferred specification throughout. Low number of observations for Black, Hispanic, and Asian hosts contribute to imprecise estimates in smaller cities (New Orleans, Nashville have less than 100 Hispanic and Asian hosts; DC and Austin have less than 200 such hosts). 
\end{tablenotes}
\end{table}

% OLD Robustness Listing Characteristics
\begin{landscape}
\begin{table}[htbp]\centering
\def\sym#1{\ifmmode^{#1}\else\(^{#1}\)\fi}
\caption{Robustness Listing Characteristics}
\begin{tabular}{l*{9}{c}}
\hline\hline
                    &\multicolumn{1}{c}{(1)}&\multicolumn{1}{c}{(2)}&\multicolumn{1}{c}{(3)}&\multicolumn{1}{c}{(4)}&\multicolumn{1}{c}{(5)}&\multicolumn{1}{c}{(6)}&\multicolumn{1}{c}{(7)}&\multicolumn{1}{c}{(8)}&\multicolumn{1}{c}{(9)}\\
                    &\multicolumn{1}{c}{Low $ LA}&\multicolumn{1}{c}{High $ LA}&\multicolumn{1}{c}{Low $ NY}&\multicolumn{1}{c}{High $ NY}&\multicolumn{1}{c}{Older Listings}&\multicolumn{1}{c}{Newer Listings}&\multicolumn{1}{c}{Apartments}&\multicolumn{1}{c}{Condos}&\multicolumn{1}{c}{Houses}\\
\hline
Black               &       8.484         &      -20.53         &       0.943         &       3.865         &      -1.747         &      -8.164\sym{***}&      -2.718         &      -7.304         &      -13.45         \\
                    &     (11.03)         &     (11.78)         &     (1.860)         &     (10.13)         &     (6.942)         &     (1.820)         &     (2.648)         &     (11.16)         &     (11.14)         \\
[1em]
Hispanic            &       3.847         &      -13.61         &      -4.802         &      -6.285         &       3.999         &       2.589         &      -3.703         &      -18.38         &       9.505         \\
                    &     (6.405)         &     (18.78)         &     (3.139)         &     (5.966)         &     (7.187)         &     (3.319)         &     (1.887)         &     (9.518)         &     (10.53)         \\
[1em]
Asian               &      -5.738\sym{**} &      -4.347         &       5.217         &      -1.648         &      -9.855\sym{***}&      -1.425         &      -2.103         &      -21.21\sym{*}  &      -10.04         \\
                    &     (1.949)         &     (16.21)         &     (8.339)         &     (8.934)         &     (2.908)         &     (3.453)         &     (3.810)         &     (10.57)         &     (6.042)         \\
[1em]
Multiracial or Unknown&      -0.227         &       3.819         &      -1.555         &       4.168         &      -2.160         &      -3.304         &      -0.414         &      -11.94         &       16.39         \\
                    &     (1.391)         &     (14.43)         &     (2.210)         &     (10.40)         &     (4.439)         &     (3.243)         &     (2.541)         &     (15.56)         &     (8.691)         \\
\hline
Location Fixed Effects&         Yes         &         Yes         &         Yes         &         Yes         &         Yes         &         Yes         &         Yes         &         Yes         &         Yes         \\
Property Fixed Effects&         Yes         &         Yes         &         Yes         &         Yes         &         Yes         &         Yes         &         Yes         &         Yes         &         Yes         \\
Host Fixed Effects  &         Yes         &         Yes         &         Yes         &         Yes         &         Yes         &         Yes         &         Yes         &         Yes         &         Yes         \\
\hline \vspace{-1.25em}&                     &                     &                     &                     &                     &                     &                     &                     &                     \\
Observations        &       17155         &        8921         &       10717         &        9700         &       15193         &       39268         &       41254         &        2903         &       22236         \\
Adjusted R2         &      0.0968         &       0.480         &      0.0513         &       0.326         &       0.547         &       0.477         &       0.312         &       0.416         &       0.461         \\

\hline\hline
\multicolumn{10}{l}{\footnotesize Standard errors in parentheses}\\
\multicolumn{10}{l}{\footnotesize \sym{*} \(p<0.05\), \sym{**} \(p<0.01\), \sym{***} \(p<0.001\)}\\
\end{tabular}
\label{table:robustlistingold}

\begin{tablenotes}
\item {\it Note:} This table breaks the effects for the combined data by high versus low price, time on market, and property type. The categories, from left to right, are: listings whose log price is below vs. above the mean predicted log price in each city, the price originally dropped, listings who have have been on the market for no more than 2 years vs. no more than 8 years, and listings of different property types, including apartments (includes apartments and lofts), condos (includes condos and townhouse), and houses. I control for my preferred specification throughout. The outcome variable is the log price of the listing.
\end{tablenotes}
\end{table}
\end{landscape}

\end{comment}
