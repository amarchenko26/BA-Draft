
% Table 1
\begin{table}[htbp]
\caption{Summary Statistics By Host Race: Listing Characteristics}
\begin{center}%
\small\begin{tabular}{l c | c | c c c c}
& \multicolumn{1}{c}{} & \multicolumn{5}{c}{Regression Sample}
\\
 \cmidrule(r){3-7}
\\
 & \multicolumn{1}{c}{Full data} & \multicolumn{1}{c}{All} & White & Black & Hispanic & Asian
\\
\hline\hline\noalign{\smallskip} 
 \textit{\textit{Outcome Variables}} & & & & & & \\ Log Price & 4.81 & 4.73 & 4.79 & 4.51 & 4.65 & 4.55 \\
 & (0.75) & (0.66) & (0.66) & (0.62) & (0.65) & (0.64) \\
 Log Number of Reviews & 2.20 & 2.16 & 2.18 & 2.14 & 2.16 & 2.04 \\
 & (1.39) & (1.38) & (1.39) & (1.36) & (1.40) & (1.36) \\
 \textit{Covariates} & & & & & & \\ \hline Property Type & & & & & & \\ \hspace{10bp}Apartments/Lofts    & 0.60 & 0.63 & 0.63 & 0.66 & 0.66 & 0.62 \\ \hspace{10bp}Townhouses/Condos   & 0.04 & 0.04 & 0.04 & 0.04 & 0.04 & 0.06 \\ \hspace{10bp}Houses                      & 0.32 & 0.30 & 0.30 & 0.27 & 0.26 & 0.30 \\ \hspace{10bp}Others                              & 0.04 & 0.03 & 0.03 & 0.03 & 0.03 & 0.03 \\Room Type &&&&&& \\ \hspace{10bp}Entire House/Apartment      & 0.58 & 0.55 & 0.58 & 0.43 & 0.51 & 0.42 \\ \hspace{10bp}Private Room                        & 0.38 & 0.42 & 0.39 & 0.50 & 0.44 & 0.53 \\ \hspace{10bp}Shared Room                         & 0.04 & 0.04 & 0.03 & 0.07 & 0.05 & 0.05 \\ Max Num. Guests & 3.44 & 3.15 & 3.24 & 2.94 & 3.06 & 2.84 \\
 & (2.41) & (2.13) & (2.15) & (2.06) & (2.15) & (2.00) \\
 Bedrooms & 1.34 & 1.26 & 1.28 & 1.19 & 1.21 & 1.18 \\
 & (0.92) & (0.80) & (0.83) & (0.69) & (0.78) & (0.72) \\
 Bathrooms & 1.30 & 1.23 & 1.24 & 1.19 & 1.21 & 1.19 \\
 & (0.69) & (0.55) & (0.56) & (0.49) & (0.52) & (0.53) \\
 Beds & 1.82 & 1.67 & 1.69 & 1.60 & 1.68 & 1.57 \\
 & (1.41) & (1.21) & (1.19) & (1.15) & (1.51) & (1.18) \\
 Cleaning Fee & 48.95 & 43.70 & 46.06 & 36.20 & 40.35 & 36.45 \\
 & (59.62) & (48.32) & (49.73) & (43.18) & (45.51) & (42.86) \\
 Extra Guests Charge & 13.74 & 13.43 & 13.26 & 15.13 & 13.94 & 12.72 \\
 & (23.65) & (22.67) & (23.00) & (22.71) & (22.48) & (20.36) \\
 Minimum Nights & 3.01 & 3.03 & 3.08 & 2.61 & 2.86 & 3.17 \\
 & (9.21) & (8.79) & (9.39) & (4.35) & (6.55) & (8.67) \\
 Availability (out of 30 days) & 11.54 & 11.04 & 10.64 & 14.19 & 11.24 & 10.79 \\
 & (10.93) & (10.91) & (10.75) & (11.49) & (10.94) & (11.01) \\
 Number of Amenities & 0.81 & 0.79 & 0.80 & 0.75 & 0.79 & 0.75 \\
 & (1.10) & (1.10) & (1.10) & (1.04) & (1.10) & (1.13) \\
 Instantly Bookable? & 0.15 & 0.15 & 0.14 & 0.21 & 0.17 & 0.16 \\
 & (0.36) & (0.36) & (0.34) & (0.41) & (0.38) & (0.37) \\
 Year of first review & 14.86 & 14.86 & 14.83 & 14.89 & 14.90 & 15.03 \\
 & (1.22) & (1.22) & (1.22) & (1.30) & (1.21) & (1.17) \\
 Strict Cancellation Policy & 0.43 & 0.40 & 0.42 & 0.42 & 0.42 & 0.42 \\\hline
Observations & \numprint{69007} & \numprint{45076} & \numprint{32934} & \numprint{4354} & \numprint{2913} & \numprint{4875} 
\\
\hline\hline\noalign{\smallskip} \end{tabular} 
\begin{minipage}{6in}
{\it Note:} The values in the table are means and standard deviations of listing-level data in my full sample. Summary statistics for selected covariates are listed in the table. Categorical variables such as room type do not have standard deviations. Property types are explicitly listed if more than 1.5\% of listings are that type. Strict cancellation policy is the most common out of 4 possible policies:  strict (43\%), flexible (31\%), moderate (25\%) and other (1\%) Year of first review is a proxy for the time on the market - 14.86 indicates that the first review of the mean listing in the full sample occurred in October of 2014.
\end{minipage}
\end{center}
\end{table}

\newpage

% Table 2
\begin{table}[htbp]
\caption{Summary Statistics By Host Race: Host Demographics}
\begin{center}%
\small\begin{tabular}{l c | c | c c c c}
& \multicolumn{1}{c}{} & \multicolumn{5}{c}{Regression Sample}
\\
 \cmidrule(r){3-7}
\\
 & \multicolumn{1}{c}{Full data} & \multicolumn{1}{c}{All} & White & Black & Hispanic & Asian
\\
\hline\hline\noalign{\smallskip} 
 \textit{Race} &&&&&& \\
 \hspace{10bp}White & 0.64 & 0.73 &  1.00 & 0.00 &  0.00 & 0.00 \\  \hspace{10bp}Black & 0.07 & 0.10 &  0.00 & 1.00 &  0.00 & 0.00 \\  \hspace{10bp}Hispanic & 0.05 & 0.06 &  0.00 & 0.00 &  1.00 & 0.00 \\  \hspace{10bp}Asian & 0.09 & 0.11 &  0.00 & 0.00 &  0.00 & 1.00 \\  \hspace{10bp}Unknown & 0.15 & {0.00} & {0.00} &  {0.00}  & {0.00}  & {0.00} \\  \textit{Sex} &&&&&& \\
 \hspace{10bp}Male & 0.31 & 0.45 &  0.45 & 0.40 &  0.50 & 0.44 \\  \hspace{10bp}Female & 0.38 & 0.55 &  0.55 & 0.60 &  0.50 & 0.56 \\  \hspace{10bp}Unknown & 0.31 & {0.00} & {0.00} &  {0.00}  & {0.00}  & {0.00} \\  \textit{Age} &&&&&& \\
 \hspace{10bp}Young ($<30$) & 0.43 & 0.51 &  0.49 & 0.54 &  0.52 & 0.61 \\  \hspace{10bp}Middle-aged & 0.42 & 0.46 &  0.48 & 0.45 &  0.47 & 0.38 \\  \hspace{10bp}Old ($>65$) & 0.02 & 0.02 &  0.03 & 0.00 &  0.01 & 0.01 \\  \hspace{10bp}Unknown & 0.13 & {0.00} & {0.00} &  {0.00}  & {0.00}  & {0.00} \\ \hline
Observations & \numprint{69000} & \numprint{45076} & \numprint{32934} & \numprint{4354} & \numprint{2913} & \numprint{4875} 
\\
\hline\hline\noalign{\smallskip} \end{tabular} 
\begin{minipage}{6in}
\label{table:host_demographics}
{\it Note:} The values in the table are summaries of host demographics in the host-level data. Column 1 is the summary statistics for the full, unrestricted data set across 7 cities. Columns 2 - 6 are the restricted data used in the analysis. Column 2 is the full regression sample, and columns 3 - 6 break down the regression sample by host race. The “Unknown” category was dropped from the regression and is therefore zero throughout columns 2 - 6. White refers only to Non-Hispanic Whites.\end{minipage}
\end{center}
\end{table}

\newpage

% Table 3
\begin{table}[htbp]
\caption{Summary Statistics By Host Race: Host Characteristics}
\begin{center}%
\small\begin{tabular}{l c | c | c c c c}
& \multicolumn{1}{c}{} & \multicolumn{5}{c}{Regression Sample}
\\
 \cmidrule(r){3-7}
\\
 & \multicolumn{1}{c}{Full data} & \multicolumn{1}{c}{All} & White & Black & Hispanic & Asian
\\
\hline\hline\noalign{\smallskip} 
 \textit{\textit{Outcome Variables}} & & & & & & \\ Host Listings Count & 6.38 & 2.23 & 2.16 & 2.38 & 2.49 & 2.44 \\
 & (36.54) & (2.59) & (2.50) & (2.83) & (3.03) & (2.61) \\
 \textit{Covariates} & & & & & & \\ \hline Review scores rating & 93.56 & 93.68 & 94.18 & 91.91 & 92.80 & 92.26 \\
 & (8.13) & (7.90) & (7.33) & (9.44) & (8.71) & (9.27) \\
 Host is a Superhost & 0.13 & 0.13 & 0.14 & 0.09 & 0.11 & 0.10 \\
 & (0.34) & (0.33) & (0.34) & (0.28) & (0.31) & (0.30) \\
 Response rate & 0.77 & 0.76 & 0.76 & 0.78 & 0.76 & 0.74 \\
 & (0.38) & (0.39) & (0.39) & (0.37) & (0.39) & (0.40) \\
 Acceptance rate & 0.47 & 0.45 & 0.46 & 0.35 & 0.49 & 0.44 \\
 & (0.46) & (0.46) & (0.46) & (0.45) & (0.47) & (0.47) \\
 Polarity of Summary & 0.30 & 0.30 & 0.30 & 0.29 & 0.30 & 0.29 \\
 & (0.17) & (0.17) & (0.17) & (0.16) & (0.17) & (0.17) \\
 Subjectivity of Summary & 0.53 & 0.54 & 0.54 & 0.53 & 0.54 & 0.53 \\
 & (0.15) & (0.15) & (0.15) & (0.15) & (0.15) & (0.15) \\
 Host's Identity Verified? & 0.70 & 0.70 & 0.71 & 0.66 & 0.68 & 0.69 \\
 & (0.46) & (0.46) & (0.45) & (0.47) & (0.47) & (0.46) \\
 Guest Pic Required? & 0.04 & 0.04 & 0.04 & 0.06 & 0.04 & 0.04 \\
 & (0.19) & (0.19) & (0.19) & (0.23) & (0.19) & (0.19) \\
 Guest Phone Required? & 0.05 & 0.05 & 0.05 & 0.06 & 0.04 & 0.04 \\
 & (0.22) & (0.21) & (0.21) & (0.24) & (0.20) & (0.20) \\
 Response time $<$ 1 hour & 0.41 & 0.40 & 0.39 & 0.44 & 0.41 & 0.41 \\\hline
Observations & \numprint{69010} & \numprint{45076} & \numprint{32934} & \numprint{4354} & \numprint{2913} & \numprint{4875}
\\
\hline\hline\noalign{\smallskip} \end{tabular} 
\begin{minipage}{6in}
\label{table:host_summary}
{\it Note:} The values in the table are means and standard deviations of host-level data in the full sample. Summary statistics for selected covariates are listed in the table. Categorical variables such as response time do not have standard deviations. Statistics for only the most frequent response time (\say{within an hour}) are included. White refers only to non-Hispanic whites. Polarity of \say{Summary} and Subjectivity of \say{Summary} refer to the scores from a natural language processing algorithm that measures the sentiment and objectivity of that field. These two measures were also calculated for the description, space, neighborhood overview, notes, and transit fields, but were not included in the table for the sake of clarity and because they follow a similar pattern as the \say{Summary} field.
\end{minipage}
\end{center}
\end{table}

\newpage

% Table 4
\begin{table}[htbp]
\caption{Summary Statistics By Race: Reviewer Characteristics}
\begin{center}%
\small\begin{tabular}{l c | c | c c c c}
& \multicolumn{1}{c}{} & \multicolumn{5}{c}{Reviewer Race in Chicago data} 
\\
 \cmidrule(r){3-7}
\\
 & \multicolumn{1}{c}{Full data} & \multicolumn{1}{c}{All} & White & Black & Hispanic & Asian
\\
\hline\hline\noalign{\smallskip} 
 Reviewer Race  & 1.00 & 1.00 & 0.66 & 0.03 & 0.04 & 0.11 \\\\
 Host race & & & & & & \\ \hspace{10bp}White &     0.73 & 0.83 & 0.84 & 0.70 & 0.75 & 0.75 \\ \hspace{10bp}Black &     0.06 & 0.06& 0.05 & 0.17 & 0.07 & 0.06 \\ \hspace{10bp}Hispanic &  0.04 & 0.05& 0.05 & 0.06 & 0.10 & 0.08 \\ \hspace{10bp}Asian &     0.05 & 0.05& 0.05 & 0.08 & 0.08 & 0.11 \\ \hspace{10bp}Unknown &   0.12 & 0.00& 0.00 & 0.00 & 0.00 & 0.00 \\\\
 Review Sentiment & 0.51 & 0.51 & 0.51 & 0.50 & 0.47 & 0.53 \\
 & (0.26) & (0.26) & (0.25) & (0.23) & (0.30) & (0.25) \\
\\
 Listing Sentiment & 0.51 & 0.51 & 0.51 & 0.50 & 0.50 & 0.51 \\
 & (0.07) & (0.07) & (0.07) & (0.07) & (0.07) & (0.09) \\
\\
\hline
Observations & \numprint{17050} &  \numprint{10573} & \numprint{6929} & \numprint{319} & \numprint{402} & \numprint{1153}
\\
\hline\hline\noalign{\smallskip} \end{tabular} 
\begin{minipage}{6in}
\label{table:reviewer_demographics}
{\it Note:} The values in this table are means and standard deviations of reviewer-level data who left reviews for a randomly chosen set of hosts in Chicago. Column 1 presents the means for the full reviewer data. Column 2 presents the means of the sample used in Table \ref{table:sentiment}. Columns 3 - 6 partition Column 2 by reviewer race. Row 1, \say{Reviewer race} indicates the proportion of the different reviewer races in the data coded. Row 2, \say{Host race} indicates the marginal probability of a host race given a reviewer race. The review sentiment is the sentiment of each review, the listing sentiment is the average sentiment per listing. Observations in Columns 2 - 5 do not add up to 17,050 because multiracial or unidentifiable reviewer pictures are excluded. White refers only to non-Hispanic Whites.
\end{minipage}
\end{center}
\end{table}

\newpage

% Table 5
\begin{table}[htbp]\centering
	\def\sym#1{\ifmmode^{#1}\else\(^{#1}\)\fi}
	\caption{Main result: Estimates of effect of host’s race and gender on price}
	\begin{tabular}{l*{5}{c}}
		\hline\hline
		                    &\multicolumn{1}{c}{(1)}&\multicolumn{1}{c}{(2)}&\multicolumn{1}{c}{(3)}&\multicolumn{1}{c}{(4)}\\
                    &\multicolumn{1}{c}{Model 1}&\multicolumn{1}{c}{Model 2}&\multicolumn{1}{c}{Model 3}&\multicolumn{1}{c}{Model 4}\\
\hline
White Female        &     -0.0236\sym{*}  &     -0.0138         &     0.00201         &     0.00298         \\
                    &    (0.0106)         &   (0.00854)         &   (0.00496)         &   (0.00484)         \\
[1em]
Black Male          &      -0.276\sym{***}&     -0.0828\sym{**} &     -0.0360\sym{**} &     -0.0328\sym{**} \\
                    &    (0.0315)         &    (0.0259)         &    (0.0123)         &    (0.0123)         \\
[1em]
Black Female        &      -0.299\sym{***}&     -0.0586\sym{**} &     -0.0196         &     -0.0167         \\
                    &    (0.0296)         &    (0.0188)         &    (0.0102)         &   (0.00996)         \\
[1em]
Hispanic Male       &      -0.153\sym{***}&     -0.0521\sym{**} &     -0.0233\sym{*}  &     -0.0200         \\
                    &    (0.0259)         &    (0.0191)         &    (0.0113)         &    (0.0113)         \\
[1em]
Hispanic Female     &      -0.150\sym{***}&     -0.0653\sym{**} &     -0.0196         &     -0.0202         \\
                    &    (0.0280)         &    (0.0202)         &    (0.0115)         &    (0.0114)         \\
[1em]
Asian Male          &      -0.221\sym{***}&     -0.0987\sym{***}&     -0.0425\sym{**} &     -0.0446\sym{***}\\
                    &    (0.0336)         &    (0.0225)         &    (0.0134)         &    (0.0135)         \\
[1em]
Asian Female        &      -0.283\sym{***}&      -0.131\sym{***}&     -0.0409\sym{***}&     -0.0396\sym{***}\\
                    &    (0.0299)         &    (0.0161)         &   (0.00874)         &   (0.00893)         \\
[1em]
Constant            &       4.802\sym{***}&       4.979\sym{***}&       3.891\sym{***}&       4.003\sym{***}\\
                    &    (0.0300)         &     (0.398)         &     (0.343)         &     (0.344)         \\
\hline
Location Controls   &                     &         Yes         &         Yes         &         Yes         \\
Property Controls   &                     &                     &         Yes         &         Yes         \\
Host Controls       &                     &                     &                     &         Yes         \\
\hline \vspace{-1.25em}&                     &                     &                     &                     \\
Observations        &       45073         &       45073         &       45073         &       45073         \\
Adjusted R2         &      0.0263         &       0.246         &       0.716         &       0.720         \\

		\hline\hline
		\multicolumn{5}{l}{\footnotesize Standard errors in parentheses}\\
		\multicolumn{5}{l}{\footnotesize \sym{*} \(p<0.05\), \sym{**} \(p<0.01\), \sym{***} \(p<0.001\)}\\
	\end{tabular}	
\label{table:price}
	\begin{tablenotes}
		
		\item {\it Note:} This table presents the impact of host race on the price of a listing. The dependent variable is the log price. The omitted category is White males. The unit of observation is a listing. The sample is the sample of listings across 7 US cities. Model 1 is the baseline effect of host demographics on price. Model 2 controls for listing location to the neighborhood level and demographic and economic health characteristics on the zipcode-level. Model 3 adds listing characteristics such as the property type and size. Model 4 adds host characteristics such as response and acceptance rates and measures of host effort.  
	\end{tablenotes}
\end{table}

% Table 6
\begin{table}[htbp]\centering
	\def\sym#1{\ifmmode^{#1}\else\(^{#1}\)\fi}
	
	\caption{Robustness check with controls from Edelman \& Luca (2014)}
	\begin{tabular}{l*{1}{c}}
		\hline\hline
		                    &\multicolumn{1}{c}{(1)}\\
                    &\multicolumn{1}{c}{Price per night}\\
\hline
Black               &      -0.117\sym{***}\\
                    &    (0.0107)         \\
Accommodates        &      0.0684\sym{***}\\
                    &   (0.00288)         \\
Bedrooms            &       0.129\sym{***}\\
                    &   (0.00724)         \\
Review Scores Location&      -0.488\sym{***}\\
                    &    (0.0434)         \\
Review Scores Location Squared&      0.0363\sym{***}\\
                    &   (0.00249)         \\
Review Scores Checkin&   -0.000735         \\
                    &   (0.00683)         \\
Review Scores Communication&    -0.00366         \\
                    &   (0.00718)         \\
Review Scores Cleanliness&      0.0230\sym{***}\\
                    &   (0.00417)         \\
Review Scores Accuracy&     -0.0186\sym{**} \\
                    &   (0.00574)         \\
Host's Identity Verified?&      0.0233\sym{**} \\
                    &   (0.00801)         \\
Private room        &      -0.627\sym{***}\\
                    &   (0.00826)         \\
Shared room         &      -1.123\sym{***}\\
                    &    (0.0183)         \\
\hline
Location Controls   &         Yes         \\
Property Controls   &         Yes         \\
Host Controls       &         Yes         \\
\hline \vspace{-1.25em}&                     \\
Observations        &       11999         \\
Adjusted R2         &       0.619         \\
 
		\hline\hline
		\multicolumn{2}{l}{\footnotesize Standard errors in parentheses}\\
		\multicolumn{2}{l}{\footnotesize \sym{*} \(p<0.05\), \sym{**} \(p<0.01\), \sym{***} \(p<0.001\)}\\
	\end{tabular}
\label{table:edelman_new}
	\begin{tablenotes}
		\item {\it Note:} This table presents the effect on log price of controlling for Edelman \& Luca's (2014) full specification using my NYC data. The omitted category for race is White hosts. The omitted category for room type is Entire Apartment. I could not control for host social media accounts as a proxy for host reliability like Edelman \& Luca did, because Airbnb no longer provides this information. Instead, I controlled for ``host verified", a dummy for whether Airbnb has the host's phone number and email. I was not able to control for ``picture quality" either, but picture quality did not significantly influence price in Edelman \& Luca's regression.
	\end{tablenotes}
\end{table}


% Table 7
\begin{table}[htbp]\centering
	\def\sym#1{\ifmmode^{#1}\else\(^{#1}\)\fi}
	\caption{Estimates of effect of host’s race and gender on number of reviews}
	\begin{tabular}{l*{4}{c}}
		\hline\hline
		                    &\multicolumn{1}{c}{(1)}&\multicolumn{1}{c}{(2)}&\multicolumn{1}{c}{(3)}&\multicolumn{1}{c}{(4)}\\
                    &\multicolumn{1}{c}{Model 1}&\multicolumn{1}{c}{Model 2}&\multicolumn{1}{c}{Model 3}&\multicolumn{1}{c}{Model 4}\\
\hline
White Female        &     -0.0646\sym{**} &     -0.0489\sym{*}  &     -0.0651\sym{***}&     -0.0516\sym{**} \\
                    &    (0.0246)         &    (0.0232)         &    (0.0164)         &    (0.0162)         \\
[1em]
Black Male          &     -0.0618         &     -0.0528         &     -0.0867\sym{**} &     -0.0619         \\
                    &    (0.0636)         &    (0.0542)         &    (0.0334)         &    (0.0321)         \\
[1em]
Black Female        &     -0.0626         &     -0.0633         &      -0.144\sym{***}&      -0.104\sym{***}\\
                    &    (0.0634)         &    (0.0575)         &    (0.0301)         &    (0.0271)         \\
[1em]
Hispanic Male       &     -0.0920         &     -0.0398         &     -0.0567         &     -0.0534         \\
                    &    (0.0530)         &    (0.0503)         &    (0.0353)         &    (0.0315)         \\
[1em]
Hispanic Female     &    0.000356         &      0.0468         &     -0.0277         &      0.0191         \\
                    &    (0.0589)         &    (0.0565)         &    (0.0377)         &    (0.0364)         \\
[1em]
Asian Male          &     -0.0872         &      0.0114         &     -0.0145         &     -0.0147         \\
                    &    (0.0542)         &    (0.0468)         &    (0.0374)         &    (0.0319)         \\
[1em]
Asian Female        &      -0.182\sym{***}&     -0.0662         &     -0.0998\sym{***}&     -0.0529\sym{*}  \\
                    &    (0.0523)         &    (0.0412)         &    (0.0278)         &    (0.0251)         \\
[1em]
Constant            &       1.251\sym{***}&       1.591\sym{**} &       4.876\sym{***}&       3.910\sym{***}\\
                    &     (0.231)         &     (0.558)         &     (0.531)         &     (0.416)         \\
\hline
Location Controls   &                     &         Yes         &         Yes         &         Yes         \\
Property Controls   &                     &                     &         Yes         &         Yes         \\
Host Controls       &                     &                     &                     &         Yes         \\
\hline \vspace{-1.25em}&                     &                     &                     &                     \\
Observations        &       35734         &       35734         &       35734         &       35734         \\
Adjusted R2         &      0.0102         &      0.0742         &       0.455         &       0.559         \\

		\hline\hline
		\multicolumn{5}{l}{\footnotesize Standard errors in parentheses}\\
		\multicolumn{5}{l}{\footnotesize \sym{*} \(p<0.05\), \sym{**} \(p<0.01\), \sym{***} \(p<0.001\)}\\
	\end{tabular}
\label{table:num_reviews}

	\begin{tablenotes}
		\item {\it Note:} The dependent variable is the log number of reviews of the listing. The omitted category for race is White males. The unit of observation is an Airbnb listing, so hosts who have multiple listings are treated separately each time. The sample is the sample of listings across 7 US cities. The specification is the same as Table \ref{table:price}.	
	\end{tablenotes}
\end{table}


% Table 8
\begin{table}[htbp]\centering
	\def\sym#1{\ifmmode^{#1}\else\(^{#1}\)\fi}
	\caption{Effect of host’s race on listing availability out of 30 days}
	\begin{tabular}{l*{1}{c}}
		\hline\hline
		                    &\multicolumn{1}{c}{(1)}&\multicolumn{1}{c}{(2)}\\
                    &\multicolumn{1}{c}{Number of vacant days out of 30}&\multicolumn{1}{c}{Log number of reviews}\\
\hline
White Female        &      -0.888\sym{***}&     -0.0516\sym{**} \\
                    &     (0.110)         &    (0.0162)         \\
[1em]
Black Male          &       2.338\sym{***}&     -0.0619         \\
                    &     (0.258)         &    (0.0321)         \\
[1em]
Black Female        &       1.775\sym{***}&      -0.104\sym{***}\\
                    &     (0.227)         &    (0.0271)         \\
[1em]
Hispanic Male       &      -0.185         &     -0.0534         \\
                    &     (0.272)         &    (0.0315)         \\
[1em]
Hispanic Female     &     -0.0908         &      0.0191         \\
                    &     (0.274)         &    (0.0364)         \\
[1em]
Asian Male          &      -0.130         &     -0.0147         \\
                    &     (0.234)         &    (0.0319)         \\
[1em]
Asian Female        &      -1.106\sym{***}&     -0.0529\sym{*}  \\
                    &     (0.215)         &    (0.0251)         \\
[1em]
Constant            &      -11.46         &       3.910\sym{***}\\
                    &     (11.12)         &     (0.416)         \\
\hline
Location Controls   &         Yes         &         Yes         \\
Property Controls   &         Yes         &         Yes         \\
Host Controls       &         Yes         &         Yes         \\
\hline \vspace{-1.25em}&                     &                     \\
Observations        &       45076         &       35734         \\
Adjusted R2         &       0.228         &       0.559         \\

		\hline\hline
		\multicolumn{2}{l}{\footnotesize Standard errors in parentheses}\\
		\multicolumn{2}{l}{\footnotesize \sym{*} \(p<0.05\), \sym{**} \(p<0.01\), \sym{***} \(p<0.001\)}\\
	\end{tabular}
\label{table:availability}
	\begin{tablenotes}
		\item {\it Note:} This table presents the effect of host race on listing availability out of 30 days, controlling for my preferred specification in Table \ref{price}. When a listing is booked, this availability metric is updated on the Airbnb website to reflect that booking. Therefore, this measure actually represents the number of days out of the total available days that listings were vacant, relative to a White male host.
	\end{tablenotes}
\end{table}


% Table 9
\begin{table}[htbp]\centering
	\def\sym#1{\ifmmode^{#1}\else\(^{#1}\)\fi}
	\caption{Robustness City}
	\begin{tabular}{l*{7}{c}}
		\hline\hline
		                    &\multicolumn{1}{c}{(1)}&\multicolumn{1}{c}{(2)}&\multicolumn{1}{c}{(3)}&\multicolumn{1}{c}{(4)}&\multicolumn{1}{c}{(5)}&\multicolumn{1}{c}{(6)}&\multicolumn{1}{c}{(7)}\\
                    &\multicolumn{1}{c}{LA}&\multicolumn{1}{c}{NYC}&\multicolumn{1}{c}{Austin}&\multicolumn{1}{c}{Chicago}&\multicolumn{1}{c}{New Orleans}&\multicolumn{1}{c}{DC}&\multicolumn{1}{c}{Nashville}\\
\hline
Female              &    -0.00241         &     0.00823         &      0.0152         &     -0.0230         &      0.0259         &      0.0337\sym{**} &      0.0194         \\
                    &   (0.00636)         &   (0.00618)         &    (0.0153)         &    (0.0162)         &    (0.0189)         &    (0.0122)         &    (0.0169)         \\
[1em]
Black               &     -0.0290\sym{*}  &    -0.00470         &     -0.0770         &     -0.0283         &     -0.0588         &     -0.0628         &     -0.0612         \\
                    &    (0.0118)         &   (0.00944)         &    (0.0677)         &    (0.0307)         &    (0.0451)         &    (0.0337)         &    (0.0496)         \\
[1em]
Hispanic            &     -0.0315\sym{**} &     -0.0165         &      0.0220         &     -0.0478\sym{*}  &     0.00187         &    -0.00452         &      -0.144\sym{*}  \\
                    &    (0.0100)         &    (0.0138)         &    (0.0261)         &    (0.0205)         &    (0.0469)         &    (0.0250)         &    (0.0581)         \\
[1em]
Asian               &     -0.0310\sym{**} &     -0.0357\sym{*}  &      -0.104\sym{*}  &      -0.104\sym{**} &     -0.0161         &     -0.0550\sym{**} &     -0.0645         \\
                    &    (0.0108)         &    (0.0141)         &    (0.0459)         &    (0.0315)         &    (0.0618)         &    (0.0186)         &    (0.0579)         \\
\hline
\textit{Fixed Effects:}&                     &                     &                     &                     &                     &                     &                     \\
Location Controls   &         Yes         &         Yes         &         Yes         &         Yes         &         Yes         &         Yes         &         Yes         \\
Property Controls   &         Yes         &         Yes         &         Yes         &         Yes         &         Yes         &         Yes         &         Yes         \\
Host Controls       &         Yes         &         Yes         &         Yes         &         Yes         &         Yes         &         Yes         &         Yes         \\
\hline \vspace{-1.25em}&                     &                     &                     &                     &                     &                     &                     \\
Observations        &       16824         &       14765         &        3635         &        3255         &        2562         &        2285         &        1747         \\
Adjusted R2         &       0.754         &       0.735         &       0.722         &       0.730         &       0.675         &       0.675         &       0.773         \\

		\hline\hline
		\multicolumn{8}{l}{\footnotesize Standard errors in parentheses}\\
		\multicolumn{8}{l}{\footnotesize \sym{*} \(p<0.05\), \sym{**} \(p<0.01\), \sym{***} \(p<0.001\)}\\
	\end{tabular}
\label{table:robustcity}

	\begin{tablenotes}
		\item {\it Note:} This table breaks down the effects for the combined data in Table \ref{table:price} across the 7 cities in the sample. Each set of coefficients represents the coefficient on host race, with log price as the outcome variable. I control for my preferred specification throughout. Low number of observations for Black, Hispanic, and Asian hosts contribute to imprecise estimates in smaller cities (New Orleans, Nashville have less than 100 Hispanic and Asian hosts; DC and Austin have less than 200 such hosts). 
	\end{tablenotes}
\end{table}


% Table 10
\begin{landscape}
	\begin{table}[htbp]\centering
		\def\sym#1{\ifmmode^{#1}\else\(^{#1}\)\fi}
		\caption{Robustness Listing Characteristics}
		\begin{tabular}{l*{9}{c}}
			\hline\hline
			                    &\multicolumn{1}{c}{(1)}&\multicolumn{1}{c}{(2)}&\multicolumn{1}{c}{(3)}&\multicolumn{1}{c}{(4)}&\multicolumn{1}{c}{(5)}&\multicolumn{1}{c}{(6)}&\multicolumn{1}{c}{(7)}&\multicolumn{1}{c}{(8)}&\multicolumn{1}{c}{(9)}\\
                    &\multicolumn{1}{c}{Low \$ LA}&\multicolumn{1}{c}{High \$ LA}&\multicolumn{1}{c}{Low \$ NY}&\multicolumn{1}{c}{High \$ NY}&\multicolumn{1}{c}{Older Listings}&\multicolumn{1}{c}{Newer Listings}&\multicolumn{1}{c}{Apartments}&\multicolumn{1}{c}{Condos}&\multicolumn{1}{c}{Houses}\\
\hline
Black               &     -0.0278\sym{*}  &     -0.0360         &      0.0154         &     -0.0597\sym{***}&     -0.0337\sym{*}  &     -0.0364\sym{***}&     -0.0246\sym{**} &     0.00348         &     -0.0441\sym{*}  \\
                    &    (0.0127)         &    (0.0302)         &   (0.00952)         &    (0.0156)         &    (0.0156)         &   (0.00901)         &   (0.00902)         &    (0.0407)         &    (0.0179)         \\
[1em]
Hispanic            &     -0.0293\sym{**} &     -0.0491         &     -0.0152         &    -0.00604         &     -0.0349\sym{*}  &     -0.0170         &     -0.0217\sym{*}  &     -0.0331         &     -0.0380\sym{*}  \\
                    &    (0.0104)         &    (0.0291)         &    (0.0190)         &    (0.0150)         &    (0.0153)         &   (0.00970)         &   (0.00925)         &    (0.0441)         &    (0.0170)         \\
[1em]
Asian               &     -0.0301\sym{**} &     -0.0579\sym{*}  &     -0.0370\sym{*}  &     -0.0354\sym{*}  &     -0.0252\sym{*}  &     -0.0408\sym{***}&     -0.0388\sym{***}&     -0.0575         &     -0.0469\sym{**} \\
                    &    (0.0109)         &    (0.0227)         &    (0.0170)         &    (0.0152)         &    (0.0119)         &    (0.0113)         &    (0.0100)         &    (0.0387)         &    (0.0145)         \\
\hline
Location Controls   &         Yes         &         Yes         &         Yes         &         Yes         &         Yes         &         Yes         &         Yes         &         Yes         &         Yes         \\
Property Controls   &         Yes         &         Yes         &         Yes         &         Yes         &         Yes         &         Yes         &         Yes         &         Yes         &         Yes         \\
Host Controls       &         Yes         &         Yes         &         Yes         &         Yes         &         Yes         &         Yes         &         Yes         &         Yes         &         Yes         \\
\hline \vspace{-1.25em}&                     &                     &                     &                     &                     &                     &                     &                     &                     \\
Observations        &       13005         &        3819         &        8271         &        6494         &        9846         &       25882         &       28408         &        1854         &       13509         \\
Adjusted R2         &       0.560         &       0.544         &       0.428         &       0.525         &       0.770         &       0.763         &       0.684         &       0.786         &       0.795         \\

			\hline\hline
			\multicolumn{10}{l}{\footnotesize Standard errors in parentheses}\\
			\multicolumn{10}{l}{\footnotesize \sym{*} \(p<0.05\), \sym{**} \(p<0.01\), \sym{***} \(p<0.001\)}\\
		\end{tabular}
\label{table:robustlisting}
	
		\begin{tablenotes}
			\item {\it Note:} This table breaks the effects for the combined data by high versus low price, time on market, and property type. The categories, from left to right, are: listings whose log price is below vs. above the mean predicted log price in each city, the price originally dropped, listings who have have been on the market for no more than 2 years vs. no more than 8 years, and listings of different property types, including apartments (includes apartments and lofts), condos (includes condos and townhouse), and houses. I control for my preferred specification throughout. The outcome variable is the log price of the listing.
		\end{tablenotes}
	\end{table}
\end{landscape}

% Table 11
\begin{landscape}
	\begin{table}[htbp]\centering
		\def\sym#1{\ifmmode^{#1}\else\(^{#1}\)\fi}
		\caption{Estimates of effect of host demographics on review sentiment, by reviewer demographics}
		\begin{tabular}{l *{8}{c}}
			\hline\hline
			&\multicolumn{8}{c}{Reviewers} \\
			\cmidrule(r){2-9}\\
			                    &\multicolumn{1}{c}{(1)}&\multicolumn{1}{c}{(2)}&\multicolumn{1}{c}{(3)}&\multicolumn{1}{c}{(4)}&\multicolumn{1}{c}{(5)}&\multicolumn{1}{c}{(6)}&\multicolumn{1}{c}{(7)}&\multicolumn{1}{c}{(8)}\\
                    &\multicolumn{1}{c}{White M}&\multicolumn{1}{c}{White F}&\multicolumn{1}{c}{Black M}&\multicolumn{1}{c}{Black F}&\multicolumn{1}{c}{Hispanic M}&\multicolumn{1}{c}{Hispanic F}&\multicolumn{1}{c}{Asian M}&\multicolumn{1}{c}{Asian F}\\
\hline
White Female        &     -0.0780         &      0.0371         &     -0.0475         &       2.352\sym{***}&      -0.326         &      -0.912\sym{***}&       0.139         &      0.0114         \\
                    &    (0.0733)         &    (0.0471)         &    (0.0774)         &     (0.435)         &     (0.276)         &  (2.58e-13)         &     (0.356)         &     (0.209)         \\
Black Male          &      -0.175         &      -0.164         &     -0.0635         &       1.419         &      -0.172         &       1.230\sym{***}&       0.932         &      -3.941\sym{*}  \\
                    &     (0.205)         &     (0.296)         &     (0.369)         &     (0.822)         &     (1.052)         &  (1.28e-12)         &     (0.823)         &     (1.613)         \\
Black Female        &     -0.0793         &      0.0249         &      0.0551         &      -5.562\sym{***}&      0.0769         &       0.350\sym{***}&       0.379         &      0.0576         \\
                    &     (0.177)         &     (0.110)         &     (0.134)         &     (1.034)         &     (0.665)         &  (1.54e-12)         &     (0.533)         &     (0.660)         \\
Hispanic Male       &     -0.0350         &      0.0716         &      -0.337\sym{*}  &      -0.756         &       0.803         &       0.521\sym{***}&      -0.630         &      -0.572         \\
                    &     (0.104)         &     (0.135)         &     (0.131)         &     (0.545)         &     (0.618)         &  (1.63e-14)         &     (0.641)         &     (1.059)         \\
Hispanic Female     &      0.0119         &     -0.0751         &      0.0352         &       9.364\sym{***}&      -1.363         &      -1.933\sym{***}&      -1.098         &       1.345\sym{*}  \\
                    &     (0.360)         &    (0.0722)         &     (0.226)         &     (1.293)         &     (2.832)         &  (1.54e-12)         &     (0.899)         &     (0.497)         \\
Asian Male          &      -0.329         &      -0.248         &       0.211         &       9.200\sym{***}&       0.306         &       0.853\sym{***}&    -0.00444         &      -1.307         \\
                    &     (0.240)         &     (0.169)         &     (0.261)         &     (1.510)         &     (0.799)         &  (1.03e-12)         &     (1.091)         &     (1.864)         \\
Asian Female        &      -0.282         &      -0.269         &      -0.388         &       13.96\sym{***}&       0.985         &      -0.960\sym{***}&      -0.986         &      -0.725         \\
                    &     (0.167)         &     (0.147)         &     (0.228)         &     (1.832)         &     (0.609)         &  (1.80e-12)         &     (0.893)         &     (0.546)         \\
\hline
Location Controls   &         Yes         &         Yes         &         Yes         &         Yes         &         Yes         &         Yes         &         Yes         &         Yes         \\
Property Controls   &         Yes         &         Yes         &         Yes         &         Yes         &         Yes         &         Yes         &         Yes         &         Yes         \\
Host Controls       &         Yes         &         Yes         &         Yes         &         Yes         &         Yes         &         Yes         &         Yes         &         Yes         \\
\hline \vspace{-1.25em}&                     &                     &                     &                     &                     &                     &                     &                     \\
Observations        &        2665         &        2527         &        1737         &         121         &         171         &          27         &         198         &         142         \\
Adjusted R2         &      0.0504         &      0.0454         &      0.0657         &       0.826         &       0.622         &       0.970         &       0.500         &       0.685         \\
	
			\hline\hline
			\multicolumn{9}{l}{\footnotesize Standard errors in parentheses}\\
			\multicolumn{9}{l}{\footnotesize \sym{*} \(p<0.05\), \sym{**} \(p<0.01\), \sym{***} \(p<0.001\)}\\
		\end{tabular}
	\label{table:sentiment}
	
		\begin{tablenotes}
			
			\item {\it Note:} This table measures the quality of a review that reviewers leave for hosts in Chicago. The columns are the demographics of the reviewers (male is \say{M}, female is \say{F}), and the rows are the demographics of the host. The outcome variable is the sentiment of the review. Each coefficient is the standardized sentiment of a review. Review sentiment measures how positive or negative the review is. Reviews that are numerically positive are of positive sentiment and numerically negative are negative sentiment, relative to the mean sentiment score for each host type. The unit of observation is a single review. The data is a subsample of the Chicago hosts and their reviewers. I control for my preferred specification throughout. 
			
		\end{tablenotes}
		
	\end{table}
\end{landscape}



% Table 12
\begin{table}[htbp]\centering
	\def\sym#1{\ifmmode^{#1}\else\(^{#1}\)\fi}
	\caption{Estimates of effect of host's race and gender on yearly revenue}
	\begin{tabular}{l*{4}{c}}
		\hline\hline
		                    &\multicolumn{1}{c}{(1)}&\multicolumn{1}{c}{(2)}&\multicolumn{1}{c}{(3)}&\multicolumn{1}{c}{(4)}\\
                    &\multicolumn{1}{c}{Model 1}&\multicolumn{1}{c}{Model 2}&\multicolumn{1}{c}{Model 3}&\multicolumn{1}{c}{Model 4}\\
\hline
White Female        &      -199.0\sym{***}&      -156.5\sym{***}&      -151.9\sym{***}&       92.25         \\
                    &     (48.54)         &     (46.80)         &     (39.72)         &     (296.6)         \\
[1em]
Black Male          &      -655.0\sym{***}&      -329.5\sym{***}&      -261.8\sym{***}&       290.8         \\
                    &     (98.27)         &     (96.15)         &     (59.77)         &     (608.6)         \\
[1em]
Black Female        &      -814.7\sym{***}&      -365.0\sym{***}&      -319.5\sym{***}&       575.1         \\
                    &     (96.68)         &     (78.69)         &     (51.94)         &     (763.7)         \\
[1em]
Hispanic Male       &      -209.0         &      -44.57         &      -25.43         &      -328.9         \\
                    &     (112.5)         &     (97.48)         &     (88.24)         &     (766.3)         \\
[1em]
Hispanic Female     &      -280.8\sym{*}  &      -79.57         &      -118.8         &      3219.4         \\
                    &     (140.1)         &     (120.6)         &     (108.1)         &    (3499.2)         \\
[1em]
Asian Male          &      -360.5\sym{**} &      -95.81         &      -15.85         &       180.6         \\
                    &     (129.4)         &     (115.4)         &     (88.42)         &    (1378.4)         \\
[1em]
Asian Female        &      -676.6\sym{***}&      -329.2\sym{***}&      -183.7\sym{**} &     -1684.3         \\
                    &     (98.12)         &     (74.92)         &     (62.31)         &     (866.0)         \\
[1em]
Constant            &      2301.2\sym{***}&      3975.9\sym{***}&      1097.5\sym{***}&     -6935.1         \\
                    &     (109.2)         &     (36.27)         &     (169.1)         &    (4465.5)         \\
\hline
Location Controls   &                     &         Yes         &         Yes         &         Yes         \\
Property Controls   &                     &                     &         Yes         &         Yes         \\
Host Controls       &                     &                     &                     &         Yes         \\
\hline \vspace{-1.25em}&                     &                     &                     &                     \\
Observations        &       45072         &       45072         &       45072         &         356         \\
Adjusted R2         &     0.00628         &      0.0959         &       0.361         &       0.538         \\

		\hline\hline
		\multicolumn{5}{l}{\footnotesize Standard errors in parentheses}\\
		\multicolumn{5}{l}{\footnotesize \sym{*} \(p<0.05\), \sym{**} \(p<0.01\), \sym{***} \(p<0.001\)}\\
	\end{tabular}
\label{revenue}
	\begin{tablenotes}
		\item {\it Note:} The dependent variable is a constructed measure of yearly host revenue, as measured by (price * number of reviews per month * 12) for each listing. The omitted category for race is White males, so all coefficients are relative to that group. The unit of observation is an Airbnb listing, so hosts who have multiple listings are treated separately each time. The sample is the sample of listings across 7 US cities. The specification is the same as Table \ref{table:price}.
	\end{tablenotes}
\end{table}
