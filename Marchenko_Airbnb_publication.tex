\documentclass[11pt, oneside]{article}   	
\usepackage[margin = 1 in]{geometry}                		
\geometry{letterpaper}                   		
\edef\restoreparindent{\parindent=\the\parindent\relax} % this and the next 2 lines parskip AND indent
\usepackage{parskip}
\restoreparindent
%\usepackage[parfill]{parskip}    %Begin paragraphs with an empty line rather than an indent
\usepackage{graphicx}				
\usepackage{setspace}								
\usepackage{amssymb}
\doublespacing
%\geometry{footnotesep=2\baselineskip}
\interfootnotelinepenalty=10000 %prevents long footnotes from overflowing to new page
\usepackage{longtable}
\usepackage{caption} %Put caption* in a table to remove the enumeration
\usepackage{rotating} % To rotate the reviewer table
\usepackage{natbib}   %activates bibliography
\usepackage [english]{babel} %something for bibliography
\usepackage{verbatim} %to be able to use \begin{comment}
\usepackage{booktabs}
\usepackage{array}
\usepackage{float} 
\usepackage{pdflscape} %allows for one page to be landscape (lscape makes word landscape, pdflscape makes page landscape)
\usepackage{threeparttable} %allows tablenotes
\usepackage{amsmath}
\usepackage{import} % to use import commands relative to the file path
\usepackage[autolanguage]{numprint} % prints numbers with thousandths separator 
\usepackage{dirtytalk} % quotations shortcut

\begin{document}
		
	\title{The Impact of Host Race and Gender on Prices on Airbnb\footnote
		{I thank Steven D. Levitt for valuable advice. Kirthi Bellamkonda, Jacob Dorn, Michal Dzitko, Michael Galperin, Sonia Jaffe, Ezra Karger, Sylvia Klosin, Victor Lima, and Kotaro Yoshida provided helpful feedback and comments. Abdallah Aboelela, Melody Jih, and Thiago Resende provided excellent research assistance. I thank Mahathi Ayyagari, Melody Jih, Fong Chai, Lydia Sum, Joseph Day, Leah Umanskiy, Cristian Raygoza, Putter Thepkanjana, Spencer Kang, and Tony Song for coding host demographics. All remaining errors are my own. Data was provided by InsideAirbnb.com, founded and maintained by Murray Cox. Airbnb was not consulted in this research.}}
	\author{Anya Marchenko\footnote{University of Chicago, Becker Friedman Institute}}
	\maketitle
	
	\begin{abstract}
		This study investigates the impact of host race and gender on Airbnb property prices. I use an existing dataset of Airbnb listings and visually inspect 70,000 host profile pictures to code host demographics. I estimate that Asian hosts earn 4 -- 5\%, and Black male hosts 3\%, less than White males for the same type of property. However, controlling for more observables weakens the effects, requiring a cautious interpretation of these point estimates. I use two proxies for the number of bookings a listing has to estimate whether a demand or supply shift is responsible for the price disparity. I find that despite the lower prices they charge for listings, minority hosts face lower demand. These findings are consistent with, but not conclusive of, the presence of discrimination. 
		\\\\
		\textit{Keywords}: Airbnb; Discrimination; Race; Online marketplace
		
	\end{abstract}
	%		 the point estimates suggest a price disparity, 
	
	\newpage
	
	\singlespacing
	\section{Introduction}
	
African-Americans experience pervasively worse outcomes in the housing market as a result of historic and current racial discrimination \citep{krysan}. Even after the gains during the Civil Rights Era, such as the landmark Fair Housing Act of 1968, discrimination in the housing market is widely documented by social scientists. African-American renters are told that there are 30\% fewer available housing units than white renters \citep{yinger1}. African-American families face higher barriers when raising capital to purchase a home \citep{pope}. Emails sent to landlords from home-seekers with African-American-typical names receive lower response rates than emails sent by those with names commonly associated with whites \citep{hanson}.%
	\footnote{Some studies have found evidence that discrimination in rental markets is statistical in nature (that is, landlords use race as a proxy for income). For African-Americans who imply that they are of a higher social class when applying for an apartment, discrimination is virtually not present \citep{hanson}.}

%\citep{yinger1}
%Landlords renting out apartments discriminate both because of their own prejudice and in response to the prejudice of their white renters \citep{ondrich}.

Economists have primarily studied discrimination against African-American tenants. There is little research on the other side of the market - when African-Americans are supplying, rather than demanding, housing. Because property ownership cannot be randomized, it is difficult to disentangle true discrimination from systematic differences in the housing owned by African-Americans and white landlords.%
	\footnote{One would also expect black landlords to fare worse than white landlords in this area as well. Properties owned by African-Americans tend to be less expensive than those owned by white Americans. The average black household still has less mean wealth than a white household \citep{oliver}. Even middle-class black and Hispanic households still live in neighborhoods with median incomes similar to those of very poor white neighborhoods \citep{reardon}.} 
THIAGO FIX THIS - FIND EXAMPLE OF DISRCRIMINATION AGAINST AA LANDLORDS THAT'S NOT THIS FOLLOWING THING Some studies have found anecdotal evidence of discrimination against African-American landlords. For instance, the spike in subprime lending and the ensuing foreclosure crisis was found to be causally linked to residential segregation, and evidence suggests that specifically black residential dissimilarity and spatial isolation were important predictors of foreclosures across U.S. metropolitan areas \citep{foreclosure}.

This paper leverages the rise of sharing economies and the data and standardization their platforms provide to credibly measure discrimination against minority landlords. Airbnb is a sharing economy platform that allows people to rent out their apartment, house, or a single room to short-term lodgers. Since Airbnb simply provides an online platform for a market that already exists, it is reasonable to assume that agents will discriminate on Airbnb in a similar way that they discriminate in the real world. This means that studying discrimination in sharing economies could be an important way to learn about discrimination in traditional housing markets as well.
%Sharing economies are convenient for research in several distinct ways. First, all agents on the platform obey a uniform set of rules and observe the same set of information about one another. Second, sharing economies allow for cleaner data collection, as well as both price and a proxy for quantity demand for each listing. 

Measurements of discrimination on Airbnb are still potentially confounded by many other factors that affect a listing's price. Edelman and Luca (2014) estimated the effect of host race on the price of their listing in New York City with a small set of controls. Their results suggest that non-black hosts on Airbnb have prices roughly 12\% higher than black hosts. However, they only control for a few property characteristics, the quality of the host's reviews, and a measure of the reliability of the host. They leave out many other unobservables such as the type of listing (which is important if black hosts own single rooms and white hosts own entire houses) and proxies for the quality of the host themselves (which could also vary by race). 

In this paper, I use a previously unexploited dataset from a webscrape of Airbnb to empirically measure discrimination. Using price and quantity information for 70,000 Airbnb hosts throughout the country, I address the limitations of Edelman and Luca's research by controlling for location, additional property characteristics, comprehensive measures of listing size, and text analyses of host-written descriptions of the listing. 

I find that non-white hosts, both male and female, have lower prices than white hosts. The biggest effect is for Asian female hosts, whose prices are roughly \$9 less per day than white male hosts who own the same type of listing. The second biggest effect is for black males, whose price is lower by \$7, followed by black women and Asian men with price disparities of \$6 per day, and Hispanic females with a coefficient of \$5.%
	\footnote{This effect is statistically significant at the p $<$ .001 level for black hosts and Asian women, the p $<$ .01 level for Asian male hosts, and p $<$ .05 level for Hispanic women.} 
For Hispanic men the effect is small, around \$2, and is not statistically significant. 

I have data not just on the price, but the number of reviews and the vacancy rate of a listing, two measures of the quantity demanded. This is important because knowing both quantity demaned and price allows me to distinguish whether the price disparity between minority and white hosts on Airbnb is due to a demand shift (consistent with discrimination) or a supply shift (consistent with a difference in marginal cost for the hosts, or other hypothesis). 

Both measures of quantity demanded - the number of reviews and the vacancy rate - confirm that despite lower prices, minority hosts have a lower quantity demanded. 

%FIX ANYA WITH MORE PRECISE LANGUAGE. Quantity demanded measures are also lower for minority hosts - black females receive half as many reviews as white males, and the coefficients for black males and Asian hosts are less statistically significant, but are also negative, indicating fewer reviews for their properties as well.

My first measure of quantity demanded is the number of reviews. I find that minority hosts have either the same or lower review numbers than white hosts for a listing that spent the same amount of time on the market. Most coefficients are either roughly zero, or in the range of 1-2 reviews less than white hosts. THIAGO - FIX RESULTS DESCRIPTION PREVIOUS SENTENCE In general, my results suggest that minority hosts see a lower quantity demanded than white male hosts.%
	\footnote{It is important to keep in mind that this conclusion is only salient if the total number of reviews is a reasonable proxy for the demand of a listing. Yet, one can imagine that if reviewers systematically under-review minority hosts relative to white hosts, these groups would have lower numbers of reviews that do not necessarily represent a lower quantity demanded. There is no way to tell apart these mechanisms in my data. A recent study found that reviews left by hosts on guests’ pages can significantly reduce discrimination and render acceptance rates of guests with White-sounding names and African American-sounding names statistically indistinguishable \citep{cui}, but it remains unknown whether or not reviewers discriminate against minorities in leaving reviews \citep{ye}. If reviewers systematically under-review minority hosts, this itself could be evidence of discrimination. My working assumption is that even if not every guest leaves a review, the review proportion is similar across host race, and a lower number of reviews therefore indicates a real difference between quantity demanded of minority hosts and white hosts. }
	
One potential explanation for their lower number of reviews is that minority hosts make their listing available to guests less frequently than white hosts. A host controls how many days of the month they offer their listing for rent via an availability calendar on the listing page. When a guest books their listing, the booked days disappear from the availability calendar. Therefore, the measure of availability is actually a measure of true vacancy for the listing. If minority hosts have lower numbers of reviews, perhaps this is because they offer their listing for fewer days of the month than white hosts. To test this, I regress a listing's availability out of 30 days on host race, controlling for the preferred specification. Contrary to this hypothesis, regression results show that the listings of black hosts stay vacant 1 - 2 days per month longer than the listings of white hosts. However, there is evidence that Asian hosts do choose to make their listings available less frequently, which would contribute to their lower number of reviews. 

I also conduct several robustness checks of this result. I find that prices are lower for black hosts across all cities in my sample, while prices for Hispanic and Asian hosts are generally lower as well (with a couple exceptions in cities with a small sample size of Hispanic and Asian hosts, such as Nashville and DC), indicating that the price disparity is not driven by a single city. Secondly, I break up my sample by various listing characteristics (price, type of property, and location), and find that the price disparity holds across all types of listings. 

Thirdly, I investigate whether or not the price disparity is driven by minority hosts owning listings of worse quality, or simply being worse hosts, than whites. I consider the quality of a host's reviews as a proxy for the quality of the listing and host. I use the race and gender of the reviewer and the host to compare the sentiment (how favorable or unfavorable the review is) of the reviews that guests leave for white and for minority hosts.%
	\footnote{Since it required hand-coding, demographic information of the reviewers is only available for a randomly-chosen subset of hosts in Chicago.} 
Rather than observing that minority hosts uniformly had lower quality reviews, which the hypothesis would predict, the significance of the result was either negligible, or depended on the demographics of the reviewer and host. While there is some evidence that male reviewers tend to rate male hosts higher, there is little within-race preference between reviewers and hosts. Taken as a whole, sentiment analysis suggests that minority hosts do not have lower quality reviews. 

\subsection{About Airbnb} 
Airbnb is an online marketplace founded in 2008 that allows hosts to rent their private dwellings to guests as temporary accommodation. As of 2017, it has more than 3 million listings, more than Marriott's 1.2 million rooms worldwide \citep{aboutus}. Just like traditional hotel chains, guests on Airbnb can browse listings by city and property type, and book a stay based on prices, location, past reviews, pictures of the listing, size, and amenities. Unlike traditional hotel chains, however, hosts create a profile for themselves and a page for each listing they are renting. Each listing page includes the name and picture of the host, the reviews left by previous guests, and those guests' profile pictures. Guests can therefore infer demographic information about the host through a host's picture and name, creating the opportunity for discrimination. Fureigures 1-2 present screenshots of a listing in a Chicago neighborhood, illustrating some of the information that would be available to a potential guest. Figures 3-6 display an example of reviews, a sample host profile, and sample location maps, and can be found in the Appendix.


\subsection{Previous Literature} 
Most relevant to this paper is \cite{edelman}, the first paper to identify and measure the extent of anti-host discrimination on Airbnb. They explore the effect of host race on the price of their listing using a snapshot of roughly 3,800 New York City hosts in 2012. Controlling for several confounders that influence price, their findings indicate that non-black hosts on Airbnb have prices roughly 12\% higher than black hosts. I build on Edelman and Luca's research in several important ways. First, their sample was relatively narrow and confined to a single city. My sample includes seven large urban centers that cover each geographic region in the US. INSERT MORE DETAILS ABOUT REPLICABILITY OF SAMPLE. Discrimination in a large, cosmopolitan city with a highly diverse population such as New York might look different from discrimination in Nashville, which is more racially homogenous.%
	\footnote{MOVE DEMOGRAPHIC DETAILS TO DATA APPENDIX. According to the U.S. Census Bureau, Nashvile is 60.4\% white, 28.4\% black, 10.0\% Hispanic, and 2.5\% Asian. New York City is 44\% white, 25.5\% black, 12.7\% Asian, and 28.6\% Hispanic.} 
Second, their set of controls was limited by the relatively sparse listing information available on the Airbnb website (Airbnb has added more comprehensive data on each listing page since 2013). Their covariates only included a listing's location, the number of people the listing accommodates, the rating, the number of bedrooms, and whether or not the whole apartment is available to the guest. After confirming their results with their model, I then control for a more complete set of covariates (see Section 2.2 and Section 3.1).%
	\footnote{See Table 6 for the results of my regression using their covariates. See Section 3.1 for a discussion of my controls.} 
Most importantly, I also test alternative hypotheses for these price disparities, which Edelman and Luca are unable to address. 

\cite{becker} proposed the idea that discrimination against a group can be reflected in that group's market prices. In the Airbnb market, Becker's market discrimination would be reflected in the price that the guest (buyer) pays to the host (seller) to stay with them. If the guest (buyer) is discriminating, then given two comparable listings, they would choose not to stay in the one owned by a minority host (seller). Responding to the lowered demand curve, hosts in minority groups rationally post a lower price and, despite this, face a lower quantity demanded. 

Becker (1957) was concerned with discrimination arising from face-to-face interactions between minority and majority groups. Since then, there has been a large amount of research indicating that Becker's theory holds for people participating in online markets for labor, lending, rental, and products. In these cases, participants simply bring their prejudices online and use names and photos to discriminate. A canonical example is the \cite{bertrand} study, which found that resumes with white sounding names received 50\% more callbacks from potential employers than identical resumes with African-American sounding names. \cite{doleac} examined market outcomes when selling an iPod on various online marketplaces. In some pictures, a dark-skinned hand was holding the iPod, signaling a black seller, while in others, a light-skinned hand was holding the iPod, signaling a white seller. Hands which indicates a black seller received fewer and lower offers than white sellers. In sharing economies, a similar pattern occurs. Uber riders with distinctively African-American names experience longer wait times and more frequent cancellations than riders who use distinctively white names \citep{knittel}. A later study by \cite{edelman2} found a similar result: guests with distinctively African-American names receive 16\% fewer responses from Airbnb hosts than those with white names. These examples suggest that users of online platforms transfer their biases from the real world into the online world.  

\begin{comment}
Structurally, African-Americans were denied Federal Housing Adminstration mortgages at the low interest rates that were offered to white families, redlining districts , predatory , among many other structural moving north during the Great Migration were met with  Shut out from One reason is  
Many efforts have been made to curb discrimination in the housing sector against African-Americans. Landmark federal legislation such as the Fair Housing Act of 1968 prohibits housing discrimination based on race, the enforcement of anti-discrimination legislation is difficult on the local level. Residential preferences, differences in family structure and availability of affordable housing contribute to these disparities. Discrimination in housing has also been cited as one of the primary causes of these inequities. 

%Even though the accurate identification and measurement of discrimination by social scientists is vital to creating policies and statutes to combat it, measuring discrimination is difficult. Unobservable variables in the error term make it hard to isolate the effect of discrimination on the outcome variable of interest. Audit studies are one way that researchers can isolate the effect of race, sex, or other demographic on the outcome of interest. However, these types of experiments are not always possible due to the large organizational, manpower, or time costs associated with them. In the absence of an experimental set-up, regression models with a carefully chosen set of controls can aid in the accurate measurement of discrimination.  
Discrimination is difficult to measure. In the real world. Economists and other social scientists have long been concerned with the fact that minorities, especially African-Americans, have experienced pervasively lower living standards in the United States. One potential cause of this is discrimination in the housing market. 

Moreover, many small players have entered these markets who would have otherwise been unable to participate in traditional markets. Managing a room or home with Airbnb has much lower barriers to entry than being the landlord of a large apartment building. In just the 10 years since its founding, Airbnb has surpassed Marriott nearly three-fold in the number of rooms offered worldwide \citep{sharing}. 
\end{comment}

%As more people have entered these new markets, they have become increasingly dependent on the supplementary income they provide. Hosting with Airbnb, a platform that allows people to rent out their apartment, house, or single room to short-term lodgers, is one such opportunity. A 2017 report released by Airbnb states that in rural areas, hosts get as much as 5 - 20\% of their income from their listing \citep{rural}. Airbnb's fastest growing demographic of hosts, women over 60 years of age, earn \$6,000 a year on average from hosting, often relying on that income to supplement retirement savings \citep{elderly}\citep{nyt2}. 

%Some residents of areas of New York City have started relying on hosting with Airbnb to pay for rent or fund retirement \citep{nyt1}. 

%The extent to which hosts have grown to rely on Airbnb as a source of income makes discrimination on the platform a relevant topic of research. The economic consequences of discrimination are substantial - hosts who are discriminated against would face lower demand, have higher vacancies, and earn less revenue from their listing. While one previous paper found evidence of discrimination against New York City hosts using data from 2013, no other more recent or comprehensive research has been done on this type of discrimination on Airbnb. 

%In this paper, I empirically investigate the existence and extent of anti-host discrimination in Airbnb. I start by measuring the effect of host race and sex on the price of the listing and on a constructed measure of host revenue. I use data from a webscrape of around 70,000 Airbnb listings across 7 U.S. cities.\footnote{The scrape includes all of the property, host, and review information on a listing profile. To see what information would be available, see Figures 1-5 for screenshots of a sample listing. All of the information seen on the sample listing is included as variables in the data set.} For each of the 70,000 listings, the race, sex, and age of the host from their profile picture was coded. 

%Next, I construct a measure of host revenue by multiplying the price a host charges by the total number of reviews for that listing (a proxy for the quantity demanded). Using this measure of revenue, I estimate that White female hosts, Black male hosts, Black female hosts, and Asian female hosts lose about \$100-\$300 in revenue over the course of a year as compared to White male hosts who own similar listings. The exact revenue loss depends on the coefficients on price and number of reviews of a particular host.\footnote{See Table 5 and Section 3.2 for the exact effects on revenue.} These effects are statistically significant at the p $<$ .05 level or higher, and significant at the p $<$ .001 level for White females and Black females. There are also negative effects on revenue for Hispanic hosts and Asian males, but they are not significant. In Section 4, I also conduct several robustness checks and show that these results hold across various cities, price ranges, time on the market, and property types.\footnote{See Tables 6, 7 and discussion in Section 4.}

% discrimination is hard to measure, 
% Understanding discrimination in this new housing market is important because it ties into racial discrepancies in housing widely observed by economists and other social scientists. 

% However, it is difficult to separate the effect of current racial discrimination from the confounding effect of these other economic realities. It is therefore unclear to what extent current discrimination, especially in the housing market, contributes to these long-standing economic disparities. 

%Even though the accurate identification and measurement of discrimination by social scientists is vital to creating policies and statutes to combat it, measuring discrimination is difficult. Unobservable variables in the error term make it hard to isolate the effect of discrimination on the outcome variable of interest. Audit studies are one way that researchers can isolate the effect of race, sex, or other demographic on the outcome of interest. However, these types of experiments are not always possible due to the large organizational, manpower, or time costs associated with them. In the absence of an experimental set-up, regression models with a carefully chosen set of controls can aid in the accurate measurement of discrimination.  

% Economists and other social scientists have long documented the poor housing outcomes for minorities, particularly African-Americans, in the housing market.
	\label{intro}

\newpage
	\section{Data}
	\subsection{Source} 
	\label{source}
My data are taken from the website Inside Airbnb, which provides cleaned and aggregated data on Airbnb listings in 43 cities across the world \citep{insideairbnb}. The data provided on the website are sourced from a webscrape of publicly available information on the Airbnb website. Inside Airbnb is not run by or affiliated with Airbnb itself.%
	\footnote{Airbnb's host profiles and listings are publicly available information, and no private data was accessed in the scrape. The cleaned data is under a Creative Commons Public Domain Dedication.} 
The intent of Inside Airbnb is to inform the public on how Airbnb competes with the residential housing market in their areas. 

The scrape of the Airbnb website was conducted throughout 2015, and provides a point-in-time snapshot of all of the listings available in a particular city. This includes all of the information that would be available to an Airbnb user browsing through listings at the time of the scrape.%
	\footnote{Inside Airbnb provides some time-series information on prices, but since each listing's price was not scraped daily, there are often week-long or month-long gaps in the time-series price data. A cursory glance at the time-series prices reveals that hosts do not change prices often, and if they do, they often reflect predictable weekend or holiday seasonality. There is therefore reason to believe that the prices posted at the time of the scrape are representative of a listing's price throughout the year. Because of the incompleteness of the time-series data set, I focus on the cross-sectional data for the main analysis.} 

A total of 70,000 host pictures across seven US cities were coded - Chicago, Los Angeles, New York City, Austin, Washington, D.C., and New Orleans.%
	\footnote{For every city but New York, every single Airbnb listing that existed in that city at the time of the scrape was coded. In New York, which had the most listings in the sample, half of the existing 40,000 listings were randomly chosen.} 
Large cities with racial diversity which were geographically dispersed were chosen. Time, effort, and monetary constraints prohibited the coding of all 16 US cities whose data was available on InsideAirbnb.com. This approach limits the applicability of my findings to urban areas, discounting the roughly one-fifth of Airbnb's listings which are located in rural areas.%
	\footnote{A 2017 report released by Airbnb stated that 18.4\% of all active listings are located in rural areas, and there was 138\% year-in-year growth in Airbnb guest arrivals at rural listings.} 
In addition to main host data, demographic data was also coded for 16,000 reviewers who stayed with a subset of the Chicago hosts.%
	\footnote{This represents about 23\% of the total number of reviewers in Chicago. Not all reviewers could be coded due to time and labor constraints. A random subset of Chicago hosts was chosen such that the 16,000 reviews represent all of the reviews left for those hosts. Each review has a unique reviewer id, host id, listing id, the date of the review, the review text, and the coded race, sex, and age of the reviewer.}
For those hosts in Chicago, it is thus possible to study the interaction of reviewer demographics, host demographics, and review quality.


\subsection{Data Summary}
	\label{summary}
Summary statistics of listing characteristics, host demographics, and host characteristics are displayed in Tables \ref{table:listing}-\ref{table:covariates}. Histograms of prices, number of reviews, and review sentiment are included in Figures \ref{fig:prices} - \ref{fig:sentiment}. There is significant variation in both sex and race of the hosts on Airbnb. Roughly a third of the sample are single females (38\%), and a third are single males (31\%), with the rest being couples or groups (31\%). About two-thirds of the hosts are white (64\%), and less than a tenth are black (7\%), Hispanic (5\%), or Asian (9\%).%
	\footnote{The rest of the profile pictures were either pictures of groups, pictures without a human face, or multiracial couples, all of which were put in the ``Unknown/Multiracial" category in Table \ref{table:demographics}.} 
%Black hosts in the sample are underrepresented relative to the proportion of African-Americans in the national population (13\%). Hispanic hosts are similarly very underrepresented relative to the proportion of Hispanics in the population (16\%). One explanation for this could be that people self-identify as Hispanic for census data, while Airbnb hosts were identified by RAs who might have mistakenly coded Hispanic hosts as other categories. Asian-American hosts (9\%) are overrepresented by a few percentage points relative to the 5.6\% of Asian-Americans in the national average \cite{census}. 

The prices of listings owned by white hosts are dramatically higher than those of other hosts. The mean price per night of a listing is \$178 per night for white, \$125 for black, \$160 for Hispanic, and \$131 for Asian hosts. Minority hosts also have lower median prices and lower standard deviations, indicating that not only do minority hosts own cheaper listings on average, but their listings are more concentrated around the lower mean.\footnote{The median price of a listing owned by a white hosts is \$115 per night, \$90 for black hosts, \$99 for Hispanic hosts, and \$90 for Asian hosts.} 

It is reasonable to expect that a large portion of the price differences described above are driven by differences in property characteristics. Table \ref{table:listing} shows that white hosts own the most houses and the fewest apartments or lofts. They have the most bedrooms, bathrooms, beds, and amenities in their properties. In most of these measures of property quality, the listings owned by Hispanic hosts come the closest in quality to white hosts. Both black and Asian hosts have properties with the worst listing characteristics. 

While white hosts' listings are of higher quality in terms of property characteristics, this is not the case for host characteristics. Black hosts do well in categories where the host can personally influence their desirability: responding on time, writing long descriptions, or making their listing available for more days out of the month. %They have the highest response rate at 77\%, with white and Hispanic hosts behind them at 75.6\%. They make their listings available an average of 14 days a month, a full 4 days more than white hosts. However, black hosts have the lowest acceptance rate, accepting only 36\% of guests who ask to stay with them. Hispanic hosts have the highest acceptance rate at nearly 50\%. 
Black and Hispanic hosts also do well in some of my constructed measures of ``host quality" (for example, they describe their listings using as many or more words associated with quality descriptions, such as ``airy," ``beautiful," and ``clean", as white hosts).%
	\footnote{See Figure \ref{fig:property} for an example of host-written descriptions on a real listing profile.} 
	
%The difference between white and Asian hosts increases as the fields get less prominent on the profile. At most the difference in the length of descriptions that white and Asian hosts write is 13 words. While white hosts write the longest descriptions in every host-written field, black hosts are, on average, only four words behind white hosts in this metric. Asian hosts write the shortest descriptions in every host written field. 

White hosts also have the highest number of reviews, and the highest review ratings. Airbnb designates especially experienced, highly-rated hosts as ``Superhosts". Users on Airbnb are willing to pay more to stay with a ``Superhost," which is likely due to the perception that they own listings of higher quality. Because Airbnb assigns Superhost status based on the number of stays a host has, the quality of their reviews, and their response rate, white hosts are Superhosts most frequently. %13.4\% of white hosts are Superhosts, while the next runner-up, Hispanic hosts, are at 10.8\% Superhosts.

The reviewers in Chicago have some interesting characteristics, displayed in Table \ref{table:reviewer}. The reviewers have similar gender diversity as the overall host population but significantly less racial diversity. Importantly, the measure of review quality externally assigned by Sentimentr to the text of each review generally matches up with the numeric scores reviewers gave. While all hosts have on average very positive reviews, white hosts have the most positive review sentiment, and black hosts the worst review sentiment.%
	\footnote{See Section \ref{empirical} for details of sentiment analysis. } 
%However, 67\% of reviewers are white, with only 6\% being African American and Hispanic, and 12\% Asian.

	\label{data}
	
	\section{Empirical Approach}
	% EMPIRICAL APPROACH

I use OLS to estimate the impact of host race on listing price on Airbnb, controlling for location, property, and host characteristics. My main specification is of the form:

\[ log(\text{Price}_{i,j}) = \alpha_i + \beta_1 \text{Race}_{i} \times \text{Sex}_i + \beta_2 \text{Age}_i + \beta_3 \text{Location} +  \beta_4 x_{i,j} \]

Where $log(\text{Price}_{i,j})$ is the log of host $i$'s price from their Airbnb listing $j$. For hosts with multiple listings, each listing is a separate observation. $\text{Race}_{i}  \times \text{Sex}_i$ is the interaction of the race and sex of the host, where White males are the omitted category throughout. $\text{Age}_i$ is a dummy variable for whether the host is young, middle-aged, or senior. $x_{i,j}$ is a vector of property and host controls that grows additively in each model. Together, I control for all features of the listing that are available to a potential guest, as well as additional metrics that aim to capture unobservable differences between hosts. Each column of Tables \ref{table:price} controls for all covariates in the previous columns, plus a new set of covariates, as detailed below. All other tables control only for the full specification, Model 4. Standard errors are clustered by neighborhood throughout.

Table \ref{table:price} presents OLS estimates of the effect of host race and gender on the listing price according to the following models. 

\begin{enumerate}
	\item \textit{Model 1} presents the raw effect of host race and sex on the price of a listing. These coefficients are consistent with the mean listing prices by race presented in Table \ref{table:listing_summary}, but further broken down by male and female hosts within each racial category.
	
	\item \textit{Model 2} adds city and neighborhood fixed effects, as well as zip code-level Census information on demographics and various indicators of economic health. 
	
	Adding neighborhood fixed effects removes any variation in prices that are due to a property's location in a neighborhood with better amenities or proximity downtown. However, if Airbnb listings are clustered in certain areas of each city, neighborhood controls might not be very informative. To this end, I add data on the property values, proxies for the desirability of the neighborhood, including the unemployment rate, the occupancy rate, and other measures of poverty. See full details in Section 2. I also include population density and commuting time to work as proxies for distance to downtown. 
		
	\item \textit{Model 3} adds controls for listing-specific characteristics, as detailed in Section 2. See Table \ref{table:listing_summary} for a full list of property controls. I also control for the listing's duration on the market by proxying with fixed effects for the month and year of the listing's first review.
	
	\item \textit{Model 4} represents my full specification. I add all remaining host-dependent fields on the listing page, such as Superhost status, the host's response time, and their cancellation policy. I also include my constructed host quality controls in the form of sentiment analysis of the text on the listing page to control for hosts who write more objective descriptions of more positive valence. See Table \ref{table:host_summary} for a full list of these controls and Section 2 for details of construction of host effort variables. 
\end{enumerate}





	\label{empirical}
	
	\section{Results}
	\label{result1}
	

Table \ref{table:price} presents OLS estimates of the effect of host race and gender on the listing price according to the models described in Section \ref{empirical}. 

Model 1 is a naive regression and presented solely to show baseline effects of host race and gender on price. Model 2 adds the property's location. Since a listing's price is strongly correlated with its location, it is unsurprising that a large amount of variation in Airbnb prices between racial groups can be explained by these location controls. This is consistent with the idea that individuals of a particular race are not randomly distributed across neighborhoods of a given city for a variety of reasons.%
	\footnote{These include persistent urban segregation and in-group preferences [MELODY CITE HERE].}
For example, Black hosts had the largest drop in point estimate with the addition of location controls. Consistent with this result, it is well-documented that Blacks in urban populations are nearly four times more likely than Whites to live in neighborhoods where the poverty rate is 40\% or higher \citep{firebaugh}. The effects for Asian hosts behave similarly. While this may be surprising, as urban Asian populations [MELODY - is there a paper or stat about urban asian poverty? Like something that shows that asian americans live in worse neighborhoods that white people?can you cite  it, and insert a sentence here  like "despite the stereotype of asian americans as the model  minority, there are persistent gaps in the income/wealth/quality of neighborhoods ofasians and whites.]

Property characteristics have the most explanatory power in accounting for price differences, as the $R^2$ increases from around .25 to .72 with the addition of listing controls in Model 3. Relative to apartments, houses are around 10\% more expensive, bed and breakfasts are 15\% more expensive, and condos are 7\% more expensive (as opposed to tents, which are 100\% cheaper than apartments!). Relative to renting out the entire home, private rooms and shared room are 50\% and nearly 100\% less expensive, respectively. SOME SHIT. 

My results are stable to the addition of host chara`cteristic controls while still clustering standard errors at the neighborhood level. The inclusion of host characteristics does not improve the fit of the model substantially, increasing the $R^2$ by only .004. It is therefore unlikely that adding more sophisticated measures of host quality or effort would significantly help explain price disparities. 

The results in Model 4 indicate that Asian hosts, both male and female, earn lower prices from their Airbnb listing than White hosts. These effects are 4\% for Asian females and 4.5\% for Asian male hosts. The second biggest effect is for Black males, whose prices are roughly 3.5\% lower.%
	\footnote{This effect is statistically significant at the p $<$ .001 level for Asian female and male hosts, and the p $<$ .01 level for Black male hosts.} 
The estimates are negative but not statistically significant for Hispanic and Black female hosts. Overall, my point estimates suggest an effect, but as I control for more observables, these effects are weakening. This requires a cautious interpretation of these point estimates. 

These effects of Asian host race on listing price are lower than measured by \cite{wang} (a 20\% disparity) and \cite{kakar} (an 8\% disparity). This is unsurprising, as it is unlikely that we would measure the same disparity when there are important structural differences between this paper and previous literature. My sample is national, while both previous papers considered Airbnb listings in the Bay Area only. There are also vast difference in our sample sizes: \cite{wang} has a sample size of 101 observations, while \cite{kakar} has 715 observations. Similarly, the effects of Black host race that I measure are smaller than those of \cite{edelman}, who measured a price disparity of 12\%. But for similar reasons, this comparison is uninformative. I can confirm their results using my data, with the results presented in Table \ref{table:edelman_new}. 

My large sample size and [presence of reviewer data] gives me enough power to include more comprehensive controls to account for potential omitted variable bias, such as demographics of neighborhood, occupancy rates, the sentiment of the descriptions and reviews. 

In short, moving from left to right in the table, the coefficients become less negative. This raises the question of whether or not if I controlled for all unobservables, it would further reduce my estimates. I follow the logic of Oster (2017) and Altonji, Elder, and Taber (2005) to explore this question [MELODY ADD THESE CITES TO BIB].

I use the results from Model 3 and Model 4 in my calculations. The change in estimates from Model 1 to Model 2 is largely not relevant to the question of whether there is a price disparity on Airbnb and if it could be caused by discrimination. This is because the location controls account for a different kind of variation - the fact rental prices are highly correlated with location, and cities are highly segregated by race. 

%The move from Model 2 to Model 3 is similar, in the sense that Model 3 adds objective property qualities. The room for omitted variable bias is therefore in Model 4, 

I therefore use Model 2 and Model 3 as my comparison. Under the most extreme assumption that all of the variation in the outcome variable can be attributed to unobservables, indeed, I can no longer reject the Null hypothesis of no price disparity. However, as Oster (2017) notes, this is a very high standard. In her review of papers published in the American Economic Review, Journal of Political Economy, and the Quarterly Journal of Economics, only 40\% of the positive results for non-randomized studies survive this standard, and 30\% of randomized results. 

Using the change in coefficients going from Model 2 to 3, I lose statistical significance at Oster’s maximum $R^2$ of XX. It is worth noting that the pattern observed in my paper is remarkably similar to \cite{kakar}. For example, comparing Column 3 to 6 in Table 3 of their paper, their point estimate implied by Oster's analysis is 0. 






There are several possible sources of unexplained variation that could remain. One is the quality of the text, such as the reviews and descriptions. Another potential source is the quality of pictures that the host takes, both as a profile picture and of the listing. 

I control for all possible property-specific variables that Airbnb shows on the listing page, as well as extension demographic, property values, and occupancy controls, so it is unlikely that there remain important property characteristics that drive the price differences. 











I also break up the effects of host race on listing price by city and controlled for my preferred specification. The results are in Table \ref{table:robustcity}. In general, no single city is driving all of the variation in my data. The effects on price are mostly negative for minority hosts, with a few positive coefficients in cities with fewer observations, none of which are significant. 

I use two different measures of quantity demanded - the number of reviews and the vacancy rate. Both confirm that Black hosts and White females have lower quantity demanded than White males. The results for Asian and Hispanic hosts are mixed. 

My first measure of quantity demanded is the number of reviews. In Table \ref{table:num_reviews}, I regress the number of reviews on host race, controlling for the same set of models as Table \ref{table:price} (including the listing's time on the market, an important driver of the number of reviews). I find that minority hosts have either the same or lower review numbers than White hosts for a listing that spends the same amount of time on the market. Specifically, Black hosts, White females, and Hispanic Males have 7 to 8\% fewer reviews than White males. Coefficients are roughly zero for Hispanic females and Asian hosts. 

A second proxy for quantity demanded is the number of days per month a listing remains vacant. Minority hosts may have a lower quantity demanded because they offer up their listing for fewer days of the month, not because they face lower demand. In order to test this, I regress the availability of the listing on host race, controlling for my preferred specification. The availability of a listing is a measure of vacancy for the following reason: availability is controlled by the host, who can update their availability calendar on their listing page. Potential guests can then see on which days the listing is available and book accordingly. When a guest books an available day, that day is removed from the availability calendar. Therefore, the availability out of 30 days measure is a true measure of the vacancy of a listing. It is  important to note that days could be unavailable on a listing's calendar for two reasons: either a host marks them as unavailable, or a guest books on that day. Therefore, if a host has few vacancies, it is not possible to tell if they have high demand, or simply no time to manage their listing. This is problematic only if minority and White hosts mark days as unavailable at different rates, a plausible conclusion if 

Vacancy rates can also be affected by the number of properties a host owns - if a host does not live in the property they rent out, they might make more days available on their calendar. If White hosts are more likely to own second or third properties, they would have more availability on their calendars, which would underestimate the gap in availability between White hosts and minority hosts.  

The results, presented in Table \ref{table:availability}, are striking. I find that the listings of Black hosts spend about 20\% more time vacant on the market than the listings of White males. The effect is statistically significant, and amounts to about 2 - 3 days per month in real units. Interestingly, White females make their listing less available than White males, with a statistically significant difference of one day. Asian females are similar to White females, making their listing available one day less than White males. 

Overall, there is suggestive evidence to show that even though Black hosts offer their listings for more days and charge lower prices, fewer guests stay with them. Controlling for occupancy rates mitigates some, but not all, concerns that minority hosts are less likely to have a second property that they can rent out full time, increasing vacancy rates. 

Female Asian and female White hosts, on the other hand, have lower vacancies than White hosts. Lower availability is therefore a possible explanation for why these groups have a lower number of reviews. Unfortunately, I am not able to further distinguish between the effect of lower availability and potential discrimination in my data. 















	\label{results}
		
	\section{Conclusion \& Further Work}
	
Long after the passage of anti-discrimination laws in the housing sector, pervasive disparities remain between minority and white landlords. In the absence of an experiment that randomizes property ownership, it is difficult for economists to measure the extent of discrimination empirically. In this paper, I measure the minority-white price disparity in the P2P short-term housing market of Airbnb. I find that Asian hosts earn 5\% less per day, Black hosts 3\%, and Hispanic hosts 2.5\%, for the same type of listing as White hosts. Since prices only matter to hosts to the extent that they affect revenue, I calculate the impact of the price disparity on hosts' annual revenue, presented in Table \ref{table:revenue}. The biggest yearly revenue loss in the entire sample is for Black females, who could expect to earn around \$350 per year less than a White male operating the same listing, and Black males, who would lose about \$300 throughout the year. 

Airbnb itself can do much to address issues of discrimination on the platform. In response to media outcry about allegations of discrimination, Airbnb updated its Discrimination Policy in September 2016, increasing instant bookings (the opportunity for guests to book without waiting for host approval) and making host profile pictures smaller. Evaluating Airbnb's efforts to address discrimination is therefore a relevant extension of this research. Since InsideAirbnb.com is continually being updated, there is now data available from webscrapes of listings after Airbnb's new discrimination policy took effect in September 2016. Future work can explore whether the policy helped curb discrimination on the platform by measuring the extent of discrimination before and after the policy took effect. If user interface really does influence the extent of discrimination in Airbnb, then the prices of minority and white hosts should start converging. 


%Sharing economies platforms create a particularly complex environment for regulating discrimination. On the one hand, agents are constrained by certain features of the user interface - if Airbnb never provided guests with a picture or the name of the host, there would be little opportunity to discriminate. On the other hand, users ultimately have nearly full control of the transactions they engage in. For example, drivers on Uber can choose not to accept certain trips or turn the app on or off at their convenience. 
	\label{conclusion}
	\newpage
	
	\bibliographystyle{elsarticle-harv}
	\bibliography{bibliography2}
	
	\newpage
	\begin{figure}[!ht]\centering
	\includegraphics[width=.8\textwidth]{figures/sample1-cover}
	\caption{Sample listing page}
	\label{fig:listing}
\end{figure}

\begin{figure}[!ht]\centering
	\includegraphics[width=.8\textwidth]{figures/sample2-property}
	\caption{Listing information}
	\label{fig:property}
\end{figure}

\begin{figure}[!ht]\centering
\includegraphics[width=.8\textwidth]{figures/sample3-reviews}
\caption{Review information}
	\label{fig:reviewinfo}
\end{figure}

\begin{figure}\centering
\includegraphics[width=.9\textwidth]{figures/sample4-host}
\caption{Host information}
	\label{fig:host}
\end{figure}

\begin{figure}\centering
\includegraphics[width=.8\textwidth]{figures/sample5-location}
\caption{Location information}
	\label{fig:location}
\end{figure}

\begin{figure}\centering
\includegraphics[width=.8\textwidth]{figures/chicago_city_neighborhoods}
\caption[City of Chicago neighborhoods]{City of Chicago neighborhoods, showing level of granularity of neighborhood controls}
	\label{fig:chicago}
\end{figure}



	
	\newpage
	
% Summary Stats by Host Race: Listing Characteristics
\begin{table}[htbp]
\caption{Summary Statistics By Host Race: Listing Characteristics}
\begin{center}%
\small\begin{tabular}{l c | c | c c c c}
& \multicolumn{1}{c}{Full Data} & \multicolumn{5}{c}{Regression Sample}
\\
 \cmidrule(r){3-7}
\\
 & \multicolumn{1}{c}{Full data} & \multicolumn{1}{c}{All} & White & Black & Hispanic & Asian
\\
\hline\hline\noalign{\smallskip} 
 \textit{\textit{Outcome Variables}} & & & & & & \\ Price & 175.72 & 142.77 & 151.46 & 112.22 & 131.45 & 118.08 \\
 & (294.14) & (116.11) & (121.49) & (89.64) & (106.15) & (94.91) \\
 Number of Reviews & 17.51 & 16.87 & 17.46 & 15.39 & 16.75 & 14.23 \\
 & (31.86) & (31.23) & (32.37) & (27.52) & (29.92) & (26.77) \\
 \textit{Covariates} & & & & & & \\ \hline Property Type & & & & & & \\ \hspace{10bp}Apartments/Lofts    & 0.60 & 0.63 & 0.63 & 0.66 & 0.66 & 0.62 \\ \hspace{10bp}Townhouses/Condos   & 0.04 & 0.04 & 0.04 & 0.04 & 0.04 & 0.06 \\ \hspace{10bp}Houses                      & 0.32 & 0.30 & 0.30 & 0.27 & 0.26 & 0.30 \\ \hspace{10bp}Others                              & 0.04 & 0.03 & 0.03 & 0.03 & 0.03 & 0.03 \\Room Type &&&&&& \\ \hspace{10bp}Entire House/Apartment      & 0.58 & 0.55 & 0.58 & 0.43 & 0.51 & 0.42 \\ \hspace{10bp}Private Room                        & 0.38 & 0.42 & 0.39 & 0.50 & 0.44 & 0.53 \\ \hspace{10bp}Shared Room                         & 0.04 & 0.04 & 0.03 & 0.07 & 0.05 & 0.05 \\ Max Num. Guests & 3.44 & 3.15 & 3.24 & 2.94 & 3.06 & 2.84 \\
 & (2.41) & (2.13) & (2.15) & (2.06) & (2.15) & (2.00) \\
 Bedrooms & 1.34 & 1.26 & 1.28 & 1.19 & 1.21 & 1.18 \\
 & (0.92) & (0.80) & (0.83) & (0.69) & (0.78) & (0.72) \\
 Bathrooms & 1.30 & 1.23 & 1.24 & 1.19 & 1.21 & 1.19 \\
 & (0.69) & (0.55) & (0.56) & (0.49) & (0.52) & (0.53) \\
 Beds & 1.82 & 1.67 & 1.69 & 1.60 & 1.68 & 1.57 \\
 & (1.41) & (1.21) & (1.19) & (1.15) & (1.51) & (1.18) \\
 Cleaning Fee & 48.95 & 43.70 & 46.06 & 36.20 & 40.35 & 36.45 \\
 & (59.62) & (48.32) & (49.73) & (43.18) & (45.51) & (42.86) \\
 Extra Guests Charge & 13.74 & 13.43 & 13.26 & 15.13 & 13.94 & 12.72 \\
 & (23.65) & (22.67) & (23.00) & (22.71) & (22.48) & (20.36) \\
 Minimum Nights & 3.01 & 3.03 & 3.08 & 2.61 & 2.86 & 3.17 \\
 & (9.21) & (8.79) & (9.39) & (4.35) & (6.55) & (8.67) \\
 Availability (out of 30 days) & 11.54 & 11.04 & 10.64 & 14.19 & 11.24 & 10.79 \\
 & (10.93) & (10.91) & (10.75) & (11.49) & (10.94) & (11.01) \\
 Number of Amenities & 0.81 & 0.79 & 0.80 & 0.75 & 0.79 & 0.75 \\
 & (1.10) & (1.10) & (1.10) & (1.04) & (1.10) & (1.13) \\
 Instantly Bookable? & 0.15 & 0.15 & 0.14 & 0.21 & 0.17 & 0.16 \\
 & (0.36) & (0.36) & (0.34) & (0.41) & (0.38) & (0.37) \\
 Year of first review & 14.86 & 14.86 & 14.83 & 14.89 & 14.90 & 15.03 \\
 & (1.22) & (1.22) & (1.22) & (1.30) & (1.21) & (1.17) \\
 Strict Cancellation Policy & 0.43 & 0.40 & 0.40 & 0.41 & 0.41 & 0.40 \\\hline
Observations & \numprint{69007} & \numprint{45076} & \numprint{26391} & \numprint{3346} & \numprint{2274} & \numprint{3719}
\\
\hline\hline\noalign{\smallskip} \end{tabular} 
\begin{minipage}{6in}
{\it Note:} The values in the table are means and standard deviations of listing-level data in my full sample. Summary statistics for selected covariates are listed in the table. Categorical variables such as room type do not have standard deviations. Property types are explicitly listed if more than 1.5\% of listings are that type. Only the most popular cancellation policy type is listed - in the full sample, 99\% of listings have strict (43\%), flexible (31\%) or moderate (25\%) cancellation policies. Year of first review is a proxy for the time on the market - 14.86 indicates that the first review of the mean listing in the full sample occurred in October of 2014.
\end{minipage}
\end{center}
\end{table}

\newpage

% Summary Stats by Host Race: Host Demographics
\begin{table}[htbp]
\caption{Summary Statistics By Host Race: Host Demographics}
\begin{center}%
\small\begin{tabular}{l c | c | c c c c}
& \multicolumn{1}{c}{} & \multicolumn{5}{c}{Regression Sample}
\\
 \cmidrule(r){3-7}
\\
 & \multicolumn{1}{c}{Full data} & \multicolumn{1}{c}{All} & White & Black & Hispanic & Asian
\\
\hline\hline\noalign{\smallskip} 
 Race &&&&&& \\
 \hspace{10bp}White & 0.735 & 0.731 &     1 &    0 &     0 &    0 \\  \hspace{10bp}Black & 0.096 & 0.097 &     0 &    1 &     0 &    0 \\  \hspace{10bp}Hispanic & 0.063 & 0.065 &     0 &    0 &     1 &    0 \\  \hspace{10bp}Asian & 0.106 & 0.108 &     0 &    0 &     0 &    1 \\  \hspace{10bp}Unknown & 0.000 & {0.000} & {0} &  {0}  & {0}  & {0} \\  Sex &&&&&& \\
 \hspace{10bp}Male & 0.449 & 0.448 &  0.451 & 0.403 &  0.498 & 0.436 \\  \hspace{10bp}Female & 0.550 & 0.552 &  0.549 & 0.597 &  0.502 & 0.564 \\  \hspace{10bp}Unknown & 0.000 & {0} & {0} &  {0}  & {0}  & {0}\\  Age &&&&&& \\
 \hspace{10bp}Young($\<30$) & 0.507 & 0.513 &  0.494 & 0.542 &  0.518 & 0.612 \\  \hspace{10bp}Middle-aged & 0.469 & 0.464 &  0.478 & 0.449 &  0.471 & 0.378 \\  \hspace{10bp}Old ($/>65$) &              0.020 & 0.021 &  0.026 & 0.005 &  0.010 & 0.008 \\  \hspace{10bp}Unknown & 0.000 & {0} & {0} &  {0}  & {0}  & {0} \\ \hline
Observations & 46930 & 45076 & 32934 & 4354 & 2913 & 4875 
\\
\hline\hline\noalign{\smallskip} \end{tabular} 
\begin{minipage}{6in}
{\it Note:} The values in the table are summaries of host demographics inthe host-level data. Column 1 is the summary statistics for the full,unrestricted data set across 7 cities. Columns 2 $-$ 6 are the restricteddata used in the analysis. Column 2 is the full regression sample, andcolumns 3 $-$ 6 break down the regression sample by host race. The“Unknown” category was dropped from the regression and is therefore zerothroughout columns 2 $-$ 6. White refers only to Non-Hispanic Whites.\end{minipage}
\end{center}
\end{table}

\newpage

% Summary Stats by Host Race: Host Characteristics
\begin{table}[htbp]
\caption{Summary Statistics By Host Race: Host Characteristics}
\begin{center}%
\small\begin{tabular}{l c | c | c c c c}
& \multicolumn{1}{c}{} & \multicolumn{5}{c}{Regression Sample}
\\
 \cmidrule(r){3-7}
\\
 & \multicolumn{1}{c}{Full data} & \multicolumn{1}{c}{All} & White & Black & Hispanic & Asian
\\
\hline\hline\noalign{\smallskip} 
 \textit{\textit{Outcome Variables}} & & & & & & \\ Host Listings Count & 6.38 & 2.23 & 2.16 & 2.38 & 2.49 & 2.44 \\
 & (36.54) & (2.59) & (2.50) & (2.83) & (3.03) & (2.61) \\
 \textit{Covariates} & & & & & & \\ \hline Review scores rating & 93.56 & 93.68 & 94.18 & 91.91 & 92.80 & 92.26 \\
 & (8.13) & (7.90) & (7.33) & (9.44) & (8.71) & (9.27) \\
 Host is a Superhost & 0.13 & 0.13 & 0.14 & 0.09 & 0.11 & 0.10 \\
 & (0.34) & (0.33) & (0.34) & (0.28) & (0.31) & (0.30) \\
 Response rate & 0.77 & 0.76 & 0.76 & 0.78 & 0.76 & 0.74 \\
 & (0.38) & (0.39) & (0.39) & (0.37) & (0.39) & (0.40) \\
 Acceptance rate & 0.47 & 0.45 & 0.46 & 0.35 & 0.49 & 0.44 \\
 & (0.46) & (0.46) & (0.46) & (0.45) & (0.47) & (0.47) \\
 Polarity of Summary & 0.30 & 0.30 & 0.30 & 0.29 & 0.30 & 0.29 \\
 & (0.17) & (0.17) & (0.17) & (0.16) & (0.17) & (0.17) \\
 Subjectivity of Summary & 0.53 & 0.54 & 0.54 & 0.53 & 0.54 & 0.53 \\
 & (0.15) & (0.15) & (0.15) & (0.15) & (0.15) & (0.15) \\
 Host's Identity Verified? & 0.70 & 0.70 & 0.71 & 0.66 & 0.68 & 0.69 \\
 & (0.46) & (0.46) & (0.45) & (0.47) & (0.47) & (0.46) \\
 Guest Pic Required? & 0.04 & 0.04 & 0.04 & 0.06 & 0.04 & 0.04 \\
 & (0.19) & (0.19) & (0.19) & (0.23) & (0.19) & (0.19) \\
 Guest Phone Required? & 0.05 & 0.05 & 0.05 & 0.06 & 0.04 & 0.04 \\
 & (0.22) & (0.21) & (0.21) & (0.24) & (0.20) & (0.20) \\
 Response time $<$ 1 hour & 0.41 & 0.40 & 0.39 & 0.44 & 0.41 & 0.41 \\\hline
Observations & \numprint{69010} & \numprint{45076} & \numprint{32934} & \numprint{4354} & \numprint{2913} & \numprint{4875}
\\
\hline\hline\noalign{\smallskip} \end{tabular} 
\begin{minipage}{6in}
\label{table:host_summary}
{\it Note:} The values in the table are means and standard deviations of host-level data in the full sample. Summary statistics for selected covariates are listed in the table. Categorical variables such as response time do not have standard deviations. Statistics for only the most frequent response time (\say{within an hour}) are included. White refers only to non-Hispanic whites. Polarity of \say{Summary} and Subjectivity of \say{Summary} refer to the scores from a natural language processing algorithm that measures the sentiment and objectivity of that field. These two measures were also calculated for the description, space, neighborhood overview, notes, and transit fields, but were not included in the table for the sake of clarity and because they follow a similar pattern as the \say{Summary} field.
\end{minipage}
\end{center}
\end{table}

\newpage

% Summary Stats by Race: Reviewer Characteristics
\begin{table}[htbp]
\caption{Summary Statistics By Race: Reviewer Characteristics}
\begin{center}%
\small\begin{tabular}{l c | c | c c c c}
& \multicolumn{1}{c}{} & \multicolumn{5}{c}{Reviewer Race in \say{All} data} 
\\
 \cmidrule(r){3-7}
\\
 & \multicolumn{1}{c}{Full data} & \multicolumn{1}{c}{All} & White & Black & Hispanic & Asian
\\
\hline\hline\noalign{\smallskip} 
 Reviewer Race  & 1.00 & 1.00 & 0.66 & 0.03 & 0.04 & 0.11 \\\\
 Host race & & & & & & \\ \hspace{10bp}White &     0.73 & 0.83 & 0.84 & 0.70 & 0.75 & 0.75 \\ \hspace{10bp}Black &     0.06 & 0.06& 0.05 & 0.17 & 0.07 & 0.06 \\ \hspace{10bp}Hispanic &  0.04 & 0.05& 0.05 & 0.06 & 0.10 & 0.08 \\ \hspace{10bp}Asian &     0.05 & 0.05& 0.05 & 0.08 & 0.08 & 0.11 \\ \hspace{10bp}Unknown &   0.12 & 0.00& 0.00 & 0.00 & 0.00 & 0.00 \\\\
 Review Sentiment & 0.51 & 0.51 & 0.51 & 0.50 & 0.47 & 0.53 \\
 & (0.26) & (0.26) & (0.25) & (0.23) & (0.30) & (0.25) \\
\\
 Listing Sentiment & 0.51 & 0.51 & 0.51 & 0.50 & 0.50 & 0.51 \\
 & (0.07) & (0.07) & (0.07) & (0.07) & (0.07) & (0.09) \\
\\
\hline
Observations & \numprint{17050} &  \numprint{10573} & \numprint{6929} & \numprint{319} & \numprint{402} & \numprint{1153}
\\
\hline\hline\noalign{\smallskip} \end{tabular} 
\begin{minipage}{6in}
{Note:} This table demonstratates the summary statistics for data used in the \say{Estimates of effect of host demographics on review sentiment, by reviewer demographics} table. Column 1 contains statistics on the raw data. Column 2 contains statistics on the data used in the estimations. Columns 3 - 6 section Column 2 by reviewer race. Row 1: Reviewer race, indicates the proportion of the different races in the reviewer data. Row 2: Host race, indicates the marginal probability of a host race given reviewer race. The values in the table are means and standard deviations of reviewer-level data who left reviews for a randomly chosen set of hosts in Chicago. The review sentiment is the sentiment of each review, the listing sentiment is the average sentiment per listing. Observations in columns 2 - 5 do not add up to 17,050 because multiracial or unidentifiable reviewer pictures are excluded. White refers only to non-Hispanic whites.
\end{minipage}
\end{center}
\end{table}

\newpage

% Price
\begin{table}[htbp]\centering
	\def\sym#1{\ifmmode^{#1}\else\(^{#1}\)\fi}
	\caption{Main result: Estimates of effect of host race and gender on listing price}
	\begin{tabular}{l*{5}{c}}
		\hline\hline
		                    &\multicolumn{1}{c}{(1)}&\multicolumn{1}{c}{(2)}&\multicolumn{1}{c}{(3)}&\multicolumn{1}{c}{(4)}\\
                    &\multicolumn{1}{c}{Model 1}&\multicolumn{1}{c}{Model 2}&\multicolumn{1}{c}{Model 3}&\multicolumn{1}{c}{Model 4}\\
\hline
White Female        &     -0.0236\sym{*}  &     -0.0138         &     0.00201         &     0.00298         \\
                    &    (0.0106)         &   (0.00854)         &   (0.00496)         &   (0.00484)         \\
[1em]
Black Male          &      -0.276\sym{***}&     -0.0828\sym{**} &     -0.0360\sym{**} &     -0.0328\sym{**} \\
                    &    (0.0315)         &    (0.0259)         &    (0.0123)         &    (0.0123)         \\
[1em]
Black Female        &      -0.299\sym{***}&     -0.0586\sym{**} &     -0.0196         &     -0.0167         \\
                    &    (0.0296)         &    (0.0188)         &    (0.0102)         &   (0.00996)         \\
[1em]
Hispanic Male       &      -0.153\sym{***}&     -0.0521\sym{**} &     -0.0233\sym{*}  &     -0.0200         \\
                    &    (0.0259)         &    (0.0191)         &    (0.0113)         &    (0.0113)         \\
[1em]
Hispanic Female     &      -0.150\sym{***}&     -0.0653\sym{**} &     -0.0196         &     -0.0202         \\
                    &    (0.0280)         &    (0.0202)         &    (0.0115)         &    (0.0114)         \\
[1em]
Asian Male          &      -0.221\sym{***}&     -0.0987\sym{***}&     -0.0425\sym{**} &     -0.0446\sym{***}\\
                    &    (0.0336)         &    (0.0225)         &    (0.0134)         &    (0.0135)         \\
[1em]
Asian Female        &      -0.283\sym{***}&      -0.131\sym{***}&     -0.0409\sym{***}&     -0.0396\sym{***}\\
                    &    (0.0299)         &    (0.0161)         &   (0.00874)         &   (0.00893)         \\
[1em]
Constant            &       4.802\sym{***}&       4.979\sym{***}&       3.891\sym{***}&       4.003\sym{***}\\
                    &    (0.0300)         &     (0.398)         &     (0.343)         &     (0.344)         \\
\hline
Location Controls   &                     &         Yes         &         Yes         &         Yes         \\
Property Controls   &                     &                     &         Yes         &         Yes         \\
Host Controls       &                     &                     &                     &         Yes         \\
\hline \vspace{-1.25em}&                     &                     &                     &                     \\
Observations        &       45073         &       45073         &       45073         &       45073         \\
Adjusted R2         &      0.0263         &       0.246         &       0.716         &       0.720         \\

		\hline\hline
		\multicolumn{5}{l}{\footnotesize Standard errors in parentheses}\\
		\multicolumn{5}{l}{\footnotesize \sym{*} \(p<0.05\), \sym{**} \(p<0.01\), \sym{***} \(p<0.001\)}\\
	\end{tabular}	
\label{table:price}
	\begin{tablenotes}
		
		\item {\it Note:} This table presents the impact of host race on the price of a listing. The dependent variable is the log price. The omitted category is White males. The unit of observation is a listing. The sample is listings across seven US cities, whose prices are no more than \$800 per night, and whose hosts own no more than twenty properties. Model 1 is the baseline effect of host demographics on price. Model 2 includes fixed effects for the neighborhood of the listing and Census demographic, economic health characteristics, and occupancy rates on the zipcode-level. Model 3 adds listing characteristics such as the property type and size. Model 4 adds host characteristics such as response and acceptance rates, and measures of host effort.  
		
	\end{tablenotes}
\end{table}



% ALL MEASURES of QD
\begin{table}[htbp]\centering
	\def\sym#1{\ifmmode^{#1}\else\(^{#1}\)\fi}
	\caption{Effect of host race on two proxies of a listing's number of bookings}
	\begin{tabular}{l*{2}{c}}
		\hline\hline
		\input{code/tables/tex_output/individual_tables/quantity_demanded}
		\hline\hline
		\multicolumn{2}{l}{\footnotesize Standard errors in parentheses}\\
		\multicolumn{2}{l}{\footnotesize \sym{*} \(p<0.05\), \sym{**} \(p<0.01\), \sym{***} \(p<0.001\)}\\
	\end{tabular}
\label{table:quantity_demanded}
	\begin{tablenotes}
		
		\item {\it Note:} This table presents the effect of host race on two proxies for the quantity demanded of a listing: its number of reviews and its availability out of 30 days. The availability metric represents the number of days out of the total days available for booking that a listing is vacant. The omitted category is White males. I control for the specification in Table \ref{table:price}, Model 4.
		
	\end{tablenotes}
\end{table}



% Host demographics on review sentiment, by reviewer demographics
\begin{landscape}
	\begin{table}[htbp]\centering
		\def\sym#1{\ifmmode^{#1}\else\(^{#1}\)\fi}
		\caption{Estimates of effect of host demographics on review sentiment, by reviewer demographics}
		\begin{tabular}{l *{9}{c}}
			\hline\hline
			&\multicolumn{9}{c}{Reviewers} \\
			\cmidrule(r){3-10}\\
			                    &\multicolumn{1}{c}{(1)}&\multicolumn{1}{c}{(2)}&\multicolumn{1}{c}{(3)}&\multicolumn{1}{c}{(4)}&\multicolumn{1}{c}{(5)}&\multicolumn{1}{c}{(6)}&\multicolumn{1}{c}{(7)}&\multicolumn{1}{c}{(8)}&\multicolumn{1}{c}{(9)}\\
                    &\multicolumn{1}{c}{Full sample}&\multicolumn{1}{c}{White M}&\multicolumn{1}{c}{White F}&\multicolumn{1}{c}{Black M}&\multicolumn{1}{c}{Black F}&\multicolumn{1}{c}{Hispanic M}&\multicolumn{1}{c}{Hispanic F}&\multicolumn{1}{c}{Asian M}&\multicolumn{1}{c}{Asian F}\\
\hline
White Female        &    -0.00268         &     -0.0246         &      0.0640         &     -0.0718         &       2.215\sym{***}&      -1.032\sym{***}&      -0.898\sym{***}&       4.660\sym{***}&      -1.823\sym{***}\\
                    &    (0.0468)         &    (0.0758)         &    (0.0575)         &    (0.0902)         &  (2.03e-11)         &  (2.01e-10)         &  (2.73e-13)         &     (0.416)         &  (2.77e-08)         \\
Black Male          &      -0.172\sym{**} &      -0.176         &      -0.155         &      -0.168         &      -15.79\sym{***}&       22.87\sym{***}&       0.186\sym{***}&      -4.616         &      -18.00\sym{***}\\
                    &    (0.0621)         &     (0.192)         &     (0.312)         &     (0.459)         &  (1.20e-10)         &  (1.80e-09)         &  (5.36e-13)         &     (3.232)         &(0.000000824)         \\
Black Female        &       0.104         &     -0.0509         &      0.0863         &       0.144         &      -2.253\sym{***}&       6.886\sym{***}&       0.345\sym{***}&      -4.269\sym{***}&       3.430\sym{***}\\
                    &    (0.0685)         &     (0.185)         &     (0.126)         &     (0.195)         &  (3.76e-11)         &  (4.07e-10)         &  (1.60e-12)         &     (0.883)         &(0.000000275)         \\
Hispanic Male       &      -0.115\sym{**} &     -0.0477         &     -0.0130         &      -0.436\sym{***}&       6.431\sym{***}&       19.07\sym{***}&      -0.512\sym{***}&      -7.871\sym{***}&       6.453\sym{***}\\
                    &    (0.0411)         &    (0.0967)         &     (0.108)         &     (0.113)         &  (6.66e-11)         &  (1.11e-09)         &  (8.00e-13)         &     (1.597)         &(0.000000210)         \\
Hispanic Female     &      0.0711         &     0.00719         &      0.0117         &      0.0858         &      -37.98\sym{***}&       85.18\sym{***}&      -2.929\sym{***}&       6.073\sym{***}&      -4.928\sym{***}\\
                    &    (0.0951)         &     (0.401)         &     (0.139)         &     (0.187)         &  (2.99e-10)         &  (4.27e-09)         &  (8.01e-13)         &     (0.888)         &(0.000000109)         \\
Asian Male          &      0.0219         &      -0.281         &      -0.168         &       0.182         &       6.200\sym{***}&      -21.97\sym{***}&       0.792\sym{***}&       8.107\sym{***}&       11.59\sym{***}\\
                    &     (0.162)         &     (0.231)         &     (0.141)         &     (0.271)         &  (1.05e-10)         &  (1.28e-09)         &  (8.01e-13)         &     (0.736)         &(0.000000331)         \\
Asian Female        &      -0.147         &      -0.224         &      -0.321         &     -0.0774         &      -8.758\sym{***}&      -11.16\sym{***}&      -0.993\sym{***}&       7.884\sym{***}&      -2.325\sym{***}\\
                    &    (0.0882)         &     (0.201)         &     (0.162)         &     (0.307)         &  (6.98e-11)         &  (7.24e-10)         &  (1.60e-12)         &     (0.797)         &  (8.11e-08)         \\
\hline
Location Controls   &         Yes         &         Yes         &         Yes         &         Yes         &         Yes         &         Yes         &         Yes         &         Yes         &         Yes         \\
Property Controls   &         Yes         &         Yes         &         Yes         &         Yes         &         Yes         &         Yes         &         Yes         &         Yes         &         Yes         \\
Host Controls       &         Yes         &         Yes         &         Yes         &         Yes         &         Yes         &         Yes         &         Yes         &         Yes         &         Yes         \\
\hline \vspace{-1.25em}&                     &                     &                     &                     &                     &                     &                     &                     &                     \\
Observations        &       10573         &        2665         &        2527         &        1737         &         121         &         171         &          27         &         198         &         142         \\
Adjusted R2         &      0.0238         &      0.0548         &      0.0525         &      0.0922         &       0.838         &       0.719         &       0.970         &       0.642         &       0.786         \\
	
			\hline\hline
			\multicolumn{10}{l}{\footnotesize Standard errors in parentheses}\\
			\multicolumn{10}{l}{\footnotesize \sym{*} \(p<0.05\), \sym{**} \(p<0.01\), \sym{***} \(p<0.001\)}\\
		\end{tabular}
		\label{table:sentiment}
		
		\begin{tablenotes}
			
			\item {\it Note:} This table presents the quality of reviews that reviewers leave for hosts in Chicago. The columns are the demographics of the reviewers (male is \say{M}, female is \say{F}), and the rows are the demographics of the host, consistent with previous tables. The outcome variable is the standardized sentiment of the review, as assigned by a machine learning algorithm. Reviews that are numerically positive are of positive sentiment and numerically negative are negative sentiment, relative to the mean sentiment score for each host type. The unit of observation is a single review. The data is a subsample of the Chicago hosts and their reviewers. I control for the specification in Table \ref{table:price}, Model 4.
			
		\end{tablenotes}
		
	\end{table}
\end{landscape}


% Robustness city
\begin{table}[htbp]\centering
	\def\sym#1{\ifmmode^{#1}\else\(^{#1}\)\fi}
	\caption{Main results by city}
	\begin{tabular}{l*{7}{c}}
		\hline\hline
		\input{code/tables/tex_output/individual_tables/price_by_city}
		\hline\hline
		\multicolumn{8}{l}{\footnotesize Standard errors in parentheses}\\
		\multicolumn{8}{l}{\footnotesize \sym{*} \(p<0.05\), \sym{**} \(p<0.01\), \sym{***} \(p<0.001\)}\\
	\end{tabular}
	\label{table:robustcity}
	\begin{tablenotes}
		
		\item {\it Note:} This table estimates the main results in Table \ref{table:price} separately across the 7 cities in the sample. Each set of coefficients represents the coefficient on host race in a regression with log price as the outcome variable. Low number of observations for Black, Hispanic, and Asian hosts contribute to imprecise estimates in smaller cities (New Orleans, Nashville have less than 100 Hispanic and Asian hosts; DC and Austin have less than 200 such hosts). The omitted category is White males. I control for the specification in Table \ref{table:price}, Model 4.
		
	\end{tablenotes}
\end{table}


\begin{landscape}
	% Robustness Listing Characteristics 
	\begin{table}[htbp]\centering
		\def\sym#1{\ifmmode^{#1}\else\(^{#1}\)\fi}
		\caption{Main results by listing characteristics}
		\begin{tabular}{l*{8}{c}}
			\hline\hline
			\input{code/tables/tex_output/individual_tables/price_by_listing_type}
			\hline\hline
			\multicolumn{9}{l}{\footnotesize Standard errors in parentheses}\\
			\multicolumn{9}{l}{\footnotesize \sym{*} \(p<0.05\), \sym{**} \(p<0.01\), \sym{***} \(p<0.001\)}\\
		\end{tabular}
		\label{table:robustlisting}
		\begin{tablenotes}
			
			\item {\it Note:} This table estimates the main results in Table \ref{table:price} separately for listings of different price points, review numbers, age, and property types. The categories, from left to right, are: listings whose price is above the cutoff price in the original sample, listings of all prices, listings with more than 5 reviews, listings who have have been on the market for no more than 2 years versus no more than 8 years, and listings of different property types, including apartments (includes apartments and lofts), condos (includes condos and townhouse), and houses. The omitted category is White males. I control for the specification in Table \ref{table:price}, Model 4. The outcome variable is the log price of the listing.
			
		\end{tablenotes}
	\end{table}
\end{landscape}


%% APPENDIX REALLY 

% Edelman & Luca
\begin{table}[htbp]\centering
	\def\sym#1{\ifmmode^{#1}\else\(^{#1}\)\fi}
	
	\caption{Robustness check with controls from Edelman \& Luca (2014)}
	\begin{tabular}{l*{1}{c}}
		\hline\hline
		                    &\multicolumn{1}{c}{(1)}\\
                    &\multicolumn{1}{c}{Edelman}\\
\hline
0                   &           0         \\
                    &         (.)         \\
[1em]
Black               &      -14.68         \\
                    &     (73.64)         \\
[1em]
Hispanic            &      -8.702         \\
                    &     (73.73)         \\
[1em]
Asian               &       5.019         \\
                    &     (73.64)         \\
[1em]
Multiracial or Unknown&       6.590         \\
                    &     (73.62)         \\
\hline
Location Fixed Effects&         Yes         \\
Property Fixed Effects&         Yes         \\
Host Fixed Effects  &         Yes         \\
\hline \vspace{-1.25em}&                     \\
Observations        &       16592         \\
Adjusted R2         &       0.238         \\
 
		\hline\hline
		\multicolumn{2}{l}{\footnotesize Standard errors in parentheses}\\
		\multicolumn{2}{l}{\footnotesize \sym{*} \(p<0.05\), \sym{**} \(p<0.01\), \sym{***} \(p<0.001\)}\\
	\end{tabular}
	\label{table:edelman}
	\begin{tablenotes}
		
		\item {\it Note:} This table presents the effect on log price of controlling for \cite{edelman}'s full specification using my NYC data. The omitted category for race is White hosts. The omitted category for room type is Entire Apartment. I could not control for host social media accounts as a proxy for host reliability like \cite{edelman} did, because Airbnb no longer provides this information. Instead, I controll for ``host verified", a dummy for whether Airbnb has the host's phone number and email. I similarly can not control for ``picture quality", but picture quality did not significantly influence price in \cite{edelman}'s regression.
		
	\end{tablenotes}
\end{table}


% Robustness host listings count
\begin{table}[htbp]\centering
	\def\sym#1{\ifmmode^{#1}\else\(^{#1}\)\fi}
	\caption{Estimates of effect of host race on price, by host's listing count}
	\begin{tabular}{l*{6}{c}}
		\hline\hline
		&\multicolumn{6}{c}{Cutoff for the number of listings operated by a host} \\
		\cmidrule(r){2-7}\\
		\input{code/tables/tex_output/individual_tables/robustness_listings_count}
		\hline\hline
		\multicolumn{5}{l}{\footnotesize Standard errors in parentheses}\\
		\multicolumn{5}{l}{\footnotesize \sym{*} \(p<0.05\), \sym{**} \(p<0.01\), \sym{***} \(p<0.001\)}\\
	\end{tabular}
	\label{table:robustness_listings_count}
	\begin{tablenotes}

		\item {\it Note:} This table estimates the main results in Table \ref{table:price} separately for hosts who operate different numbers of listings. The categories, from left to right, are: listings operated by hosts who own 1 property on Airbnb, listings operated by hosts who have 2 or fewer properties on Airbnb, etc. The omitted category is White males. I control for the specification in Table \ref{table:price}, Model 4. The outcome variable is the log price of the listing.
		
	\end{tablenotes}
\end{table}




\begin{comment}

% Price, no interaction
\begin{table}[htbp]\centering
	\def\sym#1{\ifmmode^{#1}\else\(^{#1}\)\fi}
	\caption{Main result: Estimates of effect of host’s race and gender on price [Without interaction]}
	\begin{tabular}{l*{5}{c}}
		\hline\hline
		\input{code/tables/tex_output/individual_tables/price_no_int}
		\hline\hline
		\multicolumn{5}{l}{\footnotesize Standard errors in parentheses}\\
		\multicolumn{5}{l}{\footnotesize \sym{*} \(p<0.05\), \sym{**} \(p<0.01\), \sym{***} \(p<0.001\)}\\
	\end{tabular}
	\label{table:price_no_int}
	\begin{tablenotes}
		
		\item {\it Note:} The dependent variable is the log price of the listing. The omitted category for host race and gender is White hosts. The unit of observation is a listing. The sample is the sample of listings across 7 US cities. Model 1 is the baseline effect of host demographics on price. Model 2 controls for listing location to the neighborhood level and demographic and economic health characteristics on the zipcode-level. Model 3 adds listing characteristics such as the property type and size. Model 4 adds host characteristics such as response and acceptance rates and measures of host effort.  
	\end{tablenotes}
\end{table}


% Yearly revenue
\begin{table}[htbp]\centering
	\def\sym#1{\ifmmode^{#1}\else\(^{#1}\)\fi}
	\caption{Estimates of effect of host's race and gender on yearly revenue}
	\begin{tabular}{l*{4}{c}}
		\hline\hline
		                    &\multicolumn{1}{c}{(1)}&\multicolumn{1}{c}{(2)}&\multicolumn{1}{c}{(3)}&\multicolumn{1}{c}{(4)}\\
                    &\multicolumn{1}{c}{Model 1}&\multicolumn{1}{c}{Model 2}&\multicolumn{1}{c}{Model 3}&\multicolumn{1}{c}{Model 4}\\
\hline
White Female        &      -206.2\sym{***}&      -157.8\sym{**} &      -150.4\sym{**} &      -137.8\sym{**} \\
                    &     (51.41)         &     (52.02)         &     (45.74)         &     (47.49)         \\
[1em]
White Two people or Unknown&       408.9\sym{***}&       335.2\sym{***}&       123.1\sym{**} &       10.57         \\
                    &     (76.17)         &     (51.60)         &     (47.52)         &     (46.72)         \\
[1em]
Black Male          &      -710.1\sym{***}&      -334.5\sym{***}&      -232.4\sym{***}&      -138.1\sym{*}  \\
                    &     (101.6)         &     (95.95)         &     (57.99)         &     (58.38)         \\
[1em]
Black Female        &      -815.5\sym{***}&      -345.1\sym{***}&      -238.6\sym{***}&      -195.7\sym{***}\\
                    &     (102.4)         &     (87.33)         &     (54.95)         &     (52.66)         \\
[1em]
Black Two people or Unknown&      -363.9\sym{*}  &       105.7         &      -195.7         &      -224.9\sym{*}  \\
                    &     (161.7)         &     (166.3)         &     (102.8)         &     (105.1)         \\
[1em]
Hispanic Male       &      -209.8         &      -35.64         &       4.983         &       26.26         \\
                    &     (114.4)         &     (101.2)         &     (92.49)         &     (88.10)         \\
[1em]
Hispanic Female     &      -250.5         &      -59.61         &      -126.9         &      -62.61         \\
                    &     (138.1)         &     (119.0)         &     (114.5)         &     (113.7)         \\
[1em]
Hispanic Two people or Unknown&      -307.3         &       124.0         &      -16.35         &      -44.26         \\
                    &     (198.0)         &     (192.5)         &     (158.5)         &     (162.3)         \\
[1em]
Asian Male          &      -377.6\sym{**} &      -71.74         &       7.878         &       12.88         \\
                    &     (128.3)         &     (110.4)         &     (87.06)         &     (83.89)         \\
[1em]
Asian Female        &      -705.9\sym{***}&      -295.9\sym{***}&      -165.8\sym{*}  &      -141.7\sym{*}  \\
                    &     (105.8)         &     (82.04)         &     (66.30)         &     (66.03)         \\
[1em]
Asian Two people or Unknown&      -207.9         &       241.2         &      -8.691         &      -122.2         \\
                    &     (144.1)         &     (128.7)         &     (94.61)         &     (96.72)         \\
[1em]
Multiracial or Unknown Male&      -60.11         &       10.06         &       13.69         &       13.33         \\
                    &     (253.7)         &     (210.9)         &     (180.2)         &     (176.9)         \\
[1em]
Multiracial or Unknown Female&      -246.8         &      -266.6         &      -247.5         &      -198.3         \\
                    &     (239.1)         &     (193.0)         &     (193.8)         &     (177.0)         \\
[1em]
Multiracial or Unknown Two people or Unknown&       194.3         &       254.1\sym{*}  &       53.01         &      -22.06         \\
                    &     (108.8)         &     (108.8)         &     (84.67)         &     (80.58)         \\
\hline
Location Fixed Effects&                     &         Yes         &         Yes         &         Yes         \\
Property Fixed Effects&                     &                     &         Yes         &         Yes         \\
Host Fixed Effects  &                     &                     &                     &         Yes         \\
\hline \vspace{-1.25em}&                     &                     &                     &                     \\
Observations        &       68984         &       68981         &       68981         &       68951         \\
Adjusted R2         &     0.00847         &      0.0863         &       0.330         &       0.374         \\

		\hline\hline
		\multicolumn{5}{l}{\footnotesize Standard errors in parentheses}\\
		\multicolumn{5}{l}{\footnotesize \sym{*} \(p<0.05\), \sym{**} \(p<0.01\), \sym{***} \(p<0.001\)}\\
	\end{tabular}
	\label{revenue}
	\begin{tablenotes}
		\item {\it Note:} The dependent variable is a measure of yearly host revenue, as measured by for each listing. The omitted category for race is White males, so all coefficients are relative to that group. The unit of observation is an Airbnb listing, so hosts who have multiple listings are treated separately each time. The sample is the sample of listings across 7 US cities. The specification is the same as Table \ref{table:price}.
	\end{tablenotes}
\end{table}

%  Number of reviews
\begin{table}[htbp]\centering
\def\sym#1{\ifmmode^{#1}\else\(^{#1}\)\fi}
\caption{Estimates of effect of host’s race and gender on number of reviews}
\begin{tabular}{l*{4}{c}}
\hline\hline
                    &\multicolumn{1}{c}{(1)}&\multicolumn{1}{c}{(2)}&\multicolumn{1}{c}{(3)}&\multicolumn{1}{c}{(4)}\\
                    &\multicolumn{1}{c}{Model 1}&\multicolumn{1}{c}{Model 2}&\multicolumn{1}{c}{Model 3}&\multicolumn{1}{c}{Model 4}\\
\hline
White Female        &     -0.0646\sym{**} &     -0.0489\sym{*}  &     -0.0651\sym{***}&     -0.0516\sym{**} \\
                    &    (0.0246)         &    (0.0232)         &    (0.0164)         &    (0.0162)         \\
[1em]
Black Male          &     -0.0618         &     -0.0528         &     -0.0867\sym{**} &     -0.0619         \\
                    &    (0.0636)         &    (0.0542)         &    (0.0334)         &    (0.0321)         \\
[1em]
Black Female        &     -0.0626         &     -0.0633         &      -0.144\sym{***}&      -0.104\sym{***}\\
                    &    (0.0634)         &    (0.0575)         &    (0.0301)         &    (0.0271)         \\
[1em]
Hispanic Male       &     -0.0920         &     -0.0398         &     -0.0567         &     -0.0534         \\
                    &    (0.0530)         &    (0.0503)         &    (0.0353)         &    (0.0315)         \\
[1em]
Hispanic Female     &    0.000356         &      0.0468         &     -0.0277         &      0.0191         \\
                    &    (0.0589)         &    (0.0565)         &    (0.0377)         &    (0.0364)         \\
[1em]
Asian Male          &     -0.0872         &      0.0114         &     -0.0145         &     -0.0147         \\
                    &    (0.0542)         &    (0.0468)         &    (0.0374)         &    (0.0319)         \\
[1em]
Asian Female        &      -0.182\sym{***}&     -0.0662         &     -0.0998\sym{***}&     -0.0529\sym{*}  \\
                    &    (0.0523)         &    (0.0412)         &    (0.0278)         &    (0.0251)         \\
[1em]
Constant            &       1.251\sym{***}&       1.591\sym{**} &       4.876\sym{***}&       3.910\sym{***}\\
                    &     (0.231)         &     (0.558)         &     (0.531)         &     (0.416)         \\
\hline
Location Controls   &                     &         Yes         &         Yes         &         Yes         \\
Property Controls   &                     &                     &         Yes         &         Yes         \\
Host Controls       &                     &                     &                     &         Yes         \\
\hline \vspace{-1.25em}&                     &                     &                     &                     \\
Observations        &       35734         &       35734         &       35734         &       35734         \\
Adjusted R2         &      0.0102         &      0.0742         &       0.455         &       0.559         \\

\hline\hline
\multicolumn{5}{l}{\footnotesize Standard errors in parentheses}\\
\multicolumn{5}{l}{\footnotesize \sym{*} \(p<0.05\), \sym{**} \(p<0.01\), \sym{***} \(p<0.001\)}\\
\end{tabular}
\label{table:num_reviews}

\begin{tablenotes}
\item {\it Note:} The dependent variable is the log number of reviews of the listing. The omitted category for race is White males. The unit of observation is an Airbnb listing, so hosts who have multiple listings are treated separately each time. The sample is the sample of listings across 7 US cities. The specification is the same as Table \ref{table:price}.	
\end{tablenotes}
\end{table}


% GARBAGE Chicago price, ML sentiment controls
\begin{table}[htbp]\centering
\def\sym#1{\ifmmode^{#1}\else\(^{#1}\)\fi}
\caption{Main result: Estimates of effect of Chicago host’s race and gender on price, ML sentiment controls}
\begin{tabular}{l*{5}{c}}
\hline\hline
\input{code/tables/tex_output/individual_tables/chicago_price_sentiment}
\hline\hline
\multicolumn{5}{l}{\footnotesize Standard errors in parentheses}\\
\multicolumn{5}{l}{\footnotesize \sym{*} \(p<0.05\), \sym{**} \(p<0.01\), \sym{***} \(p<0.001\)}\\
\end{tabular}	
\label{table:chiprice}
\begin{tablenotes}

\item {\it Note:} This table presents the impact of host race on the price of a listing. The dependent variable is the log price. The omitted category is White males. The unit of observation is a listing. The sample is the sample of listings in Chicago. Model 1 is the baseline effect of host demographics on price. Model 2 controls for listing location to the neighborhood level and demographic and economic health characteristics on the zipcode-level. Model 3 adds listing characteristics such as the property type and size. Model 4 adds host characteristics such as response and acceptance rates and measures of host effort.  
\end{tablenotes}
\end{table}


% OLD Robustness City
\begin{table}[htbp]\centering
\def\sym#1{\ifmmode^{#1}\else\(^{#1}\)\fi}
\caption{Robustness City}
\begin{tabular}{l*{7}{c}}
\hline\hline
                    &\multicolumn{1}{c}{(1)}&\multicolumn{1}{c}{(2)}&\multicolumn{1}{c}{(3)}&\multicolumn{1}{c}{(4)}&\multicolumn{1}{c}{(5)}&\multicolumn{1}{c}{(6)}&\multicolumn{1}{c}{(7)}\\
                    &\multicolumn{1}{c}{LA}&\multicolumn{1}{c}{NYC}&\multicolumn{1}{c}{Austin}&\multicolumn{1}{c}{Chicago}&\multicolumn{1}{c}{New Orleans}&\multicolumn{1}{c}{DC}&\multicolumn{1}{c}{Nashville}\\
\hline
Black               &      -2.140         &       0.216         &      -22.44         &      -5.119         &      -24.97\sym{*}  &      -15.82\sym{**} &      -16.53         \\
                    &     (8.483)         &     (3.988)         &     (20.55)         &     (5.020)         &     (9.445)         &     (5.621)         &     (10.24)         \\
[1em]
Hispanic            &       8.654         &      -3.570         &       12.48         &     -0.0855         &      -4.189         &      -2.363         &      -40.36\sym{**} \\
                    &     (8.023)         &     (2.893)         &     (11.39)         &     (5.714)         &     (12.29)         &     (7.185)         &     (13.55)         \\
[1em]
Asian               &       0.939         &       1.329         &      -45.73\sym{***}&      -16.85\sym{***}&      -3.150         &      -14.33\sym{*}  &      -5.671         \\
                    &     (5.852)         &     (6.272)         &     (12.07)         &     (4.175)         &     (16.86)         &     (6.042)         &     (23.82)         \\
[1em]
Multiracial or Unknown&       7.168         &       1.392         &       2.283         &       13.52         &      -13.03         &       4.361         &      -43.06\sym{**} \\
                    &     (6.057)         &     (6.277)         &     (15.28)         &     (7.754)         &     (7.951)         &     (6.297)         &     (12.67)         \\
[1em]
Los Angeles         &           0         &                     &                     &                     &                     &                     &                     \\
                    &         (.)         &                     &                     &                     &                     &                     &                     \\
[1em]
New York City       &                     &           0         &                     &                     &                     &                     &                     \\
                    &                     &         (.)         &                     &                     &                     &                     &                     \\
[1em]
Austin              &                     &                     &           0         &                     &                     &                     &                     \\
                    &                     &                     &         (.)         &                     &                     &                     &                     \\
[1em]
Chicago             &                     &                     &                     &           0         &                     &                     &                     \\
                    &                     &                     &                     &         (.)         &                     &                     &                     \\
[1em]
New Orleans         &                     &                     &                     &                     &           0         &                     &                     \\
                    &                     &                     &                     &                     &         (.)         &                     &                     \\
[1em]
Washington DC       &                     &                     &                     &                     &                     &           0         &                     \\
                    &                     &                     &                     &                     &                     &         (.)         &                     \\
[1em]
Nashville           &                     &                     &                     &                     &                     &                     &           0         \\
                    &                     &                     &                     &                     &                     &                     &         (.)         \\
\hline
Location Fixed Effects&         Yes         &         Yes         &         Yes         &         Yes         &         Yes         &         Yes         &         Yes         \\
Property Fixed Effects&         Yes         &         Yes         &         Yes         &         Yes         &         Yes         &         Yes         &         Yes         \\
Host Fixed Effects  &         Yes         &         Yes         &         Yes         &         Yes         &         Yes         &         Yes         &         Yes         \\
\hline \vspace{-1.25em}&                     &                     &                     &                     &                     &                     &                     \\
Observations        &       26076         &       20417         &        5818         &        5144         &        4511         &        3722         &        3277         \\
Adjusted R2         &       0.438         &       0.280         &       0.468         &       0.366         &       0.475         &       0.450         &       0.616         \\

\hline\hline
\multicolumn{8}{l}{\footnotesize Standard errors in parentheses}\\
\multicolumn{8}{l}{\footnotesize \sym{*} \(p<0.05\), \sym{**} \(p<0.01\), \sym{***} \(p<0.001\)}\\
\end{tabular}
\label{table:robustcity_old}

\begin{tablenotes}
\item {\it Note:} This table breaks down the effects for the combined data in Table \ref{table:price} across the 7 cities in the sample. Each set of coefficients represents the coefficient on host race, with log price as the outcome variable. I control for my preferred specification throughout. Low number of observations for Black, Hispanic, and Asian hosts contribute to imprecise estimates in smaller cities (New Orleans, Nashville have less than 100 Hispanic and Asian hosts; DC and Austin have less than 200 such hosts). 
\end{tablenotes}
\end{table}

% OLD Robustness Listing Characteristics
\begin{landscape}
\begin{table}[htbp]\centering
\def\sym#1{\ifmmode^{#1}\else\(^{#1}\)\fi}
\caption{Robustness Listing Characteristics}
\begin{tabular}{l*{9}{c}}
\hline\hline
                    &\multicolumn{1}{c}{(1)}&\multicolumn{1}{c}{(2)}&\multicolumn{1}{c}{(3)}&\multicolumn{1}{c}{(4)}&\multicolumn{1}{c}{(5)}&\multicolumn{1}{c}{(6)}&\multicolumn{1}{c}{(7)}&\multicolumn{1}{c}{(8)}&\multicolumn{1}{c}{(9)}\\
                    &\multicolumn{1}{c}{Low $ LA}&\multicolumn{1}{c}{High $ LA}&\multicolumn{1}{c}{Low $ NY}&\multicolumn{1}{c}{High $ NY}&\multicolumn{1}{c}{Older Listings}&\multicolumn{1}{c}{Newer Listings}&\multicolumn{1}{c}{Apartments}&\multicolumn{1}{c}{Condos}&\multicolumn{1}{c}{Houses}\\
\hline
Black               &       8.484         &      -20.53         &       0.943         &       3.865         &      -1.747         &      -8.164\sym{***}&      -2.718         &      -7.304         &      -13.45         \\
                    &     (11.03)         &     (11.78)         &     (1.860)         &     (10.13)         &     (6.942)         &     (1.820)         &     (2.648)         &     (11.16)         &     (11.14)         \\
[1em]
Hispanic            &       3.847         &      -13.61         &      -4.802         &      -6.285         &       3.999         &       2.589         &      -3.703         &      -18.38         &       9.505         \\
                    &     (6.405)         &     (18.78)         &     (3.139)         &     (5.966)         &     (7.187)         &     (3.319)         &     (1.887)         &     (9.518)         &     (10.53)         \\
[1em]
Asian               &      -5.738\sym{**} &      -4.347         &       5.217         &      -1.648         &      -9.855\sym{***}&      -1.425         &      -2.103         &      -21.21\sym{*}  &      -10.04         \\
                    &     (1.949)         &     (16.21)         &     (8.339)         &     (8.934)         &     (2.908)         &     (3.453)         &     (3.810)         &     (10.57)         &     (6.042)         \\
[1em]
Multiracial or Unknown&      -0.227         &       3.819         &      -1.555         &       4.168         &      -2.160         &      -3.304         &      -0.414         &      -11.94         &       16.39         \\
                    &     (1.391)         &     (14.43)         &     (2.210)         &     (10.40)         &     (4.439)         &     (3.243)         &     (2.541)         &     (15.56)         &     (8.691)         \\
\hline
Location Fixed Effects&         Yes         &         Yes         &         Yes         &         Yes         &         Yes         &         Yes         &         Yes         &         Yes         &         Yes         \\
Property Fixed Effects&         Yes         &         Yes         &         Yes         &         Yes         &         Yes         &         Yes         &         Yes         &         Yes         &         Yes         \\
Host Fixed Effects  &         Yes         &         Yes         &         Yes         &         Yes         &         Yes         &         Yes         &         Yes         &         Yes         &         Yes         \\
\hline \vspace{-1.25em}&                     &                     &                     &                     &                     &                     &                     &                     &                     \\
Observations        &       17155         &        8921         &       10717         &        9700         &       15193         &       39268         &       41254         &        2903         &       22236         \\
Adjusted R2         &      0.0968         &       0.480         &      0.0513         &       0.326         &       0.547         &       0.477         &       0.312         &       0.416         &       0.461         \\

\hline\hline
\multicolumn{10}{l}{\footnotesize Standard errors in parentheses}\\
\multicolumn{10}{l}{\footnotesize \sym{*} \(p<0.05\), \sym{**} \(p<0.01\), \sym{***} \(p<0.001\)}\\
\end{tabular}
\label{table:robustlistingold}

\begin{tablenotes}
\item {\it Note:} This table breaks the effects for the combined data by high versus low price, time on market, and property type. The categories, from left to right, are: listings whose log price is below vs. above the mean predicted log price in each city, the price originally dropped, listings who have have been on the market for no more than 2 years vs. no more than 8 years, and listings of different property types, including apartments (includes apartments and lofts), condos (includes condos and townhouse), and houses. I control for my preferred specification throughout. The outcome variable is the log price of the listing.
\end{tablenotes}
\end{table}
\end{landscape}

\end{comment}


	
	
	\begin{comment}
		
	In this section, I consider alternative explanations for the price differences I found between minority and white hosts. In particular, I examine three relevant mechanisms that are not discrimination that could explain my results. I will then argue that these mechanisms would be insufficient in doing so. 
	
	\textbf{2. Price observed during the scrape is not the price normally set by hosts or observed by guests}
	
	The prices I used in my data analysis were prices from one particular day in 2015-2016. I ran my regressions on the assumption that this is the price that hosts set and that guests observe that drives guest booking decisions and host revenue outcomes. However, imagine that for one day, all of the white people on Airbnb raise their prices. That day, the scrape happens to take place, and the next day, the prices of White hosts all go back down. Then, the price that I observed in my data for White hosts would be high even though it doesn't incorporate real demand differences between races nor is tied to listing characteristics or host quality. While this may seem like an unlikely scenario, price hikes for the weekend or for holidays like July 4th or New Year's may give rise to a similar situation if black hosts change their prices differently than White hosts. 
	

	\textbf{3. Data clean-up disproportionately dropped lower-priced listings of White hosts or higher-priced listings of minority hosts}
	
	During my analysis, I left out hosts that had no profile picture. If hosts are aware of the potential effects of discrimination, minority hosts might be less likely to include a profile picture. If minority hosts who own higher-priced listings dropped out of the data set in this way, my coefficients might be biased downward because the sample doesn't include the higher-priced listings owned by minority hosts. A similar thing could have happened when I restricted my data set to listings with a price per night of less than \$800, and to listings owned by hosts who owned fewer than 20 listings total. If those hosts who owned a lot of listings or charged wildly high prices per night were disproportionately black, Hispanic, or Asian, I'd see a similar downward bias on my coefficients.
	
	\iffalse %comments out entire section of text
	\begin{quotation}
	``Suppose there are two groups, designated by $W$ and $N$, with members of $W$ being perfect substitutes in production for members of $N$. In the absence of discrimination and nepotism and if the labor market were perfectly competitive, the equilibrium wage rate of $W$ would equal that of $N$. Discrimination could cause these wage rates to differ; the market discrimination coefficient between $W$ and $N$ [...] is defined as the proportional difference between these wage rates" \end{quotation}
	\fi
	\end{comment}
	
	
\end{document}  
